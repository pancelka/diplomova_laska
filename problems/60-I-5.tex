\problem{60-I-5}{
Dokážte, že najmenší spoločný násobok $[a, b]$ a najväčší spoločný deliteľ $(a, b)$ ľubovoľných dvoch kladných celých čísel $a, b$ spĺňajú nerovnosť
$$a \cdot (a, b) + b \cdot [a, b] \geq 2ab.$$
Zistite, kedy v~tejto nerovnosti nastane rovnosť.
}{
\rieh Nerovnosť by bolo ľahké dokázať, ak by niektorý z~dvoch sčítancov na ľavej strane bol sám osebe aspoň taký, ako pravá strana. Číslo $[a, b]$ je zjavne násobkom čísla $a$. Ak $[a, b] \geq 2a$, tak $b[a, b] \geq 2ab$ a v~zadanej nerovnosti platí dokonca ostrá nerovnosť, lebo číslo $a(a, b)$ je kladné. Ak $[a, b] < 2a$, tak neostáva iná možnosť ako $[a, b] = a$. To však nastane iba v~prípade, keď $b \mid a$. V~tomto prípade $(a, b) = b$ a v~zadanej nerovnosti
nastane rovnosť.\\
\\
\textbf{Iné riešenie.} Označme $d = (a, b)$, takže $a = ud$ a $b = vd$ pre nesúdeliteľné prirodzené čísla $u, v$. Z~toho hneď vieme, že $[a, b] = uvd$. Keďže
$$a \cdot (a, b) + b \cdot [a, b] = ud^2+ uv^2d^2= u(1 + v^2
)d^2,$$
$$2ab = 2uvd^2,$$
je vzhľadom na $ud^2 > 0$ nerovnosť zo zadania ekvivalentná s~nerovnosťou $1 + v^2 \geq 2v$, čiže $(v - 1)^2 \geq0$, čo platí pre každé $v$. Rovnosť nastane práve vtedy, keď $v = 1$, čiže $b \mid a$.\\
\\
\textbf{Iné riešenie.} Označme $d = (a, b)$. Je známe, že $[a, b] \cdot (a, b) = ab$. Po vyjadrení $[a, b]$ z~tohto vzťahu, dosadení do zadanej nerovnosti a ekvivalentnej úprave dostaneme ekvivalentnú nerovnosť $d^2 + b^2 \geq 2bd$, ktorá platí, lebo $(d - b)^2 \geq 0$. Rovnosť nastáva
pre $d = b$, čiže v~prípade $b \mid a$.\\
\\
\kom Na úspešné zvládnutie úlohy je opäť potrebná znalosť z~predchádzajúceho seminára o~nerovnostiach a taktiež ponúka široké spektrum prístupov, takže bude zaujímavé sledovať, ako k~nej študenti pristúpia.
}
