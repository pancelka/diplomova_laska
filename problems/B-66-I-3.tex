% Do not delete this line (pandoc magic!)

\problem{B-66-I-3}{
Na kružnici $k$ sú zvolené body $A$, $B$, $C$, $D$, $E$ (v tomto poradí) tak, že platí $|AB| = |CD| = |DE|$. Dokážte, že ťažiská trojuholníkov $ABD$, $ACD$ a $BDE$ ležia na kružnici sústrednej s kružnicou $k$.
}{
\rieh Vzhľadom na podmienky úlohy je $|\ma CBD| = |\ma ADB| = |\ma EAD|$, pretože všetky tri uvažované uhly vytínajú na kružnici $k$ zhodné tetivy. Keďže $|\ma CBD|= |\ma ADB|$, je $AD \parallel BC$. 

Tetivový štvoruholník $ABCD$ je teda rovnoramenný lichobežník či pravouholník, v ktorom sú (zhodné) trojuholníky $DAB$ a $ADC$ súmerne združené podľa spoločnej osi $o_1$ strán $AD$, $BC$. Tá však prechádza stredom $O$ kružnice $k$. V uvedenej súmernosti si tak zodpovedajú aj ťažiská $K$ a $L$ oboch zhodných trojuholníkov $DAB$ a $ADC$ \todo{(obr. 4)}. Os úsečky $KL$ teda prechádza stredom $O$ kružnice $k$. Navyše oba body $K$ a $L$ sú rôzne, pretože zodpovedajúce si ťažnice z (rôznych) vrcholov $B$ a $C$ sa pretínajú v strede spoločnej strany $AD$ oboch zhodných trojuholníkov, zatiaľ čo ťažiská sú vnútornými bodmi oboch úsečiek.\\
\\
\todo{DOPLNIŤ Obr. 4}\\
\\Analogicky dokážeme, že aj $ABDE$ je rovnoramenný lichobežník či pravouholník. Pre ťažiská $K$, $M$ zhodných trojuholníkov $DAB$ a $BED$ preto platí, že aj os úsečky $KM$ prechádza stredom $O$ kružnice $k$. Odtiaľ $|OL| = |OK| = |OM| > 0$ (body $K$ a $L$ sú rôzne), takže ťažiská všetkých troch uvažovaných trojuholníkov ležia na kružnici
sústrednej s $k$. Tým je dôkaz ukončený.
}
