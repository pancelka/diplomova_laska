% Do not delete this line (pandoc magic!)

\problem{62-I-4}{
Rozhodnite, či z ľubovoľných siedmich vrcholov daného pravidelného 19-uholníka možno vždy vybrať štyri, ktoré sú vrcholmi lichobežníka.
}{
\rieh Označme $S$ stred daného pravidelného 19-uholníka $A_ 1, A_2 \ldots A_{19}$. Os každej úsečky $A_iA_j$ je priamka, ktorá okrem bodu $S$ prechádza ešte niektorým vrcholom $A_k$ (to vďaka tomu, že číslo 19 je nepárne). Preto sa dajú všetky úsečky $A_ iA_j$ rozdeliť na 19 skupín, pričom v každej skupine budú navzájom rovnobežné úsečky so spoločnou osou, ktorou je vždy jedna z priamok $SA_ k$. V každej skupine je pritom zrejme $(19 -1) : 2 = 9$ úsečiek a každé dve z nich sú základňami lichobežníka (nemôže sa jednať o rovnobežník, lebo žiadna z úsečiek $A_iA_j$ neprechádza stredom $S$, opäť vďaka tomu, že číslo 19 je nepárne).

Počet všetkých úsečiek $A_iA_j$ s krajnými bodmi v ľubovoľne vybranej sedemprvkovej množine vrcholov je $(7 \cdot 6) : 2 = 21 > 19$, takže dve z týchto úsečiek ležia v rovnakej z 19 opísaných skupín. Tým je existencia žiadaného lichobežníka dokázaná, nech už je sedemprvková množina vrcholov zvolena akokoľvek.\\
\textit{Poznámka.} Úvodnú úvahu o osi úsečky $A_iA_j$ možno vynechať. Namiesto toho môžeme rovno opísať uvedených 19 deväťprvkových skupín navzájom rovnobežných úsečiek a potom skonštatovať, že ide o všetky možné úsečky $A_iA_j$, lebo tých je $(19 \cdot 18) : 2 = 19 \cdot 9$, teda práve toľko, koľko je úsečiek v opísaných 19 skupinách.\\
\textbf{Iné riešenie.} Zatiaľ čo v prvom riešení sme uvažovali o základniach hľadaného lichobežníka, teraz sa zameriame na jeho ramená alebo uhlopriečky. V oboch prípadoch to musia byť dve zhodné úsečky, lebo každý lichobežník, ktorému možno opísať kružnicu, je rovnoramenný. Osi jeho základní totiž musia prechádzať stredom opísanej kružnice, takže splývajú a tvoria tak os súmernosti celého lichobežníka. Naopak každé dve tetivy jednej kružnice, ktoré majú rovnakú dĺžku kratšiu ako priemer kružnice, nie sú rovnobežné a nemajú spoločný krajný bod, tvoria buď ramená, alebo uhlopriečky (rovnoramenného) lichobežníka (stačí si uvedomiť, že ľubovoľné dve zhodné tetivy jednej kružnice sú súmerne združené podľa priamky prechádzajúce stredom uvedenej kružnice a priesečníkom prislúchajúcich sečníc).

V pravidelnom 19-uholníku $A_1A_2 \ldots A_{19}$ majú zrejme všetky úsečky $A_iA_j$ dokopy len 9 rôznych dĺžok. Vo vybranej sedemprvkovej množine vrcholov má oba krajné body celkom $(7 \cdot 6) : 2 = 21$ úsečiek. Keďže $21 > 2 \cdot 9$, podľa Dirichletovho princípu niektoré tri z týchto úsečiek majú rovnakú dĺžku (t. j. sú zhodné). Keby každé dve z týchto troch úsečiek mali spoločný vrchol (a vieme, že z ľubovoľného vrcholu vychádzajú nanajvýš dve zhodné strany či uhlopriečky), vytvorili by tieto tri úsečky rovnostranný trojuholník, čo nie je možné, lebo $3 \nmid  19$. Preto niektoré dve z týchto troch zhodných úsečiek nemajú spoločný krajný bod, takže to sú buď ramená, alebo uhlopriečky rovnoramenného lichobežníka (protiľahlé strany rovnobežníka to byť nemôžu).
}