% Do not delete this line (pandoc magic!)

\problem{63-II-4}{Daný je konvexný štvoruholník $ABCD$ s bodom$E$ vnútri strany $AB$ tak, že platí $|\ma ADE| = |\ma DEC| = |\ma ECB|$. Obsahy trojuholníkov $AED$ a $CEB$ sú postupne 18\,cm$^2$ a 8\,cm$^2$ . Určte obsah trojuholníka $ECD$.
}{
\rieh Hľadaný obsah trojuholníka $ECD$ označme $S$. Uhol $DEC$ je striedavý s uhlami $ADE$ a $ECB$, odtiaľ $AD  \parallel EC$ a $ED \parallel BC$ (\ref{fig:63II4}). Trojuholníky $EDA$
\begin{figure}[h]
    \centering
    \includegraphics{images/63K41\imagesuffix}
    \caption{}
    \label{fig:63II4}
\end{figure}
a $EDC$ majú spoločnú stranu $ED$, pomer ich obsahov je teda rovný pomeru prislúchajúcich výšok. Ak navyše postupne označíme $P, Q$ a $R$ kolmé priemety vrcholov $A, B$ a $C$ na priamku $DE$ a označíme $v = |AP|$, $w = |BQ| = |CR|$, dostaneme z podobných
pravouhlých trojuholníkov $AEP$ a $BEQ$ úmeru
$$\frac{18}{S}=\frac{v}{w}=\frac{|AE|}{|EB|}.$$
Analogicky pre trojuholníky $ECD$ a $ECB$ zistíme, že
$$\frac{8}{S}=\frac{|EB|}{|AE|}.$$
(V \ref{fig:63II4} sú prislúchajúce priemety iba naznačené, ale jedná sa o ten istý výpočet
ako v predošlom odseku, len v ňom zameníme zodpovedajúce body $A\leftrightarrow B, C \leftrightarrow D$ a prislúchajúce obsahy trojuholníkov $AED$ a $BEC$.) Dokopy teda je $S : 8 = 18 : S$ čiže $S^2 = 144$, takže trojuholník $ECD$ má obsah $S = 12$\,cm$^2$ .
}