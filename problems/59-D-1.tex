\problem{59-D-1}{
Erika a Klárka hrali hru ”slovný logik“ s týmito pravidlami: Hráč $A$ si myslí slovo zložené z piatich rôznych písmen. Hráč $B$ vysloví ľubovoľné slovo zložené z piatich rôznych písmen a hráč $A$ mu prezradí, koľko písmen uhádol na správnej pozícii a koľko na nesprávnej. Písmená považujeme za rôzne, aj keď sa líšia iba mäkčeňom alebo dĺžňom (napríklad písmena $A$, \textit{Á} sú rôzne). Keby si hráč $A$ myslel napríklad slovo \textit{LOĎKA} a $B$ by vyslovil slovo \textit{KOLÁČ}, odpovie hráč $A$, že jedno písmeno uhádol hráč $B$ na správnej pozícii a dve na nesprávnej. Skrátene oznámi \uv{1 + 2}, lebo sa naozaj obe slová zhodujú iba v písmene $O$ vrátane pozície (druhej zľava) a v písmenách $K$ a $L$, ktorých pozície sú odlišné. Erika si myslela slovo z piatich rôznych písmen a Klárka vyslovila slová \textit{KABÁT, STRUK, SKOBA, CESTA} a \textit{ZÁPAL}. Erika na tieto slová v danom poradí odpovedala 0 + 3, 0 + 2, 1 + 2, 2 + 0 a 1 + 2. Zistite, aké slovo si Erika mohla myslieť.
}{
\rieh Slová \textit{ZÁPAL} a $STRUK$ nemajú spoločné písmená. Preto sa, ako vyplýva z odpovedí 1 + 2 a 0 + 2, medzi ich písmenami, ktoré dokopy tvoria množinu $M= \{Z$, \textit{Á}, $P, A, L, S, T, R, U, K\}$, nachádza všetkých päť písmen hľadaného slova. V slove $SKOBA$ majú byť práve tri z hľadaných písmen. Sú to teda písmená $S, K, A$. (Zvyšné písmená $B$ a $O$ totiž do množiny $M$ nepatria.) V slove $CESTA$ majú byť len dve z hľadaných písmen, a obe na správnej pozícii. Sú to už nájdené $S$ a $A$, ktoré teda patria na tretie, resp. piate miesto hľadaného slova (a písmeno $T$ môžeme z množiny $M$ ”vylúčiť“). Písmeno $K$ nemôže byť ani na prvom, ani na druhom mieste: vyplýva to z odpovedí pre slová \textit{KABÁT} (0 + 3) a $SKOBA$ (1 + 2). Takže je na štvrtom mieste a ostáva určiť prvé dve písmená. V slove $STRUK$ sú len dve z hľadaných písmen (musia to teda byť $S$ a $K$), obe na nesprávnych pozíciách. Preto z množiny $M$ \uv{ vylúčime} aj písmená $R, U$ (a $T$, ak sme to doteraz neurobili). Zvyšné dve hľadané písmená potom patria do množiny $\{Z$, \textit{Á}, $P, L\}$. Z podmienok pre slovo \textit{KABÁT} vyplýva, že jedno z nich je \textit{Á}. V slove \textit{ZÁPAL} je práve jedno písmeno na správnej pozícii. Keby to bolo $Z$, nemali by sme kam uložiť písmeno \textit{Á}. Takže \textit{Á} je na druhom mieste a navyše môžeme vylúčiť písmeno $Z$. Na prvom mieste hľadaného slova môže byť $L$ alebo $P$. Ľahko sa presvedčíme, že nájdené slová \textit{LÁSKA} aj \textit{PÁSKA} vyhovujú všetkým podmienkam úlohy.\\
\\
\kom Úloha opäť nevyžaduje žiadne matematické znalosti, je však výbornou previerkou toho, ako sú študenti schopní narábať s veľkým množstvom informácií, nestratiť v nich prehľad a využiť ich na zdarné vyriešenie zadaného problému. \\
\\
}
