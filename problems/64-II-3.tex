% Do not delete this line (pandoc magic!)

\problem{64-II-3}{
Daný je lichobežník $ABCD$ so základňami $AB$, $CD$, pričom $2|AB| = 3|CD|$.

a) Nájdite bod $P$ vnútri lichobežníka tak, aby obsahy trojuholníkov $ABP$ a $CDP$ boli v~pomere $3 : 1$ a aj obsahy trojuholníkov $BCP$ a $DAP$ boli v~pomere $3 : 1$.

b) Pre nájdený bod $P$ určte postupný pomer obsahov trojuholníkov $ABP$, $BCP$, $CDP$ a $DAP$.
}{
\rieh Predpokladajme, že bod $P$ má požadované vlastnosti. Priamka rovnobežná so základňami lichobežníka a prechádzajúca bodom $P$ pretína ramená $AD$ a $BC$ postupne v~bodoch $M$ a $N$ (obr. 6). Označme $v$ výšku daného lichobežníka, $v_1$ výšku trojuholníka $CDP$ a $v_2$ výšku trojuholníka $ABP$.
\begin{center}
\includegraphics{images/64K3\imagesuffix}

Obr. 6
\end{center}
a) Keďže obsahy trojuholníkov $ABP$ a $CDP$ sú v~pomere $3 : 1$, platí
$$\frac{|AB|v_2}{2}:\frac{|CD|v_1}{2}= 3 : 1, \ \ \ \ \text{čiže} \ \ \ \ \frac{v_1}{v_2}=\frac{1}{3}\cdot \frac{|AB|}{|CD|}=\frac{1}{3}\cdot \frac{3}{2}=\frac{1}{2}.$$
Z~vyznačených dvojíc podobných pravouhlých trojuholníkov vyplýva, že v~práve určenom pomere $2 : 1$ výšok $v_2$ a $v_1$ delí aj bod $M$ rameno $AD$ a bod $N$ rameno $BC$ (v~prípade pravého uhla pri jednom z~vrcholov $A$ či $B$ je to zrejmé rovno). Tým je konštrukcia bodov $M$ a $N$, a teda aj úsečky $MN$ určená. Teraz zistíme, v~akom pomere ju delí uvažovaný bod $P$.

Keďže obsahy trojuholníkov $BCP$ a $DAP$ sú v~pomere $3 : 1$, platí
$$ \bigg(\frac{|NP|v_1}{2}+\frac{|NP|v_2}{2}\bigg): \bigg(\frac{|MP|v_1}{2}+\frac{|MP|v_2}{2}\bigg)= 3 : 1,$$
$$ \frac{|NP|(v_1 + v_2 )}{2}: \frac{|MP|(v_1 + v_2 )}{2}= 3 : 1, \ \ \ \  |NP| : |MP| = 3 : 1.$$
Tým je konštrukcia (jediného) vyhovujúceho bodu $P$ úplne opísaná.

b) Doplňme trojuholník $DAC$ na rovnobežník $DAXC$. Jeho strana $CX$ delí priečku $MN$ na dve časti, a keďže $v_1 =\frac{1}{3}v$, môžeme dĺžku priečky $MN$ vyjadriť ako $|MN| = |MY | + |Y N| = |AX| +\frac{1}{3} |XB| = |CD| +\frac{1}{3} (|AB| - |CD|) = \frac{1}{3}|AB| +\frac{2}{3}|CD| = \frac{7}{6}|CD|$, lebo podľa zadania platí $|AB| =\frac{3}{2}|CD|$. Preto
$$|MP| =\frac{1}{4}|MN| =\frac{1}{4} \cdot \frac{7}{6}|CD| = \frac{7}{24}|CD|,$$
takže pre pomer obsahov trojuholníkov $CDP$ a $DAP$ platí
$$ \frac{|CD|v_1}{2}:\frac{|MP|(v_1 + v_2 )}{2}= (|CD|v_1 ) : \bigg( \frac{7}{24}\cdot |CD| \cdot 3v_1\bigg)= 1 :\frac{7}{8} = 8 : 7.$$
Pomer obsahov trojuholníkov $BCP$ a $CDP$ je teda $21 : 8$ a pomer obsahov trojuholníkov $ABP$ a $BCP$ je tak $24 : 21$. Postupný pomer obsahov trojuholníkov $ABP$, $BCP$, $CDP$ a $DAP$ je preto $24 : 21 : 8 : 7$.\\
\\
\kom Táto komplexná úloha je vrcholom tohto seminárneho stretnutia. Vyžaduje umnú prácu s~pomermi obsahov, podobnými trojuholníkmi aj netriviálny nápad doplnenia trojuholníka $DAC$ na rovnobežník. Je tak vhodné skôr než samostatne úlohu riešiť spoločne na tabuľu. Študentom tiež pripomenieme, že podobne ako v~úvodnej úlohe, aj tu našlo vhodné rozdelenie zadaného útvaru svoje opodstatnenie a prispelo k~úspešnému rozklúsknutiu problému. \\
\\
}
