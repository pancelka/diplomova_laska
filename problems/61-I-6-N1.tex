% Do not delete this line (pandoc magic!)

\problem{61-I-6-N1}{}{
Na hracej ploche $m\times n$ tvorenej bielymi štvorcovými políčkami sa Monika a Tamara striedajú v~ťahoch jednou figúrkou pri nasledujúcej hre. Najskôr Monika položí figúrku na ľubovoľné políčko a toto políčko zafarbí namodro. Ďalej vždy hráčka, ktorá je na ťahu, urobí s~figúrkou skok na políčko, ktoré je doposiaľ biele a zafarbí toto políčko namodro. Pritom pod skokom rozumieme ťah šachovou vežou,  t.\,j. presuny figúrky v~smere riadkov alebo v~smere stĺpcov hracej dosky (o~ľubovoľný počet políčok). Hráčka, ktorá je na rade a už nemôže urobiť ťah, prehráva. Rozhodnite, ktoré z~hráčok môže hrať tak, že vyhrá nezávisle na ťahoch druhej hráčky?
}{
\rieh Ak sú obe čísla $m$ a $n$ nepárne, má víťaznú stratégiu prvá hráčka, ak je aspoň jedno z~čísel $m, n$ párne, má víťaznú stratégiu druhá hráčka. V~oboch prípadoch si uvedená hráčka vopred v~duchu rozdelí všetky políčka hracej dosky do dvojíc (v~prvom prípade jedno políčko ostane, naň potom hráčka položí figúrku v~úvodnom ťahu), a to tak, aby v~každom zostavenom páre boli políčka navzájom dosiahnuteľné jedným skokom (pre ťahy vežou je to ľahké, stačí párovať len susedné políčka riadku alebo stĺpca); v~priebehu hry potom táto hráčka môže vždy skočiť z~jedného políčka na druhé políčko toho istého páru, takže vyhrá.\\
\\
\kom Úloha je netriviálna, jej zvládnutie je však výborným predpokladom na úspešné vyriešenie domácej práce. Ak nám ostane na konci seminára dostatok času, môžeme študentov najprv odohrať pár kôl hry pre nimi zvolené rozmery tabuľky a na základe poznatkov z hry potom presne popísať víťaznú stratégiu. \\
\\
}
