% Do not delete this line (pandoc magic!)

\problem{~\cite{holton2010}, úloha 20, str. 110}{}{
Nájdite všetky prirodzené dvojciferné čísla, ktoré sa rovnajú dvojnásobku súčinu svojich cifier.
}{
\rieh Označme naše hľadané číslo $\overline{ab}$. Zadanie potom môžeme napísať ako $10a+b=2ab$, pričom $a\neq 0$ (inak by hľadané číslo nebolo dvojciferné). Taktiež $b\neq 0$, pretože $10a+b=2ab>0$. Keďže čísla $10a$ a $2ab$ sú párne, musí byť párne aj $b$, teda $b=2k$, $k \in \{1, 2, 3, 4 \}$. Využitím tohto poznatku môžeme rovnicu z úvodu riešenia upraviť na tvar $10a=b(2a-1)=2k(2a-1)$, teda $5a=k(2a-1)$ Preto buď $5 \mid k$ alebo $5 \mid (2a-1)$. Keďže ale $k \in \{1, 2, 3, 4\}$, platí $5\nmid k$ a preto musí platiť $5\mid(2a-1)$. Jediné možnosti, ktoré túto podmienku spĺňajú, sú $a=3$ alebo $a=8$. Ak je $a=8$, dostávame zo zadania $10\cdot 8 + b=2\cdot8\cdot b$, teda $b=16/3$. To však riešením úlohy byť nemôže, pretože $b$ musí byť celé jednociferné číslo. Preto ostáva len možnosť $a=3$, odkiaľ odvodíme $b=6$. Skúškou overíme, že nájdené číslo 36 vyhovuje zadaniu.
\\
\kom Úloha má krátke a jednoduché zadanie, jej riešenie však vyžaduje uplatnenie znalostí o deliteľnosti aj zápis čísla v rozvinutom tvare.\\
\\
}
