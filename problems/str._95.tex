% Do not delete this line (pandoc magic!)

\problem{\cite{thiele1986}, príklad 3, str. 95}{seminar20,prvocisla}{
Nájdite celočíselné riešenia rovnice $$\frac{1}{x}+\frac{1}{y}=\frac{1}{p},$$ kde $p$ je pevne dané prvočíslo.
}{
\rieh Ak existujú vôbec nejaké riešenia vyšetrovanej rovnice, potom sú nenulové. Preto môžeme rovnicu upraviť na ekvivalentný tvar $yx-px-py=0$, resp. $(x-p)(y-p)-p^2=0$, a teda $$(x-p)(y-p)=p^2.$$ Odtiaľ je vidieť, že celočíselné riešenia môžeme dostať len vtedy, ak $x-p$ prebehne všetkých deliteľov čísla $p^2$, pričom $y-p$ prebehne doplnkové delitele. Pretože je $p$ prvočíslo, musí byť nutne $$x-p \in \{1, p, p^2, -1, -p, -p^2\}.$$ Pretože $x\neq 0$, odpadá $x-p=-p$. Ostáva teda $$x \in \{1+p, 2p, p+p^2, p-1, p-p^2\} \ \ \ \ \text{a teda} \ \ \ \ y \in \{p+p^2, 2p, 1+p, p-p^2, p-1\}.$$ Tieto hodnoty sú skutočne riešením, o~čom sa môžeme presvedčiť skúškou.\\
\\
\kom Úloha, v~ktorej opäť predtým, než uplatníme znalosti o~deliteľnosti, príp. prvočíslach, musíme umne upraviť východiskový tvar rovnice.
}
