\seminar{14}{Matematická súťaž v~riešení úloh \textit{Náboj}}

\teachernote{
\subsection*{Príprava a prevedenie súťaže}
Predvianočné seminárne stretnutie sa ponesie v~odľahčenom duchu. Pre študentov bude pripravená tímová aktivita v~štýle súťaže \textit{Náboj}.

Pred seminárom je potrebné vytlačiť a pripraviť zadania úloh pre študentov. Úlohy sú dostupné na \cite{Naboj} v~rôznych jazykových mutáciách a voľne prístupné na použitie. Každý tím bude mať vlastnú sadu zadaní, rozstrihaných na jednotlivé úlohy. Okrem toho je potrebné mať pripravenú aj verziu pre
opravovateľov, kde nájdu úlohy spolu s~výsledkami a stručným riešením.

Posledným prípravným krokom je osloviť kolegov, ktorí by nám boli ochotní s~organizáciou pomôcť, keďže pre jedného človeka je kontrolovanie výsledkov, vydávanie nových úloh a zapisovanie priebežného poradia dosť náročné, ak nie nemožné. S~dobrovoľnými pomocníkmi sa potom pred seminárom dohodneme na rozdelení rolí, aby si prípadní opravovatelia mali čas prejsť zadania úloh.

Na seminárnom stretnutí potom rozdelíme študentov do štvorčlenných až päťčlenných tímov, vysvetlíme pravidlá, zodpovieme prípadné otázky, rozdáme zadania a súťaž môže začať. Ak máme študentov na seminári menej, je možné súťažiť aj v~trojčlenných tímoch.
}
