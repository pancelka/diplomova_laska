% Do not delete this line (pandoc magic!)

\problem{B-58-II-1}{}{
V obore reálnych čísel riešte sústavu rovníc
\begin{align*}
    x + y & = 1,\\
    x - y & = a,\\
    -4ax + 4y & = z^2 + 4
\end{align*}
s neznámymi $x$, $y$, $z$ a reálnym parametrom $a$.
}{
\rieh Sčítaním prvej a druhej rovnice danej sústavy dostaneme $2x = 1 + a$, odčítaním druhej rovnice od prvej $2y = 1 - a$. Odtiaľ
\begin{equation} \label{eq:B58II1_1}
    x =\frac{1}{2}(1 + a), \ \ \ \  y =\frac{1}{2}(1 - a).
\end{equation}
Keď dosadíme za $x$ a $y$ do tretej rovnice pôvodnej sústavy, dostaneme rovnicu
$$-2a(1 + a) + 2(1 - a) = z^2 + 4,\ \ \ \ \text{čiže} \ \ \ \ z^2 + 2a^2 + 4a + 2 = 0,$$
ktorú upravíme na tvar
$$z^2 + 2(a + 1)^2 = 0.$$
Oba sčítance na ľavej strane poslednej rovnice sú nezáporné čísla. Ich súčet je 0 práve vtedy, keď $z = 0$, $a = -1$. Dosadením týchto hodnôt do \ref{eq:B58II1_1} dostaneme $x = 0, y = 1$.

\textit{Záver.} Daná sústava rovníc má riešenie iba pre $a = -1$, a to $x = 0$, $y = 1$, $z = 0$. Skúška pri tomto postupe nie je nutná.
\\
\\
\kom Úloha vyžaduje umné narábanie so sústavou troch rovníc tak, aby bolo možné uskutočniť záverečnú diskusiu o existencii riešenia pre rôzne hodnoty parametra $a$. Je tiež vhodné so študentami prediskutovať, prečo v tomto prípade nie je nutné robiť skúšku správnosti.\\
\\
}
