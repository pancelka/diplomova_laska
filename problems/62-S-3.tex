% Do not delete this line (pandoc magic!)

\problem{62-S-3}{kombi,ofarbovanie,skolskekolo}{
Každý vrchol pravidelného devätnásťuholníka je ofarbený jednou zo šiestich farieb. Dokážte, že niektorý tupouhlý trojuholník má všetky vrcholy ofarbené rovnakou farbou.
}{
\rieh Keďže $19 > 6 \cdot 3$, majú rovnakú farbu niektoré štyri vrcholy, ktoré označíme $A, B, C, D$ v poradí na opísanej kružnici. Tie tvoria vrcholy konvexného štvoruholníka, ktorého vnútorné uhly majú súčet $360^{\circ}$, takže nemôžu byť všetky menšie ako $90^{\circ}$. Zároveň je zrejmé, že žiadny z nich nemôže byť rovný $90^{\circ}$, pretože číslo 19 je nepárne. Aspoň jeden z uhlov $ABC, BCD, CDA, DAB$ je teda väčší ako $90^{\circ}$, a preto je príslušný trojuholník tupouhlý.
}