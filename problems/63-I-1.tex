% Do not delete this line (pandoc magic!)

\problem{63-I-1}{
 Určte, akú najmenšiu hodnotu môže nadobúdať výraz $V = (a-b)^2 +(b-c)^2 +(c-a)^2$, ak reálne čísla $a, b, c$ spĺňajú dvojicu podmienok
\begin{align*}
a + 3b + c &= 6,\\
-a + b - c &= 2.
\end{align*}
}{
\rieh Sčítaním oboch rovníc z~podmienky zistíme, že $b = 2$. Dosadením za $b$ do niektorej z~nich vyjde $c = -a$. Platí teda $V = (a - 2)^2 + (2 + a)^2 + (-2a)^2$. Po umocnení a sčítaní zistíme, že $V = 6a^2 + 8 \geq 8$. Rovnosť nastane práve vtedy, keď $a = 0$, $b = 2$ a $c = 0$.

Hľadaná najmenšia hodnota výrazu $V$ je teda rovná 8.\\

\kom Riešenie úlohy vyžaduje prácu so sústavou dvoch rovníc, avšak manipulácia tejto sústavy nie je až taká zložitá, takže zaradenie úlohy bez toho, aby sme sa systematicky venovali riešeniu sústav rovníc, nepokladáme za problematické.
}