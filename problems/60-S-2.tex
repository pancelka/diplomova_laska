% Do not delete this line (pandoc magic!)

\problem{60-S-2}{seminar23,netrgeo}{Daný je štvorec so stranou dĺžky 6\,cm. Nájdite množinu stredov všetkých priečok štvorca, ktoré ho delia na dva štvoruholníky, z~ktorých jeden má obsah 12\,cm$^2$. (Priečka štvorca je úsečka, ktorej krajné body ležia na stranách štvorca.)
}{
\rieh Ak priečka delí štvorec na dva štvoruholníky, musia ich koncové body ležať na protiľahlých stranách štvorca. V~takom prípade sú oba štvoruholníky lichobežníkmi alebo pravouholníkmi (pre potreby tohto riešenia budeme pravouholník považovať za špeciálny lichobežník). Označme daný štvorec $ABCD$, koncové body priečky označme $K$ a $L$. Predpokladajme, že bod $K$ leží na strane $AD$, potom bod $L$ leží na strane $BC$. Jeden zo štvoruholníkov $KABL$ a $KDCL$ má podľa zadania obsah 12\,cm$^2$; nech je to napr. lichobežník $KABL$.

Obsah lichobežníka vypočítame ako súčin jeho výšky s~dĺžkou strednej priečky. Výška je v~našom prípade rovná dĺžke strany štvorca čiže 6\,cm. Jeho stredná priečka má teda dĺžku 2\,cm. Z~toho vyplýva, že stred úsečky $KL$ musí ležať na osi strany $AB$ vo
\begin{figure}[h]
    \centering
    \begin{minipage}{0.45\textwidth}
        \centering
        \includegraphics[width=0.9\textwidth]{images/60S21\imagesuffix} % first figure itself
        \caption{}
        \label{fig:60S2_1}
    \end{minipage}\hfill
    \begin{minipage}{0.45\textwidth}
        \centering
        \includegraphics[width=0.9\textwidth]{images/60S22\imagesuffix} % second figure itself
        \caption{}
        \label{fig:60S2_2}
    \end{minipage}
\end{figure}
vzdialenosti 2\,cm od stredu strany $AB$ (obr.~\ref{fig:60S2_1}). Platí to aj naopak: Ak stred úsečky $KL$ leží v~opísanej polohe, bude štvoruholník $KABL$ lichobežník s~obsahom 12\,cm$^2$.

Ak budeme namiesto lichobežníka $KABL$ uvažovať lichobežník $KDCL$, vyjde stred priečky $KL$ na osi úsečky $CD$ vo vzdialenosti 2\,cm od stredu strany $CD$.

Ak priečka $KL$ spája body na stranách $AB$ a $CD$, dostaneme ďalšie dva možné body ležiace na spojnici stredov úsečiek $AD$ a $BC$. Hľadanú množinu teda tvoria štyri body, ktoré ležia na priečkach spájajúcich stredy protiľahlých strán štvorca vo vzdialenosti 1\,cm od jeho stredu (obr.~\ref{fig:60S2_2}).
}