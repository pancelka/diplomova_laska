% Do not delete this line (pandoc magic!)

\problem{63-I-6}{}{
Šachového turnaja sa zúčastnilo 8 hráčov a každý s každým odohral jednu partiu. Za víťazstvo získal hráč 1 bod, za remízu pol bodu, za prehru žiadny bod. Na konci turnaja mali všetci účastníci rôzne počty bodov. Hráč, ktorý skončil na 2. mieste, získal rovnaký počet bodov ako poslední štyria dokopy. Určte výsledok partie medzi 4. a 6. hráčom v celkovom poradí.
}{
\rieh Poslední štyria hráči odohrali medzi sebou 6 partií, takže počet bodov, ktoré dokopy získali, je aspoň 6. Hráč, ktorý skončil na 2. mieste, teda získal aspoň 6 bodov. Keby získal viac ako 6, teda aspoň 6,5 bodov, musel by najlepší hráč (vďaka podmienke rôznych počtov) získať všetkých 7 možných bodov; porazil by tak i hráča na 2. mieste, ktorý by v dôsledku toho získal menej ako 6,5 bodov, a to je spor. Hráč v poradí druhý preto získal práve 6 bodov. Presne toľko ale získali dokopy i poslední štyria, a tak mohli tieto body získať len zo vzájomných partií, čo znamená, že prehrali všetky partie s hráčmi z prvej polovice výsledného poradia. Hráč, ktorý skončil na 6. mieste, preto prehral partiu s hráčom, ktorý skončil na 4. mieste.
}