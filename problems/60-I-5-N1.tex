% Do not delete this line (pandoc magic!)

\problem{60-I-5-N1}{seminar08,nsdnsn}{
Nech $d$ je najväčší spoločný deliteľ prirodzených čísel $a$ a $b$. Ukážte, že čísla $a/d$ a $b/d$ sú celé a nesúdeliteľné.
}{
\rie Ak je $d$ najväčším spoločným deliteľom čísel $a$ a $b$, potom existujú prirodzené čísla $u$ a $v$ také, že $a=ud$ a $b=vd$, čím sme dokázali prvú časť tvrdenia. Druhú dokážeme sporom. Predpokladajme, že $a/d$ a $b/d$ nie sú nesúdeliteľné. Potom existuje ich najväčší spoločný deliteľ $d_1$. Číslo $d_1$ však potom delí aj čísla $a$ a $b$, čo je spor s~predpokladom, že $d=(a,b)$.\\
\\
\kom Táto mini-úloha je prípravným krokom k~nasledujúcemu všeobecnejšiemu tvrdeniu a zároveň môže pripomenúť použitie dôkazu sporom.\\
\\
}
