\problem{66-I-4-N2}{
Určte všetky trojčleny $P (x) = ax^2+ bx + c$, pre ktoré platí $P (1) = 4$, $P (2) = 9$ a $P (3) = 18$.
}{
\rie Podobne, ako v~predchádzajúcej úlohe, zostavíme z~podmienok sústavu troch lineárnych rovníc s~tromi neznámymi $a$, $b$ a $c$:
\begin{align*}
P(1) &= a + b + c = 4, \\
P(2) &= 4a + 2b + c = 9, \\
P(3) &= 9a + 3b + c = 18.
\end{align*}
Sústava má opäť jediné riešenie $a = 2, b = -1, c = 3$, a preto existuje práve jeden trojčlen vyhovujúci zadaniu: $P(x)=2x^2-x+3$.\\
\\
\kom Predchádzajúce dve jednoduchšie úlohy majú prípravný charakter na nasledujúcu úlo\-hu a domácu prácu. Študenti si prostredníctvom nich zopakujú metódy riešenia sústav rovníc s~viacerými neznámymi. Tieto metódy by študentom mali byť známe zo ZŠ, ak však zistíme, že ich používanie nie je až také samozrejmé, je vhodné zaradiť niekoľko jednoduchších úloh, napr. z~\cite{kubat2000}.\\
}
