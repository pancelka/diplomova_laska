%

\problem{B-61-II-3}{
Pravouhlému trojuholníku $ABC$ je vpísaná kružnica, ktorá sa dotýka prepony $AB$ v bode $K$. Úsečku $AK$ otočíme o $90^\circ$ do polohy $AP$ a úsečku $BK$ otočíme o $90^\circ$ do polohy $BQ$ tak, aby body $P$, $Q$ ležali v polrovine opačnej k polrovine $ABC$.
\begin{enumerate}[a)]
    \item Dokážte, že obsahy trojuholníkov $ABC$ a $PQK$ sú rovnaké.
    \item Dokážte, že obvod trojuholníka $ABC$ nie je väčší ako obvod trojuholníka $PQK$.Kedy nastane rovnosť obvodov?
\end{enumerate}
}{
\rieh a) Označme $S$ stred a $r$ polomer kružnice vpísanej trojuholníku $ABC$ a $L$, $M$ body dotyku tejto kružnice postupne so stranami $BC$, $CA$ \todo{(obr. 1)}. Ak označíme $|AK| = x$, $|BK| = y$, tak $|AP| = |AM| = x$, $|KP| = x\sqrt{2}$, $|BQ| = |BL| = y$, $|KQ| = y\sqrt{2}$. Keďže oba uhly $AKP$, $BKQ$ majú veľkosť $45^\circ$, je trojuholník $PQK$ pravouhlý, takže jeho obsah je 
$$S_{PQK} =\frac{x\sqrt{2}y\sqrt{2}}{2}=xy.$$

\todo{DOPLNIŤ Obr. 1}\\
\\
Štvoruholník $SLCM$ je štvorec so stranou dĺžky $r$ a $|AM| = x$, $|BL| = y$. Obsah trojuholníka $ABC$ je rovný súčtu obsahov trojuholníkov $ABS$, $BCS$ a $CAS$, teda
$$S_{ABC}=\frac{(x + y)r + (y + r)r + (x + r)r}{2}= (x + y + r)r.$$
Obsah trojuholníka $ABC$ je zároveň rovný
$$S_{ABC}=\frac{|AC| \cdot |BC|}{2}=\frac{(x + r)(y + r)}{2}=\frac{xy}{2}+\frac{(x + y + r)r}{2}=\frac{xy}{2}+\frac{S_{ABC}}{2}.$$
Odtiaľ dostávame $S_{ABC} = xy$, čiže $S_{ABC} = S_{PQK}$, čo sme mali dokázať.


b) V trojuholníku $ABC$ sú dĺžky strán $a = y + r$, $b = x + r$, $c = x + y$. Obvod trojuholníka $ABC$ je $a + b + |AB|$, obvod trojuholníka $PQK$ je $x\sqrt{2} + y\sqrt{2} + |PQ|$.

Zrejme platí $|AB| \leq |PQ|$ ($|AB|$ je vzdialenosťou rovnobežiek $AP$, $BQ$, \todo{obr. 1}). Rovnosť nastane jedine v prípade $|AP| = |BQ|$, čiže $x = y$.
Ešte dokážeme, že $a + b \leq  x\sqrt{2} + y\sqrt{2}$, teda že $a + b \leq c\sqrt{2}$. Posledná nerovnosť je ekvivalentná s nerovnosťou, ktorú dostaneme jej umocnením na druhú, pretože obe jej strany sú kladné. Dostaneme tak $a^2 +b^2 +2ab \leq 2c^2$. Keďže v pravouhlom trojuholníku $ABC$ platí $a^2 + b^2 = c^2$, máme dokázať nerovnosť $2ab \leq a^2 + b^2$, ktorá je však ekvivalentná s nerovnosťou $0 \leq (a - b)^2$. Tá platí pre všetky reálne čísla $a$, $b$ a rovnosť v nej nastane jedine pre $a = b$, t. j. $x = y$.

Celkovo vidíme, že obvod trojuholníka $ABC$ je menší alebo rovný obsahu trojuholníka $PQK$ a rovnosť nastane práve vtedy, keď je pravouhlý trojuholník $ABC$ rovnoramenný.
}