% Do not delete this line (pandoc magic!)

\problem{66-I-5-N1}{geomlah,trojuholniky,podtroj,domacekolo}{
Zopakujte si, čo viete o podobnosti dvoch trojuholníkov z učiva základnej školy: Podobnosť $\triangle A_1B_1C_1 \sim  \triangle A_2B_2C_2$ s koeficientom $k$ znamená, že pre zvyčajne označené dĺžky strán a veľkosti vnútorných uhlov oboch trojuholníkov platia rovnosti $a_2 = ka_1$, $b_2 = kb_1$, $c_2 = kc_1$, $\alpha_2 = \alpha_1$, $\beta_2 = \beta_1$, $\gamma_2 = \gamma_1$. Stačí na to, aby platilo (i) $a_2 : b_2 : c_2 = a_1 : b_1 : c_1$ (veta $sss$) alebo (ii) $\alpha_2 = \alpha_1$ a $\beta_2 = \beta_1$ (veta $uu$) alebo (iii) $a_2 : a_1 = b_2 : b_1$ a $\gamma_2 = \gamma_1$ (veta $sus$).
}{
\rie
}
