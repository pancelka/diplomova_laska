% Do not delete this line (pandoc magic!)

\problem{57-I-2}{
Štvoruholníku $ABCD$ je vpísaná kružnica so stredom $S$. Určte rozdiel $|\ma ASD|- |\ma CSD|$, ak $|\ma ASB| - |\ma BSC| = 40^\circ$
}{
\rieh Päty kolmíc spustených zo stredu $S$ vpísanej kružnice na strany $AB$, $BC$, $CD$ a $DA$ označme postupne $K$, $L$, $M$ a $N$ (obr. 1). Pravouhlé trojuholníky $ASK$ a $ASN$ sú zhodné podľa vety $Ssu$. Majú totiž spoločnú preponu $AS$ a zhodné odvesny $SK$ a $SL$, ktorých dĺžka je rovná polomeru vpísanej kružnice. Zo zhodnosti týchto trojuholníkov vyplýva jednak známe tvrdenie o~dĺžkach dotyčníc $(|AK| = |AN|)$, jednak zhodnosť uhlov $ASK$ a $ASN$, ktorých spoločnú veľkosť označíme~$\alpha$:$$|\ma ASK| = |\ma ASN| = \alpha.$$
\begin{center}
\includegraphics{images/57D2\imagesuffix}\\

Obr. 1
\end{center}
Analogicky zistíme zhodnosť trojuholníkov $SBK$ a $SBL$, ďalej $SCL$ a $SCM$, a nakoniec $SDM$ a $SDN$. Na základe uvedených zhodností môžeme položiť
$$|\ma BSK| = |\ma BSL| = \beta, \ \ \ \  |\ma CSL| = |\ma CSM| = \gamma, \ \ \ \  |\ma DSM| = |\ma DSN| = \delta.$$
Odtiaľ a z~obr. 1 potom dostávame
\begin{align*}
|\ma ASD| - |\ma CSD| &= (\alpha + \delta)- (\gamma + \delta) = \alpha - \gamma =\\
&= (\alpha + \beta) - (\gamma + \beta) = |\ma ASB| - |\ma BSC| = 40^\circ.
\end{align*}
\textit{Záver.} $|\ma ASD|  -|\ma CSD| = 40^\circ$.\\
\\
\kom Úloha je relatívne nezložitým úvodom do seminára a nadväzuje na posledné geometrické stretnutie, ktoré sa zaoberalo opísanými a vpísanými kružnicami trojuholníku. Pre úplnosť len dodajme, že štvoruholník, ktorému je možné vpísať kružnicu, sa nazýva \textit{dotyčnicový}.\\
}
