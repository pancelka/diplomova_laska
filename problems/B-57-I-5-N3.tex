% Do not delete this line (pandoc magic!)

\problem{B-57-I-5-N3}{
Nájdite všetky dvojice $(a,b)$ reálnych čísel, pre ktoré má každá z rovníc $x^2+(a-2)x+b-3=0$, $x^2+(a+2)x+3b-5=0$ dvojnásobný koreň.
}{
\rie Kvadratická rovnica má dvojnásobný koreň práve vtedy, ak jej diskriminant je rovný nule. Z tejto podmienky pre rovnice zo zadania dostávame
\begin{equation}
    \begin{aligned}
        a^2-4a-4b+16 & = 0, \\
        a^2+4a-12b+24 & =0.
    \end{aligned}
 \label{eq:B57I5N3}
\end{equation}
Odčítaním druhej rovnice od prvej máme po úprave $a=b-1$. Dosadením tohto vzťahu do jednej z rovníc v \ref{eq:B57I5N3} potom určíme možné hodnoty $b$, ktoré sú 3 a 7. K nim odpovedajúce hodnoty $a$ sú tak 2 a 6 a teda hľadané dvojice reálnych čísel $(a,b)$ sú $(2, 3)$ a $(6, 7)$.\\
\\
\kom Jednoduchá úloha na úvod, v ktorej študenti aplikujú znalosti o závislosti medzi hodnotou diskriminantu a počtom riešení kvadratickej rovnice. Ten potom vedie na riešenie sústavy dvoch rovníc s dvomi neznámymi.\\
\\
}