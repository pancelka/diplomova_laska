% Do not delete this line (pandoc magic!)

\problem{63-II-1}{}{
Nájdite všetky trojice (nie nutne rôznych) cifier $a, b, c$ také, že päťciferné čísla $\overline{6abc3}$ a $\overline{3abc6}$ sú v~pomere 63 : 36.
}{
\rieh Zostavíme a vyriešime rovnicu pre neznáme cifry $a, b, c$, ktorú vďaka tvaru zadaných čísel môžeme zapísať rovno pre jedinú neznámu $x = 100a + 10b + c$:
\begin{align*}
\frac{60000+10x+3}{30000+10x+3} &=\frac{63}{36}=\frac{7}{4},\\
40x + 240 012 &= 70x + 210 042,\\
30x &= 29 970,\\
x &= 999.
\end{align*}

\textit{Záver.} Nájdenému $x$ zodpovedá trojica cifier $a = b = c = 9$. Úloha má jediné riešenie.\\
\\
\kom Úloha prináša zaujímavú myšlienku zjednodušenia zápisu, ktorý potom vedie k~riešeniu jednoduchej lineárnej rovnice. Aj napriek tomu, že riešenie nevyžaduje mnoho počítania, ukrýva úloha záludnosť v~podobe toho, že študenti môžu prísť k~správnemu riešeniu nesprávnymi úvahami. Viac o~tejto konkrétnej úlohe a jej úskaliach je možné nájsť v~článku \cite{hana}, ktorý považujeme za hodný preštudovania.\\
\\
}
