% Do not delete this line (pandoc magic!)

\problem{58-S-2}{seminar11,anglechas,skolskekolo}{
V~pravouhlom trojuholníku $ABC$ označíme $P$ pätu výšky z~vrcholu $C$ na preponu $AB$. Priesečník úsečky $AB$ s~priamkou, ktorá prechádza vrcholom $C$ a stredom kružnice vpísanej trojuholníku $PBC$, označíme $D$. Dokážte, že úsečky $AD$ a $AC$ sú zhodné.
}{
\rieh V~pravouhlom trojuholníku $ABC$ s~preponou $AB$ pre veľkosti $\alpha, \beta$ uhlov pri vrcholoch $A$, $B$ platí $\alpha+\beta= 90^\circ$, preto $|\ma ACP| = 90^\circ -\alpha = \beta$ a $|\ma BCD| = | \ma DCP|= \frac{1}{2}(90^\circ -\beta) = \frac{1}{2}\alpha$ lebo priamka $CD$ je osou uhla $BCP$ (obr.~\ref{fig:58S2}). Pre vonkajší uhol $ADC$ trojuholníka $BCD$ tak zrejme platí $|\ma ADC| = |\ma DBC| + |\ma BCD| = \beta  +\frac{1}{2}\alpha = |\ma DCA|.$

Zistili sme, že trojuholník $ADC$ má pri vrcholoch $C, D$ zhodné vnútorné uhly, je
teda rovnoramenný, a preto $|AD| = |AC|$.
\begin{figure}[h]
    \centering
    \includegraphics{images/58S21\imagesuffix}
    \caption{}
    \label{fig:58S2}
\end{figure}
\\
\kom Úloha je zameraná na nájdenie veľkosti vhodných uhlov\footnote{V anglickej literatúre sa tejto metóde -- počítaniu veľkostí všemožných uhlov -- hovorí \textit{angle-chasing}.} a využitie poznatku, že uhly pri základni rovnoramenného trojuholníka majú rovnakú veľkosť.
}
