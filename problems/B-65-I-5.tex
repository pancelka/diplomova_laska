% Do not delete this line (pandoc magic!)

\problem{B-65-I-5}{
Vrcholy konvexného šesťuholníka $ABCDEF$ ležia na kružnici, pričom $|AB| = |CD|$. Úsečky $AE$ a $CF$ sa pretínajú v bode $G$ a úsečky $BE$ a $DF$ sa pretínajú v bode $H$. Dokážte, že úsečky $GH$, $AD$ a $BC$ sú navzájom rovnobežné. 
}{
\rieh Najskôr ukážeme, že $AD \parallel BC$. Keďže $|AB| = |CD|$, sú obvodové uhly nad tetivami $AB$ a $CD$ kružnice opísanej šesťuholníku $ABCDEF$ zhodné (obr.~\ref{fig:B65I5}), teda $|\ma ADB| = |\ma DBC|$; to sú však striedavé uhly priečky $BD$ priamok $AD$ a $BC$, preto $AD \parallel BC$. Ostáva ukázať, že $GH \parallel AD$. Využitím zhodných obvodových uhlov nad tetivami 
\begin{figure}[h]
    \centering
    \includegraphics{images/B65I5\imagesuffix}
    \caption{}
    \label{fig:B65I5}
\end{figure}
$AB$ a $CD$ pri vrcholoch $E$ a $F$ dostávame
$$|\ma GEH| = |\ma AEB| = |\ma CFD| = |\ma GFH|,$$
čo znamená, že body $E$, $F$, $G$ a $H$ ležia na jednej kružnici, pretože vrcholy zhodných uhlov $GEH$ a $GFH$ ležia v rovnakej polrovine s hraničnou priamkou $GH$. Z toho vyplýva, že uhly $EFH$ a $EGH$ nad jej tetivou $EH$ sú zhodné. To spolu so zhodnosťou uhlov $EFD$ a $EAD$ nad tetivou $ED$ pôvodnej kružnice (obr.~\ref{fig:B65I5}) vedie na zhodnosť súhlasných uhlov $EGH$ a $EAD$ priečky $AE$ priamok $GH$ a $AD$, ktoré sú teda naozaj rovnobežné. Tým je tvrdenie úlohy dokázané.
}