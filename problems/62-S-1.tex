% Do not delete this line (pandoc magic!)

\problem{62-S-1}{seminar13,vpiskca,opiskca,pytveta}{
Danému rovnostrannému trojuholníku vpíšme a opíšme kružnicu. Označme $S$ obsah vzniknutého medzikružia a $T$ obsah kruhu, ktorého priemer je zhodný s~dĺžkou strany daného trojuholníka. Ktorý z~obsahov $S$, $T$ je väčší? Svoju odpoveď zdôvodnite.
}{
\rieh Ukážeme, že sa oba obsahy rovnajú. Označme $A$, $B$, $C$ vrcholy daného trojuholníka a $r$ a $R$ zodpovedajúce polomery jeho vpísanej a opísanej kružnice; dĺžku jeho strany označme $a$. Obe uvedené kružnice majú spoločný stred $S$. Označme ešte $P$ bod dotyku vpísanej kružnice so stranou $AB$. Keďže trojuholník $ABC$ je rovnostranný, je $P$ zároveň stredom strany $AB$. Použitím Pytagorovej vety v~pravouhlom trojuholníku $PSB$ dostávame
$$R^2 - r^2=  (\tfrac{1}{2}a)^2,$$
čo je ekvivalentné s~dokazovaným tvrdením $S = \pi (R^2 - r^2) = \pi \big( \frac{1}{2}a\big)^2= T$.\\
\\
\textit{Poznámka.} Rovnostranný trojuholník so stranou $a$ má výšku veľkosti $v = \frac{1}{2}a \sqrt{3}$, takže skúmané polomery sú $R =\frac{2}{3}v \big(=\frac{1}{3}a\sqrt{3}\big)$ a $r =\frac{1}{3}v \big(=\frac{1}{6}a\sqrt{3}\big)$, a preto
$$S = \pi ( R^2 - r^2) = \pi \big( \tfrac{4}{9} -\tfrac{1}{9})v^2= \pi \cdot \tfrac{1}{3}\cdot\tfrac{3}{4}a^2= \pi \big( \tfrac{1}{2}a\big)^2= T.$$
\\
\kom Úloha je relatívne jednoduchá, využíva znalosť o~bode dotyku vpísanej kružnice a taktiež pripravuje študentov na nasledujúcu zložitejšiu analýzu. \\
\\
}
