% Do not delete this line (pandoc magic!)

\problem{60-I-3-N1}{obsahy,geomlah,domacekolo}{
Daný je lichobežník $ABCD$ s dlhšou základňou $AB$ a priesečníkom uhlopriečok $P$.
Vieme, že obsah trojuholníka $ABP$ je 16 a obsah trojuholníka $BCP$ je 10.
\begin{enumerate}[a)]
    \item Vypočítajte obsah trojuholníka $ADP$.
    \item Vypočítajte obsah lichobežníka $ABCD$.
\end{enumerate}
}{
\rieh Trojuholníky $ABC$ a $ABD$ majú spoločnú stranu $AB$ a rovnaké výšky na túto stranu, teda majú rovnaký obsah. Preto majú rovnaký obsah trojuholníky $ADP$ a $BCP$. Obsah trojuholníka $CDP$ vyrátame napríklad z jeho podobnosti s trojuholníkom $ABP$, pomer podobnosti je $| AP | / | CP | = S_{ABP}/S_{CBP}$. Dostaneme $S_{ABCD} = 169/4$.
}
