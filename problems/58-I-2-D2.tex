% Do not delete this line (pandoc magic!)

\problem{58-I-2-D2}{seminar10,geomlah,domacekolo}{
Vyjadrite obsah rovnoramenného lichobežníka $ABCD$ so základňami $AB$ a $CD$ pomocou dĺžok $a$, $c$ jeho základní a dĺžky $b$ jeho ramien.
}{
\rie Bez ujmy na všeobecnosti môžeme predpokladať, že $a>b$. Najprv vyjadríme výšku $v$ pomocou dĺžok základní a odvesien. Nech je $P$ päta výšky z~bodu $D$ na stranu $AB$. Potom $|AP|=(a-c)/2$. Použitím Pytagorovej vety v~pravouhlom trojuholníku $APD$ máme
$$\bigg(\frac{a-c}{2}\bigg)^2+v^2=b^2,$$
odkiaľ $v=\sqrt{b^2-(\frac{a-c}{2})^2}=\frac{1}{2}\sqrt{4b^2-(a-c)^2}$ a preto pre obsah lichobežníka dostávame $$S_{ABCD}=\frac{1}{2}(a+c)\cdot v=\frac{1}{4}(a+c)\sqrt{4b^2-(a-c)^2}.$$
}
