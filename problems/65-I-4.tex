% Do not delete this line (pandoc magic!)

\problem{65-I-4}{}{
Vnútri strán $AB$, $AC$ daného trojuholníka $ABC$ sú zvolené postupne body $E$, $F$, pričom $EF \parallel BC$. Úsečka $EF$ je potom rozdelená bodom $D$ tak, že platí $$p = |ED| : |DF | = |BE| : |EA|.$$
\begin{enumerate}[a)]
    \item Ukážte, že pomer obsahov trojuholníkov $ABC$ a $ABD$ je pre $p = 2 : 3$ rovnaký ako pre $p = 3 : 2$.
    \item Zdôvodnite, prečo pomer obsahov trojuholníkov $ABC$ a $ABD$ má hodnotu aspoň 4.
\end{enumerate}
}{
\rieh Pre spoločnú hodnotu $p$ oboch pomerov zo zadania platí
\begin{equation} \label{eq:65I4_1}
    |ED| = p|DF| \ \ \ \ \text{a zároveň} \ \ \ \ |BE| = p|EA|.
\end{equation}
Pred vlastným riešením oboch úloh a) a b) vyjadríme pomocou daného čísla $p$ skúmaný pomer obsahov trojuholníkov $ABC$ a $ABD$. Ten je rovný -- keďže trojuholníky majú spoločnú stranu AB -- pomeru dĺžok ich výšok $CC_0$ a $DD_0$ (obr. \ref{fig:65I4}), ktorý je rovnaký ako
\begin{figure}[h]
    \centering
    \includegraphics{images/65D4\imagesuffix}
    \caption{}
    \label{fig:65I4}
\end{figure}
pomer dĺžok úsečiek $BC$ a $ED$, a to na základe podobnosti pravouhlých trojuholníkov $BCC_0$ a $EDD_0$ podľa vety $uu$ (uplatnenej vďaka $BC \parallel ED$).\footnote{V prípade pravých uhlov $ABC$ a $AED$ to platí triviálne, lebo vtedy $B = C_0$ a $E = D_0$.} Platí teda rovnosť
\begin{equation} \label{eq:65I4_2}
    \frac{S_{ABC}}{S_{ABD}} =\frac{|BC|}{|ED|}.
\end{equation}
Vráťme sa teraz k~rovnostiam \ref{eq:65I4_1}, podľa ktorých
$$|EF| = (1 + p)|DF| \ \ \ \ \text{a} \ \ \ \ |AB| = (1 + p)|EA|,$$
a všimnime si, že trojuholníky $ABC$ a $AEF$ majú spoločný uhol pri vrchole $A$ a zhodné uhly pri vrcholoch $C$ a $F$ (pretože $BC \parallel EF$), takže sú podľa vety $uu$ podobné. Preto
pre dĺžky ich strán platí
$$\frac{|AB|}{|AE|}=\frac{|BC|}{|EF|},\ \ \ \ \text{čiže} \ \ \ \  1 + p =\frac{|BC|}{(1 + p)|DF|}, \ \ \ \ \text{odkiaľ} \ \ \ \ |BC| = (1 + p)^2 |DF|.$$
Keď vydelíme posledný vzťah hodnotou $|ED|$, ktorá je rovná $p|DF|$ podľa~\ref{eq:65I4_1}, získame podiel z~pravej strany~\ref{eq:65I4_2} a tým aj hľadané vyjadrenie
\begin{equation} \label{eq:65I4_3}
    \frac{S_{ABC}}{S_{ABD}}=\frac{(1 + p)^2}{p}.
\end{equation}

a) Algebraickou úpravou zlomku zo vzťahu~\ref{eq:65I4_3}
$$ \frac{(1 + p)^2}{p}=\frac{1 + 2p + p^2}{p}= 2 + p + \frac{1}{p}$$
zisťujeme, že hodnota pomeru $S_{ABC} : S_{ABD}$ je pre akékoľvek dve navzájom prevrátené hodnoty $p$ a $1/p$ rovnaká, teda nielen pre hodnoty $2/3$ a $3/2$, ako sme mali ukázať.

b) Podľa vzťahu~\ref{eq:65I4_3} je našou úlohou overiť pre každé $p > 0$ nerovnosť
$$\frac{(1 + p)^2}{p}\geq4,\ \ \ \ \text{čiže} \ \ \ \  (1 + p)^2\geq 4p.$$
To je však zrejme ekvivalentné s~nerovnosťou $(1 -p)^2\geq 0$, ktorá skutočne platí, nech je základ druhej mocniny akýkoľvek (rovnosť nastane jedine pre $p = 1$).

%{Dodajme, že pre iný dôkaz bolo možné využiť aj vyššie uvedené \uv{symetrické} vyjadrenie
%$$\frac{(1 + p)^2}{p}= 2 + p +\frac{1}{p}$$
%a uplatniť naň dobre známu nerovnosť $p + 1/p \geq 2$, ktorej platnosť pre každé $p > 0$ vyplýva napr. z~porovnania aritmetického a geometrického priemeru dvojice čísel $p$ a $1/p$, nazývaného AG-nerovnosť:
%$$\frac{1}{2}\bigg(p +\frac{1}{p}\bigg)\geq \sqrt{p\cdot \frac{1}{p}}= 1, \ \ \ \ \text{pretože všeobecne} \ \ \ \ \frac{a+b}{2} \geq \sqrt{a\cdot b} \ \ \ \ (\forall a, b \geq 0).$$M
}
