\problem{64-II-4}{
Na tabuli je zoznam čísel $1, 2, 3, 4, 5, 6$ a \uv{rovnica}
$$\frac{\fbox{$\phantom{7}$}}{\fbox{$\phantom{7}$}}x^2+\frac{\fbox{$\phantom{7}$}}{\fbox{$\phantom{7}$}}x + \frac{\fbox{$\phantom{7}$}}{\fbox{$\phantom{7}$}}= 0.$$
Marek s~Tomášom hrajú nasledujúcu hru. Najskôr Marek vyberie ľubovoľné číslo zo zoznamu, napíše ho do jedného z~prázdnych políčok v~\uv{rovnici} a číslo zo zoznamu zotrie. Potom Tomáš vyberie niektoré zo zvyšných čísel, napíše ho do iného prázdneho políčka a v~zozname ho zotrie. Nato Marek urobí to isté a nakoniec Tomáš doplní tri zvyšné čísla na tri zvyšné voľné políčka v~\uv{rovnici}. Marek vyhrá, ak vzniknutá kvadratická rovnica s~racionálnymi koeficientmi bude mať dva rôzne reálne korene, inak vyhrá Tomáš. Rozhodnite, ktorý z~hráčov môže vyhrať nezávisle na postupe druhého
hráča.
}{
\rieh Označme $a$, $b$, $c$ koeficienty výslednej rovnice $ax^2 + bx + c = 0$. Tá má dva rôzne reálne korene práve vtedy, keď je jej diskriminant (v~symbolickej podobe)
$$b^2 - 4ac =\bigg( \frac{{\fbox{$\phantom{7}$}}}{{\fbox{$\phantom{7}$}}} \bigg)^2-4\bigg( \frac{{\fbox{$\phantom{7}$}}}{{\fbox{$\phantom{7}$}}}\bigg) \bigg(\frac{{\fbox{$\phantom{7}$}}}{{\fbox{$\phantom{7}$}}}\bigg)$$
kladný.

Ukážeme, že vyhrávajúcu stratégiu má Marek. Najskôr do menovateľa zlomku pre koeficient $b$ napíše $1$.
\begin{enumerate}[a)]
\item Ak Tomáš obsadí vo svojom prvom ťahu iné miesto ako v~čitateli $b$, napíše do neho Marek v~nasledujúcom ťahu najväčšie zostávajúce číslo zo zoznamu (teda 5 alebo 6). Hodnota $b^2$ potom bude aspoň 25 a zo zvyšných čísel možno zostaviť výraz $4ac$ s~hodnotou nanajvýš $4\cdot  \frac{6\cdot4}{3\cdot2}= 16$. Diskriminant vzniknutej kvadratickej rovnice tak bude určite kladný.
\item Predpokladajme, že Tomáš vo svojom ťahu doplní čitateľa $b$. Marek potom v~druhom ťahu napíše najmenšie zostávajúce číslo zo zoznamu (2 alebo 3) do čitateľa $a$ (alebo $c$).
\begin{enumerate}[(i)]
\item V~prípade, že Tomáš v~prvom ťahu napísal do čitateľa $b$ číslo 2, je hodnota $b^2$ rovná 4 a najväčšia možná hodnota $4ac$ (s~prihliadnutím na druhý Marekov ťah) je $4 \cdot \frac{3\cdot 6}{4\cdot 5}=\frac{18}{5}\leq  4$, teda diskriminant vzniknutej kvadratickej rovnice bude opäť kladný.
\item  V~prípade, že Tomáš v~prvom ťahu napísal do čitateľa $b$ iné číslo ako 2, je hodnota $b^2$ aspoň 9 a hodnota $4ac$ je nanajvýš $4 \cdot \frac{2\cdot 6}{3\cdot4} = 4$, takže diskriminant
vzniknutej kvadratickej rovnice bude aj v~tomto prípade kladný.

\end{enumerate}
\end{enumerate}
\textit{Záver.} V~danej hre môže vyhrať Marek nezávisle na ťahoch Tomáša. Jeho víťazná stratégia je opísaná vyššie.
}
