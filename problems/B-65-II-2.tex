% Do not delete this line (pandoc magic!)

\problem{B-65-II-2}{
Daná je úsečka $AB$, jej stred $C$ a vnútri úsečky $AB$ bod $D$. Kružnice $k(C, |BC|)$ a $m(B, |BD|)$ sa pretínajú v bodoch $E$ a $F$. Zdôvodnite, prečo je polpriamka $FD$ osou uhla $AFE$.
}{
\rieh Kružnica $k$ je Tálesovou kružnicou nad priemerom $AB$, takže trojuholník $ABF$ je pravouhlý s pravým uhlom pri vrchole $F$. Inými slovami, priamka $AF$ je kolmá 
\begin{figure}[h]
    \centering
    \includegraphics{images/B65II2_1\imagesuffix}
    \caption{}
    \label{fig:B65II2_1}
\end{figure}
na polomer $BF$ kružnice $m$, a preto sa priamka $AF$ dotýka kružnice $m$ v bode $F$ (obr.~\ref{fig:B65II2_1}). Z rovnosti úsekového uhla zovretého tetivou $DF$ s dotyčnicou $AF$ a obvodového uhla
nad tou istou tetivou máme (ako už je vyznačené na obrázku) 
$$|\ma AFD| = |\ma DEF|.$$
Zo súmernosti úsečky $EF$ podľa osi $AB$ tak vyplýva
$$|\ma AFD| = |\ma DEF| = |\ma DFE|,$$
čo znamená, že $FD$ je osou uhla $AFE$.

\textbf{Iné riešenie*.} Označme $\beta$ veľkosť uhla $ABF$ a dopočítajme veľkosti uhlov $DFE$ a $AFE$. Trojuholník $DBF$ je rovnoramenný, lebo jeho ramená $BD$ a $BF$ sú polomery kružnice $m$, preto
$$|\ma DFB| = \frac{1}{2}(180^\circ-\beta) = 90^\circ-\frac{\beta}{2}.$$
Keďže podobne aj trojuholník $EBF$ je rovnoramenný s osou $BD$, platí
$$|\ma EFB| = 90^\circ-\beta.$$
Spojením oboch predchádzajúcich rovností tak dostávame
$$|\ma DFE| = |\ma DFB| - |\ma EFB| =\frac{\beta}{2}.$$
Z vlastností Tálesovej kružnice $k$ nad priemerom $AB$ vieme, že uhol $AFB$ je pravý. Pritom jeho časť uhol $EFB$ má, ako sme už zistili, veľkosť $90^\circ-\beta$, takže jeho druhá časť, uhol $AFE$, má veľkosť $\beta$, čo je presne dvojnásobok veľkosti uhla $DFE$. Tým sme dokázali, že priamka $FD$ je osou uhla $AFE$.

\textbf{Iné riešenie*.} Nad oblúkom $AE$ kružnice $k$ sa zhodujú uhly $ABE$ a $AFE$ (obr.~\ref{fig:B65II2_2}). Oblúku $DE$ kružnice $m$ prislúcha obvodový uhol $DFE$ a stredový uhol $DBE$. Spolu
tak dostávame
$$|\ma DFE| = \frac{1}{2} |\ma DBE| = \frac{1}{2}
|\ma ABE| = \frac{1}{2} |\ma AFE|,$$
čo dokazuje, že $FD$ je osou uhla $AFE$.
\begin{figure}[h]
    \centering
    \includegraphics{images/B65II2_2\imagesuffix}
    \caption{}
    \label{fig:B65II2_2}
\end{figure}
\\
\kom Úlohu je možné riešiť viacerými rôznymi spôsobmi, preto je to opäť vhodný priestor na to, aby ši študenti svoje riešenia porovnali a skúsili obhájiť pred spolužiakmi. V úlohe sa znova vyskytla spoločná tetiva dvoch kružníc, pekne tak nadväzuje na úlohu predchádzajúcu.\\
\\ 
}
