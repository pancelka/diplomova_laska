% Do not delete this line (pandoc magic!)

\problem{59-S-2}{}{
Kružnice $k(S; 6\,\text{cm})$ a $l(O; 4\,\text{cm})$ majú vnútorný dotyk v~bode $B$. Určte dĺžky strán trojuholníka $ABC$, pričom bod $A$ je priesečník priamky $OB$ s~kružnicou $k$ a bod $C$ je priesečník kružnice $k$ s~dotyčnicou z~bodu $A$ ku kružnici $l$.
}{
\rieh Bod dotyku kružnice $l$ s~dotyčnicou z~bodu $A$ označme $D$ (obr.~\ref{fig:59S2}). Z~vlastností dotyčnice ku kružnici vyplýva, že uhol $ADO$ je pravý. Zároveň je pravý aj uhol
\begin{figure}[h]
    \centering
    \includegraphics{images/59S2\imagesuffix}
    \caption{}
    \label{fig:59S2}
\end{figure}
$ACB$ (Tálesova veta). Trojuholníky $ABC$ a $AOD$ sú tak podobné podľa vety $uu$, lebo sa zhodujú v~uhloch $ACB$, $ADO$ a v~spoločnom uhle pri vrchole $A$. Z~uvedenej podobnosti vyplýva
\begin{equation} \label{eq:59S2}
    \frac{|BC|}{|OD|}=\frac{|AB|}{|AO|}.
\end{equation}
Zo zadaných číselných hodnôt vychádza $|OD| = |OB| = 4$\,cm, $|OS| = |SB| - |OB| = 2$\,cm, $|OA| = |OS| + |SA| = 8$\,cm a $|AB| = 12$\,cm. Podľa \ref{eq:59S2} je teda $|BC| : 4\,\text{cm} = 12 : 8$ a odtiaľ $|BC| = 6$\,cm. Z~Pytagorovej vety pre trojuholník $ABC$ nakoniec zistíme, že $|AC| = \sqrt{12^2 - 6^2}\,\text{cm}= 6$\,cm.\\
}