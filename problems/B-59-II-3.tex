% Do not delete this line (pandoc magic!)

\problem{B-59-II-3}{
V rovine je daný rovnobežník $ABCD$. Označme postupne $K$, $L$, $M$ stredy strán $AB$, $CD$, $AD$. Predpokladajme, že body $A$, $B$, $L$, $D$ ležia na jednej kružnici a súčasne aj body $K$, $L$, $D$, $M$ ležia na jednej kružnici. Dokážte, že $|AC| = 2 \cdot |AD|$.
}{
\rieh Lichobežníky $ABLD$ a $KLDM$ sú rovnoramenné, pretože sú tetivové. Odtiaľ vyplýva zhodnosť ramien $|AD| = |BL|$ a zhodnosť uhlopriečok $|KD| = |LM|$ \todo{fixni ma (obr. 2)}. Úsečky $KB$ a $DL$ sú rovnobežné a zhodné, preto je $KBLD$ rovnobežník a platí $|KD|= |BL|$. Úsečka $ML$ je strednou priečkou trojuholníka $ACD$, preto $|AC| = 2 \cdot |ML|$.
Spojením uvedených rovností máme $|AC| = 2 \cdot |ML| = 2 \cdot |KD| = 2 \cdot |BL| = 2 \cdot |AD|$.

\todo{DOPLNIŤ Obr. 2}

\textbf{Iné riešenie.} Budeme postupovať rovnako ako v druhom riešení tretej úlohy domáceho kola (je možné odvolať sa na domáce kolo bez dôkazu): Keďže $ABLD$ je tetivový (a teda rovnoramenný) lichobežník, je $|KD| = |BL| = |AD|$. Podobne je aj lichobežník $KLDM$ rovnoramenný, takže $|MK| = |DL|$ a $|DB| = 2|MK| = 2|DL| = |DC| = |AB|$. Z podobnosti rovnoramenných trojuholníkov $AKD$ a $DAB$ (zhodujú sa v uhle pri vrchole $A$ svojich základní) potom vyplýva, že $|AK|/|AD| = |DA|/|AB|$, odkiaľ po dosadení $|AK| = \frac{1}{2}|AB|$  vychádza $|DB| = |AB| =\sqrt{2}\cdot|AD|$. Ďalej využijeme
známu rovnobežníkovú rovnosť $|AC|^2 + |BD|^2 = 2 \cdot |AB|^2 + 2 \cdot |AD|^2$. Dosadením dostaneme $|AC|^2 + 2 \cdot |AD|^2 = 4 \cdot |AD|^2 + 2 \cdot |AD|^2$ a odtiaľ $|AC|^2 = 4 \cdot |AD|^2$ čiže $|AC| = 2 \cdot |AD|$.
}
