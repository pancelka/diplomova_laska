% Do not delete this line (pandoc magic!)

\problem{66-II-1}{seminar21,mismas,mnohocleny,sustavy,krajskekolo}{
Nájdite všetky mnohočleny $P(x) = ax^2 +bx+c$ s~celočíselnými koeficientami spĺňajúce
$$1 < P(1) < P(2) < P(3) \ \ \ \text{a súčasne} \ \  \
\frac{P(1) \cdot P(2) \cdot P(3)}{4}= 17^2.$$
}{
\rieh Rovnosť zo zadania je ekvivalentná rovnosti $P(1)\cdot P(2)\cdot P(3) = 4\cdot17^2$, takže čísla $P(1)$, $P(2)$, $P(3)$ môžu byť iba z~množiny deliteľov čísla $4 \cdot 17^2$ väčších ako 1:
$$2 < 4 < 17 < 2 \cdot 17 < 4 \cdot 17 < 17^2< 2 \cdot 17^2< 4 \cdot 17^2.$$

Ak by platilo $P(1) = 4$, bol by súčin $P(1)\cdot P(2)\cdot P(3)$ aspoň $4 \cdot 17 \cdot (2 \cdot 17) = 8 \cdot 17^2$, čo nevyhovuje zadaniu. Preto $P(1) = 2$ a tak je nutne $P(2) = 17$, pretože keby bolo $P(2) = 4$, musel by byť daný súčin $4 \cdot 17^2$ deliteľný číslom $P(1)\cdot P(2) = 8$, čo neplatí, a pre $P(2) = 2 \cdot 17$ by bol súčin $P(1)\cdot P(2)\cdot P(3)$ opäť príliš veľký. Pre tretiu neznámu
hodnotu $P(3)$ potom vychádza $P(3) = 4 \cdot 17^2 /(2 \cdot 17) = 2 \cdot 17$.

Hľadané koeficienty $a$, $b$, $c$ tak sú práve také celé čísla, ktoré vyhovujú sústave
\begin{align*}
P(1) &= a + b + c = 2,\\
P(2) &= 4a + 2b + c = 17,\\
P(3) &= 9a + 3b + c = 34.
\end{align*}
Jej vyriešením dostaneme $a = 1$, $b = 12$, $c = -11$.

\textit{Záver}. Úlohe vyhovuje jediný mnohočlen $P(x) = x^2 + 12x - 11$.\\
\\
\kom Úloha spája poznatky o~deliteľnosti, mnohočlenoch a na jej úspešné doriešenie je nutná aj schopnosť popasovať sa so sústavou troch rovníc s~tromi neznámymi. Zaujímavé bude tiež pozorovať, koľko študentov si spomenie, že podobnou úlohou sa už zaoberali v~seminári 6.\\
\\
}
