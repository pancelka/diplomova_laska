% Do not delete this line (pandoc magic!)

\problem{B-66-II-3}{
V rovine sú dané kružnice $k$ a $l$, ktoré sa pretínajú v bodoch $E$ a $F$. Dotyčnica ku kružnici $l$ zostrojená v bode $E$ pretína kružnicu $k$ v bode $H$ ($H \neq E$). Na oblúku $EH$ kružnice $k$, ktorý neobsahuje bod $F$, zvoľme bod $C$ ($E \neq C \neq H$) a priesečník priamky $CE$ s kružnicou $l$ označme $D$ ($D \neq E$). Dokážte, že trojuholníky $DEF$ a $CHF$
sú podobné.
}{
\rieh Z rovnosti obvodových uhlov nad tetivou $HF$ kružnice k vyplýva $|\ma HCF|= |\ma HEF|$. Uhol $HEF$ je zároveň úsekovým uhlom prislúchajúcim tetive $EF$ kružnice $l$, ktorý je však zhodný s obvodovým uhlom $EDF$ \todo{doplniť (obr. 1)}. Celkovo tak platí
$$|\ma HCF| = |\ma HEF| = |\ma EDF|. \ \ \ \ (1)$$
\todo{DOPLNIŤ Obr. 1}
Vzhľadom na to, že $CEFH$ je tetivový štvoruholník, je jeho vnútorný uhol pri vrchole $H$ zhodný s vonkajším uhlom pri jeho protiľahlom vrchole $E$. Platí teda
$$|\ma CHF| = |\ma DEF|. \ \ \ \  (2)$$
Z rovností \todo{fixni (1) a (2)} vyplýva na základe vety $uu$ podobnosť trojuholníkov $DEF$ a $CHF$. Tým je dôkaz hotový.
}
