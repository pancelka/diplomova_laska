% Do not delete this line (pandoc magic!)

\problem{60-I-4-N2}{
Každý zo šiestich žiakov istej triedy má medzi ostatnými piatimi aspoň troch kamarátov. Kamarátstvo je vzájomné. Ukážte, že vieme týchto žiakov rozdeliť do dvoch (neprázdnych) skupín tak, že každý žiak má vo svojej skupine aspoň jedného kamaráta. Vedeli by sme to spraviť aj vtedy, keby každý žiak mal presne dvoch kamarátov?
}{
\rieh Ak rozdelíme žiakov hocijakým spôsobom na dvojicu a štvoricu, tak každý žiak zo štvorice má v nej aspoň jedného kamaráta, lebo z jeho aspoň troch kamarátov sú nanajvýš dvaja v druhej skupine. Čiže stačí zobrať dvojicu kamarátov a ostatných dať do druhej skupiny. Ak má každý presne dvoch kamarátov, tiež vieme žiakov rozdeliť: vezmeme
žiaka $A$ a jeho dvoch kamarátov $B$ a $C$ a všetkých ich dáme do prvej skupiny. Zvyšní traja žiaci $D, E, F$ budú tvoriť druhú skupinu. Ak by niektorý žiak z druhej skupiny, povedzme $D$, mal za kamarátov $B$ aj $C$, tak žiaci $E$ a $F$ budú mať nanajvýš po jednom kamarátovi. Preto $D$ má za kamaráta nanajvýš jedného z $B$ a $C$, nemôže sa kamarátiť s $A$, čiže musí mať za kamaráta aspoň jedného zo žiakov $E$ a $F$. Podobne to funguje pre žiakov $E$ a $F$. O situácii so šiestimi žiakmi, kde každý má presne dvoch kamarátov, vieme povedať dokonca viac. Ak si zakreslíme žiakov ako body a kamarátsky vzťah reprezentujeme spojením bodov zodpovedajúcich dvom kamarátom, môžeme dostať len dva rôzne obrázky: dva trojuholníky, alebo šesťuholník (pri vhodnom rozmiestnení
bodov v rovine).
}