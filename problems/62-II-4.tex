% Do not delete this line (pandoc magic!)

\problem{62-II-4}{}{
Určte najmenšie celé kladné číslo $v$, pre ktoré platí: Medzi ľubovoľnými $v$ vrcholmi pravidelného dvadsaťuholníka možno nájsť tri, ktoré sú vrcholmi pravouhlého rovnoramenného trojuholníka.
}{
\rieh Nech $A_1 A_2\,\ldots A_{20}$ je pravidelný dvadsaťuholník. Podľa Tálesovej vety jedine niektorý z desiatich priemerov $A_1 A_{11}, A_2 A_{12},\,\ldots, A{10}A_{20}$ opísanej kružnice môže byť
preponou hľadaného pravouhlého trojuholníka, takže skúmané tvrdenie neplatí pre $v = 10$ (ani pre žiadne $v < 10$): stačí vybrať po jednom z vrcholov na rôznych priemeroch a nebude existovať žiadny pravouhlý trojuholník s takto vybranými vrcholmi.

V druhej časti riešenia ukážeme, že vyhovuje $v = 11$. Všetkých 20 vrcholov dvadsaťuholníka rozdelíme na päť štvoríc vrcholov štvorcov $A_1 A_6 A_{11} A_{16}, A_2 A_7 A_{12} A_{17}, A_3 A_8 A_{13} A_{18}, A_4 A_9 A_{14} A_{19}$ a $A_5 A_{10} A_{15} A_{20}$. Ak teraz vyberieme ľubovoľne 11 vrcholov, budú vďaka nerovnosti $11 > 5\cdot 2$ medzi vybranými aspoň tri vrcholy niektorého z piatich uvedených štvorcov (Dirichletov princíp). Ostáva dodať, že akékoľvek tri vrcholy štvorca zrejme tvoria pravouhlý rovnoramenný trojuholník.\\
\textit{Odpoveď.} Hľadané najmenšie číslo v je rovné číslu 11.
}