% Do not delete this line (pandoc magic!)

\problem{61-S-2}{seminar13,vpiskca,pytveta}{
Označme $S$ stred základne $AB$ daného rovnoramenného trojuholníka $ABC$. Predpokladajme, že kružnice vpísané trojuholníkom $ACS$, $BCS$ sa dotýkajú priamky $AB$ v~bodoch, ktoré delia základňu $AB$ na tri zhodné diely. Vypočítajte pomer $|AB| : |CS|$.
}{
\rieh Vďaka súmernosti podľa priamky $CS$ sa obe vpísané kružnice dotýkajú výšky $CS$ v~rovnakom bode, ktorý označíme $D$. Body dotyku týchto kružníc s~úsečkami $AS$, $BS$, $AC$, $BC$ označíme postupne $E$, $F$, $G$, $H$ (obr.~\ref{fig:61S2}). Pre vyjadrenie všetkých potrebných dĺžok ešte zavedieme označenie $x = |SD|$ a $y = |CD|$.
\begin{figure}[h]
    \centering
    \includegraphics{images/61S2\imagesuffix}
    \caption{}
    \label{fig:61S2}
\end{figure}
Vzhľadom na symetriu dotyčníc z~daného bodu k~danej kružnici platia rovnosti
$$|SD| = |SE| = |SF| = x \ \ \ \ \text{a} \ \ \ \ |CD| = |CG| = |CH| = y.$$
Úsečka $EF$ má preto dĺžku $2x$, ktorá je podľa zadania zároveň dĺžkou úsečiek $AE$ a $BF$, a teda aj dĺžkou úsečiek $AG$ a $BH$ (opäť vďaka symetrii dotyčníc). Odtiaľ už bezprostredne vyplývajú rovnosti
$$|AB| = 6x, \ \ \ \ |AC| = |BC| = 2x + y \ \ \ \ \text{a} \ \ \ \  |CS| = x + y.$$

Závislosť medzi dĺžkami $x$ a $y$ zistíme použitím Pytagorovej vety pre pravouhlý trojuholník $ACS$ (s~odvesnou $A$ dĺžky $3x$):
$$(2x + y)^2= (3x)^2+ (x + y)^2.$$
Roznásobením a ďalšími úpravami odtiaľ dostaneme ($x$ a $y$ sú kladné hodnoty)
\begin{align*}
4x^2+ 4xy + y^2 &= 9x^2+ x^2+ 2xy + y^2,\\
2xy & = 6x^2,\\
y &= 3x.
\end{align*}
Hľadaný pomer tak má hodnotu
$$|AB| : |CS| = 6x : (x + y) = 6x : 4x = 3 : 2.$$
Poznamenajme, že prakticky rovnaký postup celého riešenia možno zapísať aj pri štandardnom označení $c = |AB|$ a $v = |CS|$. Keďže podľa zadania platí $|AE| =\frac{1}{3}c$, a teda $|SE| =\frac{1}{6}c$, z~rovnosti $|SD| = |SE|$ vyplýva $|CD| = |CS|-|SD| = v-\frac{1}{6}c$, odkiaľ
$$|AC| = |AG| + |CG| = |AE| + |CD| =\tfrac{1}{3}c + (v-\tfrac{1}{6}c) = v~+\tfrac{1}{6}c,$$
takže z~Pytagorovej vety pre trojuholník $ACS$,
$$(v +\tfrac{1}{6}c)^2= (\tfrac{1}{2}c)^2+ v^2,$$
vychádza $3v = 2c$, čiže $c : v~= 3 : 2$.\\
\\
\kom Úloha vychádza z~poznatku, ktorý si študenti osvojili v~úlohe predchádzajúcej a pridáva k~nemu ešte prácu s~Pytagorovou vetou a manipuláciu s~algebraickými výrazmi, takže tvorí prirodzené pokračovanie úlohy predchádzajúcej.\\
\\
}
