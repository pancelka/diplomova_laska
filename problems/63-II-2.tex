% Do not delete this line (pandoc magic!)

\problem{63-II-2}{
Šachového turnaja sa zúčastnilo 5 hráčov a každý s každým odohral jednu partiu. Za prvenstvo získal hráč 1 bod, za remízu pol bodu, za prehru žiadny bod. Poradie hráčov na turnaji sa určuje podľa počtu získaných bodov. Jediným ďalším kritériom rozhodujúcim o konečnom umiestnení hráčov v prípade rovnosti bodov je počet výhier (kto má viac výhier, je na tom v umiestnení lepšie). Na turnaji získali všetci hráči rovnaký počet bodov. Vojto porazil Petra a o prvé miesto sa delil s Tomášom. Ako dopadla partia medzi Petrom a Martinom?
}{
\rieh Každý hráč odohral po jednej partii so zvyšnými štyrmi. Bolo teda odo hraných celkom $\frac{1}{2}\cdot5 \cdot 4 = 10$ partií, takže každý hráč získal práve 2 body. Sú len tri možnosti, ako získať odohraním štyroch partií 2 body, a podľa toho obsahovala celková tabuľka nanajvýš tri rovnocenné skupiny hráčov. Tieto skupiny, $A, B$ a $C$, uvádzame v poradí, v ktorom by sa v konečnej tabuľke umiestnili:

Skupina $A$ obsahuje všetkých hráčov, ktorí majú po dvoch výhrach a dvoch prehrách. Skupina $B$ pozostáva z hráčov s jednou výhrou, jednou prehrou a dvoma remízami. Skupina $C$ obsahuje hráčov so štyrmi remízami.

Vojto a Tomáš sú jediní víťazi, preto nepatria do skupiny $C$. Nepatria ani do skupiny $B$, pretože v opačnom prípade by s nimi museli všetci traja hráči zo skupiny $C$ s horším výsledkom remizovať (a každý hráč skupiny B má len dve remízy).

Z toho vyplýva, že Vojto a Tomáš majú po dvoch výhrach a dvoch prehrách a skupina $C$ je prázdna. Zvyšní traja hráči tak majú po jednej výhre, jednej prehre a dvoch remízach, ktoré museli uhrať navzájom medzi sebou.\\
\textit{Záver}. Peter a Martin spolu remizovali.\\

\textbf{Iné riešenie*.} Využijeme (nadbytočný) údaj, že Vojto porazil Petra: Keby mali Vojto a Tomáš po jednej výhre, jednej prehre a dvoch remízach, musel by aj Peter patriť medzi víťazov turnaja. Jediný v poradí nižší celkový výsledok sú totiž štyri remízy, Peter však jednu partiu prehral, a tak musel aj jednu vyhrať. Vojto a Tomáš majú 1preto po dvoch výhrach a dvoch prehrách. Ak Peter prehral s Vojtom, musel poraziť Tomáša. (Nemohol mať dve prehry, keďže bol v poradí nižšie ako Tomáš a Vojto. Ani nemohol s Tomášom, ktorý žiadnu remízu nemá, remizovať.) Potrebný druhý bod získal dvoma remízami -- s Martinom a nepomenovaným piatym hráčom.\\
\textit{Záver}. Peter a Martin spolu remizovali.
}