% Do not delete this line (pandoc magic!)

\problem{61-I-4}{seminar18,nerovnosti2,odhady}{
Reálne čísla $a, b, c, d$ vyhovujú rovnici $ab + bc + cd + da = 16$.
\begin{enumerate}[a)]
\item Dokážte, že medzi číslami $a, b, c, d$ sa nájdu dve so súčtom najviac 4.
\item Akú najmenšiu hodnotu môže mať súčet $a^2 + b^2 + c^2 + d^2$?
\end{enumerate}
}{
\rieh a) Z rovnosti $16 = ab + bc + cd + da = (a + c)(b + d)$ vyplýva, že obidva súčty $a + c$ a $b + d$ nemôžu byť väčšie ako 4 súčasne, lebo v opačnom prípade by bol ich súčin väčší ako 16. Preto vždy aspoň jeden zo súčtov $a + c$ alebo $b + d$ má požadovanú
vlastnosť.
b) Využijeme všeobecnú rovnosť $$a^2+ b^2+ c^2+ d^2=\frac{1}{2}(a - b)^2+\frac{1}{2}(b - c)^2+\frac{1}{2}(c - d)^2+\frac{1}{2}(d - a)^2+ ab + bc + cd + da,$$
o platnosti ktorej sa ľahko presvedčíme úpravou pravej strany. Vzhľadom na nezápornosť druhých mocnín $(a-b)^2 $, $(b-c)^2$ , $(c-d)^2$ a $(d-a)^2$ dostávame pre ľavú stranu rovnosti odhad $$a^2+ b^2+ c^2+ d^2\geq ab + bc + cd + da = 16.$$
Je nájdené číslo 16 najmenšou hodnotou uvažovaných súčtov? Ináč povedané: nastane pre niektorú vyhovujúcu štvoricu v odvodenej nerovnosti rovnosť? Z nášho postupu je jasné, že musíme rozhodnúť, či pre niektorú z uvažovaných štvoríc platí $a - b = b - c = c - d = d - a = 0$, čiže $a = b = c = d$. Pre takú štvoricu má rovnosť $ab + bc + cd + da = 16$ tvar $4a^2 = 16$, čomu vyhovuje $a = \pm 2$. Pre vyhovujúce štvorice $a = b = c = d = 2$ a $a = b = c = d = -2$ má súčet $a^2 + b^2 + c^2 + d^2$ naozaj hodnotu 16, preto ide o hľadané minimum.\\
\\
\kom Ďalšia úloha, ktorá nepracuje s priamym dokazovaním nerovností, avšak využíva fakt, ktorý sme si osvojili už pri prvom nerovnostnom seminári, a to že druhá mocnina ľubovoľného reálneho čísla je vždy nezáporná. To nám potom pomohlo uskutočniť odhad hodnoty súčtu zo zadania úlohy. Na tomto mieste považujeme sa vhodné študentom zmieniť, že odhadovanie hodnôt je ďalším miestom, kde sa znalosti o nerovnostiach výborne uplatnia, ako ukáže aj nasledujúca úloha.
\\
\\
}