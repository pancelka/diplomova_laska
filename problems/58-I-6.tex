% Do not delete this line (pandoc magic!)

\problem{66-I-1-D3, resp. 58-I-6}{seminar05,nerovnosti}{
Dokážte, že pre ľubovoľné rôzne kladné čísla $a, b$ platí
$$\frac{a+b}{2}<\frac{2(a^2 + ab + b^2 )}{3(a+b)}<\sqrt{\frac{a^2+b^2}{2}}.$$
}{
\rieh Ľavú nerovnosť dokážeme ekvivalentnými úpravami:
\begin{align*}
\frac{a+b}{2}&<\frac{2(a^2 + ab + b^2 )}{3(a+b)}, \ \ | \cdot 6(a+b)\\
3(a+b)^2&<4(a^2+ab+b^2),\\
0&<(a-b)^2.
\end{align*}
Posledná nerovnosť vzhľadom na predpoklad $a\neq b$ platí. Aj pravú nerovnosť zo zadania budeme ekvivalentne upravovať, začneme umocnením každej strany na druhú:
\begin{align*}
\frac{4(a^2 + ab + b^2 )^2}{9(a + b)^2}&<\frac{a^2 + b^2}{2}, \ \ | \cdot 18(a + b)^2\\
8(a^2 + ab + b^2 )^2 &< 9(a^2 + b^2 )(a + b)^2,\\
8(a^4 + b^4 + 2a^3 b + 2ab^3 + 3a^2 b^2 ) &< 9(a^4 + b^4 + 2a^3 b + 2ab^3 + 2a^2 b^2 ),\\
6a^2 b^2 &< a^4 + b^4 + 2a^3 b + 2ab^3.
\end{align*}
Posledná nerovnosť je súčtom nerovností $2a^2 b^2 < a^4 + b^4$ a $4a^2 b^2 < 2a^3 b + 2ab^3$, ktoré obe platia, lebo po presune členov z~ľavých strán na pravé dostaneme po rozklade už zrejmé nerovnosti $0 < (a^2- b^2)^2$, resp. $0 < 2ab(a - b)^2$.
}
