% Do not delete this line (pandoc magic!)

\problem{B-64-S-2}{skolskekolo}{
Označme $P$ priesečník uhlopriečok konvexného štvoruholníka $ABCD$. Vypočítajte jeho obsah, ak obsahy trojuholníkov $ABC$, $BCD$ a $DAP$ sú postupne $8$\,cm$^2$, $9$\,cm$^2$, $10$\,cm$^2$.
}{
\rieh Označme $S_{XYZ}$ obsah trojuholníka $XYZ$ vyjadrený v cm$^2$ a ďalej označme $S = S_{ABP}$. Podľa zadania platí $S_{ADP} = 10$, $S + S_{BCP} = 8$, $S_{BCP} + S_{CDP} = 9$. Z druhej rovnosti vyplýva $S_{BCP} = 8 - S$, dosadením do tretej rovnosti potom vyjde $S_{CDP} = 1 + S$ \todo{(obr. 1)}.\\
\\
\todo{DOPLNIŤ Obr. 1}\\
\\
Trojuholníky $ABP$ a $ADP$ majú zhodnú výšku z vrcholu $A$. Pre pomer ich obsahov preto platí $S : S_{ADP} = |BP| : |DP|$. Podobne pre trojuholníky $BCP$ a $CDP$ dostaneme $S_BCP : S_{CDP} = |BP| : |DP|$. Z toho už vyplýva $S : S_{ADP} = S_{BCP} : S_{CDP}$, čo vzhľadom na odvodené vzťahy znamená
$$ \frac{S}{10}=\frac{8 - S}{1 + S}.$$
Po úprave tak dostaneme pre $S$ kvadratickú rovnicu
$$S^2+ 11S - 80 = (S + 16)(S - 5) = 0,$$
ktorá má dva korene $-16$ a $5$. Keďže obsah $S$ trojuholníka $ABP$ je nezáporné číslo, vyhovuje iba $S = 5$. Odtiaľ už ľahko dopočítame z vyššie uvedených vzťahov $S_{BCP} = 3$ a $S_{CDP} = 6$. Obsah celého štvoruholníka $ABCD$ vyjadrený v cm$^2$ teda je
$$S + S_{BCP} + S_{CDP} + S_{ADP} = 5 + 3 + 6 + 10 = 24.$$
\textit{Záver.} Obsah štvoruholníka $ABCD$ je 24\,cm$^2$.
}
