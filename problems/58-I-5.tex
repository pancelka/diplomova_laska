% Do not delete this line (pandoc magic!)

\problem{58-I-5}{
Z~množiny $\{1, 2, 3, \ldots, 99\}$ vyberte čo najväčší počet čísel tak, aby súčet žiadnych dvoch vybraných čísel nebol násobkom jedenástich. (Vysvetlite, prečo zvolený výber má požadovanú vlastnosť a prečo žiadny výber väčšieho počtu čísel nevyhovuje.)
}{
\rieh Čísla od 1 do 99 rozdelíme podľa ich zvyšku po delení číslom 11 do jedenástich deväťprvkových skupín $T_0, T_1 ,\ldots, T_{10}$:
\begin{center}
\begin{align*}
T_0 &= \{11, 22, 33, . . . , 99\},\\
T_1 &= \{1, 12, 23, . . . , 89\},\\
T_2 &= \{2, 13, 24, . . . , 90\},\\
\vdots\\
T_{10} &= \{10, 21, 32, . . . , 98\}.\\
\end{align*}
\end{center}
Ak vyberieme jedno číslo z~$T_0$ (viac ich ani vybrať nesmieme) a všetky čísla z~$T_1, T_2, T_3, T_4$ a $T_5$, dostaneme vyhovujúci výber $1 + 5 \cdot 9 = 46$ čísel, lebo súčet dvoch čísel z~$0, 1, 2, 3, 4, 5$ je deliteľný jedenástimi jedine v~prípade 0 + 0, z~množiny $T_0$ sme však vybrali iba jedno číslo.

Na druhej strane, v~ľubovoľnom vyhovujúcom výbere je najviac jedno číslo zo skupiny $T_0$ a najviac 9 čísel z~každej zo skupín
$$ T_1 \cup T_{10}, \ \ T_2 \cup T_9, \ \ T_3 \cup T_8, \ \  T_4 \cup T_7, \ \ T_5 \cup T_6,$$
lebo pri výbere 10 čísel z~niektorej skupiny $T_i \cup T_{11-i}$ by medzi vybranými bolo niektoré číslo zo skupiny $T_i$ a aj niektoré číslo zo skupiny $T_{11-i}$; ich súčet by potom bol deliteľný jedenástimi. Celkom je teda vo výbere najviac $1 + 5 \cdot 9 = 46$ čísel.

\textit{Poznámka}. Možno uvedené \uv{učesané} riešenie vyzerá príliš trikovo. Avšak počiatočné úvahy každého riešiteľa k~nemu rýchlo vedú: iste záleží len na zvyškoch vybraných čísel, takže rozdelenie na triedy $T_i$ a vyberanie z~nich je prirodzené. Je jasné, že z~$T_0$ môže byť vybrané len jedno číslo a všetko ďalšie, o~čo sa musíme starať, je požiadavka, aby sme nevybrali zároveň po čísle zo skupiny $T_i$ aj zo skupiny $T_{11-i}$. Ak je už vybrané niektoré číslo z~triedy $T_i$, kde $i\neq 0$, môžeme pokojne vybrať všetky čísla z~$T_i$, to už skúmanú vlastnosť nepokazí. Je preto dokonca jasné, ako všetky možné výbery najväčšieho počtu čísel vyzerajú.\\
\\
\kom Je veľmi vhodné sa k tejto úlohe v nasledujúcom seminári vrátiť s poznámkou, že množiny $T_1, \ldots, T_{10}$ nazývame zvyškové triedy. \\
\\
}
