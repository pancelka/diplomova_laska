% Do not delete this line (pandoc magic!)

\problem{58-I-1-D1}{
Dokážte, že ľubovoľné prirodzené číslo $n \geq 3$, ktoré nie je mocninou čísla 2, možno
vyjadriť ako súčet niekoľkých po sebe idúcich prirodzených čísel.
}{
\rieh $n = \frac{n-1}{2}+ \frac{n+1}{2}$ pre
$n$ nepárne, $n = ( \frac{n}{p}-\frac{p-1}{2}+1)+(\frac{n}{p}-\frac{p-1}{2}+ 1) +\,\ldots + ( \frac{n}{p}+\frac{p-1}{2})$ pre $n = p\cdot q$, kde $p > 1$ je nepárny deliteľ
}
