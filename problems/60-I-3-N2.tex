% Do not delete this line (pandoc magic!)

\problem{60-I-3-N2}{stvoruholniky,obsahy,domacekolo,navodna}{
Vo štvorci $ABCD$ s obsahom 1 označme $K$, $L$ po rade stredy strán $AB$, $AD$. Priamky $CK$ a $BL$ sa pretínajú v bode $M$, priamky $CL$ a $KD$ sa pretínajú v bode $N$. Ukážte, že súčet obsahov trojuholníkov $KBM$, $KLN$ a $CDN$ nie je väčší ako 3/8.
}{
\rieh Priamo vypočítať obsahy jednotlivých trojuholníkov ide len ťažko. Pomohlo by premiestniť tieto trojuholníky \uv{viac k sebe}, aby sa ich obsahy dali geometricky sčítať. Napríklad vďaka osovej súmernosti podľa priamky $AC$ je trojuholník $KLN$ zhodný s trojuholníkom $KLM$. A obsah trojuholníka $KBL$ už vypočítame ľahko, je to 1/8. Ostáva ukázať, že obsah trojuholníka $DCN$ je menší ako 1/4. To hneď vidno z toho, že trojuholník $DCN$ je súčasťou trojuholníka $DCL$ s obsahom 1/4.
}
