% Do not delete this line (pandoc magic!)

\problem{66-I-4}{seminar06,rovnice,sustavy,mnohocleny}{
Nájdite všetky trojčleny $P(x)=ax^2+bx+c$ s~celočíselnými koeficientami $a, b, c$, pre ktoré platí $P(1) < P(2) < P(3)$ a zároveň $$(P(1))^2+ (P(2))^2+ (P(3))^2= 22.$$
}{
\rieh Keďže $a, b, c$ sú podľa zadania celé čísla, sú také aj hodnoty $P(1), P(2)$ a $P(3)$. Ich druhé mocniny, čiže čísla  $P(1)^2 , P(2)^2$ a $P(3)^2$, sú preto druhými mocninami
celých čísel, teda tri (nie nutne rôzne) čísla z~množiny $\{0, 1, 4, 9, 16, 25, . . .\}$. Ich súčet je podľa zadania rovný 22, takže každý z~troch sčítancov je menší ako šieste možné číslo 25. Akými spôsobmi možno vôbec zostaviť súčet 22 z~troch čísel vybraných z~množiny  $\{0, 1, 4, 9, 16\}$?
Systematickým rozborom rýchlo zistíme, že rozklad čísla 22 na súčet troch druhých mocnín je (až na poradie sčítancov) iba jeden, a to $22 = 4+9+9$. Dve z~čísel $P(1), P(2)$ a $P(3)$ majú teda absolútnu hodnotu 3 a tretie 2, a keďže
$P(1) < P(2) < P(3)$, musí nutne platiť $P(1) = -3$, $P(3) = 3$ a $P(2) \in \{-2, 2\}$. Pre každú z~oboch vyhovujúcich
trojíc $(P(1), P(2), P(3)) = (-3, -2, 3)$ a $(P(1), P(2), P(3)) = (-3, 2, 3)$ určíme koeficienty $a, b, c$ príslušného trojčlena $P(x)$ tak, že nájdené hodnoty dosadíme do pravých strán rovníc
\begin{align*}
a + b + c &= P(1),\\
4a + 2b + c &= P(2),\\
9a + 3b + c &= P(3)
\end{align*}
a výslednú sústavu troch rovníc s~neznámymi $a, b, c$ vyriešime. Tento jednoduchý výpočet tu vynecháme, v~oboch prípadoch vyjdú celočíselné trojice $(a, b, c)$, ktoré zapíšeme rovno ako koeficienty trojčlenov, ktoré sú jedinými dvoma riešeniami danej úlohy:
$$P_1(x) = 2x^2 -5x \ \ \ \  \text{a} \ \ \ \ P_2 (x) = -2x^2+ 11x -12.$$
}
