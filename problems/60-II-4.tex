% Do not delete this line (pandoc magic!)

\problem{60-II-4}{seminar18,nerovnosti2,spor}{
Nech $x, y, z$ sú kladné reálne čísla. Ukážte, že aspoň jedno z čísel $x + y + z - xyz$ a $xy + yz + zx - 3$ je nezáporné.
}{
\rieh Ukážeme, že ak je číslo $xy + yz + zx - 3$ záporné, je číslo $x + y + z - xyz$ kladné.
Ak $xy + yz + zx < 3$, je aspoň jedno z čísel $xy$, $yz$, $zx$ menšie ako 1, napr. $xy$. Potom $x + y + z - xyz = x + y + z(1 - xy)$ je zjavne súčet troch kladných čísel.

\textbf{Iné riešenie*.} Ukážeme, že ak je číslo $x+y +z -xyz$ záporné, tak číslo $xy +yz +zx-3$ je kladné.
Predpokladajme, že $x + y + z < xyz$. Tým skôr $x < xyz$. Po skrátení kladného čísla $x$ dostaneme $yz > 1$. Podobne odvodíme odhady $xy > 1$ a $zx > 1$. Teraz ich stačí sčítať a máme $xy + yz + zx > 3$.

\textbf{Iné riešenie*.} Tvrdenie úlohy dokážeme sporom. Predpokladajme, že $x + y + z < xyz$ a zároveň $xy + yz + zx < 3$. Obe tieto nerovnosti sú symetrické, preto môžeme predpokladať, že čísla $x, y, z$ sú označené tak, že $z$ je najmenšie. Z druhej nerovnosti
dostaneme, že $xy < 3$. Potom však $x + y + z < xyz < 3z$, teda $x + y < 2z$. To je však spor s tým, že číslo z je najmenšie.\\
\\
\kom Úloha je zaujímavá tým, že na prvý pohľad nemusí vyzerať ako úloha o nerovnostiach a tiež študentom nemusí byť úplne jasné, z ktorého konca úlohu uchopiť. Môže to byť tiež dobrá príležitosť na ukážku toho, ako sa dá uplatniť dôkaz sporom.\\
\\
}