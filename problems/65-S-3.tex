% Do not delete this line (pandoc magic!)

\problem{65-S-3}{seminar23,netrgeo,skolskekolo}{V~kružnici so stredom $S$ zostrojíme priemer $AB$ a ľubovoľnú naň kolmú tetivu $CD$. Zdôvodnite, prečo je obvod trojuholníka $ACD$ menší ako dvojnásobok obvodu trojuholníka $SBC$.
}{
\rieh Želaný vzťah medzi obvodmi trojuholníkov $ACD$ a $SBC$ vyplynie, keď pre dĺžky ich strán objavíme nerovnosti
$$|AC| < 2|SB|,\ \ \ \  |AD| < 2|SC|\ \ \ \  \text{a} \ \ \ \  |CD| < 2|BC|.$$
Prvé dve z~nich sú dôsledkom toho, že tetivy $AC$ a $AD$ danej kružnice sú kratšie ako jej priemer $AB$ (obr.~\ref{fig:65S3}), tretia nerovnosť zapísaná v~tvare $\frac{1}{2}|CD| < |BC|$ je nerovnosťou medzi dĺžkami odvesny a prepony dvoch zhodných pravouhlých trojuholníkov, na ktoré je trojuholník $BCD$ rozdelený priamkou $AB$, ktorá je totiž (vďaka predpokladu $AB \perp CD$) osou tetivy $CD$. Dodajme, že rovnako dobre možno využiť aj trojuholníkovú nerovnosť $|CD| < |BC| + |BD| = 2|BC|$.
\begin{figure}[h]
    \centering
    \includegraphics{images/65S3\imagesuffix}
    \caption{}
    \label{fig:65S3}
\end{figure}
\textbf{Iné riešenie*.} Označme $\alpha$ veľkosti vnútorných uhlov pri základni $AC$ rovnoramenného trojuholníka $SAC$. Potom jeho vonkajší uhol pri vrchole $S$, čiže uhol $CSB$, má veľkosť $2\alpha$, ktorú má aj uhol $CAD$, pretože polpriamka $AB$ je jeho osou (obr.~\ref{fig:65S3}). Rovnoramenné trojuholníky $ACD$ a $SCB$ sa tak zhodujú vo vnútorných uhloch pri svojich hlavných vrcholoch $A$ a $S$, a sú teda podobné. Preto je pomer ich obvodov rovný pomeru dĺžok ich ramien, a ten má naozaj hodnotu menšiu ako 2, lebo ramená trojuholníka $ACD$ sú kratšie ako priemer danej kružnice, zatiaľ čo ramená trojuholníka $SCB$ majú dĺžku jej polomeru.\\
}