% Do not delete this line (pandoc magic!)

\problem{66-I-2-N1}{seminar07,delitelnost}{
Dokážte, že v~nekonečnom rade čísel
$$ 1 \cdot 2 \cdot 3, 2 \cdot 3 \cdot 4, 3 \cdot 4 \cdot 5, 4 \cdot 5 \cdot 6,\,\ldots ,$$
je číslo prvé deliteľom všetkých čísel ďalších.
}{
\rie Prvé číslo v~nekonečnom rade je číslo 6. Dokazujeme tak, že všetky výrazy tvaru $n(n+1)(n+2)$, kde $n\geq 2$ je prirodzené číslo, sú deliteľné šiestimi. To ale zjavne platí, keďže z~troch po sebe idúcich čísel je vždy práve jedno deliteľné tromi a minimálne jedno z~nich je tiež párne. Deliteľnosť dvomi a tromi zároveň nám tak zaručí deliteľnosť šiestimi a požadované tvrdenie je dokázané.\\
\\
\kom Táto jednoduchá úloha zoznámi žiakov s~poznatkom často využívaným v~úlohách zameraných na dokazovanie deliteľnosti číslom, ktoré je násobkom troch: z~troch po sebe idúcich prirodzených čísel je vždy práve jedno deliteľné tromi.\\
\\
}
