% Do not delete this line (pandoc magic!)

\problem{60-I-2-D2}{
Dokážte, že ku každému celému číslu $x$ existuje celé číslo $y$ také, že $19x+3y$ je deliteľné 50.
}{
Číslo $19x$ dáva po delení 50 zvyšok, ktorý označíme $z$. Chceme ukázať, že pre
ľubovoľné $z$ vieme nájsť $y$ tak, aby číslo $3y$ dávalo zvyšok $50 - z$. Vezmime si čísla
$3 \cdot 1$, $3 \cdot 2$, $3 \cdot 3$, $\ldots$, $3 \cdot 50$. Keby dve z týchto čísel, povedzme $3i$ a $3j$, dávali rovnaký
zvyšok, musí byť ich rozdiel $3(i - j)$ deliteľný 50. Pritom 3 a 50 sú nesúdeliteľné, preto $50 \mid i - j$. To však nie je možné, lebo $1 \leq  i - j \leq 49$. Preto vymenované čísla dávajú všetky možné rôzne zvyšky po delení 50, a teda jedno z nich dáva zvyšok $50 - z$.
}
