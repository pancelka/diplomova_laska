% Do not delete this line (pandoc magic!)

\problem{66-I-1}{
Dokážte, že pre ľubovoľné reálne číslo a platí nerovnosť $$a^2+\frac{1}{a^2-a+1}\geq a+1.$$ Určte, kedy nastáva rovnosť.
}{
\rieh Úpravou dvojčlena $a^2 - a$ doplnením na štvorec a využitím faktu že druhá mocnina reálneho čísla je nezáporná ukážeme, že menovateľ zlomku v~nerovnosti je kladný:
$$a^2-a+1=\bigg(a^2-a+\frac{1}{4}\bigg) +\frac{3}{4}=\bigg(a-\frac{1}{2}\bigg)^2+\frac{3}{4}\geq \frac{3}{4}>0.$$
Ak ním teda obe strany dokazovanej nerovnosti vynásobíme, dostaneme ekvivalentnú nerovnosť
$$a^2(a^2-a+1)+1\geq (a+1)(a^2-a+1).$$
Po roznásobení a zlúčení rovnakých mocnín a dôjdeme k~nerovnosti
$$ a^4-2a^3+a^2\geq 0,$$
ktorá však platí, pretože jej ľavá strana má rozklad $a^2 (a - 1)^2$ s~nezápornými činiteľmi $a^2$ a $(a - 1)^2$. Tým je pôvodná nerovnosť pre každé reálne číslo a dokázaná. Zároveň sme zistili, že rovnosť vo výslednej, a teda aj v~pôvodnej nerovnosti nastane práve vtedy, keď platí $a^2 (a - 1)^2 = 0$, teda jedine vtedy, keď $a = 0$ alebo $a = 1$.\\
\\
\textbf{Iné riešenie*.} Danú nerovnosť môžeme prepísať na tvar $$ (a^2 - a + 1) + \frac{1}{a^2-a+1}\geq 2 \ \ \ \ \text{čiže} \ \ \ \  u~+\frac{1}{u}\geq 2,$$ pričom $u = a^2 -a + 1$. Využitím faktu, že posledná nerovnosť platí pre každé kladné
reálne číslo $u$ a že prechádza v~rovnosť jedine pre $u = 1$.

Na dôkaz pôvodnej nerovnosti ostáva už len overiť, že výraz $u = a^2 - a + 1$ je kladný pre každé reálne číslo $a$. To možno spraviť rovnako ako v~prvom riešení, alebo prepísať nerovnosť $a^2 - a + 1 > 0$ na tvar $$a(a -1) > -1$$ a uskutočniť krátku diskusiu: Posledná nerovnosť platí ako pre každé $a \geq 1$, tak pre každé $a\leq 0$, lebo v~oboch prípadoch máme dokonca $a(a - 1) \geq 0$; pre zvyšné hodnoty $a$, teda pre $a \in (0, 1)$, je súčin $a(a - 1)$ síce záporný, avšak určite väčší ako $-1$, pretože oba činitele $a$, $a - 1$ majú absolútnu hodnotu menšiu ako 1. Prepísaná nerovnosť je
tak dokázaná pre každé reálne číslo $a$, a tým je podmienka pre použitie nerovnosti $u + \frac{1}{u} \geq 2$ pre $u = a^2 + a + 1$ overená.

Ako sme už uviedli, rovnosť $u + \frac{1}{u} = 2$ nastane jedine pre $u = 1$. Pre rovnosť v~nerovnosti zo zadania úlohy tak dostávame podmienku $a^2 -a+1 = 1$, čiže $a(a-1)= 0$, čo je splnené iba pre $a = 0$ a pre $a = 1$.\\
\\
\kom Úloha využíva spojenie viacerých poznatkov -- faktu, že druhá mocnina akéhokoľvek reálneho čísla je nezáporná, úpravu na štvorec, ekvivalentné úpravy nerovností a tiež známu nerovnosť $u+\frac{1}{u} \geq 2$ pre každé kladné reálne $u$. Je síce náročnejšia ako úlohy, ktorými sme sa doteraz zaoberali, ale považujeme ju za vhodnú ilustráciu toho, ako nám rozšírený arzenál metód pomôže v~úspešnom zvládnutí zložitejších problémov. Úloha tiež demonštruje, že k~správnemu riešeniu častokrát vedú viaceré cesty.\\
\\
}
