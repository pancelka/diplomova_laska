% Do not delete this line (pandoc magic!)

\problem{B-59-I-3}{
V rovine je daná úsečka $AB$. Zostrojte rovnobežník $ABCD$, pre ktorého stredy strán $AB$, $CD$, $DA$ označené postupne $K$, $L$, $M$ platí: body $A$, $B$, $L$, $D$ ležia na jednej kružnici a aj body $K$, $L$, $D$, $M$ ležia na jednej kružnici.
}{
\rieh Označme $a = |AB|$, $b = |AD|$ dĺžky strán hľadaného rovnobežníka \todo{doplniť (obr. 1)}. Lichobežníku $ABLD$ sa dá opísať kružnica, preto je rovnoramenný, a teda $|BL|= b$. Keďže sú úsečky $KB$ a $DL$ rovnobežné a zhodné, je $KBLD$ rovnobežník, a preto $|KD| = |BL| = b$. To znamená, že trojuholník $AKD$ je rovnoramenný, takže bod $D$
musí ležať na osi jeho základne $AK$.

\todo{DOPLNIŤ Obr. 1}
Úsečka $KL$ je strednou priečkou rovnobežníka $ABCD$, preto $KL \parallel MD$; $KLDM$ je teda lichobežník, a pretože sa mu dá opísať kružnica, je rovnoramenný; odtiaľ $|KM|= |DL| = \frac{1}{2}a$. Keďže $KM$ je stredná priečka trojuholníka $BDA$, má strana $BD$ dĺžku $2 \cdot |KM| = a$. Bod $D$ teda leží na kružnici so stredom $B$ a polomerom $a$.

\textit{Konštrukcia.} Zostrojíme stred $K$ úsečky $AB$, os $o$ úsečky $AK$ a kružnicu $k$ so stredom $B$ a polomerom $|AB|$. Priesečník tejto kružnice s osou úsečky $AK$ je bod $D$. Bod $C$ je potom priesečník priamok vedených bodmi $D$ a $B$ rovnobežne s priamkami $AB$ a $AD$.

\textit{Dôkaz správnosti konštrukcie.} Štvoruholník $ABCD$ má protiľahlé strany rovnobežné, je to teda rovnobežník. Označíme $L$ a $M$ stredy úsečiek $CD$ a $AD$. Z toho, že bod $D$ leží na osi úsečky $AK$, vyplýva $|KD| = |AD|$. Keďže $KBLD$ je rovnobežník, platí $|BL| = |KD| = |AD|$. Lichobežník $ABDL$ je teda rovnoramenný, a preto body $A$, $B$, $L$, $D$ ležia na jednej kružnici. Úsečka $KM$ je stredná priečka trojuholníka $BDA$, preto $|KM| = \frac{1}{2}|BD|= \frac{1}{2}|AB| = |DL|$; $KLDM$ je teda rovnoramenný lichobežník, a preto jeho vrcholy ležia na jednej kružnici.

\textit{Diskusia.} Priamka $o$ má od bodu $B$ menšiu vzdialenosť ako bod $A$, takže pretína kružnicu $k$ v dvoch bodoch. Úloha má teda v každej polrovine s hraničnou priamkou $AB$
jedno riešenie. 

\textbf{Iné riešenie.} Tak ako v prvom riešení dokážeme, že $|KD| = |AD|$ a $|DB| = |AB|$. Trojuholníky $AKD$ a $DAB$ sú teda rovnoramenné, a keďže sa zhodujú v uhle pri vrchole $A$, sú podobné. Preto $|AK|/|AD| = |DA|/|AB|$, čiže $\frac{1}{2}a/b = b/a$ a odtiaľ $b = \frac{1}{2}a\sqrt{2}$. Bod $D$ je teda priesečníkom kružníc so stredmi $A$ a $K$ a polomerom $\frac{1}{2}a\sqrt{2}$.
}
