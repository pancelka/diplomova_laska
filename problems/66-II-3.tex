% Do not delete this line (pandoc magic!)

\problem{66-II-3}{Dokážte, že obdĺžnik s~rozmermi $32 \times 120$ sa dá zakryť siedmimi zhodnými štvorcami so stranou 30.

}{
\rieh Štyrmi štvorcami so stranou 30 zrejme zakryjeme obdĺžnik $30\times 120$. Zvyšnú časť $2 \times 120$ rozdelíme na tri zhodné časti, konkrétne obdĺžniky $2 \times 40$, a ukážeme, ako každý z~nich (rovnako) pokryť jedným z~troch zvyšných štvorcov so stranou 30. Dosiahneme to, keď štvorec položíme na obdĺžnik tak, že obe uhlopriečky štvorca budú ležať na osiach súmernosti dotyčného obdĺžnika. Stačí potom ukázať, že obdĺžnik so stranou 2 vpísaný do štvorca podľa \ref{fig:66II3} má druhú stranu dlhšiu ako 40. Jej dĺžka je zrejme $30\sqrt{2}-2$ (od uhlopriečky štvorca odčítame na každej strane 1 ako veľkosť výšky
\begin{figure}[h]
    \centering
    \includegraphics{images/66K3\imagesuffix}
    \caption{}
    \label{fig:66II3}
\end{figure}
pravouhlého trojuholníka so stranami $2, \sqrt{2}, \sqrt{2}$, pozri zväčšenú časť \ref{fig:66II3}), takže stačí ukázať, že $30\sqrt{2}-2\geq 40$. To je ekvivalentné s~nerovnosťou $5\sqrt{2}\geq 7$, čiže $50 \geq 49$, čo je splnené. Daný obdĺžnik $32 \times 120$ teda naozaj možno zakryť siedmimi štvorcami so stranou 30.\\
}