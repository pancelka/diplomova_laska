% Do not delete this line (pandoc magic!)

\problem{60-II-4}{
Nech $x, y, z$ sú kladné reálne čísla. Ukážte, že aspoň jedno z čísel $x + y + z - xyz$ a $xy + yz + zx - 3$ je nezáporné.
}{
\rieh Ukážeme, že ak je číslo $xy + yz + zx - 3$ záporné, je číslo $x + y + z - xyz$ kladné.
Ak $xy + yz + zx < 3$, je aspoň jedno z čísel $xy$, $yz$, $zx$ menšie ako 1, napr. $xy$. Potom $x + y + z - xyz = x + y + z(1 - xy)$ je zjavne súčet troch kladných čísel.

\textbf{Iné riešenie.} Ukážeme, že ak je číslo $x+y +z -xyz$ záporné, tak číslo $xy +yz +zx-3$ je kladné.
Predpokladajme, že $x + y + z < xyz$. Tým skôr $x < xyz$. Po skrátení kladného čísla $x$ dostaneme $yz > 1$. Podobne odvodíme odhady $xy > 1$ a $zx > 1$. Teraz ich stačí sčítať a máme $xy + yz + zx > 3$.

\textbf{Iné riešenie.} Tvrdenie úlohy dokážeme sporom. Predpokladajme, že $x + y + z < xyz$ a zároveň $xy + yz + zx < 3$. Obe tieto nerovnosti sú symetrické, preto môžeme predpokladať, že čísla $x, y, z$ sú označené tak, že $z$ je najmenšie. Z druhej nerovnosti
dostaneme, že $xy < 3$. Potom však $x + y + z < xyz < 3z$, teda $x + y < 2z$. To je však spor s tým, že číslo z je najmenšie.\\
\\
\textbf{Iné riešenie.} Aritmetický priemer $c$ čísel $a, b$ má tú vlastnosť, že sa od neho obe čísla líšia o rovnakú hodnotu $d$. Ak nahradíme premenné $a, b$ v daných nerovnostiach premennými $c, d$, zápis nerovností aj dôkaz oboch vzťahov sa zjednoduší. Položme teda $c = \frac{1}{2}(a + b)$, potom $a = c + d$ a $b = c - d$ (pričom $d =\frac{1}{2}(a - b)$, ako sa ľahko môžeme presvedčiť). Takže $a^2 + b^2 = 2(c^2 + d^2)$, $ab = c^2 - d^2$ , odkiaľ $a^2 + 3ab + b^2 = 5c^2 - d^2$.
Označme ešte písmenami $m$ a $n$ ľavú a pravú stranu prvej z dokazovaných nerovností. Potom
$$m=\sqrt{am}=\sqrt{c^2-d^2},$$
$$ n=\frac{2(a^2+3ab+b^2}{5(a+b)}=\frac{2(5c^2-d^2}{5\cdot 2c}=c-\frac{d^2}{5c}=\sqrt{\bigg(c-\frac{d^2}{5c} \bigg)^2}=\sqrt{\bigg(c-d^2\frac{2}{5}-\frac{d^2}{25c^2} \bigg)}.$$ Keďže z vyjadrenia kladnej hodnoty $m$ vidíme, že $d^2 < c^2$ , pre výraz v poslednej zátvorke pod odmocninou platí
$$1>\frac{2}{5}\geq\frac{2}{5}-\frac{d^2}{25c^2}>\frac{2}{5}-\frac{1}{25}=\frac{9}{25}>0,$$ čo znamená, že výraz pod odmocninou leží v uzavretom intervale medzi číslami $c^2 - d^2$ a $c^2$. Odtiaľ vyplýva $m \leq n \leq c$, pričom rovnosť nastane práve vtedy, keď $d = 0$, t. j. keď $a = b$.

\textit{Poznámka.} Z výsledkov súťažnej úlohy vyplýva, že rozdiel medzi aritmetickým a geometrickým priemerom dvoch kladných čísel možno zdola odhadnúť nezáporným lomeným výrazom takto:
$$\frac{a+b}{2}- \sqrt{ab}\geq\frac{a+b}{2}-\frac{2(a^2+3ab+b^2}{5(a+b)}=\frac{(a-b)^2}{10(a+b)}.$$
Umocnením osamostatnenej odmocniny a ďalšími úpravami môžeme dokázať silnejší odhad rovnakého typu
$$\frac{a+b}{2}- \sqrt{ab}\geq\frac{(a-b)^2}{4(a+b)}.$$
Inú metódu dôkazov spolu s ďalšími podobnými nerovnosťami nájdete v článku J. Šimšu \textit{Dolní odhady rozdílu průměrů} v časopise Rozhledy matematicko-fyzikální 65 (1986/87), číslo 10, str. 403 -- 407.
}