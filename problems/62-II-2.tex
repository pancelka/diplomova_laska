% Do not delete this line (pandoc magic!)

\problem{62-II-2}{seminar12,obsahy,geompoc,krajskekolo}{
Vnútri rovnobežníka $ABCD$ je daný bod $K$ a v~páse medzi rovnobežkami $BC$ a $AD$ v~polrovine opačnej k~$CDA$ je daný bod $L$. Obsahy trojuholníkov $ABK, BCK, DAK$ a $DCL$ sú $S_{ABK} = 18$\,cm$^2$, $S_{BCK} = 8$\,cm$^2$, $S_{DAK} = 16$\,cm$^2$, $S_{DCL} = 36$\,cm$^2$. Vypočítajte obsahy trojuholníkov $CDK$ a $ABL$.
}{
\rieh Trojuholníky $ABK$ a $CDK$ majú zhodné strany $AB$ a $CD$ a súčet ich výšok $v_1$ a $v_2$ (vzdialeností bodu $K$ od priamky $AB$, resp. $CD$) je rovný výške v~rovnobežníka $ABCD$ (vzdialenosti rovnobežných priamok $AB$ a $CD$, obr.~\ref{fig:62II2}). Preto súčet ich obsahov dáva polovicu súčtu obsahu daného rovnobežníka:
$$S_{ABK} + S_{CDK} = \frac{1}{2} |AB|v_1 +\frac{1}{2} |CD|v_2 = \frac{1}{2}|AB| \cdot (v_1 + v_2 ) =\frac{1}{2}|AB| \cdot v~=\frac{1}{2} S_{ABCD}.$$
Podobne aj $S_{BCK} + S_{DAK} =\frac{1}{2} S_{ABCD}$, teda
$$S_{CDK} = S_{BCK} + S_{DAK} - S_{ABK}= 6\,\text{cm}^2.$$
\begin{figure}[h]
    \centering
    \includegraphics{images/62K2\imagesuffix}
    \caption{}
    \label{fig:62II2}
\end{figure}
Trojuholníky $ABL$ a $DCL$ majú zhodné strany $AB$ a $CD$. Ak $v_3$ označuje príslušnú výšku druhého z~nich, je výška prvého z~nich rovná $v + v_3$, takže pre rozdiel obsahov týchto trojuholníkov platí
\begin{align*}
S_{ABL} - S_{DCL} &= \frac{1}{2} |AB| \cdot (v~+ v_3 ) - \frac{1}{2}|CD|\cdot v_3 =\frac{1}{2} |AB| \cdot (v~+ v_3 - v_3 ) =\\
&= \frac{1}{2} |AB| \cdot v~= \frac{1}{2} S_{ABCD} = S_{BCK} + S_{DAK}.
\end{align*}
Odtiaľ vyplýva
$$S_{ABL} = S_{BCK} + S_{DAK} + S_{DCL} = 60\,\text{cm}^2.$$
\\
\kom Úloha precvičuje použitie tvrdenia, ktoré sme dokázali v~prvom geometrickom seminári, a to, že ak majú dva trojuholníky základňu rovnakej dĺžky, potom ich obsahy sú v~rovnakom pomere ako ich výšky na túto základňu.\\
\\
}
