\seminar{29}

\subsection*{Téma}
Algebraické výrazy a rovnice VI -- Kvadratické rovnice

\teachernote{
\subsection*{Ciele}
Precvičiť metódy používané pri práci s~kvadratickými rovnicami

\subsubsection*{Úvodný komentár}

Na začiatku seminára si spolu so študentami osviežime znalosti o kvadratických rovniciach, počte ich riešení a vzťahoch medzi reálnymi koreňmi a koeficientmi (Viètove vzorce). V čase konania seminára už študenti pravdepodobne budú mať za sebou preberanie tohto učiva na hodinách matematiky, takže by opakovanie nemalo zabrať priveľa času.}

\subsection*{Úlohy a riešenia}


% Do not delete this line (pandoc magic!)

\problem{B-57-I-5-N3}{
Nájdite všetky dvojice $(a,b)$ reálnych čísel, pre ktoré má každá z rovníc $x^2+(a-2)x+b-3=0$, $x^2+(a+2)x+3b-5=0$ dvojnásobný koreň.
}{
\rie Kvadratická rovnica má dvojnásobný koreň práve vtedy, ak jej diskriminant je rovný nule. Z tejto podmienky pre rovnice zo zadania dostávame
\begin{equation}
    \begin{aligned}
        a^2-4a-4b+16 & = 0, \\
        a^2+4a-12b+24 & =0.
    \end{aligned}
 \label{eq:B57I5N3}
\end{equation}
Odčítaním druhej rovnice od prvej máme po úprave $a=b-1$. Dosadením tohto vzťahu do jednej z rovníc v \ref{eq:B57I5N3} potom určíme možné hodnoty $b$, ktoré sú 3 a 7. K nim odpovedajúce hodnoty $a$ sú tak 2 a 6 a teda hľadané dvojice reálnych čísel $(a,b)$ sú $(2, 3)$ a $(6, 7)$.
}
\\
\\
\kom Jednoduchá úloha na úvod, v ktorej študenti aplikujú znalosti o závislosti medzi hodnotou diskriminantu a počtom riešení kvadratickej rovnice. Ten potom vedie na riešenie sústavy dvoch rovníc s dvomi neznámymi.

% Do not delete this line (pandoc magic!)

\problem{B-57-I-5}{seminar30,kvadr}{
Určte všetky dvojice $a, b$ reálnych čísel, pre ktoré má každá z~kvadratických rovníc
$$ax^2 + 2bx + 1 = 0, \ \ \ \ bx^2 + 2ax + 1 = 0$$
dva rôzne reálne korene, pričom práve jeden z~nich je spoločný obom rovniciam.
}{
\rieh Zo zadania vyplýva, že $a \neq 0$, $b \neq 0$ (inak by rovnice neboli kvadratické)
a $a \neq b$ (inak by rovnice boli totožné, a ak by mali dva reálne korene, boli by oba
spoločné).

Označme $x_0$ spoločný koreň oboch rovníc, takže
$$ax_0^2+ 2bx_0 + 1 = 0,\ \ \ bx_0^2+ 2ax_0 + 1 = 0.$$
Odčítaním oboch rovníc dostaneme $(a - b)(x_0^2- 2x_0 ) = x_0 (a - b)(x_0 - 2) = 0$. Keďže $a \neq b$ a 0 zrejme koreňom daných rovníc nie je, musí byť spoločným koreňom číslo $x_0 = 2$. Dosadením do daných rovníc tak dostaneme jedinú podmienku $4a + 4b + 1 = 0$, čiže
$$b = -a -\frac{1}{4}.$$


Diskriminant druhej z~daných rovníc je potom $4a^2 - 4b = 4a^2 + 4a + 1 = (2a + 1)^2$, takže rovnica má dva rôzne reálne korene pre ľubovoľné $a \neq -\frac{1}{2}$. Podobne diskriminant prvej z~daných rovníc je $4b^2- 4a = 4b^2 + 4b +1 = (2b +1)^2$. Rovnica má teda dva rôzne reálne korene pre ľubovoľné $b \neq -\frac{1}{2}$, čiže $a\neq  \frac{1}{4}$

Z~uvedených predpokladov však zároveň vyplýva $a \neq -\frac{1}{4}$ $(b \neq 0)$ a $a \neq - \frac{1}{8}$ $(a \neq b)$.

\textit{Záver.} Vyhovujú všetky dvojice $(a, -a - \frac{1}{4})$, kde $a \in \RR \ \{-\frac{1}{2}, -\frac{1}{4}, -\frac{1}{8}, 0, \frac{1}{4}\}$.\\
\\
\kom V úlohe sa k správnemu riešeniu dostaneme pomocou vhodného odčítania dvoch rovníc (a potom vhodnou úpravou takto vzniknutej rovnice). Považujeme za vhodné študentov na tento \uv{trik} upozorniť, keďže nájde uplatnenie nielen v nasledujúcej úlohe, ale aj v rôznych iných príkladoch.
}


\kom V úlohe sa k správnemu riešeniu dostaneme pomocou vhodného odčítania dvoch rovníc (a potom vhodnou úpravou takto vzniknutej rovnice). Považujeme za vhodné študentov na tento \uv{trik} upozorniť, keďže nájde uplatnenie nielen v nasledujúcej úlohe, ale aj v rôznych iných príkladoch.

% Do not delete this line (pandoc magic!)

\problem{B-57-II-1}{
Uvažujme dve kvadratické rovnice
$$x^2-ax-b = 0,\ \ \ \  x^2-bx-a = 0$$
s~reálnymi parametrami $a$, $b$. Zistite, akú najmenšiu a akú najväčšiu hodnotu môže nadobudnúť súčet $a + b$, ak existuje práve jedno reálne číslo $x$, ktoré súčasne vyhovuje obom rovniciam. Určte ďalej všetky dvojice $(a, b)$ reálnych parametrov, pre ktoré tento súčet tieto hodnoty nadobúda.
}{
\rieh Odčítaním oboch daných rovníc dostaneme rovnosť $(b-a)x+a-b = 0$, čiže $(b-a)(x-1) = 0$. Odtiaľ vyplýva, že $b = a$ alebo $x = 1$.

Ak $b = a$, majú obidve rovnice tvar $x^2-ax-a = 0$. Práve jedno riešenie existuje práve vtedy, keď diskriminant $a^2 + 4a$ je nulový. To platí pre $a = 0$ a pre $a = -4$. Pretože $b = a$, má súčet $a + b$ v~prvom prípade hodnotu $0$ a v~druhom prípade hodnotu $-8$.

Ak $x = 1$, dostaneme z~daných rovníc $a + b = 1$, teda $b = 1-a$. Rovnice potom majú tvar
$$x^2-ax + a-1 = 0 \ \ \ \ \text{a} \ \ \ \ x^2 + (a-1)x-a = 0.$$
Prvá má korene $1$ a $a-1$, druhá má korene $1$ a $-a$. Práve jedno spoločné riešenie tak dostaneme vždy s~výnimkou prípadu, keď $a-1 = -a$, čiže $a = \frac{1}{2}$ -- vtedy sú spoločné riešenia dve.

\textit{Záver.} Najmenšia hodnota súčtu $a + b$ je $-8$ a je dosiahnutá pre $a = b = -4$. Najväčšia hodnota súčtu $a + b$ je $1$; túto hodnotu má súčet $a + b$ pre všetky dvojice $(a, 1-a)$, kde $a\neq \frac{1}{2}$ je ľubovoľné reálne číslo.\\
\\\kom Úloha nadväzuje na predchádzajúcu, opäť rovnice v zadaní sčítame. Viac ako náročnosťou výpočtu je úloha zaujímavá svojim rozborom, kde je potrebné dať pozor na to, aby študenti správne zvážili oba prípady ($a=b$, $x=1$).\\
\\
}

\kom Úloha nadväzuje na predchádzajúcu, opäť rovnice v zadaní sčítame. Viac ako náročnosťou výpočtu je úloha zaujímavá svojim rozborom, kde je potrebné dať pozor na to, aby študenti správne zvážili oba prípady ($a=b$, $x=1$).

% Do not delete this line (pandoc magic!)

\problem{B-62-II-1}{}{
Pre ľubovoľné reálne čísla $k\neq \pm 1$, $p \neq 0$ a $q$ dokážte tvrdenie: Rovnica
$$x^2+ px + q = 0$$
má v~obore reálnych čísel dva korene, z~ktorých jeden je $k$-násobkom druhého, práve vtedy, keď platí $kp^2 = (k~+ 1)^2 q$.
}{
\rieh Čísla $x_1$, $x_2$ sú koreňmi danej kvadratickej rovnice práve vtedy, keď platí
\begin{equation} \label{eq:B62II1}
    x_1 + x_2 = -p \ \ \ \ \text{a} \ \ \ \  x_1 x_2 = q.
\end{equation}
Predpokladajme, že daná kvadratická rovnica má reálne korene $x_1 = \alpha$, $x_2 = k\alpha$. Dosadením do~\ref{eq:B62II1} dostaneme $(k + 1)\alpha = -p$ a $k\alpha^2 = q$. Pre obe strany dokazovanej rovnosti $kp^2 = (k~+ 1)^2 q$ odtiaľ vyplýva
$$kp^2= k(-(k + 1)\alpha)^2= k(k + 1)^2\alpha^2,$$
$$(k + 1)^2q = (k~+ 1)^2 \cdot k\alpha^2= k(k + 1)^2\alpha^2,$$
teda daná rovnosť skutočne platí.

Nech naopak pre reálne čísla $p$, $q$ a $k \neq -1$ platí $kp^2 = (k+1)^2q$. Uvažujme dvojicu
reálnych čísel
$$x_1 =\frac{-kp}{k + 1} \ \ \ \ \text{a} \ \ \ \  x_2 =\frac{-p}{k + 1}.$$
Také čísla (pre ktoré platí $x_1 = kx_2$) sú koreňmi danej kvadratickej rovnice, ak spĺňajú obe rovnosti~\ref{eq:B62II1}. Overenie urobíme dosadením:
\begin{align*}
x_1 + x_2 &= \frac{-kp}{k + 1}+\frac{-p}{k + 1}=\frac{-(k + 1)p}{k + 1}= -p,\\
x_1 x_2 &=\frac{-kp}{k + 1}\cdot\frac{-p}{k + 1}=\frac{kp^2}{(k + 1)^2}=\frac{(k + 1)^2q}{(k + 1)^2}= q.
\end{align*}
Tým je celý dôkaz hotový.\\
\\
}



%\kom Je dôležité, aby si študenti uvedomili, že v tomto prípade dokazujú ekvivalenciu, teda je potrebné dokázať obe implikácie a dávať si pozor na, čo sú predpoklady, a čo je tvrdenie, ktoré sa snažíme dokázať. Prvá časť riešenia je relatívne priamočiara, v druhej \todo{XXX}.


% Do not delete this line (pandoc magic!)

\problem{B-59-I-6}{
Reálne čísla $a$, $b$ majú túto vlastnosť: rovnica $x^2 -ax+b-1 = 0$ má v~množine reálnych čísel dva rôzne korene, ktorých rozdiel je kladným koreňom rovnice $x^2 - ax + b + 1 = 0$.

a) Dokážte nerovnosť $b > 3$.

b) Pomocou $b$ vyjadrite korene oboch rovníc.
}{
\rieh  Označme $x_1$ menší a $x_2$ väčší koreň prvej rovnice. Potom platí $x_1 + x_2 = a$, $x_1 x_2 = b - 1$. Druhá rovnica má koreň $x_2 - x_1$, a keďže súčet oboch koreňov je $a$, musí byť druhý koreň $a - (x_2 - x_1 ) = x_1 + x_2 - x_2 + x_1 = 2x_1$. Súčin koreňov druhej rovnice je $(x_2 -x_1 )\cdot2x_1 = b+1$. Odtiaľ dostávame $b = -1+2x_1 x_2 -2x_1^2= -1+2(b-1)-2x_1^2$, a teda
$$b = 3 + 2x_1^3> 3, \ \ \ \  (1)$$
lebo z~rovnosti $x_1 = 0$ by vyplývalo $b + 1 = b - 1 = 0$.

Keďže $x_2 - x_1 > 0$ a $b + 1 > 0$, musí byť aj $x_1 > 0$; z~(1) máme $x_1 =\sqrt{(b - 3)/2}$ a ďalej
$$x_2 =\frac{b-1}{x_1}=\frac{(b - 1)\sqrt{2}}{\sqrt{b-3}}.$$
Korene druhej rovnice sú potom
$$x_2 - x_1 = \frac{b+1}{} \ \ \ \ \text{a} \ \ \ \  2x_1=\sqrt{2(b - 3)}.$$
\\
\textbf{Iné riešenie*.} Korene prvej rovnice sú
$$x_1 = \frac{a -\sqrt{a^2 - 4b + 4}}{2}, \ \ \ \  x_2 =\frac{a +\sqrt{a^2 - 4b + 4}}{2},$$
pričom pre diskriminant máme
$$D = a^2 - 4(b - 1) > 0. \ \ \ \  (2)$$
Rozdiel koreňov $x_2 - x_1 =\sqrt{a^2 - 4b + 4}$ je koreňom druhej rovnice, a preto
\begin{align*}
a^2 - 4b + 4 - a \sqrt{a^2 - 4b + 4} + b + 1 &= 0,\\
a^2 - 3b + 5 &= a\sqrt{a 2 - 4b + 4}, \ \ \ \  (3)\\
a^4 + 2a^2 (5 - 3b) + (3b - 5)^2 &= a^4 - 4a^2 b + 4a^2,\\
(3b - 5)^2 &= a^2 (2b - 6).
\end{align*}
Rovnosť  $a = 0$ nastáva práve vtedy, keď $3b - 5 = 0$; potom by ale neplatilo (2). Preto $a^2 > 0$, $(3b - 5)^2 > 0$, a teda aj $2b - 6 > 0$, čiže $b > 3$. Z~(2) a (3) potom vyplýva $a > 0$, a teda $a = (3b - 5)/\sqrt{2(b - 3)}$; ďalej potom
\begin{align*}
x_1 &=\frac{1}{2}\bigg( \frac{3a-5}{\sqrt{2(b-3)}}-\sqrt{\frac{(3b-5)^2}{2(b-3)}}-4b+4\bigg)=\sqrt{\frac{b-3}{2}},\\
x_2 &=\frac{1}{2} \bigg( \frac{3a-5}{\sqrt{2(b-3)}}+\sqrt{\frac{(3b-5)^2}{2(b-3)}}-4b+4\bigg)=\frac{(b-1)\sqrt{2}}{\sqrt{b-3}}.
\end{align*}
Druhá rovnica má korene
\begin{align*}
x_3 &=\frac{a-\sqrt{a^2-4b-4}}{2}=\frac{b+1}{\sqrt{2(b-3)}}=x_2-x_1\\
x_4 &=\frac{a+\sqrt{a^2-4b-4}}{2}=\sqrt{2(b-3)}.
\end{align*}
}


\kom Úloha sa dá vyriešiť relatívne \uv{netrikovo} vyjadrením koreňov prvej rovnice, dosadením ich rozdielu do druhej rovnice a odpovedajúcou diskusiou. Takýto prístup je síce zrozumiteľný, avšak dosť pracný. Ak študenti neprídu na prvý spôsob riešenia, považujeme za vhodné im ho ukázať ako dobrý príklad toho, ako nám použitie Viètovych vzorcov môže výraznej zjednodušiť výpočet.

% Do not delete this line (pandoc magic!)

\problem{B-64-II-4}{seminar30,kvadr,hra,krajskekolo}{
Na tabuli je zoznam čísel $1, 2, 3, 4, 5, 6$ a \uv{rovnica}
$$\frac{\fbox{$\phantom{7}$}}{\fbox{$\phantom{7}$}}x^2+\frac{\fbox{$\phantom{7}$}}{\fbox{$\phantom{7}$}}x + \frac{\fbox{$\phantom{7}$}}{\fbox{$\phantom{7}$}}= 0.$$
Marek s~Tomášom hrajú nasledujúcu hru. Najskôr Marek vyberie ľubovoľné číslo zo zoznamu, napíše ho do jedného z~prázdnych políčok v~\uv{rovnici} a číslo zo zoznamu zotrie. Potom Tomáš vyberie niektoré zo zvyšných čísel, napíše ho do iného prázdneho políčka a v~zozname ho zotrie. Nato Marek urobí to isté a nakoniec Tomáš doplní tri zvyšné čísla na tri zvyšné voľné políčka v~\uv{rovnici}. Marek vyhrá, ak vzniknutá kvadratická rovnica s~racionálnymi koeficientmi bude mať dva rôzne reálne korene, inak vyhrá Tomáš. Rozhodnite, ktorý z~hráčov môže vyhrať nezávisle na postupe druhého
hráča.
}{
\rieh Označme $a$, $b$, $c$ koeficienty výslednej rovnice $ax^2 + bx + c = 0$. Tá má dva rôzne reálne korene práve vtedy, keď je jej diskriminant (v~symbolickej podobe)
$$b^2 - 4ac =\bigg( \frac{{\fbox{$\phantom{7}$}}}{{\fbox{$\phantom{7}$}}} \bigg)^2-4\bigg( \frac{{\fbox{$\phantom{7}$}}}{{\fbox{$\phantom{7}$}}}\bigg) \bigg(\frac{{\fbox{$\phantom{7}$}}}{{\fbox{$\phantom{7}$}}}\bigg)$$
kladný.

Ukážeme, že vyhrávajúcu stratégiu má Marek. Najskôr do menovateľa zlomku pre koeficient $b$ napíše $1$.
\begin{enumerate}[a)]
\item Ak Tomáš obsadí vo svojom prvom ťahu iné miesto ako v~čitateli $b$, napíše do neho Marek v~nasledujúcom ťahu najväčšie zostávajúce číslo zo zoznamu (teda 5 alebo 6). Hodnota $b^2$ potom bude aspoň 25 a zo zvyšných čísel možno zostaviť výraz $4ac$ s~hodnotou nanajvýš $4\cdot  \frac{6\cdot4}{3\cdot2}= 16$. Diskriminant vzniknutej kvadratickej rovnice tak bude určite kladný.
\item Predpokladajme, že Tomáš vo svojom ťahu doplní čitateľa $b$. Marek potom v~druhom ťahu napíše najmenšie zostávajúce číslo zo zoznamu (2 alebo 3) do čitateľa $a$ (alebo $c$).
\begin{enumerate}[(i)]
\item V~prípade, že Tomáš v~prvom ťahu napísal do čitateľa $b$ číslo 2, je hodnota $b^2$ rovná 4 a najväčšia možná hodnota $4ac$ (s~prihliadnutím na druhý Marekov ťah) je $4 \cdot \frac{3\cdot 6}{4\cdot 5}=\frac{18}{5}\leq  4$, teda diskriminant vzniknutej kvadratickej rovnice bude opäť kladný.
\item  V~prípade, že Tomáš v~prvom ťahu napísal do čitateľa $b$ iné číslo ako 2, je hodnota $b^2$ aspoň 9 a hodnota $4ac$ je nanajvýš $4 \cdot \frac{2\cdot 6}{3\cdot4} = 4$, takže diskriminant
vzniknutej kvadratickej rovnice bude aj v~tomto prípade kladný.

\end{enumerate}
\end{enumerate}
\textit{Záver.} V~danej hre môže vyhrať Marek nezávisle na ťahoch Tomáša. Jeho víťazná stratégia je opísaná vyššie.\\
\\
\kom Posledná úloha je zaujímavým spojením hľadania víťaznej stratégie a analýzy vlastností diskriminantu kvadratickej rovnice. Študentov necháme riešenie úlohy hľadať samostatne a potom ich vyzveme, aby stratégiu, ktorú našli, použili pri hre so spolužiakmi. Bude zaujímavé pozorovať, či nastane situácia, v ktorej aj neoptimálna stratégia zvíťazí.\\
\\
}


\kom Posledná úloha je zaujímavým spojením hľadania víťaznej stratégie a analýzy vlastností diskriminantu kvadratickej rovnice. Študentov necháme riešenie úlohy hľadať samostatne a potom ich vyzveme, aby stratégiu, ktorú našli, použili pri hre so spolužiakmi. Bude zaujímavé pozorovať, či nastane situácia, v ktorej aj neoptimálna stratégia zvíťazí.


\subsection*{Domáca práca}

% Do not delete this line (pandoc magic!)

\problem{B-57-S-2}{
Určte všetky dvojice $(a, b)$ reálnych čísel, pre ktoré majú rovnice
$$x^2 + (3a + b)x + 4a = 0, \ \ \ \  x^2 + (3b + a)x + 4b = 0$$
spoločný reálny koreň.
}{
\rieh Nech $x_0$ je spoločný koreň oboch rovníc. Potom platí
$$x_0^2+ (3a + b)x_0 + 4a = 0, \ \ \ \  x_0^2+ (3b + a)x_0 + 4b = 0.$$
Odčítaním týchto rovníc dostaneme $(2a-2b)x_0 +4(a-b) = 0$, odkiaľ po úprave získame $(a - b)(x_0 + 2) = 0$.

Rozoberieme dve možnosti:

Ak $a = b$, majú obidve dané rovnice rovnaký tvar $x^2 + 4ax + 4a = 0$. Aspoň jeden koreň (samozrejme spoločný) existuje práve vtedy, keď je diskriminant $16a^2-16a$ nezáporný, teda $a \in (-\infty, 0\rangle \cup \langle 1, \infty)$.

Ak $x_0 = -2$, dostaneme z~prvej aj z~druhej rovnice $4-2a-2b = 0$, teda $b = 2-a$. Dosadením do zadania dostaneme rovnice
$$x^2 + (2a + 2)x + 4a = 0, \ \ \ \ x^2 + (6-2a)x + 8-4a = 0,$$
ktoré majú pri ľubovoľnej hodnote parametra a spoločný koreň $-2$.

\textit{Záver.} Dané rovnice majú aspoň jeden spoločný koreň pre všetky dvojice $(a, a)$, kde $a \in (-\infty, 0\rangle \cup \langle 1, \infty)$, a pre všetky dvojice tvaru $(a, 2-a)$, kde $a$ je ľubovoľné.\\
\\
\ul{30.3} [57-II-1]  Uvažujme dve kvadratické rovnice
$$x^2-ax-b = 0,\ \ \ \  x^2-bx-a = 0$$
s~reálnymi parametrami $a$, $b$. Zistite, akú najmenšiu a akú najväčšiu hodnotu môže nadobudnúť súčet $a + b$, ak existuje práve jedno reálne číslo $x$, ktoré súčasne vyhovuje obom rovniciam. Určte ďalej všetky dvojice $(a, b)$ reálnych parametrov, pre ktoré tento súčet tieto hodnoty nadobúda.

\rieh Odčítaním oboch daných rovníc dostaneme rovnosť $(b-a)x+a-b = 0$, čiže $(b-a)(x-1) = 0$. Odtiaľ vyplýva, že $b = a$ alebo $x = 1$.

Ak $b = a$, majú obidve rovnice tvar $x^2-ax-a = 0$. Práve jedno riešenie existuje práve vtedy, keď diskriminant $a^2 + 4a$ je nulový. To platí pre $a = 0$ a pre $a = -4$. Pretože $b = a$, má súčet $a + b$ v~prvom prípade hodnotu $0$ a v~druhom prípade hodnotu $-8$.

Ak $x = 1$, dostaneme z~daných rovníc $a + b = 1$, teda $b = 1-a$. Rovnice potom majú tvar
$$x^2-ax + a-1 = 0 \ \ \ \ \text{a} \ \ \ \ x^2 + (a-1)x-a = 0.$$
Prvá má korene $1$ a $a-1$, druhá má korene $1$ a $-a$. Práve jedno spoločné riešenie tak dostaneme vždy s~výnimkou prípadu, keď $a-1 = -a$, čiže $a = \frac{1}{2}$ -- vtedy sú spoločné riešenia dve.

\textit{Záver.} Najmenšia hodnota súčtu $a + b$ je $-8$ a je dosiahnutá pre $a = b = -4$. Najväčšia hodnota súčtu $a + b$ je $1$; túto hodnotu má súčet $a + b$ pre všetky dvojice $(a, 1-a)$, kde $a\neq \frac{1}{2}$ je ľubovoľné reálne číslo.\\
\\
}


% Do not delete this line (pandoc magic!)

\problem{B-59-S-1}{}{
Určte všetky hodnoty reálnych parametrov $p, q$, pre ktoré má každá z~rovníc
$$x(x - p) = 3 + q, \ \ \ \ x(x + p) = 3 - q$$
v~obore reálnych čísel dva rôzne korene, ktorých aritmetický priemer je jedným z~koreňov
zvyšnej rovnice.
}{
\rieh Z~Viètových vzťahov pre korene kvadratickej rovnice (ktoré vyplývajú z~rozkladu daného kvadratického trojčlena na súčin koreňových činiteľov) ľahko zistíme, že súčet koreňov prvej rovnice je $p$, takže ich aritmetický priemer je $\frac{1}{2}p$. Toto číslo má byť
koreňom druhej rovnice, preto
\begin{equation} \label{eq:B59S1_1}
    \frac{p}{2}\cdot \frac{3p}{2}= 3 - q.
\end{equation}
Podobne súčet koreňov druhej rovnice je $-p$, ich aritmetický priemer je $-\frac{1}{2}p$, a preto
\begin{equation} \label{eq:B59S1_2}
    -\frac{p}{2}\cdot \bigg(- \frac{3p}{2}\bigg)= 3 + q.
\end{equation}
Porovnaním oboch vzťahov~\ref{eq:B59S1_1} a~\ref{eq:B59S1_2} máme $3 - q = 3 + q$, čiže $q = 0$ a z~\ref{eq:B59S1_1} potom vyjde $p = 2$ alebo $p = -2$.

Z~oboch nájdených riešení dostaneme tú istú dvojicu rovníc $x(x - 2) = 3$, $x(x + 2) = 3$. Korene prvej z~nich sú čísla $-1$ a $3$, ich aritmetický priemer je $1$. Korene druhej rovnice sú čísla $1$ a $-3$, ich aritmetický priemer je $-1$.
}



%\teachernote{
%\subsection*{Doplňujúce zdroje a materiály}



%}