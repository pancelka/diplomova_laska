\problem{62-I-2-N2, upravené}{
Škriatok sa pohybuje v~tabuľke $10 \times 15$ skokmi o~jedno políčko nahor alebo o~jedno políčko doprava. Koľkými rôznymi cestami sa môže dostať z~ľavého dolného do pravého horného políčka?\
}{
\rieh Škriatok urobí 9 skokov nahor a 14 skokov doprava. Jeho cestu určíme, keď v~poradí všetkých 23 skokov vyberieme tých deväť, ktoré povedú nahor. Počet týchto výberov 9 prvkov z~daných 23 je rovný zlomku $\frac{23 \cdot 22 \cdots 16 \cdot 15}{9 \cdot 8 \cdots2\cdot 1}$, teda číslu $817 190$.\\
\\
\kom Ako sme už spomínali, táto úloha je tiež prípravou na domácu prácu. Je tiež vhodným miestom, kde môžeme prípadným tápajúcim študentom pripomenúť metódu riešenia, s ktorou sme sa už stretli: pokúsiť sa vypozorovať, ako sa úloha správa pre menšie rozmery, napr. tabuľku $3\times 3$ a potom objavené výsledky zovšeobecniť. \\
\\
}
