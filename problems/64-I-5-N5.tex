% Do not delete this line (pandoc magic!)

\problem{64-I-5-N5}{seminar08,nsdnsn,domacekolo,navodna}{
Ak majú prirodzené čísla $a, b$ najväčšieho spoločného deliteľa $d$, majú rovnakého najväčšieho spoločného deliteľa aj čísla $a$, $b$, $a - b$, $a + b$. Dokážte. Platí rovnaké tvrdenie pre najmenší spoločný násobok?
}{
\rie Najväčší spoločný deliteľ týchto štyroch čísel nebude určite väčší ako $d$ (ak by bol, potom by $d$ nebol najväčší spoločný deliteľ čísel $a$ a $b$, čo by bolo v~spore s~predpokladom úlohy). Stačí teda ukázať, že $d$ delí $a+b$ a $a-b$. Ak zapíšeme $a$ a $b$ v~tvare $a=ud$ a $b=vd$, pričom pre prirodzené čísla $u$, $v$ platí $(u,v)=1$, bude potom $a+b=ud+vd=(u+v)d$, $a-b=ud-vd=(u-v)d$. Vidíme, že $d$ delí súčet aj rozdiel čísel $a$ a $b$, tvrdenie je teda dokázané.

Tvrdenie pre najmenší spoločný násobok neplatí, uvedieme protipríklad. Pre čísla $a=12$, $b=8$, $a+b=20$, $a-b=4$, $[12,8]=24$, avšak $[12,8,20,4]=120$.\\
\\
\kom Úloha precvičuje dôkaz všeobecného tvrdenia a opäť prináša protipríklad ako dostatočný argument.\\
\\
}
