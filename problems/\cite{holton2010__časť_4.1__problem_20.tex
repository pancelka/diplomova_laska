\problem{\cite{holton2010}, časť 4.1, problem 20}{
Nájdite všetky prirodzené dvojciferné čísla, ktoré sa rovnajú dvojnásobku súčinu svojich cifier.
}{
\rie Nech je naše hľadané číslo $\overline{ab}$. Zadanie potom môžeme napísať ako $10a+b=2ab$, pričom $a\neq 0$ (inak by hľadané číslo nebolo dvojciferné). Taktiež $b\neq 0$, pretože $10a+b=2ab>0$. Keďže čísla $10a$ a $2ab$ sú párne, musí byť párne aj $b$, teda $b=2k$, $k \in \{1,2,3,4\}$. Využitím tohto poznatku môžeme zadanie úlohy upraviť na tvar $10a=b(2a-1)=2k(2a-1)$, teda $5a=k(2a-1)$. Preto buď $5\mid k$ alebo $5\mid (2a-1)$. Keďže je ale $k\in \{0, 1, 2, 3, 4\}$, platí $5\nmid k$ a teda určite $5 \mid (2a-1)$. Jediné možnosti, ktoré túto podmienku spĺňajú, sú $a=3$ alebo $a=8$. Ak je $a=8$, dostávame zo zadania $10\cdot 8 +b = 2\cdot 8 \cdot b$, teda $b=16/3$, čo ale nie je riešením, keďže $b$ musí byť celé číslo. Preto ostáva len možnosť $a=3$, z~ktorej potom dostávame $b=6$. Skúškou overíme, že nájdené číslo 36 vyhovuje zadaniu.\\
\\
\kom Úloha má krátke a jednoduché zadanie, jej riešenie však vyžaduje uplatnenie znalostí o~deliteľnosti aj zápis čísla v~rozvinutom tvare. \\
\\
%\ul{8.3} [59-I-6-N2 resp. 45-Z7-I-2] Nájdite všetky čísla od 1 do $1 000 000$, ktoré sa po škrtnutí prvej cifry $73$-krát zmenšia.\\
%\rie Blabla.\\
}
