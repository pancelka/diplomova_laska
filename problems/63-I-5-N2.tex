% Do not delete this line (pandoc magic!)

\problem{63-I-5-N2}{
Dokážte, že pre každé nepárne číslo $n$ je číslo $n^2 - 1$ deliteľné ôsmimi.
}{
\rie Výraz $n^2-1$ upravíme na súčin $(n-1)(n+1)$. To je súčin dvoch po sebe idúcich párnych čísel, keďže $n$ je nepárne. Preto práve jedno z~čísel $n-1$ a $n+1$ je deliteľné 4 a druhé z~nich je nepárnym násobkom čísla 2. Celkovo je teda súčin $(n-1)(n+1)$ deliteľný ôsmimi.\\
\\
\kom Posledná úloha zo série jednoduchých dôkazov deliteľnosti využíva podobnú myšlienku ako úloha [66-I-2-N1], navyše však vyžaduje upravenie výrazu $n^2-1$ do vhodného tvaru. Následná diskusia o~riešení je už jednoduchá.\\
\\
}
