% Do not delete this line (pandoc magic!)

\problem{64-I-3-N3}{seminar25,mriezsach,hra,domacekolo,navodna}{
Simona a Lenka hrajú hru. Pre dané celé číslo $k$ také, že $0 \leq k~\leq 9$, vyberie Simona $k$ políčok šachovnice $3 \times 3$ a na každé z~nich napíše číslo 1, na ostatné políčka napíše číslo 0. Lenka potom šachovnicu nejakým spôsobom pokryje tromi triminovými kockami,  t.\,j. kockami tvaru $3\times1$, a čísla pod ich políčkami vynásobí. Ak je počet kociek so súčinom 0 nepárny, vyhráva Simona, v~ostatných prípadoch vyhráva Lenka. Určte, v~koľkých percentách prípadov (vzhľadom na hodnotu $k$) má vyhrávajúcu stratégiu Simona.
}{
\rie Ukážeme, že víťaznú stratégiu má pre všetky $k$ okrem 7 a 9 Simona. Ak má Simona vyhrať, musí 1 do políčok šachovnice umiestňovať tak, aby v každom riadku a každom stĺpci nechala priestor na aspoň jednu 0.  Tým zaručí, že akokoľvek potom Lenka umiestni triminové kocky, každá z nich bude obsahovať aspoň jednu 0.  Keďže spolu máme 10 možných hodnôt $k$ a pre 8 z~nich má Simona víťaznú stratégiu, vyhrá v 80\,\% prípadov.\\
\\
\kom Zaujímavé bude sledovať, ako efektívne sa budú študenti schopní zhostiť úlohy. Keďže má úloha jednoznačný číselný výsledok, môžeme po chvíli samostatnej práce nechať študentov porovnať svoje výsledky a pokúsiť zistiť pôvod prípadných nezrovnalostí. \\
\\
}
