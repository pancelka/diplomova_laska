\seminar{20}{Teória čísel IV -- prvočísla}

\teachernote{
\subsection*{Ciele}
Precvičiť so študentmi rôzne úlohy o~prvočíslach, pri riešení ktorých sa uplatnia poznatky o~deliteľnosti nadobudnuté v~seminároch 7 a 8.

}
\subsection*{Úlohy a riešenia}

% Do not delete this line (pandoc magic!)

\problem{63-I-3-N2}{
Číslo $n$ je súčinom dvoch rôznych prvočísel. Ak zväčšíme menšie z~nich o~1 a druhé ponecháme, ich súčin sa zväčší o~7. Určte číslo $n$.
}{
\rie Označme $p<q$ prvočísla zo zadania. Potom platí $(p+1)q=pq+7$. Po roznásobení ľavej strany a odčítaní výrazu $pq$ od oboch strán rovnosti dostávame $q=7$. Prvočíslo $p$ má byť menšie ako $q$, preto $p\in \{2,3,5\}$ a hľadaným číslom $n$ je tak jedno z~čísel 14, 21 alebo 35.\\
\\
}


% Do not delete this line (pandoc magic!)

\problem{63-I-3-N4}{}{
Číslo $n$ je súčinom dvoch prvočísel. Ak zväčšíme každé z~nich o~1, ich súčin sa zväčší o~35. Určte číslo $n$.
}{
\rie Podobne ako v~predchádzajúcom prípade označme $p\leq q$ (nie nutne rôzne) prvočísla zo zadania a to prepíšme do tvaru rovnosti $(p+1)(q+1)=pq+35$. Po úprave dostávame $p+q=34$. Hľadáme teda dvojice prvočísel, ktorých súčet bude 34. Takými sú jedine 3 a 31, 5~a 29, 11 a 23, 17 a 17. Riešením úlohy je potom $n \in \{93, 145, 253, 289\}$.\\
\\
\kom Úvodné dve jednoduché úlohy majú prípravný charakter na úlohu nasledujúcu a sú skôr rozcvičkou, než náročnou aplikáciou vedomostí o~prvočíslach.\\
\\
}


\problem{63-I-3}{
Číslo $n$ je súčinom troch rôznych prvočísel. Ak zväčšíme dve menšie z~nich o~1 a najväčšie ponecháme nezmenené, zväčší sa ich súčin o~915. Určte číslo $n$.
}{
\rieh Nech $n = pqr, p < q < r$. Rovnosť $(p + 1)(q + 1)r = pqr + 915$ ekvivalentne upravíme na tvar $(p + q + 1) \cdot r = 915 = 3 \cdot 5 \cdot 61$, z~ktorého vyplýva, že prvočíslo $r$ môže nadobudnúť len niektorú z~hodnôt 3, 5 a 61. Pre $r = 3$ ale z~poslednej rovnice dostávame $(p + q + 1) \cdot 3 = 3 \cdot 5 \cdot 61$, čiže $p + q = 304$. To je spor s~tým, že $r$ je najväčšie. Analogicky zistíme, že nemôže byť ani $r = 5$. Je teda $r = 61$ a $p + q = 14$. Vyskúšaním všetkých možností pre $p$ a $q$ vyjde $p = 3$, $q = 11$, $r = 61$ a $n = 3 \cdot 11 \cdot 61 = 2 013$.\\
\\
\kom Úloha vyžaduje vhodnú manipuláciu rovnosti zo zadania a potom už len dostatočne pozornú analýzu vzniknutých možností.\\
\\
}


% Do not delete this line (pandoc magic!)

\problem{64-S-3}{seminar20,prvocisla,skolskekolo}{
Nájdite najmenšie prirodzené číslo $n$ s~ciferným súčtom 8, ktoré sa rovná súčinu troch rôznych prvočísel, pričom rozdiel dvoch najmenších z~nich je 8.
}{
\rieh Hľadané číslo $n$ je súčinom troch rôznych prvočísel, ktoré označíme $p, q, r$, $p < q < r$. Číslo $n = pqr$ má ciferný súčet 8, ktorý nie je deliteľný tromi, preto ani $n$ nie je deliteľné tromi a teda $p, q, r \neq 3$. Napokon hľadané číslo $n$ nie je deliteľné ani dvoma, pretože by muselo byť $p = 2$ a $q = p + 8 = 10$, čo nie je prvočíslo. Musí teda byť $p = 5$.

Ak je $p = 5$, je $q = p + 8 = 13$, takže $r \in \{17, 19, 23, 29, 31,\,\ldots \}$ a $n \in \{1 105,1 235, 1 495,$ $1 885, 2 015,\,\ldots\}$. V~tejto množine je zrejme najmenšie číslo s~ciferným súčtom 8 číslo $2 015$.

Ak je $p > 5$, je $p = 11$ najmenšie prvočíslo také, že aj $q = p + 8$ je prvočíslo. Preto $p = 11$, $q = 19$, a teda $r = 23$, takže pre zodpovedajúce čísla $n$ platí $n = 11 \cdot 19 \cdot 23= 4 807 > 2 015$.\\
\\
\kom Úloha príjemne spája poznatky o~deliteľnosti a prvočíslach a nemala by pre študentov byť neprekonateľnou výzvou.\\
\\
}


% Do not delete this line (pandoc magic!)

\problem{57-S-1}{seminar20,prvocisla,skolskekolo}{
Nájdite všetky dvojice prirodzených čísel $a, b$ väčších ako 1 tak, aby ich súčet aj súčin boli mocniny prvočísel.
}{
\rieh Z~podmienky pre súčin vyplýva, že $a$ aj $b$ sú mocninami toho istého prvočísla $p$: $a = p^r$, $b = p^s$, pričom $r, s$ sú celé kladné čísla. Keby bolo $p$ nepárne, bol by súčet $a + b$ deliteľný okrem čísla $p$ aj číslom 2, takže by nebol mocninou prvočísla. Teda $p = 2$. Ak $r < s$, je súčet $a + b = 2^r (1 + 2^{s-r})$ opäť číslo párne deliteľné nepárnym číslom väčším ako 1, nie je teda mocninou prvočísla. K~rovnakému záveru dôjdeme aj v~prípade, keď $r > s$. Ostáva preto jediná možnosť: $a = b = 2^r$ , pričom $r$ je celé kladné číslo. Skúška $a+b = 2^r +2^r = 2^{r+1}$ a $ab = 2^{2r}$ potvrdzuje, že riešením sú všetky dvojice $(a, b) = (2^r, 2^r)$, kde $r$ je celé kladné číslo.\\
\\
}


% Do not delete this line (pandoc magic!)

\problem{65-I-1-D2, resp. 55-II-4}{seminar20,prvocisla}{
Nájdite všetky dvojice prvočísel $p$ a $q$, pre ktoré platí $p + q^2= q + 145p^2$.
}{
\rieh Pre prvočísla $p, q$ má platiť $q(q - 1) = p(145p -1)$, takže prvočíslo $p$ delí $q(q -1)$. Prvočíslo $p$ nemôže deliť prvočíslo $q$, pretože to by znamenalo, že $p = q$, a teda $145p = p$, čo nie je možné. Preto $p$ delí $q-1$,  t.\,j. $q - 1 = kp$ pre nejaké prirodzené $k$. Po dosadení do daného vzťahu dostaneme podmienku $$p=\frac{k+1}{145-k^2}.$$ Vidíme, že menovateľ zlomku na pravej strane je kladný jedine pre $k \leq 12$, zároveň však pre $k \leq 11$ je jeho čitateľ menší ako menovateľ: $k + 1 \leq 12 < 24 \leq 145 k^2$. Iba pre $k = 12$ tak vyjde $p$ prirodzené a prvočíslo, $p = 13$. Potom $q = 157$, čo je tiež prvočíslo. Úloha má jediné riešenie.\\
\\
\kom Úloha opäť ukazuje, že upravenie podmienok zo zadania do vhodného tvaru, o~ktorom môžeme ďalej diskutovať, je často kľúčovým krokom v~riešení. V~tomto prípade ide o~podmienku $q=kp+1$ a následný rozbor hodnôt v~čitateli a menovateli zlomku. To by v~študentoch malo umocniť dojem, že zručné narábanie s~algebraickými výrazmi nájde svoje široké uplatnenie.\\
\\
}


% Do not delete this line (pandoc magic!)

\problem{62-I-5}{}{
Určte všetky celé čísla $n$, pre ktoré $2n^3 -3n^2 +n+3$ je prvočíslo.
}{
\rieh Ukážeme, že jedinými celými číslami, ktoré vyhovujú úlohe, sú $n = 0$ a $n = 1$.

Upravme najskôr výraz $V = 2n^3 - 3n^2 + n + 3$ nasledujúcim spôsobom:
$$V = (n^3 - 3n^2+ 2n) + (n^3 - n) + 3 = (n - 2)(n - 1)n + (n - 1)n(n + 1) + 3.$$
Oba súčiny $(n-2)(n-1)n$ a $(n-1)n(n+1)$ v upravenom výraze $V$ sú deliteľné tromi pre každé celé číslo $n$ (v oboch prípadoch sa jedná o súčin troch po sebe idúcich celých čísel), takže výraz $V$ je pre všetky celé čísla $n$ deliteľný tromi. Hodnota výrazu $V$ je preto prvočíslom práve vtedy, keď $V = 3$, teda práve vtedy, keď súčet oboch spomenutých súčinov je rovný nule:
$$0 = (n - 2)(n - 1)n + (n - 1)n(n + 1) = n(n - 1)[(n - 2) + (n + 1)] = n(n - 1)(2n - 1).$$
Poslednú podmienku však spĺňajú iba dve celé čísla $n$, a to $n = 0$ a $n = 1$. Tým je úloha vyriešená.
 
\textit{Poznámka.} Fakt, že výraz $V$ je deliteľný tromi pre ľubovoľné celé $n$, môžeme odvodiť aj tak, že doňho postupne dosadíme $n = 3k$, $n = 3k + 1$ a $n = 3k + 2$, pričom $k$ je celé číslo, rozdelíme teda všetky celé čísla $n$ na tri skupiny podľa toho, aký dávajú zvyšok po delení tromi.\\
\\
\kom Aj keď vzorové riešenie môže vyzerať trikovo, po vyskúšaní niekoľko málo hodnôt $n$ je vždy hodnota zo zadania deliteľná 3, čo by študentov mohlo priviesť na myšlienku skúsiť dokázať deliteľnosť čísla zo zadania tromi.


\textit{Poznámka.} Fakt, že výraz $V$ je deliteľný tromi pre ľubovoľné celé $n$, môžeme odvodiť aj tak, že doňho postupne dosadíme $n = 3k$, $n = 3k + 1$ a $n = 3k + 2$, pričom $k$ je celé číslo, rozdelíme teda všetky celé čísla $n$ na tri skupiny podľa toho, aký dávajú zvyšok po delení tromi.
}


\kom Aj keď vzorové riešenie môže vyzerať trikovo, po vyskúšaní niekoľko málo hodnôt $n$ je vždy hodnota zo zadania deliteľná 3, čo by študentov mohlo priviesť na myšlienku skúsiť dokázať deliteľnosť čísla zo zadania tromi.

% Do not delete this line (pandoc magic!)

\problem{MŘMUI TODO, 2.3, str 174}{
Nájdite všetky prvočísla, ktoré sú súčasne súčtom a rozdielom dvoch vhodných prvočísel.
}{
\rieh Predpokladajme, že prvočíslo $p$ je súčasne súčtom aj rozdielom dvoch prvočísel. Potom je však $p>2$ a teda je $p$ nepárne. Pretože je $p$ zároveň súčet aj rozdiel dvoch prvočísel, jedno z~nich musí byť vždy párne, teda 2. Takže hľadáme prvočísla $p, p_1, p_2$ tak, že $p=p_1+2=p_2-2$, teda $p_1, p, p-2$ sú tri po sebe idúce nepárne čísla a teda práve jedno z~nich je deliteľné troma (študenti by si mali rozmyslieť prečo). Avšak troma je deliteľné jediné prvočíslo 3, odkiaľ vzhľadom na to, že $p_1\geq 1$ vyplýva $p_1=3$, $p=5$ a $p_2=7$. Jediné prvočíslo vyhovujúce zadaniu je teda $p=5$.\\
\\
\kom Úloha, ktorá vyžaduje viac uvažovania, než tvrdého počítania, je zaujímavá práve jediným výsledkom.\\
\\
}


% Do not delete this line (pandoc magic!)

\problem{str. 95}{
{thiele1986} Nájdite celočíselné riešenia rovnice $$\frac{1}{x}+\frac{1}{y}=\frac{1}{p},$$ kde $p$ je pevne dané prvočíslo.
}{
\rieh Ak existujú vôbec nejaké riešenia vyšetrovanej rovnice, potom sú nenulové. Preto môžeme rovnicu upraviť na ekvivalentný tvar $yx-px-py=0$, resp. $(x-p)(y-p)-p^2=0$, a teda $$(x-p)(y-p)=p^2.$$ Odtiaľ je vidieť, že celočíselné riešenia môžeme dostať len vtedy, ak $x-p$ prebehne všetkých deliteľov čísla $p^2$, pričom $y-p$ prebehne doplnkové delitele. Pretože je $p$ prvočíslo, musí byť nutne $$x-p \in \{1, p, p^2, -1, -p, -p^2\}.$$ Pretože $x\neq 0$, odpadá $x-p=-p$. Ostáva teda $$x \in \{1+p, 2p, p+p^2, p-1, p-p^2\} \ \ \ \ \text{a teda} \ \ \ \ y \in \{p+p^2, 2p, 1+p, p-p^2, p-1\}.$$ Tieto hodnoty sú skutočne riešením, o~čom sa môžeme presvedčiť skúškou.\\
\\
\kom Úloha, v~ktorej opäť predtým, než uplatníme znalosti o~deliteľnosti, príp. prvočíslach, musíme umne upraviť východiskový tvar rovnice.
}



\home{



% Do not delete this line (pandoc magic!)

\problem{65-I-1}{seminar20,prvocisla,domacekolo}{
Nájdite všetky možné hodnoty súčinu prvočísel $p$, $q$, $r$, pre ktoré platí
$$p^2 - (q + r)^2= 637.$$
}{
\rieh Ľavú stranu danej rovnice rozložíme na súčin podľa vzorca pre $A^2 - B^2$. V~takto upravenej rovnici
$$(p + q + r)(p - q - r) = 637$$
už ľahko rozoberieme všetky možnosti pre dva celočíselné činitele naľavo. Prvý z~nich je väčší a kladný, preto aj druhý musí byť kladný (lebo taký je ich súčin), takže podľa rozkladu na súčin prvočísel čísla $637 = 7^2 \cdot 13$ ide o~jednu z~dvojíc $(637, 1)$, $(91, 7)$
alebo $(49, 13)$. Prvočíslo $p$ je zrejme aritmetickým priemerom oboch činiteľov, takže sa musí rovnať jednému z~čísel $\frac{1}{2}(637 + 1) = 319$, $\frac{1}{2}(91 + 7) = 49$, $\frac{1}{2}(49 + 13) = 31$. Prvé dve z~nich však prvočísla nie sú ($319 = 11 \cdot  29$ a $49 = 7^2$), tretie áno. Takže nutne $p = 31$ a prislúchajúce rovnosti $31 + q + r = 49$ a $31 - q - r = 13$ platia práve vtedy, keď $q + r = 18$. Také dvojice prvočísel $\{q, r\}$ sú iba $\{5, 13\}$ a $\{7, 11\}$ (stačí prebrať
všetky možnosti, alebo si uvedomiť, že jedno z~prvočísel $q$, $r$ musí byť aspoň $18 : 2 = 9$, nanajvýš však $18 - 2 = 16$). Súčin $pqr$ tak má práve dve možné hodnoty, a to $31 \cdot  5\cdot  13 = 2 015$ a $31 \cdot  7 \cdot  11 = 2 387$.\\
}

}

\teachernote{
\subsection*{Doplňujúce zdroje a materiály}
Ďalšie zaujímavé príklady je možné nájsť v  \cite{herman2011}, paragraf 2 a taktiež v \cite{holton2010}. %alebo na \todo{[PP]}.
}

