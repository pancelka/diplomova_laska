% Do not delete this line (pandoc magic!)

\problem{60-I-1-N2}{seminar06,rovnice,domacekolo,navodna}{
Máme tri čísla, o~ktorých vieme, že každé z~nich je aritmetickým priemerom niektorých dvoch z~našich troch čísel. Dokážte, že naše tri čísla sú rovnaké.
}{
\rieh Predpokladajme, že niektoré z~našich čísel je priemerom seba a iného z~našich čísel. Potom ich vieme označiť $a, b, c$ tak, že $a = (a + b)/2$. Z~tejto rovnosti vyplýva $a = b$. Číslo $c$ je buď priemerom čísel $a$ a $b$, z~čoho hneď máme, že je týmto číslam rovné, alebo je priemerom seba a niektorého z~čísel $a, b$, čiže $c = (c + a)/2$, z~toho opäť dostaneme $c = a = b$. Ak každé z~našich čísel je aritmetickým priemerom zvyšných dvoch, riešime predošlú úlohu.\\
\\
}
