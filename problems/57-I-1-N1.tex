% Do not delete this line (pandoc magic!)

\problem{57-I-1-N1}{
Ak $m, k$ a $\sqrt[k]{m}$ sú celé čísla väčšie ako 1, tak v~rozklade čísla $m$ na súčin prvočísel sa každé prvočíslo vyskytuje v~mocnine, ktorej exponent je násobkom čísla $k$. Dokážte.
}{
\rieh Rozklad čísla $m$ dostaneme, keď rozklad čísla $\sqrt[k]{m}$ umocníme na $k$-tu, každý exponent v~rozklade čísla $m$ tak bude súčinom exponentu v~rozklade čísla $\sqrt[k]{m}$ a čísla $k$. \\ %Nech je rozklad čísla $\sqrt[k]{m}=p_1^{\alpha_1}\cdot p_2^{\alpha_2}\cdots p_n^{\alpha_n}$ a rozklad čísla $m=p_1^{\beta_1}\cdot p_2^{\beta_2}\cdots p_n^{\beta_n}$, kde $p_1< \ldots < p_n$ sú prvočísla\\
\\
\kom Úloha je prípravou k~riešeniu komplexnejšieho problému, ktorý nasleduje.\\
\\
}
