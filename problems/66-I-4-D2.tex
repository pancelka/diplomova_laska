% Do not delete this line (pandoc magic!)

\problem{66-I-4-D2}{seminar06,rovnice,mnohocleny,domacekolo}{
Koeficienty $a, b, c$ trojčlena $P (x) = ax^2+ bx + c$ sú reálne čísla, pritom každá z~troch jeho hodnôt $P (1), P (2)$ a $P (3)$ je celým číslom. Vyplýva z~toho, že aj čísla $a, b, c$ sú celé, alebo je nutne celé aspoň niektoré z~nich (ktoré)?
}{
\rieh Nevyplýva. Uvážme príklad trojčlena $P (x) =\frac{1}{2}x^2+\frac{1}{2}x+1$: z~vyjadrenia $P (x) =\frac{1}{2}x(x + 1) + 1$ vyplýva, že $P (x)$ je celým číslom pre každé celé $x$, pretože súčin $x(x + 1)$ je vtedy deliteľný dvoma. Vo všeobecnej situácii je iba koeficient $c$ nutne celé číslo; vyplýva to z~vyjadrenia $c = P (0) = 3P (1) - 3P (2) + P (3)$.\\
\\
\kom Úloha je zaujímavá tým, že cesta k~riešeniu je tentoraz menej priamočiara a študenti pravdepodobne prídu na viac rôznych príkladov mnohočlenov s~neceločíselnými koeficientami, ktoré dané podmienky spĺňajú. Zaujímavá bude tiež pravdepodobne diskusia nad zdôvodnením, ktoré z~koeficientov nutne celočíselné byť musia.\\
\\
}
