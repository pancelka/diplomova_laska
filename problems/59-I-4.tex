% Do not delete this line (pandoc magic!)

\problem{59-I-4}{
Kružnica $k(S; r)$ sa dotýka priamky $AB$ v~bode $A$. Kružnica $l(T; s)$ sa dotýka priamky $AB$ v~bode $B$ a pretína kružnicu k~v~krajných bodoch $C$, $D$ jej priemeru. Vyjadrite dĺžku a úsečky $AB$ pomocou polomerov $r$, $s$. Dokážte ďalej, že priesečník $M$ priamok $CD$, $AB$ je stredom úsečky $AB$.
}{
\rieh Keďže kružnica $l$ má ako tetivu priemer $CD$ kružnice $k$ a dané kružnice nie sú totožné, platí pre ich polomery nerovnosť $s > r$. Ak označíme $P$ pätu kolmice z~bodu $S$ na úsečku $BT$ (obr. 5), tak z~Pytagorovej vety pre pravouhlé trojuholníky
\begin{center}
\includegraphics{images/59D4\imagesuffix}\\

Obr. 5
\end{center}
$CST$ a $SPT$ vyplýva
$$|ST|^2 = s^2 - r^2\ \ \ \ \text{a} \ \ \ \  |ST|^2 = |SP|^2 + (s~-r)^2. \ \ \ \  (1)$$
Odtiaľ pre veľkosť úsečky $SP$ vychádza
$$|SP|^2 = (s^2 - r^2 ) - (s~- r)^2 = 2r(s - r).$$
A~keďže $ABPS$ je pravouholník, dostávame
$$|AB| = |SP| =\sqrt{2r(s - r)}.$$

Z~pravouhlých trojuholníkov $AMS$ a $MTS$ ďalej podľa prvej rovnosti v~(1) vyplýva
$$|AM|^2 = |SM|^2 - r^2 = |MT|^2- |ST|^2 - r^2 = |MT|^2 -s^2,$$
pritom z~pravouhlého trojuholníka $MBT$ máme
$$|BM|^2 = |MT|^2 - s^2.$$
Preto $|AM| = |BM|$ a bod $M$ je teda stredom úsečky $AB$.

\textit{Poznámka.} Záver, že $M$ je stredom úsečky $AB$, vyplýva okamžite aj z~mocnosti bodu $M$ k~obom kružniciam (bod $M$ leží na tzv. chordále oboch kružníc). Tieto pojmy sú však pre súťažiacich kategórie C zväčša neznáme a nebudú nutné ani pre riešenia ďalších súťažných kôl.\\
\\
}
