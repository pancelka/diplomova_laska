% Do not delete this line (pandoc magic!)

\problem{B-62-II-4}{
V rovine sú dané kružnice $m$, $n$, ktoré sa pretínajú v bodoch $K$, $L$. Dotyčnica v bode $K$ ku kružnici $m$ pretína kružnicu $n$ v bode $A \neq K$, dotyčnica v bode $L$ ku kružnici $n$ pretína kružnicu $m$ v bode $C \neq K$. Bod $B\neq L$ je priesečník priamky $AL$ s kružnicou $m$ a bod $D \neq K$ je priesečník priamky $CK$ s kružnicou $n$. Dokážte, že štvoruholník $ABCD$ je rovnobežník.
}{
\rieh Obvodový uhol $KAL$ a úsekový uhol $CLK$ tetivy $KL$ v kružnici $n$ sú zhodné. Podobne sa zhodujú aj obvodový uhol $KCL$ a úsekový uhol $AKL$ tetivy $KL$ v kružnici $m$ \todo{doplniť (obr. 1)}. Trojuholníky $AKL$ a $LCK$ sa tak zhodujú v dvoch vnútorných uhloch, a preto sa zhodujú aj v treťom uhle. Uhly $ALK$ a $LKC$  sú teda zhodné, a preto sú zhodné aj ich doplnky do 180$^\circ$, ktorými sú obvodové uhly $ADK$, resp.
$LBC$ v uvažovaných kružniciach. Zhodnosť uhlov $ALK$ a $LKC$ dokazuje rovnobežnosť priamok $AL$ a $CK$ (teda priamok $AB$ a $CD$), ktorá spolu so zhodnosťou uhlov $ADK$ a $LBC$ znamená, že aj priamky $AD$ a $BC$ sú rovnobežné. Štvoruholník $ABCD$ je teda rovnobežník, čo sme chceli dokázať.\\
\\
\todo{DOPLNIŤ Obr. 1}\\
\\
\textit{Poznámka.} Akonáhle pomocou zhodných uhlov $ALK$ a $LKC$ zistíme, že priamky $AB$ a $CD$ sú rovnobežné, môžeme konštatovať, že oba tetivové štvoruholníky $ADLK$ a $BLKC$ sú buď pravouholníky, alebo rovnoramenné lichobežníky so zhodnými uhlami pri základniach. V oboch prípadoch to už zrejme zaručuje rovnobežnosť druhej dvojice
priamok $AD$ a $BC$.
}
