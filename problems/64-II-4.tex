% Do not delete this line (pandoc magic!)

\problem{64-II-4}{
Hovoríme, že kladné reálne číslo je copaté, ak nie je prirodzené a vo svojom dekadickom zápise obsahuje za desatinnou čiarkou iba konečne veľa nenulových cifier.

a) Nájdite dve copaté čísla $a, b$ také, že $a \cdot b =2015$.

b) Rozhodnite, či existujú tri copaté čísla $a, b, c$ také, že čísla $a \cdot b$, $b \cdot c$ a $c \cdot a$ sú
všetky prirodzené.
}{
\rieh a) Takých dvojíc copatých čísel je nekonečne veľa. Je to napr. dvojica $$a = 2 015 \cdot \frac{5}{2}=5037,5, \ \ \ b =\frac{2}{}5= 0,4.$$

Podobne vyhovuje každá z nekonečne veľa dvojíc
$$a = 2 015 \cdot \frac{5^m}{2^n},\ \ \  b =\frac{2^n}{5^m},$$
pričom $m$ a $n$ sú ľubovoľné prirodzené čísla. Uvedené číslo $a$ má $n$ desatinných miest, číslo $b$ ich má $m$.

b) Taká trojica copatých čísel neexistuje. Každé copaté číslo, ktoré má za desatinnou čiarkou poslednú nenulovú cifru na $k$-tom mieste, t. j. na mieste rádu $10^{-k}$, môžeme pre vhodné prirodzené číslo s zapísať ako $s \cdot 10^{-k} (k \geq 1)$. Pritom $s$ nie je deliteľné desiatimi, môže teda byť deliteľné iba  jedným z prvočísel 2 alebo 5, a to ľubovoľnou jeho mocninou.

Súčinom dvoch copatých čísel $a = s/10^k$ a $b = t/10^l$ dostaneme prirodzené číslo iba vtedy, keď je súčin $st$ deliteľný $10^{k+l}$ , čiže keď jedno z čísel $s, t$ je deliteľné $2^k+l$ a druhé $5^k+l$, pričom $k + l = 2$. Ak sú teda $a = s/10^k$, $b = t/10^l$, $c = u/10^m$ ľubovoľné copaté čísla také, že súčiny $a \cdot b$ a $a \cdot c$ sú prirodzené čísla, je z predchádzajúcej úvahy zrejmé, že obe čísla $t$ aj $u$ musia byť buď obe nepárne a deliteľné piatimi, alebo naopak obe párne a nedeliteľné piatimi, takže ich súčin tu nemôže byť deliteľný desiatimi, teda súčin $bc = tu/10^{l+m}$ nemôže byť celý.
}