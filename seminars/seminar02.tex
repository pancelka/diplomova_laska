\seminar{2}

\subsection*{Téma}
Prístup k riešeniu matematických úloh, typy dôkazov

\teachernote{
\subsection*{Ciele}
Prediskutovať so študentmi rôzne prístupy k riešeniu neznámych problémov, zopakovať a/alebo zoznámiť s typmi dôkazov používaných v matematike.

\subsection*{Úlohy a riešenia}

\problem{ZM 43}{
V tramtárii chcú zaviesť novú základnú peňažnú jednotku: toliar. V spojitosti s reformou sa vynorila otázka: aké majú mať hodnoty mince, ktoré pripraví mincovňa? Starí obyvatelia Tramtárie považovali 3 a 5 za šťastné čísla. Preto popredný historik navrhol, aby sa obmedzili na razenie troj- a päťtoliarových mincí. Svoj návrh zdôvodňoval takto: troj- a päťtoliarovými mincami možno vyplatiť hociktorú celočíselnú sumu a to skoro vždy presne, bez vrátenia. Ten, kto peniaze dostáva, musí niečo vrátiť iba vtedy, ak je vyplácaná suma 1, 2, 4 alebo 7 toliarov. Bola by to pravda?
}{
}

\problem{ZM 44}{
K návrhu z predchádzajúcej úlohy došiel upravujúci návrh. Istý občan píše: Uznávam síce, že troj- a päťtoliarovými mincami možno opísaným spôsobom naozaj zaplatiť každú sumu, ale sú to primalé hodnoty. Navrhujem, aby sa razili radšej päť- a osemtoliarové mince. Je zrejmé -- zdôvodňoval ďalej --, že berúc do úvahy aj výdavok z peňazí, všetko, čo možno vyplatiť 3 a 5 toliarmi, možno vyplatiť aj 5 a 8 toliarmi. Tiež je zrejmé, že od istého čísla -- vlastne v prípade väčších súm toliarov -- ani tu netreba vrátiť mince. Je to skutočne také zrejmé?
}{
}

\problem{ZM 45}{
Diskusia, o ktorej sme v predchádzajúcich dvoch úlohách referovali, pokračovala aj ďalej. Boli aj takí, ktorí 5 a 8 toliarové mince považovali za primalé hodnoty a navrhovali väčšie. Návrhy na dve hodnoty toliarov boli napokon tieto:
\begin{enumerate}[a)]
\item 8 a 13, 
\item 21 a 34, 
\item 144 a 233.
\end{enumerate}
Autori všetkých troch návrhov tvrdili, že týmito dvojicami hodnôt možno vyplatiť hocikoľko (celočíselných) súm toliarov a keď vyplácaná suma je dosť veľká, tak netreba ani vrátiť peniaze. Tvrdenie ktorého z týchto autorov je správne a ktorého nie?
}{
}

\problem{ZM 46}{
Tri návrhy, ktoré sme uviedli v predchádzajúcej úlohe, neboli náhodné. Starí obyvatelia uctievali nielen čísla 3 a 5, ale aj všetky čísla Fibonacciho postupnosti. Pravdepodobne dospeli až k variáciám týchto čísel, lebo boli aj také návrhy, aby hodnoty dvoch mincí boli
\begin{enumerate}[a)]
\item 2 a 8, 
\item 3 a 21, 
\item 21 a 144 toliarov.
\end{enumerate}
Ktorý z týchto troch návrhov je vhodný na zaplatenie hocijakej (celočíselnej) sumy toliarov, dokonca tak, aby sa pri dostatočne veľkej sume nemuselo nič vrátiť?
}{
}

\problem{ZM 47}{
Diskusia o peňažnej reforme v Tramtárii nadobúdala čoraz väčšie rozmery. Mnohí jej účastníci tvrdili, že návrhy z úlohy 45 sa ukazujú byť správne, lebo rešpektujú aj poradie magických čísel, zatiaľ čo v úlohe 46 sú neúspešné, pretože toto pradie nerešpektovali. Ako skutočná bomba preto zapôsobil návrh, aby sa vydali mince v hodnote 4 a 11 toliarov. Ani jedna hodnota sa nenachádza medzi \uv{magickými} číslami! Napriek tomu autor tvrdil, že aj tieto hodnoty spĺňajú potrebné dve požiadavky: možno nimi vyplatiť hocijaké celočíselné množstvo toliarov a keď je táto suma dostatočne veľká, tak netreba ani niť vrátiť. Mal pravdu?
}{
}

\problem{ZM 48}{
Veľká diskusia však odrazu skončila (práve vtedy, keď bola najživšia), pretože Národná banka Tramtárie na veľké prekvapenie oznámila, že zavedie dve platidlá: tri a päťtoliarové mince. Koľkorakými spôsobom možno nimi vyplatiť 49 toliarov bez vrátenia peňazí?
}{
}

\kom{V prípade, že študentov úloha zaujala, je možné ich odkázať na publikáciu \todo{ZM}, ktorá obsahuje ďalšie zovšeobecnenie riešených úloh.}

\todo{DOPLNIŤ.}%Spoločne so študentami prečítame preriešime prvú kapitolu z~\cite{holton2010}
}
