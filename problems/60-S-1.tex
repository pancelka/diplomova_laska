\problem{60-S-1}{
Po okruhu behajú dvaja atléti, každý inou konštantnou rýchlosťou. Keď bežia opačnými smermi, stretávajú sa každých 10 minút, keď bežia rovnakým smerom, stretávajú sa každých 40 minút. Za aký čas zabehne okruh rýchlejší atlét?
}{
\rieh Označme rýchlosti bežcov $v_1$ a $v_2$ tak, že $v_1 > v_2$ (rýchlosti udávame v~okruhoch za minútu). Predstavme si, že atléti vyštartujú z~rovnakého miesta, ale opačným smerom. V~okamihu ich ďalšieho stretnutia po 10 minútach bude súčet dĺžok oboch prebehnutých úsekov zodpovedať presne dĺžke jedného okruhu, teda
$10v_1 +10v_2= 1$.
Ak bežia atléti z~rovnakého miesta rovnakým smerom, dôjde k~ďalšiemu stretnutiu, akonáhle rýchlejší atlét zabehne o~jeden okruh viac ako pomalší. Preto $40v_1 -40v_2 = 1$.
Dostali sme sústavu dvoch lineárnych rovníc s~neznámymi $v_1 , v_2$:
\begin{align*}
10v_1 + 10v_2 &= 1,\\
40v_1 - 40v_2 &= 1,
\end{align*}
ktorú vyriešime napríklad tak, že k~štvornásobku prvej rovnice pripočítame druhú, čím dostaneme $80v_1 = 5$, čiže $v_1 =\frac{1}{16}$. Zaujíma nás, ako dlho trvá rýchlejšiemu bežcovi prebehnúť jeden okruh, teda hodnota podielu $1/v_1$ . Po dosadení vypočítanej hodnoty $v_1$ dostaneme odpoveď: 16 minút.\\
\\
\textit{Poznámka.} Úlohu možno riešiť aj úvahou: za 40 minút ubehnú atléti spolu 4 okruhy (to vyplýva z~prvej podmienky), pritom rýchlejší o~1 okruh viac ako pomalší (to vyplýva z~druhej podmienky). To teda znamená, že prvý za uvedenú dobu ubehne 2,5 okruhu a druhý 1,5 okruhu, takže rýchlejší ubehne jeden okruh za 40/2,5 = 16 minút.\\
\\
\kom V~tomto prípade vyžaduje netriviálne úsilie správne zostavenie sústavy rovníc tak, aby skutočne zodpovedala zadaniu. Jej vyriešenie potom už zložité nie je. Za zmienku stojí, že úloha je vhodným príkladom situácie, v~ktorej si zmysluplnosť výsledku môžeme aspoň približne overiť (záporné rýchlosti, rýchlosti väčšie ako rýchlosť svetla).\\
\\
}
