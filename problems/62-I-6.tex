% Do not delete this line (pandoc magic!)

\problem{62-I-6}{}{
Vnútri pravidelného šesťuholníka $ABCDEF$ s~obsahom 30\,cm$^2$ je zvolený bod $M$. Obsahy trojuholníkov $ABM$ a $BCM$ sú postupne 3\,cm$^2$ a 2\,cm$^2$. Určte obsahy trojuholníkov $CDM$, $DEM$, $EFM$ a $FAM$.
}{
\rieh Úloha je o~obsahu šiestich trojuholníkov, na ktoré je daný pravidelný šesťuholník rozdelený spojnicami jeho vrcholov s~bodom $M$ (obr.~\ref{fig:62I6_1}). Celý šesťuholník s~daným obsahom, ktorý označíme $S$, možno rozdeliť na šesť rovnostranných trojuholníkov s~obsahom $S/6$ (obr.~\ref{fig:62I6_2}). Ak označíme $r$ ich stranu, $v$ vzdialenosť rovnobežiek $AB$, $CD$ a $v_1$ vzdialenosť bodu $M$ od priamky $AB$, dostaneme
$$S_{ABM} + S_{EDM} =\frac{1}{2}rv_1 +\frac{1}{2}r(v - v_1 ) = \frac{1}{2} rv =\frac{S}{3},$$
lebo $S/3$ je súčet obsahov dvoch vyfarbených rovnostranných trojuholníkov. Vďaka symetrii majú tú istú hodnotu $S/3$ aj súčty $S_{BCM} +S_{EFM}$ a $S_{CDM} +S_{FAM}$. Odtiaľ už dostávame prvé dva neznáme obsahy $S_DEM = S/3 - S_{ABM} = 7$\,cm$^2$ a $S_{EFM}= S/3 - S_{BCM} = 8$\,cm$^2$.
\begin{figure}[h]
    \centering
    \begin{minipage}{0.45\textwidth}
        \centering
        \includegraphics[width=0.9\textwidth]{images/62D61\imagesuffix}
        \caption{}
        \label{fig:62I6_1}
    \end{minipage}\hfill
    \begin{minipage}{0.45\textwidth}
        \centering
        \includegraphics[width=0.9\textwidth]{images/62D62\imagesuffix}
        \caption{}
        \label{fig:62I6_2}
    \end{minipage}
\end{figure}
Ako určiť zvyšné dva obsahy $S_{CDM}$ a $S_{FAM}$, keď zatiaľ poznáme len ich súčet $S/3$? Všimnime si, že súčet zadaných obsahov trojuholníkov $ABM$ a $BCM$ má významnú hodnotu $S/6$, ktorá je aj obsahom trojuholníka $ABC$ (to vyplýva opäť z~obr.~\ref{fig:62I6_2}). Taká zhoda obsahov znamená práve to, že bod $M$ leží na uhlopriečke $AC$. Trojuholníky $ABM$ a $BCM$ tak majú zhodné výšky zo spoločného vrcholu $B$ a to isté platí aj pre výšky trojuholníkov $CDM$ a $FAM$ z~vrcholov $F$ a $D$ (t.\,j. bodov, ktoré majú od priamky $AC$ rovnakú vzdialenosť). Pre pomery obsahov týchto dvojíc trojuholníkov tak dostávame
$$\frac{S_{CDM}}{S_{FAM}}=\frac{|CM|}{|AM|}=\frac{S_{BCM}}{S_{ABM}}=\frac{2}{3}.$$
V~súčte $S_{CDM} + S_{FAM}$ majúcom hodnotu $S/3$ sú teda sčítance v~pomere $2 : 3$. Preto $S_{CDM} =4$\,cm$^2$ a $S_{FAM} = 6$\,cm$^2$.\\
\\
\kom Úloha je odľahčeným a netradičným príkladom využitia princípu, na ktorom sme stavali celé toto seminárne stretnutie: súčty obsahov \uv{protiľahlých} trojuholníkov sú stále rovnaké. Posledná časť úlohy vyžaduje netriviálny nápad a študenti tak možno budú potrebovať malú radu.
}
