\problem{63-I-4-N3}{
Dokážte vety:

a) Ak majú dva trojuholníky rovnakú výšku, potom pomer ich obsahov sa rovná pomeru dĺžok príslušných základní.

b) Ak majú dva trojuholníky zhodné základne, potom pomer ich obsahov sa rovná pomeru príslušných výšok.
}{
\rie a) Označme rovnakú výšku dvoch trojuholníkov $v$. V~trojuholníku $T_1$ je táto výškou na základňu $a_1$, v~trojuholníku $T_2$ na základňu $a_2$. Pomer obsahov týchto trojuholníkov je potom $$\frac{S_{T_1}}{S_{T_2}}=\frac{\frac{1}{2}a_1v}{\frac{1}{2}a_2v}=\frac{a_1}{a_2},$$ čo sme chceli dokázať.

b) Označme zhodnú základňu dvoch trojuholníkov $z$, v~trojuholníku $T_1$ je výška na túto základňu $v_1$, v~trojuholníku $T_2$ je výška na túto základňu $v_2$. Pomer obsahov trojuholníkov $T_1$ a $T_2$ je
$$\frac{S_{T_1}}{S_{T_2}}=\frac{\frac{1}{2}zv_1}{\frac{1}{2}zv_2}=\frac{v_1}{v_2},$$
čo je pomer príslušných výšok.\\
\\
}
