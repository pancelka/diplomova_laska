% Do not delete this line (pandoc magic!)

\problem{61-I-5-N1}{seminar10,geomlah}{
Pre všeobecný trojuholník $ABC$ so stranami $a$, $b$, $c$ a obsahom $S$ platí pre polomer $r$ vpísanej kružnice vzorec $r = 2S/(a + b + c)$. Dokážte.
}{
\rieh Stred $M$ vpísanej kružnice rozdeľuje uvažovaný trojuholník $ABC$ na tri menšie trojuholníky $BCM$, $ACM$, $ABM$ s~obsahmi $\frac{1}{2}ar$, $\frac{1}{2}br$, $\frac{1}{2}cr$, ktorých súčet je $S$, odkiaľ vyplýva dokazovaný vzorec.
\begin{figure}[h]
    \centering
    \includegraphics{images/61I5N1\imagesuffix}
    \caption{}
    \label{fig:61I5N1}
\end{figure}
}
