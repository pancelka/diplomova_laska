\seminar{9}

\subsection*{Téma}
Teória čísel III -- úlohy o~ciferných zápisoch
\teachernote{
\subsection*{Ciele}
Precvičenie úloh zaoberajúcich sa cifernými zápismi, použitie rozvinutého zápisu čísla a úvah o~deliteľnosti pri riešení týchto úloh.

\subsubsection*{Úvodný komentár}

Seminár nie je náročný z~pohľadu objemu nových poznatkov, nemalé úsilie však bude vyžadovať systematická práca, využitie poznatkov o~deliteľnosti z~predchádzajúcich seminárov, zváženie všetkých možných variantov či hľadanie správneho riešenia efektívnejším spôsobom než vypísaním všetkých možností.

}
\subsection*{Úlohy a riešenia}

% Do not delete this line (pandoc magic!)

\problem{59-I-6-N1, \todo{Sedláček, J.: Co víme o~přirozených číslech, str. 7}}{
Trojciferné číslo sa končí cifrou 4. Ak túto cifru presunieme na prvé miesto (a ostatné dve cifry necháme bez zmeny), dostaneme číslo, ktoré je o~81 menšie ako pôvodné číslo. Určte pôvodné číslo.
}{
\rie Označme prvé dve cifry hľadaného trojciferného čísla $a$ a $b$. Zadanie potom môžeme prepísať do nasledujúcej rovnice $100a+10b+4=4\cdot 100 + 10a+b+81$. Po úprave a vydelení celej rovnosti deviatimi dostávame $10a+b=53$. Keďže $a$ aj $b$ sú kladné jednociferné čísla (pretože sú to cifry), je zrejmé, že $a=5$ a $b=3$. Hľadaným trojciferným číslom je tak číslo 534, čo potvrdí aj skúška správnosti. \\
\\
\kom Úloha je v~porovnaní s~tým, čo v~seminárnom stretnutí nasleduje, jednoduchá, osvieži však študentom často používanú myšlienku: číslo $\overline{abc}$ môžeme zapísať v~tvare $100a+10b+c$. Tú využijeme v~mnohých ďalších úlohách.\\
\\
}


% Do not delete this line (pandoc magic!)

\problem{~\cite{holton2010}, úloha 20, str. 110}{}{
Nájdite všetky prirodzené dvojciferné čísla, ktoré sa rovnajú dvojnásobku súčinu svojich cifier.
}{
\rieh Označme naše hľadané číslo $\overline{ab}$. Zadanie potom môžeme napísať ako $10a+b=2ab$, pričom $a\neq 0$ (inak by hľadané číslo nebolo dvojciferné). Taktiež $b\neq 0$, pretože $10a+b=2ab>0$. Keďže čísla $10a$ a $2ab$ sú párne, musí byť párne aj $b$, teda $b=2k$, $k \in \{1, 2, 3, 4 \}$. Využitím tohto poznatku môžeme rovnicu z úvodu riešenia upraviť na tvar $10a=b(2a-1)=2k(2a-1)$, teda $5a=k(2a-1)$ Preto buď $5 \mid k$ alebo $5 \mid (2a-1)$. Keďže ale $k \in \{1, 2, 3, 4\}$, platí $5\nmid k$ a preto musí platiť $5\mid(2a-1)$. Jediné možnosti, ktoré túto podmienku spĺňajú, sú $a=3$ alebo $a=8$. Ak je $a=8$, dostávame zo zadania $10\cdot 8 + b=2\cdot8\cdot b$, teda $b=16/3$. To však riešením úlohy byť nemôže, pretože $b$ musí byť celé jednociferné číslo. Preto ostáva len možnosť $a=3$, odkiaľ odvodíme $b=6$. Skúškou overíme, že nájdené číslo 36 vyhovuje zadaniu.
\\
\kom Úloha má krátke a jednoduché zadanie, jej riešenie však vyžaduje uplatnenie znalostí o deliteľnosti aj zápis čísla v rozvinutom tvare.\\
\\
}


% Do not delete this line (pandoc magic!)

\problem{61-II-2}{seminar09,cifry}{
Janko má tri kartičky, na každej je iná nenulová cifra. Súčet všetkých trojciferných čísel, ktoré možno z~týchto kartičiek zostaviť, je číslo o~6 väčšie ako trojnásobok jedného z~nich. Aké cifry sú na kartičkách?
}{
\rieh Označme $\overline{abc}$ to trojciferné číslo, o~ktorého trojnásobku sa píše v~texte úlohy. Platí tak rovnica
$$3\overline{abc} + 6 = \overline{abc} + \overline{acb}+ \overline{bac}+\overline{bca}+\overline{cab}+\overline{cba}$$  Keďže na pravej strane je každá z~cifier $a, b, c$ dvakrát na mieste jednotiek, desiatok aj
stoviek, môžeme rovnicu prepísať na tvar
$$300a + 30b + 3c + 6 = 222a + 222b + 222c, \quad \text{čiže} \quad 78a + 6 = 192b + 219c.$$
Po vydelení číslom 3 dostaneme rovnicu $26a + 2 = 64b + 73c$, z~ktorej vidíme, že $c$ je párna cifra. Platí preto $c \geq 2$, čo spolu so zrejmou nerovnosťou $b \geq 1$ (pripomíname, že všetky tri neznáme cifry sú podľa zadania nenulové) vedie k~odhadu $$ 64b + 73c \geq 64 + 146 = 210.$$
Musí preto platiť $26a + 2 \geq 210$, odkiaľ $a \geq (210 - 2) : 26 = 8$, takže cifra $a$ je buď 8, alebo 9. Pre $a = 8$ však v~nerovnosti z~predošlej vety nastane rovnosť, takže nutne $b = 1$ a $c = 2$ (a rovnica zo zadania úlohy je potom splnená). Pre $a = 9$ dostávame
rovnicu $$ 64b + 73c = 26 \cdot 9 + 2 = 236,$$
z~ktorej vyplýva, že $c$ je jednak deliteľné štyrmi, jednak je menšie ako 4, čo nemôže nastať súčasne.

\textit{Záver.} Cifry na kartičkách sú 8, 2 a 1.

\textit{Poznámka.} Riešiť odvodenú rovnicu $26a + 2 = 64b + 73c$ pre neznáme (nenulové a navzájom rôzne) cifry $a, b, c$ možno viacerými úplnými a systematickými postupmi, uviedli sme len jeden z~nich.\\
\\
\kom Úloha je zložitejšia než úvodné dve. Vychádza síce z~rozvinutého zápisu čísla, avšak vyžaduje dodatočnú netriviálnu analýzu. Bude preto iste zaujímavé diskutovať so študentmi o~tom, ako k~pristupovali k~riešeniu rovnice $26a + 2 = 64b + 73c$. \\
\\
}


% Do not delete this line (pandoc magic!)

\problem{63-II-1}{seminar09,cifry,krajskekolo}{
Nájdite všetky trojice (nie nutne rôznych) cifier $a, b, c$ také, že päťciferné čísla $\overline{6abc3}$ a $\overline{3abc6}$ sú v~pomere 63 : 36.
}{
\rieh Zostavíme a vyriešime rovnicu pre neznáme cifry $a, b, c$, ktorú vďaka tvaru zadaných čísel môžeme zapísať rovno pre jedinú neznámu $x = 100a + 10b + c$:
\begin{align*}
\frac{60000+10x+3}{30000+10x+3} &=\frac{63}{36}=\frac{7}{4},\\
40x + 240 012 &= 70x + 210 042,\\
30x &= 29 970,\\
x &= 999.
\end{align*}

\textit{Záver.} Nájdenému $x$ zodpovedá trojica cifier $a = b = c = 9$. Úloha má jediné riešenie.\\
\\
\kom Úloha prináša zaujímavú myšlienku zjednodušenia zápisu, ktorý potom vedie k~riešeniu jednoduchej lineárnej rovnice. Aj napriek tomu, že riešenie nevyžaduje mnoho počítania, ukrýva úloha záludnosť v~podobe toho, že študenti môžu prísť k~správnemu riešeniu nesprávnymi úvahami. Viac o~tejto konkrétnej úlohe a jej úskaliach je možné nájsť v~článku \cite{hana}, ktorý považujeme za hodný preštudovania.\\
\\
}


% Do not delete this line (pandoc magic!)

\problem{57-I-6-D2, resp. 53-II-4}{seminar09,cifry,domacekolo,doplnujuca}{
Žiaci mali vypočítať príklad $x + y \cdot z$ pre trojciferné číslo $x$ a dvojciferné čísla $y, z$. Martin vie násobiť a sčítať čísla zapísané v~desiatkovej sústave, ale zabudol na pravidlo prednosti násobenia pred sčítaním. Preto mu vyšlo síce zaujímavé číslo, ktoré sa píše rovnako zľava doprava ako sprava doľava, správny výsledok bol ale o~$2 004$ menší. Určte čísla $x, y, z$.
}{
\rieh Martin vypočítal hodnotu $(x + y)z$ namiesto $x + yz$, takže podľa zadania platí
$$(x + y)z - (x + yz) = 2 004, \ \ \ \ \text{čiže} \ \ \  x \cdot (z~- 1) = 2 004 = 12 \cdot 167,$$
pričom 167 je prvočíslo. Činitele $x$ a $z -1$ určíme, keď si uvedomíme, že $z$ je dvojciferné číslo, takže $9 \leq z~- 1 \leq 98$. Vidíme, že nutne $z - 1 = 12$ a  $x = 167$, odkiaľ $z = 13$. Martin teda vypočítal číslo $V = (167+y)\cdot 13$. Číslo $V$ je preto štvorciferné, a pretože sa číta spredu rovnako ako zozadu, má tvar $\overline{abba} = 1 001a + 110b$. Pretože $1 001 = 13 \cdot 77$, musí platiť rovnosť $(167 + y) \cdot 13 = 13 \cdot 77a + 110b$, z~ktorej vyplýva, že číslica $b$ je deliteľná trinástimi, takže $b = 0$. Po dosadení dostaneme (po delení trinástimi) rovnosť $167 + y = 77a$, ktorá vzhľadom na nerovnosti $10 \leq  y \leq 99$ znamená, že číslica $a$ sa rovná 3, takže $y = 64$.

V~druhej časti riešenia sme mohli postupovať aj nasledovne. Pre číslo $V = (167 + y) \cdot 13$ vychádzajú z~nerovností $10 \leq y \leq 99$ odhady $2301 \leq V~\leq 3 458$. Zistíme
preto, ktoré z~čísel $\overline{2bb2}$, kde $b \in \{3, 4, 5, 6, 7, 8, 9\}$ a čísel $\overline{3bb3}$, kde $b \in \{0, 1, 2, 3, 4\}$, sú deliteľné trinástimi. Aj keď sa týchto dvanásť čísel dá rýchlo otestovať, urobme to všeobecne ich čiastočným vydelením trinástimi:

\vspace{-25pt}
\begin{center}
\begin{align*}
 \overline{2bb2} &= 2 002 + 110b = 13 \cdot (154 + 8b) + 6b,\\
\overline{3bb3} &= 3 003 + 110b = 13 \cdot (231 + 8b) + 6b.
\end{align*}
\end{center}

Vidíme, že vyhovuje jedine číslo $\overline{3bb3}$ pre $b = 0$. Vtedy $167 + y = 231$, takže $y = 64$.

\textit{Záver.} Žiaci mali počítať príklad $167 + 64 \cdot 13$, teda $x = 167$, $y = 64$ a $z = 13$.\\
\\
\kom Na prvý pohľad sa môže zdať, že úloha nepatrí do tohto seminára, ako sa však v~priebehu riešenia ukáže, má tu svoje miesto. Taktiež zaujímavým spôsobom spája poznatky o~deliteľnosti, príp. sa v~jej riešení uplatnia odhady -- túto metódu sme už taktiež v~dnešnom seminári využili.\\
\\
}


% Do not delete this line (pandoc magic!)

\problem{58-I-3-N1, resp. 56-S-1}{seminar09,cifry,delitelnost}{
Určte počet všetkých štvorciferných prirodzených čísel, ktoré sú deliteľné šiestimi a v~ktorých zápise sa vyskytujú práve dve jednotky.
}{
\rieh  Aby číslo bolo deliteľné šiestimi, musí byť párne a mať ciferný súčet deliteľný tromi. Označme teda $b$ číslicu na mieste jednotiek (tá musí byť párna, $b \in \{0, 2, 4, 6, 8\}$) a $a$ tú číslicu, ktorá je spolu s~číslicami 1, 1 ($a \neq 1$) na prvých troch miestach štvorciferného čísla, ktoré spĺňa požiadavky úlohy. Aby bol súčet číslic $a + 1 + 1 + b$ takého čísla deliteľný tromi, musí číslo $a + b$ dávať po delení tromi zvyšok 1. Pre $b \in \{0, 6\}$ tak máme pre $a$ možnosti $a \in {4, 7}$ $(a \neq 1)$, pre $b \in \{2, 8\}$ máme $a \in \{2, 5, 8\}$ a konečne pre $b = 4$ máme $a \in \{0, 3, 6, 9\}$. Pre každé zvolené $b$ a zodpovedajúce $a \neq 0$ sú zrejme tri možnosti, ako číslice 1, 1 a $a$ na prvých troch miestach usporiadať, to je spolu $(2 \cdot 2 + 2 \cdot 3 + 3) \cdot 3 = 39$ možností, pre $a = 0$ (keď $b = 4$) potom sú len dve možnosti (číslica nula nemôže byť prvá číslica štvorciferného čísla).

Celkom existuje 41 štvorciferných prirodzených čísel, ktoré spĺňajú podmienky.

Alternatívnym postupom je vypísanie všetkých možností na základe ciferného súčtu, ktorý musí byť deliteľný troma a zároveň sa končiť párnou cifrou.\\
\\
\kom Úloha využíva poznatky o~deliteľnosti, takže pekne nadväzuje na predchádzajúce semináre. Tiež je prvou úlohou o~cifernom zápise, v~riešení ktorej nevyužijeme rozvinutý zápis čísla, ale skôr intuitívne kombinatorické úvahy.

Ak sa študenti vyberú cestou vypisovania všetkých možných kombinácií, skúsime ich povzbudiť, aby ich úsilie bolo čo najsystematickejšie a efektívne, príp. prediskutujeme, či sa riešenie dá nájsť aj inou cestou. \\
\\
}


% Do not delete this line (pandoc magic!)

\problem{58-I-3-N2, resp.54-I-5}{
Určte počet všetkých trojíc dvojciferných prirodzených čísel $a$, $b$, $c$, ktorých súčin $abc$ má zápis, v~ktorom sú všetky cifry rovnaké. Trojice líšiace sa len poradím čísel považujeme za rovnaké,  t.\,j. započítavame ich iba raz.
}{
\rieh Pre dvojmiestne čísla $a, b, c$ je súčin $abc$ číslo štvormiestne, alebo päťmiestne, alebo šesťmiestne. Ak sú teda všetky číslice čísla $abc$ rovné jednej číslici $k$, platí jedna z~rovností $abc = k~\cdot 1 111$, $abc = k~\cdot 11 111$ alebo $abc = k~\cdot 111 111$, $k \in \{1, 2,\,\ldots , 9\}$.

Čísla $1 111 = 11\cdot 101$ a $11111 = 41\cdot 271$ však majú vo svojom rozklade trojmiestne prvočísla, takže nemôžu byť súčinom dvojmiestnych čísel. Ostáva preto jediná možnosť:
$$ abc = k~\cdot 111 111 = k~\cdot 3 \cdot 7 \cdot 11 \cdot 13 \cdot 37.$$
Pozrime sa, ako môžu byť prvočísla 3, 7, 11, 13, 37 rozdelené medzi jednotlivé činitele $a, b, c$. Pretože súčiny $37 \cdot 3$ a $3 \cdot 7 \cdot 11$ sú väčšie ako 100, musí byť prvočíslo 37 samo ako jeden činiteľ a zvyšné štyri prvočísla 3, 7, 11, 13 musia byť rozdelené do dvojíc. Keďže aj súčin $11 \cdot 13$ je väčší ako 100, prichádzajú do úvahy iba rozdelenia na činitele $3 \cdot 11, 7 \cdot 13$ a 37, alebo na činitele $3 \cdot 13$, $7 \cdot 11$ a 37. K~týmto činiteľom ešte pripojíme možné činitele z~rozkladu číslice $k$ a dostaneme riešenia dvoch typov:
$$a = 33k_1, b = 91, c = 37k_2, \ \ \ \ \text {pričom} \ \ k_1 \in \{1, 2, 3\}, k_2 \in \{1, 2\},$$
$$a = 39k_1, b = 77, c = 37k_2,\ \ \ \ \text{pričom}\ \ k_1 \in \{1, 2\}, k_2 \in \{1, 2\},$$
Hľadaný počet trojíc čísel $a, b, c$ je teda $3 \cdot 2 + 2 \cdot 2 = 10$.\\
\\
\kom Posledná úloha seminára je zaujímavá myšlienkou, že čísla, ktoré majú všetky cifry rovnaké, sú násobkami čísel 11, 111, 1111,\,\ldots. Ďalšia analýza uplatní poznatky o~rozklade čísla na súčin prvočísel a je tak ďalším príkladom úlohy, v~ktorej musia študenti zapojiť vedomosti z~predchádzajúcich seminárov.
}




\subsection*{Domáca práca}

% Do not delete this line (pandoc magic!)

\problem{58-I-3}{
Nájdite všetky štvorciferné čísla $n$, ktoré majú nasledujúce tri vlastnosti: V~zápise čísla $n$ sú dve rôzne cifry, každá dvakrát. Číslo $n$ je deliteľné siedmimi. Číslo, ktoré vznikne otočením poradia cifier čísla $n$, je tiež štvorciferné a deliteľné siedmimi.
}{
\rieh V~riešení budeme označovať číslo, ktoré vznikne otočením poradia cifier čísla $n$, ako $\overline{n}$. Rozoberieme tri prípady.

(i) Číslo $n$ má tvar $aabb$, kde $a$, $b$ sú rôzne cifry. Takže $n = 1100a + 11b$ a $\overline{n} = 1100b + 11a$. Číslo 7 má deliť ako $n$, tak $\overline{n}$, teda aj ich rozdiel $n - \overline{n} = 1089(a - b)$ a súčet $n + \overline{n} = 1111(a + b)$. Keďže ani číslo 1089, ani číslo 1111 nie sú násobkom siedmich a sedem je prvočíslo, tak $7 \mid a - $b aj $7 \mid a + b$. Ak použijeme rovnakú úvahu ešte raz, vidíme, že $7 \mid (a - b) + (a + b) = 2a$ a $7 \mid (a + b) - (a - b) = 2b$, teda $7 \mid a$ a $7 \mid b$, čiže $a, b \in \{0, 7\}$. Cifry $a, b$ sú navzájom rôzne, preto jedna z~nich musí byť 0. Ale potom jedno z~čísel $aabb$, $bbaa$ nie je štvorciferné. Hľadané číslo $n$ teda nemôže mať uvedený tvar.

(ii) Číslo $n$ má tvar $abab$. Potom $7 \mid n = 1010a + 101b$ a tiež $7 \mid \overline{n} = 1010b + 101a$. Podobne ako v~predchádzajúcom prípade odvodíme, že $7 \mid n - \overline{n} = 909(a - b)$ a $7 \mid n + \overline{n} = 1111(a + b)$, a z~rovnakých dôvodov ako v~predchádzajúcom prípade zisťujeme, že $7 \mid a$, $7 \mid b$. Niektorá z~cifier by teda musela byť 0. Číslo $n$ tak nemôže mať ani tvar $abab$.

(iii) Číslo $n$ má tvar $abba$. Potom otočením poradia cifier vznikne to isté číslo, takže máme jedinú podmienku $7 \mid 1001a + 110b$. Keďže $7 \mid 1001$ a $7 \nmid 110$, je táto podmienka ekvivalentná s~podmienkou $7 \mid b$. Preto $b \in \{0, 7\}$, $a \in \{1, 2,\,\ldots, 9\}$, $a \neq b$. Vyhovuje tak všetkých 17 čísel, ktoré práve uvedené podmienky spĺňajú: 1001, 2002, 3003, 4004, 5005, 6006, 7007, 8008, 9009, 1771, 2772, 3773, 4774, 5775, 6776, 8778, 9779.\\
\\
}


\problem{57-I-6}{
Klárka mala na papieri napísané trojciferné číslo. Keď ho správne vynásobila deviatimi, dostala štvorciferné číslo, ktoré sa začínalo rovnakou číslicou ako pôvodné číslo, prostredné dve číslice sa rovnali a posledná číslica bola súčtom číslic pôvodného čísla. Ktoré štvorciferné číslo mohla Klárka dostať?
}{
\rieh Hľadajme pôvodné číslo $x = 100a + 10b + c$, ktorého cifry sú $a, b, c$. Cifru, ktorá sa vyskytuje na prostredných dvoch miestach výsledného súčinu, označme $d$. Zo zadania vyplýva
$$ 9(100a + 10b + c) = 1 000a + 100d + 10d + (a + b + c), \ \ \ \ \ \ (1)$$
pričom výraz v~poslednej zátvorke predstavuje cifru zhodnú s~poslednou cifrou súčinu $9c$. To však znamená, že nemôže byť $c \geq 5$: pre také $c$ sa totiž končí číslo $9c$ cifrou neprevyšujúcou 5, a pretože $a\neq 0$, platí naopak $a + b + c > c \geq 5$.

Zrejme tiež $c \neq 0$ (v~opačnom prípade by platilo $a = b = c = x = 0$). Ostatné
možnosti vyšetríme zostavením nasledujúcej tabuľky.
\begin{center}
\begin{tabular}{|c|c|c|c|}
\hline
$c$ &$9c$ & $a+b+c$ & $a+b$ \\
\hline
\hline
1 & 9 & 9 & 8 \\
\hline
2 & 18 & 8 & 6\\
\hline
3& 27 & 7& 4\\
\hline
4& 36 & 6 & 2\\
\hline
\end{tabular}
\end{center}

Rovnosť (1) možno prepísať na tvar
$$ 100(b - a - d) = 10d + a + 11b - 8c. (2)$$
Hodnota pravej strany je aspoň $-72$ a menšia ako 200, lebo každé z~čísel $a, b, c, d$ je najviac rovné deviatim. Takže buď $b - a - d = 0$, alebo $b - a - d = 1$.

V~prvom prípade po substitúcii $d = b - a$ upravíme vzťah (2) na tvar $8c = 3(7b - 3a)$, z~ktorého vidíme, že $c$ je násobkom troch. Z~prvej tabuľky potom vyplýva $c = 3, a = 4 - b$, čo po dosadení do rovnice $8c = 3(7b - 3a)$ vedie k~riešeniu $a = b = 2, c = 3$. Pôvodné číslo je teda $x = 223$ a jeho deväťnásobok $9x = 2 007.$

V~druhom prípade dosadíme $d = b - a - 1$ do (2) a zistíme, že $8c + 110 = 3(7b-3a)$. Výraz $8c+110$ je teda deliteľný tromi, preto číslo $c$ dáva po delení tromi zvyšok 2. Dosadením jediných možných hodnôt $c = 2$ a $b = 6 - a$ do poslednej rovnice zistíme, že $a = 0$, čo je v~rozpore s~tým, že číslo $x = 100a + 10b + c$ je trojciferné.

\textit{Záver.} Klárka dostala štvorciferné číslo 2 007.

\textit{Poznámka.} Prvá tabuľka ponúka jednoduchší, ale numericky pracnejší postup priameho dosadzovania všetkých prípustných hodnôt čísel $a, b, c$ do rovnice (1). Počet všetkých možností možno obmedziť na desať odhadom $b \geq a$, ktorý zistíme pomocou vhodnej úpravy vzťahu (1) napríklad na tvar (2). Riešenie uvádzame v~druhej tabuľke.
\begin{center}
\begin{tabular}{|r||r|r|r|r|r|r|r|r|r|r|}
\hline
$a$ & 1 & 2 & 3 & 4 & 1 & 2 & 3 & 1 & \textbf{2} & 1\\
\hline
$b$ & 7 & 6 & 5 & 4 & 5 & 4 & 3 & 3 & \textbf{2} & 1 \\
\hline
$c$ & 1 & 1 & 1 & 1 & 2 & 2 & 2 & 3 & \textbf{3} & 4 \\
\hline
$9x$ & 1539 & 2349 & 3159 & 3969 & 1368 & 2178 & 2988 & 1197 & \textbf{2007} & 1026\\
\hline
\end{tabular}
\end{center}
}



\teachernote{
\subsection*{Doplňujúce zdroje a materiály}
Výborným zdrojom úloh jednoduchších aj zložitejších je~\cite{sedlacek1961}.%\todo \footnote{\url{https://dml.cz/handle/10338.dmlcz/403433}}.

}