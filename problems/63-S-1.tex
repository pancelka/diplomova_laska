% Do not delete this line (pandoc magic!)

\problem{63-S-1}{}{
Určte, aké hodnoty môže nadobúdať výraz $V = ab + bc + cd + da$, ak reálne čísla $a,b, c, d$ spĺňajú dvojicu podmienok
\begin{align*}
2a - 5b + 2c - 5d &= 4,\\
3a + 4b + 3c + 4d &= 6.
\end{align*}
}{
\rieh Pre daný výraz $V$ platí $$V = a(b + d) + c(b + d) = (a + c)(b + d).$$
Podobne môžeme upraviť aj obe dané podmienky: \begin{equation} \label{eq:63S1}
    2(a + c) - 5(b + d) = 4 \ \ \ \ \mathrm{a} \ \ \ \  3(a + c) + 4(b + d) = 6.
\end{equation}
Ak teda zvolíme substitúciu $m = a + c$ a $n = b + d$, dostaneme riešením sústavy \ref{eq:63S1} $m = 2$ a $n = 0$. Pre daný výraz potom platí $V = mn = 0$.

\textit{Záver.} Za daných podmienok nadobúda výraz $V$ iba hodnotu 0.\\

\textbf{Iné riešenie*.} Podmienky úlohy si predstavíme ako sústavu rovníc s neznámymi $a, b$ a parametrami $c, d$. Vyriešením tejto sústavy (sčítacou alebo dosadzovacou metódou) vyjadríme $a = 2 - c, b = -d \ (c, d \in \RR )$ a po dosadení do výrazu $V$ dostávame $$V = (2 - c)(-d) - dc + cd + d(2 - c) = 0.$$
\\
\kom Zadanie úlohy opäť obsahuje sústavu dvoch rovníc. Jej riešenie sa však po substitúcii zredukuje na veľmi jednoduchú sústavu, s ktorou sa študenti stretli už na základnej škole, takže by nemala spôsobiť výrazné problémy.
}
