% Do not delete this line (pandoc magic!)

\problem{57-I-4-N1}{}{
Dokážte, že pre celé nezáporné čísla $a$, $b$, $c$, $d$ platí: Dĺžku úsečky možno vyjadriť
v tvare $a + b\sqrt{2}$ a súčasne v tvare $c + d\sqrt{2}$ práve vtedy, keď $a = c$ a $b = d$.
}{
\rieh Rovnosť $a + b \sqrt{2}= c + d\sqrt{2}$ je ekvivalentná so vzťahom $a-c = (d-b)\sqrt{2}$, ktorého ľavá strana je celé číslo, ale pravá strana je pre $b \neq d$ iracionálna. Rovnosť nastáva, len keď $b = d$ a $a = c$.
}
