% Do not delete this line (pandoc magic!)

\problem{57-S-1}{seminar20,prvocisla,skolskekolo}{
Nájdite všetky dvojice prirodzených čísel $a, b$ väčších ako 1 tak, aby ich súčet aj súčin boli mocniny prvočísel.
}{
\rieh Z~podmienky pre súčin vyplýva, že $a$ aj $b$ sú mocninami toho istého prvočísla $p$: $a = p^r$, $b = p^s$, pričom $r, s$ sú celé kladné čísla. Keby bolo $p$ nepárne, bol by súčet $a + b$ deliteľný okrem čísla $p$ aj číslom 2, takže by nebol mocninou prvočísla. Teda $p = 2$. Ak $r < s$, je súčet $a + b = 2^r (1 + 2^{s-r})$ opäť číslo párne deliteľné nepárnym číslom väčším ako 1, nie je teda mocninou prvočísla. K~rovnakému záveru dôjdeme aj v~prípade, keď $r > s$. Ostáva preto jediná možnosť: $a = b = 2^r$ , pričom $r$ je celé kladné číslo. Skúška $a+b = 2^r +2^r = 2^{r+1}$ a $ab = 2^{2r}$ potvrdzuje, že riešením sú všetky dvojice $(a, b) = (2^r, 2^r)$, kde $r$ je celé kladné číslo.\\
\\
}
