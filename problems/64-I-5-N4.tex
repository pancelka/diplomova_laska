% Do not delete this line (pandoc magic!)

\problem{64-I-5-N4}{}{
Platí pre každé tri prirodzené čísla $a, b, c$ a ich najväčší spoločný deliteľ $d$ a ich najmenší spoločný násobok $n$ rovnosť $abc = nd$?
}{
\rie Neplatí, uvedieme protipríklad. Napríklad pre čísla $15, 18$ a $24$ je $d=(15,18,24)=3$, $n=[15, 18, 24]=360$. Ďalej $15\cdot 18 \cdot 24 =6480$ a $(15,18,24)\cdot[15, 18, 24]=3\cdot 360=1080$, to však nie sú rovnaké čísla a tvrdenie neplatí. \\
\\
\kom Všeobecnejší pohľad na predchádzajúci problém by sme dostali skrz pohľad na exponenty prvočísel, z~ktorých sú čísla $a, b, c$ zložené. Skúmaná rovnosť nastane len v~prípade, že sú všetky tri čísla navzájom po dvoch nesúdeliteľné.

Zároveň úloha demonštruje riešenie uvedením protipríkladu, čo je princíp, s~ktorým sme sa v~seminároch zatiaľ nestretli a jeho spomenutie je určite vhodné.\\
\\
}
