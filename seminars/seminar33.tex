\seminar{33}{Opakovanie I -- pohľad späť na všetko, čo sme sa naučili}
\teachernote{
\subsection*{Ciele}

Zopakovať kľúčové myšlienky, postupy a poznatky, ktoré v priebehu roka študenti získali.

\subsection*{Priebeh seminára}

%\todo{mind maps}

Študentov rozdelíme do 4 skupín a každej skupine pridelíme jednu zo študovaných oblastí (algebra, teória čísel, geometria, kombinatorika). Úlohou skupín bude vytvoriť myšlienkovú mapu, ktorá zhŕňa poznatky z priradenej oblasti -- mala by obsahovať nie len konkrétne fakty (napr. \uv{Súčet veľkostí protiľahlých vnútorných uhlov tetivového štvoruholníka je $180^\circ$.}), ale aj všeobecnejšie prístupy alebo metódy (napr. \uv{Pozor na kreslenie náčrtu.}).  Ak sa študenti ešte s myšlienkovými mapami nestretli, stručne im ich vysvetlíme, príp. ukážeme niekoľko príkladov. Na vypracovanie študentom necháme 20-30 minút. Po tomto čase jednotlivé skupiny prezentujú svoje výtvory a zvyšok osadenstva prispieva svojimi komentármi a otázkami. Na jednu skupinu si odporúčame vyhradiť aspoň 15 minút. Zaujímavé bude sledovať, či sa niektoré poznatky budú vyskytovať vo viacerých skupinách, prípadne či sa študenti budú odkazovať aj na inú oblasť, než akú spracovali.

Tento zvolený spôsob upevnenia a prepojenia poznatkov pokladáme za prínosný, pretože študentom dáva priestor premýšľať trochu iným spôsobom


\subsection*{Doplňujúce materiály}

O myšlienkových mapách je možné nájsť viac na \url{\detokenize{https://www.mindmapping.com/mindmap.php}} alebo \url{https://www.mindtools.com/pages/article/newISS_01.htm}, kde je tiež k dispozícii množstvo príkladov.
}

%Keďže seminárne stretnutie bude tentokrát najmä v réžii študentov, bude prebiehať trochu inak. Každá zo štyroch skupín predstaví svoju domácu prácu a spoločne so ostatnými študentmi

%Úlohou učiteľa je výklad a myšlienky študentov vhodne korigovať tak, aby spoločne so študentami pokryli väčšinu oblastí. Cieľom tohto seminára nie je znova prechádzať všetky úlohy, ktoré sme doteraz so študentami vyriešili, ale skôr zopakovať všeobecné myšlienky a postupy, na ktoré je v jednotlivých oblastiach dobré myslieť.

%V prípade, že chce mať učiteľ istotu toho, že na seminári zaznejú všetky kľúčové myšlienky, môže študentov požiadať, aby mu domácu prácu zaslali v predstihu a skontrolovať, príp. doplniť ju.

%V prvej časti seminára prejdeme spolu so štyrmi skupinami základné oblasti problémov, ktorým sme sa na seminári venovali a v závere môžeme študentom nechať priestor na zdieľanie poznatkov alebo poučení, ktoré do žiadnej konkrétnej oblasti úloh nespadajú.



