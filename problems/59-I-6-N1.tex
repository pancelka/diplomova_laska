% Do not delete this line (pandoc magic!)

\problem{59-I-6-N1}{}{
%\todo{Sedláček, J.: Co víme o~přirozených číslech, str. 7}
Trojciferné číslo sa končí cifrou 4. Ak túto cifru presunieme na prvé miesto (a ostatné dve cifry necháme bez zmeny), dostaneme číslo, ktoré je o~81 menšie ako pôvodné číslo. Určte pôvodné číslo.
}{
\rie Označme prvé dve cifry hľadaného trojciferného čísla $a$ a $b$. Zadanie potom môžeme prepísať do nasledujúcej rovnice $100a+10b+4=4\cdot 100 + 10a+b+81$. Po úprave a vydelení celej rovnosti deviatimi dostávame $10a+b=53$. Keďže $a$ aj $b$ sú kladné jednociferné čísla (pretože sú to cifry), je zrejmé, že $a=5$ a $b=3$. Hľadaným trojciferným číslom je tak číslo 534, čo potvrdí aj skúška správnosti. \\
\\
\kom Úloha je v~porovnaní s~tým, čo v~seminárnom stretnutí nasleduje, jednoduchá, osvieži však študentom často používanú myšlienku: číslo $\overline{abc}$ môžeme zapísať v~tvare $100a+10b+c$. Tú využijeme v~mnohých ďalších úlohách.\\
\\
}
