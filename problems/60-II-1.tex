% Do not delete this line (pandoc magic!)

\problem{60-II-1}{seminar06,rovnice}{
Na tabuli sú napísané práve tri (nie nutne rôzne) reálne čísla. Vieme, že súčet ľubovoľných dvoch z~nich je tam napísaný tiež. Určte všetky trojice takých čísel.
}{
\rieh Označme čísla napísané na tabuli $a, b, c$. Súčet $a + b$ sa tiež nachádza na tabuli, je teda rovný jednému z~čísel $a, b, c$. Keby $a + b$ bolo rovné $a$ alebo $b$, bola by na tabuli aspoň jedna nula. Rozoberieme preto tri prípady podľa počtu núl napísaných na tabuli.

Ak sú na tabuli aspoň dve nuly, ľahko sa presvedčíme, že súčet každých dvoch čísel z~tabule je tam tiež. Dostávame, že trojica $t, 0, 0$ je pre ľubovoľné reálne číslo $t$ riešením úlohy.

Ak je na tabuli práve jedna nula, je tam trojica $a, b, 0,$ pričom $a$ aj $b$ sú nenulové čísla. Súčet $a + b$ teda nie je rovný ani $a$, ani $b$, musí preto byť rovný 0. Dostávame tak ďalšiu trojicu $t, -t, 0$, ktorá je riešením úlohy pre ľubovoľné reálne číslo $t$.
Ak na tabuli nie je ani jedna nula, súčet $a + b$ nie je rovný ani $a$, ani $b$, preto $a + b = c$. Z~rovnakých dôvodov je $b + c = a$ a $c + a = b$. Dostali sme sústavu troch lineárnych rovníc s~neznámymi $a, b, c$, ktorú môžeme vyriešiť. Avšak hneď z~prvých dvoch rovníc po dosadení vyjde $b + (a + b) = a$, čiže $b = 0$. To je v~spore s~tým, že na tabuli žiadna nula nie je.

\textit{Záver.} Úlohe vyhovujú trojice $t, 0, 0$ a $t, -t, 0$ pre ľubovoľné reálne číslo $t$ a žiadne iné.\\
\\
}
