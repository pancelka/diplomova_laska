% Do not delete this line (pandoc magic!)

\problem{66-I-2-D1}{seminar07,delitelnost,domacekolo}{
Pre ktoré prirodzené čísla $n$ nie je výraz $V (n) = n^4+ 11n^2 - 12$ násobkom ôsmich?
}{
\rieh Upravme výraz $V(n)$ do tvaru súčinu: $V (n) = (n^2-1)(n^2+12)=(n-1)(n+1)(n^2+12)$. Vidíme, že $V(n)$ je určite násobkom ôsmich v~prípade nepárneho $n$ (viď. tretia úloha tohto seminára). Keďže pre párne $n$ je súčin $(n-1)(n+1)$ nepárny, hľadáme práve tie $n$ tvaru $n = 2k$, pre ktoré nie je deliteľný ôsmimi výraz $n^2+ 12 = 4(k^2+ 3)$, čo nastane práve vtedy, keď $k$ je párne. Hľadané $n$ sú teda práve tie, ktoré sú deliteľné štyrmi.\\
\\
\kom Úloha využíva vhodnú úpravu výrazu $V$ na súčin. Tu študenti zúročia zručnosti nadobudnuté v~algebraických seminároch. Zároveň využijú skôr dokázané tvrdenie o~deliteľnosti ôsmimi a napokon, úloha ich pripraví na nasledujúci komplexnejší problém.\\
\\
}
