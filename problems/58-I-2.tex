% Do not delete this line (pandoc magic!)

\problem{58-I-2}{seminar11,trojuholniky,podtroj,obsahy}{
Pravouhlému trojuholníku $ABC$ s~preponou $AB$ je opísaná kružnica. Päty kolmíc z~bodov $A$, $B$ na dotyčnicu k~tejto kružnici v~bode $C$ označme $D$, $E$. Vyjadrite dĺžku úsečky $DE$ pomocou dĺžok odvesien trojuholníka $ABC$.
}{
\rieh Označme odvesny trojuholníka $ABC$ zvyčajným spôsobom $a$, $b$ a protiľahlé uhly $\alpha$, $\beta$. Stred prepony $AB$ (ktorý je súčasne stredom opísanej kružnice) označíme $O$ (obr.~\ref{fig:58I2_1}).

Výška $v = CP$ rozdeľuje trojuholník $ABC$ na trojuholníky $ACP$ a $CBP$ podobné trojuholníku $ABC$ podľa vety $uu$ ($\alpha + \beta = 90^\circ$), úsečka $OC$ je kolmá na $DE$ a navyše $|OC| = |OA| = r$ (polomer opísanej kružnice). Odtiaľ $|\ma OCA| = |\ma OAC| = \alpha$ a $|\ma DCA| = 90^\circ - |\ma OCA| = \beta$.

Pravouhlé trojuholníky $ACP$ a $ACD$ so spoločnou preponou $AC$ sa teda zhodujú aj v~uhloch pri vrchole $C$. Sú preto zhodné, dokonca súmerne združené podľa priamky $AC$. Analogicky sú trojuholníky $CBP$ a $CBE$ súmerne združené podľa $BC$. Takže $|CD|= |CE| = v$, čiže $|DE| = 2v = 2ab/\sqrt{a^2 + b^2}$, lebo z~dvojakého vyjadrenia dvojnásobku obsahu trojuholníka $ABC$ vyplýva $v = ab/|AB|$, pričom $|AB| =\sqrt{a^2 + b^2}$.

\textit{Poznámka.} Namiesto dvojakého vyjadrenia obsahu môžeme na výpočet výšky $CP$ využiť podobnosť trojuholníkov $CBP$ a $ABC$: $\sin \alpha = |CP|/|AC| = |BC|/|AB|$.
\begin{figure}[h]
    \centering
    \begin{minipage}{0.45\textwidth}
        \centering
        \includegraphics[width=0.9\textwidth]{images/58D21\imagesuffix}
        \caption{}
        \label{fig:58I2_1}
    \end{minipage}\hfill
    \begin{minipage}{0.45\textwidth}
        \centering
        \includegraphics[width=0.9\textwidth]{images/58D22\imagesuffix}
        \caption{}
        \label{fig:58I2_2}
    \end{minipage}
\end{figure}

\textbf{Iné riešenie*.} Úsečka $OC$ je strednou priečkou lichobežníka $DABE$, lebo je rovnobežná so základňami a prechádza stredom $O$ ramena $AB$. Preto $D$ je obrazom bodu $E$ v~súmernosti podľa stredu $C$. Obraz $F$ bodu $B$ v~tej istej súmernosti leží na polpriamke $AD$ za bodom $D$ (obr.~\ref{fig:58I2_2}). Máme $|CF| = |BC| = a$, uhol $ACF$ je pravý, a teda trojuholníky $AFC$ a $ABC$ sú zhodné. Vidíme, že $CD$ je výška v~trojuholníku $AFC$ zhodná s~výškou $v_c$ trojuholníka $ABC$, a $DE$ je jej dvojnásobkom. Veľkosť výšky $v_c$ dopočítame rovnako ako v~predchádzajúcom riešení.

\textit{Záver.} $|DE| = 2ab/\sqrt{a^2 + b^2}$.\\
\\
}
