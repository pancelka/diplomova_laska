\seminar{1}{Úvod do seminára, očakávania, \textit{Matematico}}

\teachernote{
\subsection*{Ciele}
Zoznámiť študentov s povahou seminára a plánom na školský rok, zoznámiť sa so študentami, motivovať na začiatok matematickou hrou.

\subsection*{Priebeh}

Keďže ide o prvý seminár, oboznámime študentov s tým, čo môžu v priebehu roka od stretnutí očakávať: aká bude forma seminárov a čo bude ich obsahom. Zároveň považujeme za vhodné porozprávať sa so študentmi o ich motivácii -- čo ich na seminár privádza a čo od neho očakávajú. Takáto informácia nám potom môže poslúžiť pri plánovaní alebo prispôsobovaní obsahu konkrétnej skupine, s ktorou budeme pracovať. Napríklad ak sa seminára budú účastniť v drvivej väčšine študenti s ambíciami na úspešné umiestnenie na Medzinárodnej matematickej olympiáde, je možné hravejšie semináre nahradiť náročnejšími matematickými partiami.

Po tomto úvode zaradíme matematicko-logickú hru \textit{Matematico}, ktorá študentov otestuje v rýchlom a optimálnom rozhodovaní sa.

\subsubsection*{Pravidlá}

Každý študent dostane tabuľku pozostávajúcu z $5\times5$ štvorčekov, do ktorej si bude zapisovať čísla, ktoré bude vedúci seminára postupne vyťahovať z balíčka. Balíček obsahuje 52 čísel, každé z čísel 1--13 sa v balíčku nachádza štyrikrát. Študenti majú vždy 7 sekúnd na to, aby číslo zapísali do práve jedného voľného políčka v tabuľke. Po vyplnení všetkých políčok si študenti spočítajú body podľa kľúča v tabuľke \ref{fujky}.
\begin{table}
\begin{tabular}{l l l}
Číselná kombinácia & v riadku alebo stĺpci & na uhlopriečke \\
\hline
dve zhodné čísla & 10 & 20\\
dva páry zhodných čísel & 20 & 40\\
tri zhodné čísla & 40 & 50 \\
tri zhodné čísla a dve zhodné čísla & 80 & 90 \\
štyri zhodné čísla & 160 & 170 \\
päť za sebou idúcich čísel & 50 & 60 \\
tri jednotky a dve trinástky & 100 & 110 \\
\end{tabular}
\caption{Body v hre \textit{Matematico}} \label{fujky}
\end{table}

Jednotlivé výsledky študentov píšeme na tabuľu, aby bolo možné vidieť rozsah nahraných bodov. Aby si sa študenti s hrou poriadne zoznámili, je vhodné zahrať aspoň dve alebo tri kolá. Po nich povzbudíme študentov, aby so spolužiakmi prediskutovali stratégiu, ktorú počas hry používajú, čo sa im overilo a čo nie, príp. podľa akého kľúča zapisujú čísla do tabuľky a či stratégiu v priebehu hry menia. Po výmene skúseností opäť odohráme dve alebo tri kolá a sledujeme, či sa priemerný počet nahraných bodov zvýšil.

Po poslednom kole vyzveme študentov, aby sa použitím čísel, ktoré boli v tomto kole vytiahnuté z balíčka, snažili maximalizovať bodový zisk, t.j. daných 25 čísel majú usporiadať do tabuľky tak, aby získali čo najviac bodov. Získané počty bodov pripíšeme na tabuľu porovnáme s predchádzajúcim kolom.

Zaujímavým pozorovaním je, že rozptyl bodov v takomto modifikovanom kole je značne menší ako v troch pôvodných. Môžeme sa so žiakmi porozprávať o dôvodoch, ktoré k tomu môžu viesť.

Na záver seminára môžeme študentov nechať maximalizovať bodový zisk použitím ľubovoľných čísel v ponuke a vyhlásiť súťaž o bonus. \\

Hra je dostupná aj online na \url{http://yetty.github.io/Matematico/}, prípadne na \url{http://matematico.cz/}

\subsubsection*{Variácie}

Študenti môžu \textit{Matematico} hrať aj v dvojiciach. Takéto usporiadanie ponúka možnosť diskutovať rozhodnutia, ktoré študenti robia a trénuje ich argumentačné a vysvetľovacie schopnosti.\\
\\
\textbf{Záverečný komentár}
Prvý seminár je časovo aj obsahovo menej hutný ako semináre nasledujúce, nie je to však nijako na škodu, keďže jeho zámer bol viac organizačný a informatívny, než matematický.
}
