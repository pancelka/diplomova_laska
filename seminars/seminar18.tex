\section*{Seminár 18}
\subsection*{Téma}
Algebraické výrazy a rovnice -- zložitejšie rovnice a ich systémy


\subsection*{Ciele}
Zoznámiť študentov s~ďalšími typmi rovníc a ich sústav (iracionálne koeficienty, dolná celá časť), tieto úlohy, spolu so slovnými úlohami precvičiť.


\subsection*{Úlohy a riešenia}
\begin{tcolorbox}[breakable,notitle,boxrule=0pt,colback=light-gray,colframe=light-gray]\ul{18.1} [59-S-1]
 Ak zväčšíme čitateľ aj menovateľ istého zlomku o~1, dostaneme zlomok o~hodnotu 1/20 väčší. Ak urobíme s~väčším zlomkom rovnakú operáciu, dostaneme zlomok o~hodnotu 1/12 väčší, ako bola hodnota zlomku na začiatku. Určte všetky tri zlomky.

\end{tcolorbox}

\rieh Označme $a/b$ pôvodný zlomok. Podľa zadania platia rovnosti
$$\frac{a+1}{b+1}-\frac{a}{b}=\frac{1}{20} \ \ \ \ \text{a} \ \ \ \ \frac{a+2}{b+2}-\frac{a}{b}=\frac{1}{12} \ \ \ \ (a,b\in \NN),$$
ktoré sú ekvivalentné so vzťahmi
$$20b(a + 1) - 20a(b + 1) = b(b + 1) \ \ \ \ \text{a} \ \ \ \ 12b(a + 2) - 12a(b + 2) = b(b + 2).$$
Tie upravíme na tvar $19b - 20a = b^2$ a $22b - 24a = b^2$. Po odčítaní oboch vzťahov zistíme, že $4a = 3b$, čo po dosadení do druhej rovnosti dá $22b - 18b = b^2$, čiže $b^2 = 4b$. Vzhľadom na podmienku $b\neq 0$ odtiaľ vyplýva $b = 4$ a $a = 3$.

Hľadané zlomky sú teda $\frac{3}{4}, \frac{4}{5}$ a $\frac{5}{6}$.\\
\\
\textbf{Iné riešenie.} Označme $a/b$ pôvodný zlomok. Zo vzťahov
$$\frac{1}{20}=\frac{1}{4\cdot5} \ \ \ \ \textrm{a} \ \ \ \  \frac{1}{12}=\frac{1}{4\cdot 3}=\frac{2}{4 \cdot 6} $$
možno odhadnúť, že riešením by mohlo byť $b = 4$. Potom
$$ \frac{4(a + 1) - 5a}{4\cdot 5}= \frac{1}{20} \ \ \ \ \textrm{a} \ \ \ \  \frac{4(a + 2) - 6a}{4\cdot 6}=\frac{1}{12},$$
čiže $a = 3$. Musíme sa však ešte presvedčiť, že úloha iné riešenie nemá. Podmienky úlohy vedú ku vzťahom
$$\frac{b-a}{b(b + 1)}=\frac{1}{4\cdot 5} \ \ \ \ \textrm{a} \ \ \ \ \frac{2(b-a)}{b(b+2)}=\frac{2}{4\cdot 6}.$$
Z~podielu ich ľavých a pravých strán potom vyplýva
$$ \frac{b+2}{b+1}=\frac{6}{5},$$ čomu vyhovuje jedine $b = 4$.\\
\textit{Poznámka.} V~úplnom riešení nesmie chýbať vylúčenie možnosti $b\neq 4$. Napríklad z~podobných rovností $1/20 = 30/24 \cdot 25$ a $1/12 = 52/24 \cdot 26$ by sme mohli hádať, že $b = 24$, čo riešením nie je.\\
\\
\kom V~prípade tejto úlohy je dôležité na začiatku správne zostaviť rovnosti. Ďalej je potrebné rovnosti vhodne upraviť. Úloha sa dá vyriešiť aj dosadzovacou metódou, tá však vedie k~riešeniu kvadratickej rovnice, ktoré mnohí študenti na klasických hodinách ešte nepreberali. Preto je vhodné študentov upozorniť na trik s~odčítaním rovníc.\\
\\
\kom V~nasledujúcej pasáži sa študenti zoznámia s~funkciou dolná celá časť. Najprv vyriešia niekoľko pomocných úloh a na záver sa pustia do boja s~úlohou domáceho kola.\\
\\
\begin{tcolorbox}[breakable,notitle,boxrule=0pt,colback=light-gray,colframe=light-gray]\ul{18.2} [59-I-3-N1] Určte $\lfloor 0 \rfloor, \lfloor 3{,}5 \rfloor,\lfloor 2{,}1\rfloor, \lfloor -4 \rfloor, \lfloor -3{,}9 \rfloor, \lfloor -0{,}2\rfloor$. Symbol $\lfloor x\rfloor$ označuje najväčšie celé číslo, ktoré nie je väčšie ako číslo $x$, tzv. dolnú celú časť reálneho čísla $x$.

\end{tcolorbox}

\rie $\lfloor 0 \rfloor = 0, \lfloor 3{,}5 \rfloor = 3,\lfloor 2{,}1\rfloor =2, \lfloor -4 \rfloor = -4, \lfloor -3{,}9 \rfloor =-4, \lfloor -0{,}2\rfloor =-1.$\\
\\
\begin{tcolorbox}[breakable,notitle,boxrule=0pt,colback=light-gray,colframe=light-gray]\ul{18.3} [59-I-3-N2] Nech $a$ je celé číslo a $t \in \langle 0; 1)$. Určte $\lfloor a \rfloor, \lfloor a+t \rfloor,\lfloor a+\frac{1}{2}t\rfloor, \lfloor a-t \rfloor, \\ \lfloor a+2t \rfloor, \lfloor a-2t\rfloor$.

\end{tcolorbox}

\rie $\lfloor a \rfloor = a, \lfloor a+t \rfloor= a$, $\lfloor a+\frac{1}{2}t\rfloor = a$, $\lfloor a-t \rfloor = a$, ak $t= 0$, resp. $\lfloor a-t \rfloor = a-1$, ak $t\neq 0$, $\lfloor a+2t \rfloor = a$, ak $t<0,5$, resp. $\lfloor a+2t \rfloor = a+1$, ak $t\geq0,5$, $\lfloor a-2t\rfloor=a$, ak $t=0$, resp. $\lfloor a-2t\rfloor=a-1$ ak $t\leq 0,5$ a $\lfloor a-2t\rfloor=a-2$ ak $t>0,5$. \\
\\
\begin{tcolorbox}[breakable,notitle,boxrule=0pt,colback=light-gray,colframe=light-gray]\ul{18.4} [59-I-3]
Určte všetky reálne čísla $x$, ktoré vyhovujú rovnici $4x - 2\lfloor x\rfloor = 5$.

\end{tcolorbox}

\rieh Položme $\lfloor x\rfloor = a$, potom $x = a + t$, pričom $t \in \langle 0, 1)$, a rovnicu $4(a + t)- 2a = 5$ ekvivalentne upravme na tvar $a=\frac{5}{2}- 2t$. Aby bolo číslo $a$ celé, musí byť $2t = k~\cdot\frac{1}{2}$, pričom $k$ je nepárne číslo. Navyše $2t \in \langle 0, 2)$. Teda buď $2t =\frac{1}{2}$ a $a = 2$, alebo $2t =\frac{3}{2}$ a $a = 1$. Pôvodná rovnica má preto dve riešenia: $x_1 = 2,25$ a $x_2 = 1,75$.\\
%\textbf{Iné riešenie.} Rovnicu upravíme na tvar $2x - \frac{5}{1}=\lfloor x \rfloor$. Jej riešením sú $x$-ové súradnice priesečníkov grafov funkcií $l: y = 2x - \frac{5}{2}$ a $p: y = \lfloor x \rfloor$. Grafy sa pretínajú v dvoch bodoch, ako vidíme na obr. XXX FIX ME. Pre prvý priesečník platí $\lfloor x \rfloor = 1$. Po dosadení do pôvodnej rovnice dostaneme $4x-2 = 5$ a odtiaľ $x_1 =\frac{7}{4}= 1,75$. Pre druhý priesečník platí $\lfloor x\rfloor = 2$, takže $4x - 4 = 5$ a $x_2 = \frac{9}{4}= 2,25.$
\\
\textbf{Iné riešenie.} Rovnicu upravíme na tvar $2x - \frac{5}{2} = \lfloor x\rfloor$. Taká rovnica bude splnená práve vtedy, keď číslo $2x - \frac{5}{2}$ bude celé a bude spĺňať nerovnosti $x - 1 < 2x - \frac{5}{2} \leq x$, ktoré sú ekvivalentné s~podmienkou $\frac{3}{2} < x \leq \frac{5}{2}$. Pre takéto $x$ zrejme hodnoty výrazu $2x - \frac{5}{2}$ vyplnia interval $( \frac{1}{2}, \frac{5}{2} \rangle $. V~ňom ležia práve dve celé čísla 1 a 2, teda hľadané $x$ nájdeme z~rovníc $2x - \frac{5}{2} = 1$ a $2x - \frac{5}{2} = 2$.\\
\\
\kom Aj napriek tomu, že funkcia dolná celá časť nie je bežným učivom preberaným v~školách, nemala by analýza úlohy robiť žiakom veľké problémy.\\
\\
\begin{tcolorbox}[breakable,notitle,boxrule=0pt,colback=light-gray,colframe=light-gray]\ul{18.5} [57-I-3-N1] Určte všetky celé čísla $n$, pre ktoré nadobúda zlomok $(4n + 27)/(n + 3)$ celočíselné
hodnoty.

\end{tcolorbox}

\rie Zlomok $(4n+27)/(n+3)$ upravíme na tvar $n+15/(n+3)$, teda číslo $n+3$ musí deliť 15. Z~toho dostávame $n \in \{-18,-8,-6,-4,-2, 0, 2,12 \}$.\\
\\
\begin{tcolorbox}[breakable,notitle,boxrule=0pt,colback=light-gray,colframe=light-gray]\ul{18.6} [57-I-3]
Máme určitý počet krabičiek a určitý počet guľôčok. Ak dáme do každej krabičky práve jednu guľôčku, ostane nám $n$ guľôčok. Keď však necháme práve $n$ krabičiek bokom, môžeme všetky guľôčky rozmiestniť tak, aby ich v~každej zostávajúcej krabičke bolo práve $n$. Koľko máme krabičiek a koľko guľôčok?

\end{tcolorbox}

\rie  Keď označíme $x$ počet krabičiek a $y$ počet guľôčok, dostaneme zo zadania sústavu rovníc
$$x + n = y \ \ \ \ \textrm{a} \ \ \ \ (x - n) \cdot n = y\ \ \ \ (1) $$
s~neznámymi $x$, $y$ a $n$ z~oboru prirodzených čísel. Vylúčením neznámej $y$ dostaneme rovnicu $x + n = (x - n) \cdot n$, ktorá pre $n = 1$ nemá riešenie. Pre $n \geq 2$ dostaneme
$$ x =\frac{n^2+n}{n-1}=n+2+\frac{2}{n-1}, \ \ \ \ (2)$$
odkiaľ vidíme, že (prirodzené) číslo $n - 1$ musí byť deliteľom čísla 2. Teda $n \in \{2, 3\}$.
Prípustné hodnoty $n$ dosadíme do (1) a sústavu vyriešime (možno tiež využiť vzťah (2)). Pre $n = 2$ dostaneme $x = 6, y = 8$ a pre $n = 3$ určíme $x = 6$ a $y = 9$.

\textit{Skúška.} Majme šesť krabičiek a osem guľôčok. Keď do každej krabičky dáme práve jednu guľôčku, ostane $n = 2$ guľôčok. Keď však odoberieme dve krabičky, môžeme do zostávajúcich štyroch rozdeliť guľôčky práve po dvoch. Podmienky úlohy sú teda splnené. Pre šesť krabičiek a deväť guľôčok urobíme skúšku rovnako ľahko.

\textit{Záver.} Buď máme šesť krabičiek a osem guľôčok, alebo šesť krabičiek a deväť guľôčok.\\
\\
\kom Úloha, spolu s~úlohou predchádzajúcou, je bežnou slovnou úlohou vedúcou na sústavu rovníc. Jej úspešné vyriešenie však vyžaduje umnú manipuláciu s~výrazmi.\\
\\
\begin{tcolorbox}[breakable,notitle,boxrule=0pt,colback=light-gray,colframe=light-gray]\ul{18.7} [57-II-4]
Nájdite všetky trojice celých čísel $x, y, z$, pre ktoré platí
$$x+y\sqrt{3}+z\sqrt{7}=y+z\sqrt{3}+x\sqrt{7}. $$
\end{tcolorbox}

\rieh Rovnicu prepíšeme na tvar
$$x-y=(z-y)\sqrt{3}+(x-z)\sqrt{7}$$
a umocníme. Po jednoduchej úprave dostaneme
$$(x - y)^2 - 3(z - y)^2 - 7(x - z)^2 = 2(x - z)(z - y)\sqrt{21}. \ \ \ \ \ \ \ (1)$$
Pre $x \neq z$ a $y \neq z$ nemôže rovnosť (1) platiť, pretože jej pravá strana je v~takom prípade číslo iracionálne, zatiaľ čo ľavá strana je číslo celé. Rovnosť teda môže nastať, len keď
$x = z$ alebo $y = z$.

V~prvom prípade po dosadení $x = z$ do pôvodnej rovnice dostaneme $z-y =\sqrt{3}(z- y)$. Odtiaľ $z = y = x$.

V~druhom prípade, keď $y = z$, dôjdeme analogicky k~rovnakému výsledku.

\textit{Záver.} Riešením danej rovnice sú všetky trojice $(x, y, z) = (k, k, k)$, kde $k$ je ľubovoľné celé číslo.\\
\\
\kom Aj napriek tomu, že vzorové riešenie úlohy vyzerá zrozumiteľne, úloha riešiteľov krajských kôl potrápila (bola najhoršie hodnotenou úlohou daného krajského kola). Záludnosti sa ukrývajú vo vytyčovaní iracionálnych čísel a nie neznámych, vhodnej úprave rovnice a diskusii o~(i)racionalite oboch strán rovnice. \\
\\

\begin{tcolorbox}[breakable,notitle,boxrule=0pt,colback=light-gray,colframe=light-gray]\ul{18.8} [64-I-1]
Určte všetky dvojice $(x, y)$ reálnych čísel, ktoré vyhovujú sústave rovníc
\begin{align*}
\sqrt{(x + 4)^2} &= 4 - y,\\
\sqrt{(y - 4)^2} &= x + 8.
\end{align*}

\end{tcolorbox}

\rie Vzhľadom na to, že pre každé reálne číslo $a$ platí $\sqrt{a^2}= |a|$, je daná sústava rovníc ekvivalentná so sústavou rovníc
\begin{align*}
|x + 4| &= 4 - y,\\
|y - 4| &= x + 8.
\end{align*}

Z~prvej rovnice vidíme, že musí byť $4 - y \geq 0$, teda $y \leq 4$. V~druhej rovnici môžeme teda odstrániť absolútnu hodnotu. Dostaneme tak $$|y - 4| = 4 - y = x + 8,\ \mathrm{ t.\,j.}\ - y = x + 4.$$
Po dosadení za $x + 4$ do prvej rovnice dostaneme
$$|-y| = |y| = 4 - y.$$
Keďže $y \leq 4$, budeme ďalej uvažovať dva prípady.

Pre $0 \leq y \leq 4$ riešime rovnicu $y = 4 - y$, a teda $y = 2$. Nájdenej hodnote $y = 2$ zodpovedá po dosadení do druhej rovnice $x = -6$.

Pre $y < 0$ dostaneme rovnicu $-y = 4 - y$, ktorá však nemá riešenie.

\textit{Záver.} Daná sústava rovníc má práve jedno riešenie, a to $(x, y) = (-6, 2)$.\\

\textbf{Iné riešenie.} Odstránením absolútnych hodnôt v~oboch rovniciach,  t.\,j. rozborom štyroch možných prípadov, keď\\
a) $(x + 4 \geq 0) \wedge (y - 4 \geq 0), \ \mathrm{ t.\,j.} \ (x \geq -4) \wedge (y \geq 4)$,\\
b) $(x + 4 \geq 0) \wedge (y - 4 < 0), \ \mathrm{ t.\,j.} \ (x \geq -4) \wedge  (y < 4)$,\\
c) $(x + 4 < 0) \wedge (y - 4 \geq 0), \ \mathrm{ t.\,j.} \ (x < -4) \wedge  (y \geq 4)$,\\
d) $(x + 4 < 0) \wedge (y - 4 < 0), \ \mathrm{ t.\,j.} \ (x < -4) \wedge  (y < 4)$,\\
zistíme, že prípady a), b), c) nedávajú (vzhľadom na uvedené obmedzenia v~jednotlivých prípadoch) žiadne reálne riešenie. V~prípade d) potom dostaneme jediné riešenie $(x, y) = (-6, 2)$ danej sústavy.\\
\\
\kom V~úvode riešenia pripomenieme vzťah $\sqrt{a^2}=|a|$, ktorý nám pomôže transformovať sústavu zo zadania na sústavu rovníc s~absolútnou hodnotou, ktorú by študenti mali byť schopní bez väčších komplikácií vyriešiť.

\subsection*{Domáca práca}
\begin{tcolorbox}[breakable,notitle,boxrule=0pt,colback=light-gray,colframe=light-gray]\ul{18.9} [59-II-4] Určte všetky dvojice reálnych čísel $x, y$, ktoré vyhovujú sústave rovníc
\begin{align*}
\lfloor x + y\rfloor &= 2 010,\\
\lfloor x\rfloor - y &= p,
\end{align*}
ak a) $p = 2$, b) $p = 3$.
Symbol $\lfloor x \rfloor$ označuje najväčšie celé číslo, ktoré nie je väčšie ako dané reálne číslo $x$ (tzv. dolná celá časť reálneho čísla $x$).

\end{tcolorbox}

\rieh Keďže číslo $p$ je celé, je aj $y = \lfloor x \rfloor-p$ celé číslo a $\lfloor x+y \rfloor = \lfloor x\rfloor+y$. Pôvodná sústava rovníc je teda ekvivalentná so sústavou
\begin{align*}
\lfloor x \rfloor + y &= 2 010,\\
\lfloor x\rfloor - y &= p,
\end{align*}
ktorú ľahko vyriešime napríklad sčítacou metódou. Dostaneme $\lfloor x \rfloor = \frac{1}{2}(2 010 + p)$ (čo môže platiť len pre párne $p$) a $y = \lfloor x \rfloor - p$.

a) Pre $p = 2$ je riešením sústavy ľubovoľné $x \in \langle 1006, 1007)$ a $y = 1 004$.\\

b) Pre $p = 3$ nemá sústava žiadne riešenie.
\\
\\
\textbf{Iné riešenie.} Položme $\lfloor x \rfloor = a$, potom $x = a + t$, pričom $t \in \langle 0, 1)$.

a) Pre $p = 2$ sústavu prepíšeme na tvar $y = a-2$ a $\lfloor 2a-2+t \rfloor = 2 010$. Z~poslednej
rovnice vyplýva $2a - 2 = 2 010$, odtiaľ  $a = 1 006$. Keďže $t \in \langle 0, 1)$, vyhovuje pôvodnej sústave každé $x \in \langle 1006, 1007)$, pričom $y = 1 004$.

b) Pre $p = 3$ dostávame $y = a - 3$ a $\lfloor 2a - 3 + t\rfloor = 2 010$. Posledná rovnica je ekvivalentná so vzťahom $2a - 3 = 2 010$, ktorému nevyhovuje žiadne celé číslo $a$. Pre $p = 3$ nemá daná sústava rovníc riešenie.\\
\\
\begin{tcolorbox}[breakable,notitle,boxrule=0pt,colback=light-gray,colframe=light-gray]\ul{18.10} [64-S-1]
V~obore reálnych čísel vyriešte sústavu rovníc
\begin{align*}
|1 - x| &= y + 1,\\
|1 + y| &= z~- 2,\\
|2 - z| &= x - x^2.
\end{align*}
\end{tcolorbox}

\rieh  Pravá strana prvej rovnice je nezáporné číslo, čo sa premietne do druhej rovnice, pričom môžeme odstrániť absolútnu hodnotu. Aj pravá strana druhej rovnice je nezáporné číslo, čo sa s~využitím rovnosti $|z -2| = |2-z|$ premietne do tretej rovnice, pričom môžeme odstrániť absolútnu hodnotu. Daná sústava má potom tvar
\begin{align*}
|1 - x| &= y + 1,\\
1 + y &= z~- 2,\\
z~- 2 &= x - x^2
\end{align*}
a odtiaľ jednoduchým porovnaním dostávame rovnicu
$$|1 - x| = x - x^2.$$
Pre $x < 1$ dostaneme rovnicu $1-x = x-x^2$ čiže $(1-x)^2 = 0$, ktorej riešenie $x = 1$ ale predpokladu $x < 1$ nevyhovuje.

Pre $x \geq 1$ vyjde rovnica $x^2 = 1$; z~jej dvoch riešení $x = -1$ a $x = 1$ predpokladu $x \geq 1$ vyhovuje iba $x = 1$.

Z~danej sústavy potom jednoducho dopočítame hodnoty $y = -1$ a $z = 2$. Sústava má teda jediné riešenie $(x, y, z) = (1, -1, 2).$\\



