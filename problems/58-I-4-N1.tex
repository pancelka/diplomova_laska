% Do not delete this line (pandoc magic!)

\problem{58-I-4-N1}{seminar10,geomlah}{
Označme $U$ priesečník uhlopriečok daného konvexného štvoruholníka $ABCD$. Dokážte, že priamky $AB$ a $CD$ sú rovnobežné práve vtedy, keď trojuholníky $ADU$ a $BCU$ majú rovnaký obsah.
}{
\rie Rovnosť obsahov trojuholníkov $ADU$ a $BCU$ je ekvivalentná s~rovnosťou obsahov trojuholníkov $ABC$ a $ABD$ so spoločnou stranou $AB$, pretože $S_{ABC}=S_{ABU}+S_{BCU}$ a $S_{ABD}=S_{ABU}+S_{AUD}$ (obr.~\ref{fig:58I4N1}). Trojuholníky $ABC$ a $ABD$ majú spoločnú základňu $AB$, takže ich obsahy budú rovnaké práve vtedy, ak výšky na túto stranu budú rovnaké, resp. ak body $C$ a $D$ budú od priamky $AB$ rovnako vzdialené. To nastane len v~prípade, ak body $C$ a $D$ ležia na priamke rovnobežnej s~priamkou $AB$, čo sme chceli dokázať.
\begin{figure}[h]
    \centering
    \includegraphics{images/58I4N1\imagesuffix}
    \caption{}
    \label{fig:58I4N1}
\end{figure}
}
