% Do not delete this line (pandoc magic!)

\problem{\cite{herman2011}, príklad 2.3, str. 174}{}{
Nájdite všetky prvočísla, ktoré sú súčasne súčtom a rozdielom dvoch vhodných prvočísel.
}{
\rieh Predpokladajme, že prvočíslo $p$ je súčasne súčtom aj rozdielom dvoch prvočísel. Potom je však $p>2$ a teda je $p$ nepárne. Pretože je $p$ zároveň súčet aj rozdiel dvoch prvočísel, jedno z~nich musí byť vždy párne, teda 2. Takže hľadáme prvočísla $p, p_1, p_2$ tak, že $p=p_1+2=p_2-2$, teda $p_1, p, p-2$ sú tri po sebe idúce nepárne čísla a teda práve jedno z~nich je deliteľné troma (študenti by si mali rozmyslieť prečo). Avšak troma je deliteľné jediné prvočíslo 3, odkiaľ vzhľadom na to, že $p_1\geq 1$ vyplýva $p_1=3$, $p=5$ a $p_2=7$. Jediné prvočíslo vyhovujúce zadaniu je teda $p=5$.\\
\\
\kom Úloha, ktorá vyžaduje viac uvažovania, než tvrdého počítania, je zaujímavá práve jediným výsledkom.\\
\\
}
