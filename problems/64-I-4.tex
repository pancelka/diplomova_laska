% Do not delete this line (pandoc magic!)

\problem{64-I-4}{
Označme $E$ stred základne $AB$ lichobežníka $ABCD$, v~ktorom platí $|AB| : |CD| = 3 : 1$. Uhlopriečka $AC$ pretína úsečky $ED$, $BD$ postupne v~bodoch $F$, $G$. Určte postupný pomer $|AF| : |FG| : |GC|$.
}{
\rieh  Keďže v~zadaní aj v~otázke úlohy sú iba pomery, môžeme si dĺžky strán lichobežníka zvoliť ako vhodné konkrétne čísla. Zvoľme teda napr. $|AB| = 6$, potom $|AE| = |BE| = 3$ a $|CD| = 2$. Hľadané dĺžky označme $|AF| = x$, $|FG| = y$, $|GC| = z$. Tieto dĺžky sme vyznačili na obr.~\ref{fig:64I4}, taktiež aj tri dvojice zhodných uhlov, ktoré teraz využijeme pri úvahách o~trojuholníkoch podobných podľa vety $uu$.

Trojuholníky $ABG$ a $CDG$ sú podobné, preto $(x + y) : z~= 6 : 2 = 3 : 1$. Aj trojuholníky $AEF$ a $CDF$ sú podobné, preto $x : (y + z) = 3 : 2$.
\begin{figure}[h]
    \centering
    \includegraphics{images/64D4\imagesuffix}
    \caption{}
    \label{fig:64I4}
\end{figure}
Odvodené úmery zapíšeme ako sústavu rovníc
\begin{align*}
x + y - 3z &= 0,\\
2x - 3y - 3z &= 0.
\end{align*}
Ich odčítaním získame rovnosť $x = 4y$, čiže $x : y = 4 : 1$. Dosadením tohto výsledku do prvej rovnice dostaneme $5y = 3z$, čiže $y : z~= 3 : 5$. Spojením oboch pomerov získame výsledok $x : y : z~= 12 : 3 : 5$.\\
\\
\kom Úloha je výborným tréningom na hľadanie vhodných dvojíc podobných trojuholníkov tak, aby sme pomocou údajov zo zadania boli schopní určiť hľadaný pomer, keďže jedna dvojica trojuholníkov na nájdenie odpovede zjavne stačiť nebude. Okrem toho tiež pozorovania z~náčrtu vedú k~sústave dvoch rovníc, takže študenti uplatnia aj svoje algebraické zručnosti.\\
\\
}
