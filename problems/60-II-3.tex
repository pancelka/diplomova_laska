% Do not delete this line (pandoc magic!)

\problem{60-II-3}{
V lichobežníku ABCD má základňa $AB$ dĺžku 18\,cm a základňa $CD$ dĺžku 6\,cm. Pre bod $E$ strany $AB$ platí $2|AE| = |EB|$. Body $K$, $L$, $M$, ktoré sú postupne ťažiskami trojuholníkov $ADE$, $CDE$, $BCE$, tvoria vrcholy rovnostranného trojuholníka. 
\begin{enumerate}[a)]
    \item Dokážte, že priamky $KM$ a $CM$ zvierajú pravý uhol.
    \item Vypočítajte dĺžky ramien lichobežníka $ABCD$.
\end{enumerate}
}{
\rieh Štvoruholník $AECD$ je rovnobežník, pretože jeho strany $AE$ a $CD$ sú rovnobežné a rovnako dlhé (obe merajú 6\,cm). Na jeho uhlopriečke $AC$ tak leží ťažnica trojuholníka $ADE$ z vrcholu $A$ aj ťažnica trojuholníka $CDE$ z vrcholu $C$, a preto na tejto priamke ležia aj body $K$ a $L$ \todo{(obr. 1)}. Navyše vieme, že ťažisko trojuholníka delí jeho ťažnice v pomere $2 : 1$, preto sú úsečky $AK$, $KL$ a $LC$ rovnako dlhé.

\todo{DOPLNIŤ Obr. 1}

Bod $L$ je stredom úsečky $KC$, preto na osi súmernosti úsečky $KM$ leží nielen výška rovnostranného trojuholníka $KLM$, ale aj stredná priečka trojuholníka $KMC$. Preto je priamka $CM$ kolmá na $KM$. Ostáva vypočítať dĺžky ramien lichobežníka $ABCD$. Označme $P$ stred úsečky $EB$. Keďže $CM$ je kolmá na $KM$, je ťažnica $CP$ kolmá na $EB$, takže trojuholník $EBC$ je rovnoramenný, a teda aj daný lichobežník $ABCD$ je rovnoramenný. Dĺžku ramena $BC$ teraz vypočítame z pravouhlého trojuholníka $PBC$, v ktorom poznáme dĺžku odvesny $PB$. Pre druhú odvesnu $CP$ zrejme platí
$$|CP| = \frac{3}{2} |CM| = 3\cdot \frac{\sqrt{3}}{2}|KM|,$$
čo jednoducho vyplýva z vlastností trojuholníka $KMC$. A keďže z podobnosti trojuholníkov $KMC$ a $APC$ máme $|KM| =\frac{2}{3}|AP|$, dostávame (počítané v centimetroch)
$$|CP| = 3\cdot \frac{\sqrt{3}}{2}|KM| = 3\cdot \frac{\sqrt{3}}{2}\frac{2}{3}|AP| =\sqrt{3}\frac{2}{3}|AB| = 12\sqrt{3}.$$
Potom
$$|BC| =\sqrt{|PB|^2 + |PC|^2} =\sqrt{36 + 12^2\cdot3} = 6\sqrt{1 + 12} = 6\sqrt{13}.$$
Ramená daného lichobežníka majú dĺžku $6\sqrt{13}$\,cm.

\textit{Alternatívny dôkaz kolmosti priamok KM a CM}. Keďže bod $L$ je stredom úsečky $KC$ a zároveň $|LK| = |LM|$, lebo trojuholník $KLM$ je rovnostranný, leží bod $M$ na Tálesovej kružnici nad priemerom $KC$, takže trojuholník $KMC$ je pravouhlý.
}
