% Do not delete this line (pandoc magic!)

\problem{B-65-S-2}{
Daná je úsečka $AB$, jej stred $C$ a vnútri úsečky $AB$ bod $D$. Kružnice $k(C, |BC|)$ a $m(B, |BD|)$ sa pretínajú v bodoch $E$ a $F$ a polpriamka $FD$ pretína kružnicu $k$ v bode $K$, $K \neq F$. Rovnobežka s priamkou $AB$ prechádzajúca bodom $K$ pretína kružnicu $k$ v bode $L$, $L\neq K$. Dokážte, že $|KL| = |BD|$. 
}{
\rieh Body $D$ aj $F$ ležia na kružnici $m$ so stredom $B$, takže trojuholník $BDF$ je rovnoramenný a platí $|\ma BFK| = |\ma BDF| > 45^\circ$, pretože trojuholník $BDF$ je navyše ostrouhlý \todo{doplniť (obr. 1)}. To ale znamená, že bod $K$ musí ležať v polrovine $oA$, pričom $o$ je os úsečky $AB$, pretože oblúk $BK$ prislúcha obvodovému uhlu väčšiemu ako $45^\circ$.

Bod $L$, ktorý je vďaka podmienke $AB \parallel KL$ súmerne združený s $K$ podľa $o$, bude preto ležať v polrovine $oB$, teda $KL$ a $AB$ (a teda aj $DB$) budú súhlasne orientované rovnobežné úsečky. Z toho vyplýva zhodnosť súhlasných uhlov $LKF$ a $BDF$. Spolu tak dostávame $|\ma LKF| = |\ma BDF| = |\ma BFK|$. 

\todo{DOPLNIŤ Obr. 1}

Priamky $FB$ a $KL$ sú potom súmerne združené podľa osi úsečky $FK$, ktorá prechádza stredom $C$ kružnice $k$, a je teda aj jej osou súmernosti. Preto aj priesečníky $B$ a $L$ týchto priamok s kružnicou $k$ sú súmerne združené podľa tejto osi, takže štvoruholník $KLBF$ je rovnoramenný lichobežník, čiže $|KL| = |BF|$. Spojením s rovnosťou $|BF|= |BD|$ polomerov kružnice $m$ tak dostávame požadovanú rovnosť $|KL| = |BF| = |BD|$.
}
