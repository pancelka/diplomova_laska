% Do not delete this line (pandoc magic!)

\problem{60-I-2}{
Dokážte, že výrazy $23x + y$, $19x + 3y$ sú deliteľné číslom 50 pre rovnaké dvojice prirodzených čísel $x$, $y$.
}{
\rieh Predpokladajme, že pre dvojicu prirodzených čísel $x, y$ platí $50 \mid 23x + y$. Potom pre nejaké prirodzené číslo $k$ platí $23x + y = 50k$. Z~tejto rovnosti dostaneme $y = 50k - 23x$, čiže $19x + 3y = 19x + 3(50k - 23x) = 150k - 50x = 50(3k - x)$, takže číslo $19x + 3y$ je násobkom čísla 50.

Podobne to funguje aj z~druhej strany. Ak pre nejakú dvojicu prirodzených čísel $x,y$ platí $50 \mid 19x + 3y$, tak $19x + 3y = 50l$ pre nejaké prirodzené číslo $l$. Z~tejto rovnosti vyjadríme číslo~$y$; dostaneme $y = (50l - 19x)/3$ (ďalší postup by bol podobný, aj keby sme vyjadrili $x$ namiesto~$y$). Po dosadení dostaneme $$23x + y = 23x + \frac{50l - 19x}{3}=\frac{69x + 50l - 19x}{3}=\frac{50 \cdot (x + l)}{3}.$$
O~výslednom zlomku vieme, že je to prirodzené číslo. Čitateľ tohto zlomku je deliteľný číslom 50. V~menovateli je len číslo 3, ktoré je nesúdeliteľné s~50, preto sa číslo 50 nemá s~čím z~menovateľa vykrátiť a teda číslo $23x + y$ je deliteľné 50.\\
\\
\textbf{Iné riešenie*.} Zrejme $3 \cdot (23x + y) - (19x + 3y) = 50x$, čiže ak 50 delí jedno z~čísel $23x + y$ a $19x + 3y$, tak delí aj druhé z~nich.\\
}
