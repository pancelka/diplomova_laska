% Do not delete this line (pandoc magic!)

\problem{66-I-3}{
Päta výšky z~vrcholu $C$ v~trojuholníku $ABC$ delí stranu $AB$ v~pomere $1 : 2$. Dokážte, že pri zvyčajnom označení dĺžok strán trojuholníka $ABC$ platí nerovnosť $$3|a - b| < c.$$
}{
\rieh Päta $D$ uvažovanej výšky je podľa zadania tým vnútorným bodom strany $AB$, pre ktorý platí $|AD| = 2|BD|$ alebo $|BD| = 2|AD|$. Obe možnosti sú znázornené na \ref{fig:66I3_1} s~popisom dĺžok strán $AC$, $BC$ a oboch úsekov rozdelenej strany $AB$.
\begin{figure}[h]
    \centering
    \includegraphics{images/66D31\imagesuffix}
    \caption{}
    \label{fig:66I3_1}
\end{figure}
Pytagorova veta pre pravouhlé trojuholníky $ACD$ a $BCD$ vedie k~dvojakému vyjadreniu druhej mocniny spoločnej odvesny $CD$, pričom v~situácii naľavo dostaneme
$$|CD|^2= b^2- \bigg(\frac{2}{3}c\bigg)^2= a^2 - \bigg(\frac{1}{3}c\bigg)^2,$$
odkiaľ po jednoduchej úprave poslednej rovnosti dostaneme vzťah
$$3(b^2 - a^2) = c^2.$$
Pre druhú situáciu vychádza analogicky
$$3(a^2 -b^2) = c^2.$$
Závery pre obe možnosti možno zapísať jednotne ako rovnosť s~absolútnou hodnotou
$$3|a^2 - b^2 | = c^2.$$
Ak použijeme rozklad $|a^2 - b^2 | = |a - b|(a + b)$ a nerovnosť $c < a + b$ (ktorú ako je známe spĺňajú dĺžky strán každého trojuholníka $ABC$), dostaneme z~odvodenej rovnosti
$$3|a - b|c < 3|a - b|(a + b) = c^2,$$
odkiaľ po vydelení kladnou hodnotou $c$ dostaneme $3|a - b| < c$, ako sme mali dokázať. Zdôraznime, že nerovnosť $3|a-b|c < 3|a-b|(a+b)$ sme správne zapísali ako ostrú -- v~prípade $a = b$ by síce prešla na rovnosť, avšak podľa nášho odvodenia by potom platilo $c^2 = 0$, čo odporuje tomu, že ide o~dĺžku strany trojuholníka.\\
\\
\textbf{Iné riešenie*.} Nerovnosť, ktorú máme dokázať, možno po vydelení tromi zapísať bez
absolútnej hodnoty ako dvojicu nerovností
$$-\frac{1}{3}c < a - b < \frac{1}{3}c.$$
Opäť ako v~pôvodnom riešení rozlíšime dve možnosti pre polohu päty $D$ uvažovanej výšky a ukážeme, že vypísanú dvojicu nerovností možno upresniť na tvar
$$-\frac{1}{3} < a - b < 0,\ \ \ \ \text{respektíve} \ \ \ \  0 < a - b <\frac{1}{3}c,$$
podľa toho, či nastáva situácia z~ľavej či pravej časti~\ref{fig:66I3_1}.

Pre situáciu z~\ref{fig:66I3_1} naľavo prepíšeme avizované nerovnosti $-\frac{1}{3}c < a - b < 0$ ako $a < b < a +\frac{1}{3}c$ a odvodíme ich z~pomocného trojuholníka $ACE$, pričom $E$ je stred úsečky $AD$, takže body $D$ a $E$ delia stranu $AB$ na tri zhodné úseky dĺžky $\frac{1}{3}c$.
\begin{figure}[h]
    \centering
    \includegraphics{images/66D32\imagesuffix}
    \caption{}
    \label{fig:66I3_2}
\end{figure}
V~\ref{fig:66I3_2} sme rovno vyznačili, že úsečka $EC$ má dĺžku $a$ ako úsečka $BC$, a to v~dôsledku zhodnosti trojuholníkov $BCD$ a $ECD$ podľa vety $sus$. Preto je pravá z~nerovností $a < b < a +\frac{1}{3}c$ porovnaním dĺžok strán trojuholníka $ACE$, ktoré má všeobecnú platnosť.

Ľavú nerovnosť $a < b$ odvodíme z~druhého všeobecného poznatku, že totiž v~každom trojuholníku oproti väčšiemu vnútornému uhlu leží dlhšia strana. Stačí nám teda zdôvodniť, prečo pre uhly vyznačené na~\ref{fig:66I3_2} platí $|\ma CAE| < |\ma AEC|$. To je však jednoduché: zatiaľ čo uhol $CAE$ je vďaka pravouhlému trojuholníku $ACD$ ostrý, uhol $AEC$ je naopak tupý, pretože k~nemu vedľajší uhol $CED$ je ostrý vďaka pravouhlému trojuholníku $CED$.

Pre prípad situácie z~\ref{fig:66I3_1} napravo možno predchádzajúci postup zopakovať s~novým bodom $E$, tentoraz stredom úsečky $BD$. Môžeme však vďaka súmernosti podľa osi $AB$ konštatovať, že z~dokázaných nerovností $-\frac{1}{3}c < a - b < 0$ pre situáciu naľavo vyplývajú nerovnosti $-\frac{1}{3}c < b - a < 0$ pre situáciu napravo, z~ktorých po vynásobení číslom $-1$ dostaneme práve nerovnosti $0 < a - b <\frac{1}{3}$, ktoré sme mali v~druhej situácii dokázať.\\
\\
\kom Nosným prvkom úlohy je opäť Pytagorova veta, väčšiu pozornosť však vyžaduje rozbor úlohy, keďže päta výšky sa môže nachádzať v~dvoch rôznych polohách.\\
\\
}
