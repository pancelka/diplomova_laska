% Do not delete this line (pandoc magic!)

\problem{66-S-3}{}{
Päta $P$ výšky z~vrcholu $C$ v~trojuholníku $ABC$ delí stranu $AB$ v~pomere $|AP| : |PB|= 1 : 3$. V~rovnakom pomere sú aj obsahy štvorcov nad jeho stranami $AC$ a $BC$.
Dokážte, že trojuholník $ABC$ je pravouhlý.
}{
\rieh Označme $d$ dĺžku úsečky $AP$ a $v$ dĺžku výšky $CP$ trojuholníka $ABC$. Dĺžky jeho strán označíme zvyčajným spôsobom $a, b, c$. Zo zadania teda vyplýva $|PB| = 3d$.
\begin{figure}
    \centering
    \includegraphics{images/66S3\imagesuffix}
    \caption{}
    \label{fig:66S3}
\end{figure}

Použitím Pytagorovej vety v~trojuholníkoch $APC$ a $PBC$ dostávame rovnosti $b^2= d^2 +v^2$ a $a^2 = 9d^2 +v^2$. Z~druhého predpokladu úlohy potom vyplýva rovnosť $a^2 = 3b^2$, čiže $9d^2 + v^2 = 3d^2 + 3v^2$, odkiaľ $v^2 = 3d^2$. Dosadením do prvých dvoch rovností tak dostávame $a^2 = 12d^2$ a $b^2 = 4d^2$. A~keďže $c = 4d$, čiže $c^2 = 16d^2$, dokázali sme, že pre dĺžky strán trojuholníka $ABC$ platí $a^2 + b^2 = c^2$.

Trojuholník $ABC$ je preto podľa obrátenej Pytagorovej vety pravouhlý.\\
\\
\textit{Poznámka.} Ak zvážime pomocný pravouhlý trojuholník s~odvesnami $a$ a $b$, tak pre jeho preponu~$c'$ podľa Pytagorovej vety platí $c' = a^2 + b^2$. Porovnaním s~odvodenou rovnosťou $c^2 = a^2 + b^2$ tak dostávame $c'= c$, takže pôvodný trojuholník je podľa vety $sss$ zhodný s~trojuholníkom pomocným, a je teda skutočne pravouhlý. Môžeme tolerovať názor, že samotná Pytagorova veta udáva nielen nutnú, ale aj postačujúcu podmienku na to, aby bol daný trojuholník pravouhlý.\\
\\
\kom Úloha relatívne priamočiaro využíva viacnásobné využitie Pytagorovej vety, je tak vhodným zahrievacím problémom tohto seminára.\\
\\
}
