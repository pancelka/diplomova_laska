% Do not delete this line (pandoc magic!)

\problem{66-I-4-N3}{seminar06,rovnice,sustavy,mnohocleny,domacekolo}{
Určte všetky dvojčleny $P (x) = ax+b$ s~celočíselnými koeficientmi $a$ a $b$, pre ktoré platí $P (1) < P (2)$ a $P (1)^2+ P(2)^2= 5$.
}{
\rieh Keďže $a$ a $b$ sú podľa zadania celé čísla, budú celými číslami aj hodnoty $P(1)$ a $P(2)$. Preto hľadáme, akými spôsobmi sa dá číslo 5 zapísať ako súčet dvoch druhých mocnín celých čísel. Ak neberieme ohľad na poradie sčítancov, je taký spôsob jediný: $5 = (\pm 1)^2+ (\pm 2)^2$. Zároveň vieme, že $P(1)<P(2)$, preto dvojicu $(P(1), P(2))$ tvoria niektoré z~nasledujúcich štyroch možností: $(1, 2), (-1, 2), (-2, -1), (-2, 1)$. Každá z~týchto štyroch dvojíc podmienok potom vedie k~sústave dvoch rovníc s~dvomi neznámymi, takže dostávame objemnejšiu variáciu prvej úlohy tohto seminára. Vyriešením systémov získame 4 vyhovujúce dvojčleny $x + 0$, $3x - 4$, $x - 3$ a $3x - 5$.\\
\\
\kom Úloha využíva takmer rovnaký princíp ako prvé dve seminárne úlohy, vyžaduje však dodatočnú analýzu plynúcu z~poslednej podmienky, čo úlohe pridáva na náročnosti. Poslednou úlohou tejto gradovanej série je domáca práca, ktorej analýza povedie k~riešeniu niekoľkých sústav troch rovníc s~tromi neznámymi.\\
\\
}
