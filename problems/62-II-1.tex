% Do not delete this line (pandoc magic!)

\problem{62-II-1}{}{
V tanečnej sa zišla skupina chlapcov a dievčat. Každý z prítomných 15 chlapcov pozná práve 4 dievčatá a každé dievča pozná práve 10 chlapcov. (Známosti sú vzájomné.) Dokážte, že ľubovoľní dvaja chlapci majú aspoň dve spoločné známe.
}{
\rieh Do každej známosti vstupuje práve jeden chlapec a každý z chlapcov má práve štyri známosti, spolu teda v tanečnej existuje $15 \cdot 4 = 60$ známostí. V každej známosti je však zastúpené práve jedno dievča a každé dievča má práve desať známostí. Ak označíme $d$ počet dievčat, tak $10 \cdot d = 60$. V tanečnej je teda 6 dievčat. Uvažujme ľubovoľného z chlapcov, povedzme Tomáša. Tomáš pozná 4 dievčatá, v tanečnej sú teda iba dve dievčatá, ktoré Tomáš nepozná. Ľubovoľný ďalší chlapec však pozná tiež štyri dievčatá, musí tak poznať aspoň dve z dievčat, ktoré pozná Tomáš.
}