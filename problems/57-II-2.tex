% Do not delete this line (pandoc magic!)

\problem{57-II-2}{}{
Klárka urobila chybu pri písomnom násobení dvoch dvojciferných čísel, a tak jej vyšlo číslo o 400 menšie, ako bol správny výsledok. Pre kontrolu vydelila číslo, ktoré dostala, menším z násobených čísel. Tentoraz počítala správne a vyšiel jej neúplný podiel 67 a zvyšok 56. Ktoré čísla Klárka násobila?
}{
\rie Označme $x$ menšie a $y$ väčšie z násobených čísel. Podľa zadania platí $xy - 400 = 67x + 56$, čiže
$$x(y - 67) = 456. (1)$$
Číslo $x$ je teda dvojciferný deliteľ čísla $456 = 2^3\cdot 3 \cdot 19$. Zo zadania navyše vyplýva,
že číslo $x$ je väčšie ako príslušný zvyšok 56. Najmenšie také $x$ je $x = 3 \cdot 19 = 57$. Pre
každý ďalší taký deliteľ platí $x \geq 4 \cdot 19 = 76$ a $y - 67 \leq 2 \cdot 3 = 6$, takže $y \leq 73 < x$, čo odporuje zvolenému označeniu $x < y$. Teda $x = 57$ a $y = 75$. Ľahko overíme, že tieto čísla vyhovujú zadaniu úlohy.

\textit{Záver.} Klárka násobila čísla 57 a 75.
}