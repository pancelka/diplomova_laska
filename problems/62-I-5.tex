% Do not delete this line (pandoc magic!)

\problem{62-I-5}{}{
Určte všetky celé čísla $n$, pre ktoré $2n^3 -3n^2 +n+3$ je prvočíslo.
}{
\rieh Ukážeme, že jedinými celými číslami, ktoré vyhovujú úlohe, sú $n = 0$ a $n = 1$.

Upravme najskôr výraz $V = 2n^3 - 3n^2 + n + 3$ nasledujúcim spôsobom:
$$V = (n^3 - 3n^2+ 2n) + (n^3 - n) + 3 = (n - 2)(n - 1)n + (n - 1)n(n + 1) + 3.$$
Oba súčiny $(n-2)(n-1)n$ a $(n-1)n(n+1)$ v upravenom výraze $V$ sú deliteľné tromi pre každé celé číslo $n$ (v oboch prípadoch sa jedná o súčin troch po sebe idúcich celých čísel), takže výraz $V$ je pre všetky celé čísla $n$ deliteľný tromi. Hodnota výrazu $V$ je preto prvočíslom práve vtedy, keď $V = 3$, teda práve vtedy, keď súčet oboch spomenutých súčinov je rovný nule:
$$0 = (n - 2)(n - 1)n + (n - 1)n(n + 1) = n(n - 1)[(n - 2) + (n + 1)] = n(n - 1)(2n - 1).$$
Poslednú podmienku však spĺňajú iba dve celé čísla $n$, a to $n = 0$ a $n = 1$. Tým je úloha vyriešená.
 
\textit{Poznámka.} Fakt, že výraz $V$ je deliteľný tromi pre ľubovoľné celé $n$, môžeme odvodiť aj tak, že doňho postupne dosadíme $n = 3k$, $n = 3k + 1$ a $n = 3k + 2$, pričom $k$ je celé číslo, rozdelíme teda všetky celé čísla $n$ na tri skupiny podľa toho, aký dávajú zvyšok po delení tromi.\\
\\
\kom Aj keď vzorové riešenie môže vyzerať trikovo, po vyskúšaní niekoľko málo hodnôt $n$ je vždy hodnota zo zadania deliteľná 3, čo by študentov mohlo priviesť na myšlienku skúsiť dokázať deliteľnosť čísla zo zadania tromi.


\textit{Poznámka.} Fakt, že výraz $V$ je deliteľný tromi pre ľubovoľné celé $n$, môžeme odvodiť aj tak, že doňho postupne dosadíme $n = 3k$, $n = 3k + 1$ a $n = 3k + 2$, pričom $k$ je celé číslo, rozdelíme teda všetky celé čísla $n$ na tri skupiny podľa toho, aký dávajú zvyšok po delení tromi.
}
