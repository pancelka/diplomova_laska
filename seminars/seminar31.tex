\seminar{31}{Geometria VII -- stredové, obvodové, úsekové uhly, tetivové štvoruholníky}

\teachernote{
\subsection*{Ciele}

Zopakovať, príp. študentov zoznámiť s vlastnosťami stredových, obvodových a úsekových uhlov a ich využitím pri riešení úloh.

\subsubsection*{Úvodný komentár}

Predtým, ako sa so študentmi pustíme do riešenia úloh, je vhodné predstaviť, príp. spoločne zopakovať vlastnosti uhlov, ktorými sa budeme v seminári zaoberať. Vhodným materiálom je \cite{kadlecek1996}, kapitola 8.
}

\subsection*{Úlohy a riešenia}

%\todo{DOPLNIŤ komentáre.}

%% Do not delete this line (pandoc magic!)

\problem{B-65-I-5-D1}{}{
Daná je tetiva $AB$ kružnice $k$ so stredom v bode $S$. Na úsečke $AB$ zvoľme bod $M$ a priesečník kružnice opísanej trojuholníku $AMS$ s kružnicou k označme $C$. Dokážte, že uhly $MCS$ a $MBS$ sú zhodné.
}{
\rieh Stačí využiť rovnosť uhlov v rovnoramennom trojuholníku $ABS$ a obvodové uhly nad $MS$ v kružnici opísanej trojuholníku $AMS$.
}


%\kom Úloha je veľmi jednoduchá, preto ju považujeme skôr za rozcvičku ako plnohodnotný matematický oriešok. Pekne však demonštruje to

% Do not delete this line (pandoc magic!)

\problem{B-66-II-3}{}{
V rovine sú dané kružnice $k$ a $l$, ktoré sa pretínajú v bodoch $E$ a $F$. Dotyčnica ku kružnici $l$ zostrojená v bode $E$ pretína kružnicu $k$ v bode $H$ ($H \neq E$). Na oblúku $EH$ kružnice $k$, ktorý neobsahuje bod $F$, zvoľme bod $C$ ($E \neq C \neq H$) a priesečník priamky $CE$ s kružnicou $l$ označme $D$ ($D \neq E$). Dokážte, že trojuholníky $DEF$ a $CHF$
sú podobné.
}{
\rieh Z rovnosti obvodových uhlov nad tetivou $HF$ kružnice k vyplýva $|\ma HCF|= |\ma HEF|$. Uhol $HEF$ je zároveň úsekovým uhlom prislúchajúcim tetive $EF$ kružnice $l$, ktorý je však zhodný s obvodovým uhlom $EDF$ (obr.~\ref{fig:B66II3}). Celkovo tak platí
\begin{equation} \label{eq:B66II3_1}
    |\ma HCF| = |\ma HEF| = |\ma EDF|.
\end{equation}

\begin{figure}[h]
    \centering
    \includegraphics{images/B66II3\imagesuffix}
    \caption{}
    \label{fig:B66II3}
\end{figure}
Vzhľadom na to, že $CEFH$ je tetivový štvoruholník, je jeho vnútorný uhol pri vrchole $H$ zhodný s vonkajším uhlom pri jeho protiľahlom vrchole $E$. Platí teda
\begin{equation} \label{eq:B66II3_2}
    |\ma CHF| = |\ma DEF|.
\end{equation}
Z rovností \ref{eq:B66II3_1} a \ref{eq:B66II3_2} vyplýva na základe vety $uu$ podobnosť trojuholníkov $DEF$ a $CHF$. Tým je dôkaz hotový.\\
\\
\kom Úloha je relatívne jednoduchou aplikáciou poznatkov o stredových, obvodových a úsekových uhloch, preto dobe poslúži ako úvodná úloha seminára. Zároveň sa v úlohe vyskytuje spoločná tetiva dvoch kružníc, ktorá je prvkom mnohých geometrických úloh v kategórii B, takže je príjemné, že sa študenti s týmto prípadom zoznámia hneď na začiatku.
}


%\kom Úloha priamočiaro využíva poznatky, ktoré sme odvodili v začiatku seminára a taktiež zoznamuje študentov s XXX, ktorý sa v úlohách vyskytuje často, totiž spoločná tetiva dvoch kružníc XXX

% Do not delete this line (pandoc magic!)

\problem{B-65-II-2}{
Daná je úsečka $AB$, jej stred $C$ a vnútri úsečky $AB$ bod $D$. Kružnice $k(C, |BC|)$ a $m(B, |BD|)$ sa pretínajú v bodoch $E$ a $F$. Zdôvodnite, prečo je polpriamka $FD$ osou uhla $AFE$.
}{
\rieh Kružnica $k$ je Tálesovou kružnicou nad priemerom $AB$, takže trojuholník $ABF$ je pravouhlý s pravým uhlom pri vrchole $F$. Inými slovami, priamka $AF$ je kolmá 
\begin{figure}[h]
    \centering
    \includegraphics{images/B65II2_1\imagesuffix}
    \caption{}
    \label{fig:B65II2_1}
\end{figure}
na polomer $BF$ kružnice $m$, a preto sa priamka $AF$ dotýka kružnice $m$ v bode $F$ (obr.~\ref{fig:B65II2_1}). Z rovnosti úsekového uhla zovretého tetivou $DF$ s dotyčnicou $AF$ a obvodového uhla
nad tou istou tetivou máme (ako už je vyznačené na obrázku) 
$$|\ma AFD| = |\ma DEF|.$$
Zo súmernosti úsečky $EF$ podľa osi $AB$ tak vyplýva
$$|\ma AFD| = |\ma DEF| = |\ma DFE|,$$
čo znamená, že $FD$ je osou uhla $AFE$.

\textbf{Iné riešenie*.} Označme $\beta$ veľkosť uhla $ABF$ a dopočítajme veľkosti uhlov $DFE$ a $AFE$. Trojuholník $DBF$ je rovnoramenný, lebo jeho ramená $BD$ a $BF$ sú polomery kružnice $m$, preto
$$|\ma DFB| = \frac{1}{2}(180^\circ-\beta) = 90^\circ-\frac{\beta}{2}.$$
Keďže podobne aj trojuholník $EBF$ je rovnoramenný s osou $BD$, platí
$$|\ma EFB| = 90^\circ-beta.$$
Spojením oboch predchádzajúcich rovností tak dostávame
$$|\ma DFE| = |\ma DFB| - |\ma EFB| =\frac{\beta}{2}.$$
Z vlastností Tálesovej kružnice $k$ nad priemerom $AB$ vieme, že uhol $AFB$ je pravý. Pritom jeho časť uhol $EFB$ má, ako sme už zistili, veľkosť $90^\circ-\beta$, takže jeho druhá časť, uhol $AFE$, má veľkosť $\beta$, čo je presne dvojnásobok veľkosti uhla $DFE$. Tým sme dokázali, že priamka $FD$ je osou uhla $AFE$.

\textbf{Iné riešenie*.} Nad oblúkom $AE$ kružnice $k$ sa zhodujú uhly $ABE$ a $AFE$ (obr.~\ref{fig:B65II2_2}). Oblúku $DE$ kružnice $m$ prislúcha obvodový uhol $DFE$ a stredový uhol $DBE$. Spolu
tak dostávame
$$|\ma DFE| = \frac{1}{2} |\ma DBE| = \frac{1}{2}
|\ma ABE| = \frac{1}{2} |\ma AFE|,$$
čo dokazuje, že $FD$ je osou uhla $AFE$.
\begin{figure}[h]
    \centering
    \includegraphics{images/B65II2_2\imagesuffix}
    \caption{}
    \label{fig:B65II2_2}
\end{figure}
}


% Do not delete this line (pandoc magic!)

\problem{B-65-I-5}{
Vrcholy konvexného šesťuholníka $ABCDEF$ ležia na kružnici, pričom $|AB| = |CD|$. Úsečky $AE$ a $CF$ sa pretínajú v bode $G$ a úsečky $BE$ a $DF$ sa pretínajú v bode $H$. Dokážte, že úsečky $GH$, $AD$ a $BC$ sú navzájom rovnobežné. 
}{
\rieh Najskôr ukážeme, že $AD \parallel BC$. Keďže $|AB| = |CD|$, sú obvodové uhly nad tetivami $AB$ a $CD$ kružnice opísanej šesťuholníku $ABCDEF$ zhodné \todo{(obr. 3)}, teda $|\ma ADB| = |\ma DBC|$; to sú však striedavé uhly priečky $BD$ priamok $AD$ a $BC$, preto $AD \parallel BC$. Ostáva ukázať, že $GH \parallel AD$. Využitím zhodných obvodových uhlov nad tetivami \\
\\
\todo{DOPLNIŤ Obr.3}\\
\\
$AB$ a $CD$ pri vrcholoch $E$ a $F$ dostávame
$$|\ma GEH| = |\ma AEB| = |\ma CFD| = |\ma GFH|,$$
čo znamená, že body $E$, $F$, $G$ a $H$ ležia na jednej kružnici, pretože vrcholy zhodných uhlov $GEH$ a $GFH$ ležia v rovnakej polrovine s hraničnou priamkou $GH$. Z toho vyplýva, že uhly $EFH$ a $EGH$ nad jej tetivou $EH$ sú zhodné. To spolu so zhodnosťou uhlov $EFD$ a $EAD$ nad tetivou $ED$ pôvodnej kružnice \todo{(obr. 3)} vedie na zhodnosť súhlasných uhlov $EGH$ a $EAD$ priečky $AE$ priamok $GH$ a $AD$, ktoré sú teda naozaj rovnobežné. Tým je tvrdenie úlohy dokázané.
}

% Do not delete this line (pandoc magic!)

\problem{B-58-I-5}{seminar31,netrgeo,obvodove,dokaz,domacekolo}{
Trojuholníku $ABC$ je opísaná kružnica $k$. Os strany $AB$ pretne kružnicu $k$ v bode $K$, ktorý leží v polrovine opačnej k polrovine $ABC$. Osi strán $AC$ a $BC$ pretnú priamku $CK$ postupne v bodoch $P$ a $Q$. Dokážte, že trojuholníky $AKP$ a $KBQ$ sú zhodné.
}{
\rieh Označme $\alpha, \beta, \gamma$ zvyčajným spôsobom veľkosti vnútorných uhlov trojuholníka $ABC$ (obr.~\ref{fig:B58I5}). Bod $K$ leží na osi úsečky $AB$, preto $|AK| = |KB|$. Trojuholník $AKB$ je rovnoramenný so základňou $AB$, jeho vnútorné uhly pri vrcholoch $A$ a $B$ sú
\begin{figure}[h]
    \centering
    \includegraphics{images/B58I5\imagesuffix}
    \caption{}
    \label{fig:B58I5}
\end{figure}
teda zhodné. Podľa vety o obvodových uhloch sú zhodné aj uhly $BCK$ a $BAK$, resp.$ACK$ a $ABK$, preto sú zhodné aj uhly $BCK$ a $ACK$. Polpriamka $CK$ je teda osou uhla $ACB$:
$$|\ma ACK| = |\ma BCK| = \frac{\gamma}{2}.$$
Keďže bod $P$ leží na osi strany $AC$, je trojuholník $ACP$ rovnoramenný a jeho vnútorné uhly pri základni $AC$ majú veľkosť $\frac{1}{2}\gamma$, takže jeho vonkajší uhol $APK$ pri vrchole $P$ má veľkosť $\frac{1}{2}\gamma + \frac{1}{2}\gamma = \gamma$. Rovnako z rovnoramenného trojuholníka $BCQ$  odvodíme, že aj veľkosť uhla $BQK$ je $\gamma$. Podľa vety o obvodových uhloch sú zhodné uhly $ABC$ a $AKC$, teda uhol $AKC$ (čiže uhol $AKP$) má veľkosť $\beta$ a -- celkom analogicky -- uhol $BKQ$ má veľkosť $\alpha$.

V každom z trojuholníkov $AKP$ a $BKQ$ už poznáme veľkosti dvoch vnútorných uhlov ($\beta$, $\gamma$, resp. $\alpha$, $\gamma$), takže vidíme, že zostávajúce uhly $KAP$ a $KBQ$ majú veľkosti$\alpha$, resp. $\beta$.

Z predošlého vyplýva, že trojuholníky $AKP$ a $KBQ$ sú zhodné podľa vety $usu$, lebo majú zhodné strany $AK$ a $KB$ aj obe dvojice k nim priľahlých vnútorných uhlov.

K uvedenému postupu dodajme, že výpočet uhlov $KAP$ a $KBQ$ cez uhly $APK$ a $BQK$ možno obísť takto: zhodnosť uhlov $KAP$ a $BAC$ (resp. $KBQ$ a $ABC$) vyplýva zo zhodnosti uhlov $KAB$ a $PAC$ (resp. $KBA$ a $QBC$).\\
\\
\kom Posledná úloha seminára pekne kombinuje vlastnosti uhlov a zhodnosť trojuholníkov, je tak dôstojným zakončením tohto geometrického stretnutia.
}


\subsection*{Domáca práca}


