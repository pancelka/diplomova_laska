% Do not delete this line (pandoc magic!)

\problem{60-I-1-D1}{seminar06,rovnice,domacekolo,doplnujuca}{
Nech $n$ je prirodzené číslo väčšie ako 2. Máme $n$ čísel so súčtom~$n$, pričom každé z~nich je aritmetickým priemerom ostatných čísel. Aké sú to čísla?
}{
\rieh Usporiadajme si naše čísla podľa veľkosti, nech $x_1 \leq x_2 \leq\,\ldots \leq x_n$. Aritmetický priemer skupiny čísel je aspoň taký, ako najmenšie z~nich. Aritmetický priemer čísel $x_2, x_3,\ldots , x_n$ je preto aspoň $x_2$, a je rovný $x_1$ len v~prípade, že žiadne z~čísel $x_3,\,\ldots , x_n$ nie je väčšie ako $x_2$. Z~toho hneď dostávame, že všetky naše čísla musia byť rovnaké a teda rovné 1.
}
