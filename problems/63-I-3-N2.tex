% Do not delete this line (pandoc magic!)

\problem{63-I-3-N2}{seminar20,prvocisla,domacekolo,navodna}{
Číslo $n$ je súčinom dvoch rôznych prvočísel. Ak zväčšíme menšie z~nich o~1 a druhé ponecháme, ich súčin sa zväčší o~7. Určte číslo $n$.
}{
\rie Označme $p<q$ prvočísla zo zadania. Potom platí $(p+1)q=pq+7$. Po roznásobení ľavej strany a odčítaní výrazu $pq$ od oboch strán rovnosti dostávame $q=7$. Prvočíslo $p$ má byť menšie ako $q$, preto $p\in \{2,3,5\}$ a hľadaným číslom $n$ je tak jedno z~čísel 14, 21 alebo 35.\\
\\
}
