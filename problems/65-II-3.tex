% Do not delete this line (pandoc magic!)

\problem{65-II-3}{seminar11,podtroj,obsahy,pomery,krajskekolo}{
V~pravouhlom lichobežníku $ABCD$ s~pravým uhlom pri vrchole $A$ základne $AB$ je bod $K$ priesečníkom výšky $CP$ lichobežníka s~jeho uhlopriečkou $BD$. Obsah štvoruholníka $APCD$ je polovicou obsahu lichobežníka $ABCD$. Určte, akú časť obsahu trojuholníka $ABC$ zaberá trojuholník $BCK$.
}{
\rieh V~pravouholníku $APCD$ označme $c = |CD| = |AP|$ a $v = |AD| = |CP|$ (obr.~\ref{fig:65II3}, pričom sme už vyznačili ďalšie dĺžky, ktoré odvodíme v~priebehu riešenia)\footnote{Keďže podľa zadania uhlopriečka $BD$ pretína výšku $CP$, musí jej päta $P$ ležať medzi bodmi $A$ a $B$, takže ide o~\uv{zvyčajný} lichobežník $ABCD$ s~dlhšou základňou $AB$ a kratšou základňou $CD$.}.
\begin{figure}[h]
    \centering
    \includegraphics{images/65K3\imagesuffix}
    \caption{}
    \label{fig:65II3}
\end{figure}
Z~predpokladu $S_{APCD} =\frac{1}{2}S_{ABCD}$ vyplýva pre druhú polovicu obsahu $ABCD$ vyjadrenie $\frac{1}{2}S_{ABCD} = S_{PBC}$, takže $S_{APCD} = S_{PBC}$ čiže $cv =\frac{1}{2}|PB|v$, odkiaľ vzhľadom na to, že $v \neq 0$, vychádza $|PB| = 2c$, v~dôsledku čoho $|AB| = 3c$.

Trojuholníky $CDK$ a $PBK$ majú pravé uhly pri vrcholoch $C$, $P$ a zhodné (vrcholové) uhly pri spoločnom vrchole $K$, takže sú podľa vety $uu$ podobné, a to s~koeficientom $|PB| : |CD| = 2c : c = 2$. Preto tiež platí $|PK| : |CK| = 2 : 1$, odkiaľ $|KP| =\frac{2}{3}v$ a $|CK| =\frac{1}{3}v$.

Posudzované obsahy trojuholníkov $ABC$ a $BCK$ tak majú vyjadrenie
$$S_{ABC} = \frac{|AB| \cdot |CP|}{2}=\frac{3cv}{2} \ \ \ \ \text{a} \ \ \ \  S_{BCK} =\frac{|CK|\cdot |BP|}{2}=\frac{\frac{1}{3}v\cdot 2c}{2}=\frac{cv}{3},$$
preto ich pomer má hodnotu
$$\frac{S_{BCK}}{S_{ABC}}=\frac{\frac{1}{3}cv}{\frac{3}{2}cv}=\frac{2}{9}.$$
\textit{Záver.} Trojuholník $BCK$ zaberá $2/9$ obsahu trojuholníka $ABC$.\\
\\
\kom Najkomplexnejšia úloha tohto seminára precvičí študentov v~používaní vlastností podobných trojuholníkov a taktiež vo vyjadrovaní obsahov trojuholníkov pomocou určiteľných hodnôt. Tvorí tak dôstojnú bodku za týmto seminárom.
}
