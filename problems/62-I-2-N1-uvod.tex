% Do not delete this line (pandoc magic!)

\problem{62-I-2-N1}{seminar18,nertaz,agner,domacekolo}{
Ukážte, že nerovnosť $\frac{1}{2} (u + v) \geq \sqrt{uv}$ medzi aritmetickým a geometrickým priemerom dvoch ľubovoľných nezáporných čísel $u$ a $v$ vyplýva zo zrejmej nerovnosti $(a - b)^2\geq 0$ vhodnou voľbou hodnoty $a$ a $b$.
}{
\rieh Zvoľme $a =\sqrt{u}$ a $b =\sqrt{v}$. Vyjdime zo zrejmej nerovnosti $(\sqrt{u}-\sqrt{v})\geq 0$ a tú ďalej ekvivalentne upravujme:
\begin{align*}
    (\sqrt{u}-\sqrt{v})^2 & \geq 0,\\
    u-2\sqrt{uv}+v\geq & 0,\\
    \frac{u+v}{2} & \geq \sqrt{uv}.
\end{align*}
\\
\kom Vyššie dokázaná nerovnosť je špeciálnym prípadom tzv. AG-nerovnosti medzi aritmetickým a geometrickým priemerom ľubovoľných nezáporných čísel. Jej všeobecnejší tvar pre viac ako dve čísla však v tomto momente nepovažujeme za dôležité so študentami pokrývať, keďže AG-nerovnosť sa v úlohách kategórie C vyskytla v posledných rokoch veľmi zriedka (a aj vtedy bolo nerovnosti možné dokázať inými metódami). Je to však veľmi užitočný nástroj, ktorému by sme sa venovali v pokračovaní seminára vo vyšších ročníkoch.

Jeho využitie budeme demonštrovať v nasledujúcej úlohe.\\
\\
}