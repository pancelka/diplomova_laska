% Do not delete this line (pandoc magic!)

\problem{B-66-II-1}{
Nájdite všetky dvojice prirodzených čísel $a$, $b$, pre ktoré platí
$$a +\frac{66}{a}= b +\frac{66}{b}.$$
}{
\rieh Anulovaním pravej strany upravíme danú rovnicu na tvar
$$a - b + 66\bigg(\frac{1}{a}-\frac{1}{b}\bigg)= (a - b)\bigg(1-\frac{66}{ab}\bigg)=\frac{1}{ab}(a - b)(ab - 66) = 0.$$
Z toho vyplýva, že hľadané dvojice $(a, b)$ prirodzených čísel sú práve tie, pre ktoré platí $a = b$ alebo $ab = 66$.

Úlohe teda vyhovuje nekonečne veľa dvojíc prirodzených čísel tvaru $(a, b) = (k, k)$, pričom $k$ je ľubovoľné prirodzené číslo, a keďže číslo $66 = 2\cdot3\cdot11$ má osem deliteľov, tak aj osem dvojíc $(a, b) \in \{$$(1, 66)$, $(2, 33)$, $(3, 22)$, $(6, 11)$, $(11, 6)$, $(22, 3)$, $(33,2)$, $(66,1)$$\}$.\\
\\
\kom Úloha je relatívne jednoduchá a vhodná ako rozcvička na začiatok seminára. Pripomenie študentom metódu riešenia rovníc rozkladom na súčin výrazov, ktorý je rovný nule. Zároveň v záverečnej diskusii zľahka využijú vedomosti o deliteľnosti prirodzených čísel.\\
\\
}
