% Do not delete this line (pandoc magic!)

\problem{64-II-2}{seminar25,mriezsach}{
V~jednom políčku šachovnice $8 \times 8$ je napísané ”$-$“ a v~ostatných políčkach ”$+$“. V~jednom kroku môžeme zmeniť na opačné súčasne všetky štyri znamienka v~ktoromkoľvek štvorci $2 \times 2$ na šachovnici. Rozhodnite, či po určitom počte krokov môže byť na šachovnici oboch znamienok rovnaký počet.
}{
\rieh Počty plusov a mínusov v~tabuľke sú na začiatku 63 a 1, teda dve nepárne čísla. V~ľubovoľnom štvorci $2 \times 2$ môžu byť zastúpené jedným zo spôsobov 2 + 2, 1 + 3 alebo 0 + 4 vo vhodnom poradí sčítancov, ktoré sa po vykonanom kroku zmenia na poradie opačné. Vidíme teda, že po jednom kroku sa celkové počty plusov a mínusov v~tabuľke buď nemenia, alebo sa oba zmenia o~2, alebo sa oba zmenia o~4, takže to stále budú dve nepárne čísla ako na začiatku. To znamená, že nikdy nemôže byť na šachovnici oboch znamienok rovnaký počet, čiže párne číslo 32.\\
\\
\kom Po krátkom experimentovaní by malo byť väčšine študentov jasné, ako sa bude šachovnica správať, a tým pádom aj aká bude odpoveď na otázku zo zadania. (Ne)náročnosti úlohy zodpovedá aj jej bodové hodnotenie v~krajskom kole, kde sa stala najlepšie hodnotenou úlohou daného ročníka.\footnote{Na Slovensku, s priemerom 3,8\,b medzi všetkými riešiteľmi a 5,5\,b medzi úspešnými riešiteľmi.}\\
\\
}
