\begin{tcolorbox}[breakable,notitle,boxrule=0pt,colback=light-gray,colframe=light-gray]\ul{3.7} [65-I-3-D1, resp. 61-B-S-1] V~obore celých čísel vyriešte rovnicu $x^2+ y^2+ x + y = 4$.

\end{tcolorbox}

\rieh Vynásobením oboch strán danej rovnice štyrmi dostaneme $$4x^2 + 4y^2 + 4x + 4y = 16.$$
Výraz na ľavej strane takto upravenej rovnice doplníme na súčet druhých mocnín dvoch dvojčlenov. Obdržíme tak
$$(4x^2 + 4x + 1) + (4y^2 + 4y + 1) = (2x + 1)^2 + (2y + 1)^2 = 18.$$
Stačí teda vyšetriť všetky rozklady čísla 18 na súčet dvoch kladných nepárnych čísel, pretože čísla $2x + 1$ a $2y + 1$ nie sú deliteľné dvoma pre žiadne celé $x$ a $y$.

Uvažujme preto nasledujúce rozklady:
$$18 = 1 + 17 = 3 + 15 = 5 + 13 = 7 + 11 = 9 + 9.$$
Medzi uvedenými súčtami je iba jeden (18 = 9 + 9) súčtom druhých mocnín dvoch celých čísel. Môžu teda nastať nasledujúce štyri prípady:
\begin{align*}
2x + 1 &= 3, &  2y + 1 &= 3, & \text{ t.\,j.}\ \  x &= 1, y = 1,\\
2x + 1 &= 3, & 2y + 1 &= -3, & \text{ t.\,j.} \ \ x &= 1, y = -2,\\
2x + 1 &= -3, & 2y + 1 &= 3, & \text{ t.\,j.} \ \ x &= -2, y = 1,\\
2x + 1 &= -3, & 2y + 1 &=-3, & \text{ t.\,j.} \ \  x &= -2, y = -2.\\\
\end{align*}
\textit{Záver.} Danej rovnici vyhovujú práve štyri dvojice celých čísel $(x, y)$, a to $(1, 1)$, $(1, -2)$, $(-2, 1)$ a $(-2, -2)$.\\
\\
\textbf{Iné riešenie.} Danú rovnicu možno upraviť na tvar $x(x + 1) + y(y + 1) = 4$, z~ktorého vidno, že číslo 4 je nutné rozložiť na súčet dvoch celých čísel, z~ktorých každé je súčinom dvoch po sebe idúcich celých čísel. Keďže najmenšie hodnoty výrazu $t(t+ 1)$ pre kladné aj záporné celé $t$ sú 0, 2, 6, 12, \ldots , do úvahy prichádza iba rozklad $4 = 2 + 2$, takže každá z~neznámych $x, y$ sa rovná jednému z~čísel 1 či $-2$ -- jediných celých čísel $t$, pre ktoré $t(t+ 1) = 2$. Navyše je jasné, že naopak každá zo štyroch dvojíc $(x, y)$ zostavených z~čísel $1, -2$ je riešením danej úlohy.