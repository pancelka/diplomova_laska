\problem{63-I-5-N1}{
Dokážte, že pre každé prirodzené $n$ je číslo $n^3+ 2n$ deliteľné tromi.
}{
\rie Každé prirodzené číslo $n$ je tvaru $n=3k$, $n=3k+1$ alebo $n=3k+2$, kde $k$ je prirodzené číslo alebo 0. Dokazované tvrdenie overíme pre každú z~týchto možností zvlášť.

a) $n=3k$: $n^3+2n=(3k)^3+2\cdot 3k=27k^3+6k=3k(9k^2+2)$, tvrdenie platí.

b) $n=3k+1$: $n^3+2n=(3k+1)^3+2(3k+1)=(27k^3+27k^2+9k+1)+(6k+2)=27k^3+27k^2+15k+3=3(9k^3+9k^2+5k+1)$, tvrdenie platí.

c) $n=3k+2$: $n^3+2n=(3k+2)^3+2(3k+2)=(27k^3+54k^2+36k+8)+(6k+4)=27k^3+54k^2+42k+12=3(9k^3+18k^2+14k+4)$, a preto $3\mid n^3+2n$ aj v~tomto prípade.\\
\\
\kom Úloha zoznamuje študentov s~ďalším možným postupom pri dokazovaní deliteľnosti výrazu daným prirodzeným číslom $m$: rozdelenie na $m$ možností podľa zvyšku po delení číslom $m$ a dokázanie tvrdenia pre každú z~týchto možností zvlášť. Je vhodné diskutovať so študentmi o~výhodnosti tejto metódy pre (ne)veľké $m$.

Zaujímavé je tiež porovnať riešenie tejto úlohy s~úlohou predchádzajúcou, keďže v~tomto prípade sa nám daný výraz nepodarilo rozložiť na súčin troch po sebe idúcich čísel, preto sme museli pristúpiť k~inému riešeniu.

Úlohu je možné dokázať použitím matematickej indukcie, avšak tá nie je štandardnou náplňou osnov nematematických gymnázií, preto sme toto riešenie nezvolili ako vzorové. Ak sa však študenti s~dôkazom použitím indukcie stretli, je vhodné s~nimi rozobrať aj tento spôsob riešenia.\\
\\
}
