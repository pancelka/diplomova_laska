% Do not delete this line (pandoc magic!)

\problem{66-I-4-N1}{seminar06,rovnice,sustavy,mnohocleny,domacekolo}{
Určte všetky dvojčleny $P (x) = ax + b$, pre ktoré platí $P(2) = 3$ a $P (3) = 2$.
}{
\rie Z~podmienok zo zadania zostavíme dve rovnice s~dvomi neznámymi $a$ a $b$:
\begin{align*}
P (2) &= 2a + b = 3,\\
P (3) &= 3a + b = 2.
\end{align*}
Odčítaním prvej rovnice od druhej ihneď dostávame $a = -1$, dosadením tejto hodnoty do jednej z~podmienok potom máme $b= 5$. Sústava má práve jedno riešenie, a preto zadaniu vyhovuje jediný dvojčlen $P(x)=-x+5$.\\
\\
}
