\section*{Seminár 3}

\subsection*{Téma}
Algebraické výrazy, rovnice a nerovnosti I~-- úprava výrazov

\subsection*{Ciele}
Zopakovanie základných poznatkov o~úprave algebraických výrazov a rovníc, ktoré by študenti mali po absolvovaní základnej školy ovládať, uplatnenie týchto poznatkov pri riešení jednoduchších úloh, úprava na štvorec.

\subsection*{Úvodný komentár}
Na začiatku seminára spolu so študentami zopakujeme základné poznatky, ktoré sú nutnou podmienkou úspešného riešenia úloh v~tomto a nasledujúcich troch seminároch.
Študenti by mali\footnote{Spracované podľa~\cite{kubat2000}}:
\begin{itemize}
\item vedieť, čo je výraz, mnohočlen, člen, koeficient,
\item vedieť mnohočleny sčítať, odčítať, násobiť a v~jednoduchších prípadoch aj deliť,
\item ovládať spamäti vzorce pre druhú (príp. tretiu) mocninu dvojčlenu,
\item vedieť rozkladať mnohočleny na súčin vynímaním alebo použitím vzorcov $(A\pm B)^2$, $A^2-B^2$ (príp. $A^3\pm B^3$),
\item vedieť, čo je lineárna rovnica a diskutovať o~počte jej riešení,
\item vedieť riešiť ekvivalentnými úpravami lineárne rovnice.\\
\end{itemize}

\subsection*{Úlohy a riešenia}

% Do not delete this line (pandoc magic!)

\problem{65-I-3-N1}{
Pre ľubovoľné reálne čísla $x, y$ a $z$ dokážte nezápornosť hodnoty každého z~výrazov $$x^2z^2+ y^2- 2xyz, \ x^2+ 4y^2+ 3z^2- 2x - 12y - 6z + 13, \ 2x^2+ 4y^2 + z^2- 4xy - 2xz$$ a zistite tiež, kedy je dotyčná hodnota rovná nule.
}{
\rieh  Prvý výraz upravíme použitím vzorca $(A-B)^2$, kde $A=xz$ a $B=y$, čím dostaneme $x^2z^2+ y^2- 2xyz=(xz-y)^2$. Keďže ide o~druhú mocninu reálneho čísla, bude hodnota výrazu vždy nezáporná a rovnať sa nule bude v~prípade $xz=y$.

Druhý výraz upravíme podobným spôsobom, obdržíme však súčet troch druhých mocnín: $x^2+ 4y^2+ 3z^2- 2x - 12y - 6z + 13= (x^2-2x+1)+(4y^2-12y+9)+3(z^2-2z+1)=(x-1)^2+(2y-3)^2+3(z-1)^2$. Všetky tri sčítance sú nezáporné a teda aj ich súčet bude nezáporný a rovný nule bude v~prípade, ak základy všetkých troch mocnín budú tiež rovné nule, teda $x=1$, $y=\frac{3}{2}$ a $z=1$.

Pohľad na posledné dva členy posledného výrazu nám napovie, že pravdepodobne opäť využijeme podobnú úpravu ako v~predchádzajúcich prípadoch, základy mocnín však budú obsahovať dve premenné. Skutočne rozpísaním člena $2x^2=x^2+x^2$ a preusporiadaním poradia členov dostávame $(x^2-4xy+4y^2)+(x^2-2xy+z^2)$, odkiaľ je už zrejmé, že výraz bude mať tvar súčtu dvoch druhých mocnín $(x-2y)^2+(x-z)^2$. Výraz je vždy nezáporný a nulovú hodnotu nadobúda pre $x=2y=z$.\\

\kom Cieľom úlohy je zopakovať použitie vzorcov $(A\pm B)^2$ v~trochu menej priamočia\-rych mnohočlenoch než na aké je priestor v~bežnom vyučovaní. Zároveň rozpísanie člena $2x^2$ na súčet dvoch rovnakých členov je trik, na ktorý je vhodné študentov upozorniť, keďže tento princíp nájde uplatnenie v~nejednej úlohe.\\
}

% Do not delete this line (pandoc magic!)

\problem{63-I-1-N1-N4}{
a) Určte najmenšiu hodnotu výrazu $V = 5 + (x - 2)^2$, $x \in \RR$. Pre ktoré $x$ ju výraz nadobúda?

b) Určte najmenšiu možnú hodnotu výrazu $W = 9 - ab$, kde $a, b$ sú reálne čísla spĺňajúce podmienku $a + b = 6$. Pre ktoré hodnoty $a, b$ je $W$ minimálne?

c) Určte najmenšiu možnú hodnotu výrazu $Y = 12-ab$, kde $a, b$ sú reálne čísla spĺňajúce podmienku $a + b = 6$. Pre ktoré hodnoty $a, b$ je $Y$ minimálne?

d) Určte najväčšiu možnú hodnotu výrazu $K = 5 + ab$, kde $a, b$ sú reálne čísla spĺňajúce podmienku $a + b = 8$. Pre ktoré hodnoty $a, b$ je $K$ maximálne?
}{
\rie

a) Výraz $(x-2)^2$ je pre všetky reálne čísla $x$ nezáporný, jeho minimálna hodnota tak je 0, v~prípade $x=2$. Najmenšia hodnota výrazu $V$ je potom 5.

b) Z~podmienky vyjadríme premennú $a$ pomocou premennej $b$, teda $a=6-b$, dosadíme do $W$ a upravíme použitím vzorca $(A-B)^2$: $W=9-ab=9-(6-b)b=(b-3)^2$. Hodnota výrazu $W$ je tak vždy nezáporná a $W_{min}=0$, čo nastane pre $b=3$ a $a=3$.

c) Všimneme si, že $Y=W+3$ a keďže aj podmienka je v~tomto prípade rovnaká, platí $Y=(b-3)^2+3$. Potom $Y_{min}=3$ pre $a=3$ a $b=3$.

d) Podobne ako v~predchádzajúcich častiach, vyjadríme z~podmienky premennú $a$ pomocou premennej $b$: $a=8-b$ a dosadíme do výrazu $K$: $K = 5 + 8a - a^2= -(a - 4)^2+ 21$, kde sme využili tzv. úpravu na štvorec. Vidíme, že časť $-(a-4)^2$ je vždy nekladná, preto $K_{max}=21$ pre $a = b = 4$.\\

\kom V~častiach c) a d) predchádzajúcej úlohy sme využili tzv. úpravu výrazu na štvorec. Aj napriek tomu, že tento úkon je v~ŠVP zaradený až v~neskoršej časti školského roka (v~časti o~kvadratických rovniciach), považujeme za vhodné študentov s~metódou zoznámiť už teraz. Je totiž prirodzene spojená s~úpravami výrazov, jej pochopenie je možné aj bez širších znalostí riešenia kvadratických rovníc a v~úlohách MO nájde uplatnenie často.
}

% Do not delete this line (pandoc magic!)

\problem{63-I-1}{seminar04,vyrazy,domacekolo}{
 Určte, akú najmenšiu hodnotu môže nadobúdať výraz $V = (a-b)^2 +(b-c)^2 +(c-a)^2$, ak reálne čísla $a, b, c$ spĺňajú dvojicu podmienok
\begin{align*}
a + 3b + c &= 6,\\
-a + b - c &= 2.
\end{align*}
}{
\rieh Sčítaním oboch rovníc z~podmienky zistíme, že $b = 2$. Dosadením za $b$ do niektorej z~nich vyjde $c = -a$. Platí teda $V = (a - 2)^2 + (2 + a)^2 + (-2a)^2$. Po umocnení a sčítaní zistíme, že $V = 6a^2 + 8 \geq 8$. Rovnosť nastane práve vtedy, keď $a = 0$, $b = 2$ a $c = 0$.

Hľadaná najmenšia hodnota výrazu $V$ je teda rovná 8.\\

\kom Riešenie úlohy vyžaduje prácu so sústavou dvoch rovníc, avšak manipulácia tejto sústavy nie je až taká zložitá, takže zaradenie úlohy bez toho, aby sme sa systematicky venovali riešeniu sústav rovníc, nepokladáme za problematické.
}

% Do not delete this line (pandoc magic!)

\problem{65-I-3}{}{
    Uvažujme výraz $$2x^2+y^2-2xy+2x+4.$$
    \begin{enumerate}[a)]

\item Nájdite všetky reálne čísla $x$ a $y$, pre ktoré daný výraz nadobúda svoju najmenšiu hodnotu.

\item Určte všetky dvojice celých nezáporných čísel $x$ a $y$, pre ktoré je hodnota daného výrazu rovná číslu 16.
\end{enumerate}
}{
\rieh Daný výraz $V (x, y)$ upravme podľa vzorcov pre $(A \pm B)^2$:$$
V(x, y) =  x^2 - 2xy + y^2 +  x^2+ 2x + 1 + 3 = (x - y)^2+ (x + 1)^2+ 3.$$

a) Prvé dva sčítance v~poslednom súčte sú druhé mocniny, majú teda nezáporné hodnoty. Minimum určite nastane v~prípade, keď pre niektoré $x$ a $y$ budú oba základy rovné nule (v~tom prípade pre inú dvojicu základov už bude hodnota výrazu $V (x, y)$ väčšia). Obe rovnosti $x - y = 0$, $x + 1 = 0$ súčasne naozaj nastanú, a to zrejme iba pre hodnoty $x = y = -1$. Dodajme (na to sa zadanie úlohy nepýta), že $V_{min} = V~(-1, -1)= 3$. Daný výraz tak nadobúda svoju najmenšiu hodnotu iba pre $x = y = -1$.\\

b) Podľa úpravy z~úvodu riešenia platí $$V (x, y) = 16 \Leftrightarrow (x -y)^2+ (x + 1)^2+ 3 = 16 \Leftrightarrow (x -y)^2+ (x + 1)^2= 13.$$
Oba sčítance $(x - y)^2$ a $(x + 1)^2$ sú (pre celé nezáporné čísla $x$ a $y$) z~množiny $\{0, 1, 4, 9, 16,\,\ldots \}$. Jeden preto zrejme musí byť 4 a druhý 9. Vzhľadom na predpoklad $x \geq 0$ je základ $x+1$ mocniny $(x+1)^2$ kladný, musí preto byť rovný 2 alebo 3 (a nie $-2$ či $-3$). V~prvom prípade,  t.\,j. pre $x = 1$, potom pre základ mocniny $(x - y)^2$ dostávame podmienku $1 - y = \pm $, teda $y = 1 \pm 3$, čiže $y = 4$ (hodnota $y = -2$ je zadaním časti b) vylúčená). V~druhom prípade, keď $x = 2$, dostaneme podobne z~rovnosti $x - y = 2 - y = \pm 2$ dve vyhovujúce hodnoty $y = 0$ a $y = 4$. Celkovo teda všetky hľadané dvojice $(x, y)$ sú $(1, 4), (2, 0)$ a $(2, 4)$.\\
\\
\kom Záverečná úloha stavia na poznatkoch z~predchádzajúcich úloh, avšak vyžaduje dodatočnú analýzu, preto sme ju zvolili ako vyvrcholenie prvého algebraického seminára.
}

\subsection*{Domáca práca}

% Do not delete this line (pandoc magic!)

\problem{65-II-1}{seminar04,vyrazy,odhady,krajskekolo}{
Nájdite najmenšiu možnú hodnotu výrazu $$3x^2 - 12xy + y^4,$$
v~ktorom $x$ a $y$ sú ľubovoľné celé nezáporné čísla.
}{
\rieh Označme daný výraz $V$ a upravme ho dvojakým doplnením na štvorec: $$V = 3x^2 - 12xy + y^4= 3(x - 2y)^2 - 12y^2+ y^4= 3(x - 2y)^2+ (y^2 - 6)^2 - 36.$$
Zrejme platí $(x-2y)^2\geq0$, takže najmenšiu hodnotu výrazu V~pri pevnom $y$ dostaneme, keď položíme $x = 2y$. Ostáva preto nájsť najmenšiu možnú hodnotu mocniny $(y^2 - 6)^2$
s~nezápornou celočíselnou premennou $y$. Keďže $y^2 \in \{0, 1, 4, 9, 16, 25,\,\ldots\}$ a číslo 6 padne medzi čísla 4 a 9 tejto množiny, platí pre každé celé číslo $y$ nerovnosť $$(y^2 - 6)^
2\geq \mathrm{min} ((4 - 6)^2, (9 - 6)^2) = \mathrm{min}\{4, 9\} = 4.$$ Pre ľubovoľné celé čísla $x$ a $y$ tak dostávame odhad
$$V \geq 3 \cdot 0 + 4 - 36 = -32,$$
pritom rovnosť $V = -32$ nastáva pre $y = 2$ a $x = 2y = 4$.\\
\textit{Záver.} Hľadaná najmenšia možná hodnota daného výrazu je -32.\\
}

% Do not delete this line (pandoc magic!)

\problem{65-I-3-D1, resp. B-61-S-1}{V~obore celých čísel vyriešte rovnicu $$x^2+ y^2+ x + y = 4.$$
}{
\rieh Vynásobením oboch strán danej rovnice štyrmi dostaneme $$4x^2 + 4y^2 + 4x + 4y = 16.$$
Výraz na ľavej strane takto upravenej rovnice doplníme na súčet druhých mocnín dvoch dvojčlenov. Obdržíme tak
$$(4x^2 + 4x + 1) + (4y^2 + 4y + 1) = (2x + 1)^2 + (2y + 1)^2 = 18.$$
Stačí teda vyšetriť všetky rozklady čísla 18 na súčet dvoch kladných nepárnych čísel, pretože čísla $2x + 1$ a $2y + 1$ nie sú deliteľné dvoma pre žiadne celé $x$ a $y$.

Uvažujme preto nasledujúce rozklady:
$$18 = 1 + 17 = 3 + 15 = 5 + 13 = 7 + 11 = 9 + 9.$$
Medzi uvedenými súčtami je iba jeden (18 = 9 + 9) súčtom druhých mocnín dvoch celých čísel. Môžu teda nastať nasledujúce štyri prípady:
\begin{align*}
2x + 1 &= 3, &  2y + 1 &= 3, & \text{ t.\,j.}\ \  x &= 1, y = 1,\\
2x + 1 &= 3, & 2y + 1 &= -3, & \text{ t.\,j.} \ \ x &= 1, y = -2,\\
2x + 1 &= -3, & 2y + 1 &= 3, & \text{ t.\,j.} \ \ x &= -2, y = 1,\\
2x + 1 &= -3, & 2y + 1 &=-3, & \text{ t.\,j.} \ \  x &= -2, y = -2.\\\
\end{align*}
\textit{Záver.} Danej rovnici vyhovujú práve štyri dvojice celých čísel $(x, y)$, a to $(1, 1)$, $(1, -2)$, $(-2, 1)$ a $(-2, -2)$.\\
\\
\textbf{Iné riešenie*.} Danú rovnicu možno upraviť na tvar $x(x + 1) + y(y + 1) = 4$, z~ktorého vidno, že číslo 4 je nutné rozložiť na súčet dvoch celých čísel, z~ktorých každé je súčinom dvoch po sebe idúcich celých čísel. Keďže najmenšie hodnoty výrazu $t(t+ 1)$ pre kladné aj záporné celé $t$ sú 0, 2, 6, 12,\,\ldots , do úvahy prichádza iba rozklad $4 = 2 + 2$, takže každá z~neznámych $x, y$ sa rovná jednému z~čísel 1 či $-2$ -- jediných celých čísel $t$, pre ktoré $t(t+ 1) = 2$. Navyše je jasné, že naopak každá zo štyroch dvojíc $(x, y)$ zostavených z~čísel $1, -2$ je riešením danej úlohy.
}

\subsection*{Doplňujúce zdroje a materiály}
V~prípade, že na seminári zistíme, že študenti potrebujú získať zručnosť v~rozkladaní výrazov na súčin, je možné odkázať ich na rôzne stredoškolské zbierky príkladov, napr.\cite{kubat2000} alebo \cite{busek1999}, kde nájdu dostatočné množstvo príkladov na precvičenie.
