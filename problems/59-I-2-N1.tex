% Do not delete this line (pandoc magic!)

\problem{59-I-2-N1}{netrgeo,domacekolo,navodna}{
Na hárok papiera tvaru obdĺžnika narysujte podľa \todo{obr. 3} pravouholník $ABCD$ tak,\\
\\
\todo{DOPLNIŤ Obr. 3}\\
\\
aby jeho strany $AB$ a $AD$ splývali s okrajom papiera. Potom zostrojte priamku, aby mala s pravouholníkom spoločný len bod $C$ a jej prienik s hárkom papiera tvoril úsečku $MN$, pozdĺž ktorej papier rozstrihnite. Vzniknutý papierový model trojuholníka $AMN$ s narysovaným obdĺžnikom $ABCD$ preložte pozdĺž úsečiek $BC$ a $DC$. Túto činnosť niekoľkokrát opakujte, pritom pre rovnaký pravouholník $ABCD$ voľte rôzne dĺžky úsečky $BM$. Čo možno z výsledku usúdiť o pomere obsahov trojuholníka $AMN$ a pravouholníka $ABCD$? Hypotézu dokážte.
}{
\rie
}
