% Do not delete this line (pandoc magic!)

\problem{66-I-5-prvá časť}{seminar22,stvoruholniky,domacekolo}{
Ak označíme $X$ a $Y$ postupne stredy základní $RS$ a $TU$ všeobecného lichobežníka $RSTU$, tak na úsečke $XY$ leží priesečník $P$ uhlopriečok $RT$ a $SU$, a to tak, že $|PX| : |PY | = |RS| : |TU|$. Na priamke $XY$ leží tiež priesečník $Q$ predĺžených ramien $RU$ a $ST$, a to tak, že $|QX| : |QY | = |RS| : |TU|$ (obr.~\ref{fig:66I5_1}). Dokážte.
}{
\rieh
\begin{figure}[h]
    \centering
    \includegraphics{images/66D51\imagesuffix}
    \caption{}
    \label{fig:66I5_1}
\end{figure}
Napriek tomu, že sa podľa obrázka zdá, že bod $P$ na úsečke $XY$ naozaj leží, musíme tento poznatok dokázať, teda \textit{odvodiť} argumentáciou nezávislou na presnosti nášho rysovania. Na to určite stačí preukázať, že obe úsečky $PX$, $PY$ zvierajú s~priamkou $RT$ zhodné uhly (na obrázku vyznačené otáznikmi). Všimnime si, že tieto úsečky sú ťažnicami trojuholníkov $RSP$ a $TUP$, ktoré sa zhodujú vo vnútorných uhloch (vyznačených oblúčikmi) pri rovnobežných stranách $RS$ a $TU$, takže ide o~trojuholníky podobné, a to s~koeficientom $k = |TU|/|RS|$. S~rovnakým koeficientom platí aj podobnosť \uv{polovíc} týchto trojuholníkov vyťatých ich ťažnicami, presnejšie podobnosť $RXP \sim TYP$. Z~nej už želaná zhodnosť uhlov $RPX$ a $TPY$ aj želaná rovnosť $|PY | = k|PX|$ (vďaka rovnakému koeficientu) vyplýva. Všetko o~bode $P$ je tak dokázané; podobne sa overia aj obe vlastnosti bodu $Q$ - ukáže sa, že úsečky $QX$ a $QY$ zvierajú ten istý uhol s~priamkou $RQ$ a ich dĺžky sú zviazané rovnosťou $|QY | = k|QX|$, a to vďaka tomu, že $QX$ a $QY$ sú ťažnice v~dvoch navzájom podobných trojuholníkoch $RSQ$ a $UTQ$.\\
\\
\kom Úloha je prípravou na riešenie záverečného problému tohto seminára a pripomína študentom metódu dôkazu toho, že bod $P$ leží na priamke úsečke $XY$.
}