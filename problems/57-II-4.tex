\problem{57-II-4}{
Nájdite všetky trojice celých čísel $x, y, z$, pre ktoré platí
$$x+y\sqrt{3}+z\sqrt{7}=y+z\sqrt{3}+x\sqrt{7}. $$
}{
\rieh Rovnicu prepíšeme na tvar
$$x-y=(z-y)\sqrt{3}+(x-z)\sqrt{7}$$
a umocníme. Po jednoduchej úprave dostaneme
$$(x - y)^2 - 3(z - y)^2 - 7(x - z)^2 = 2(x - z)(z - y)\sqrt{21}. \ \ \ \ \ \ \ (1)$$
Pre $x \neq z$ a $y \neq z$ nemôže rovnosť (1) platiť, pretože jej pravá strana je v~takom prípade číslo iracionálne, zatiaľ čo ľavá strana je číslo celé. Rovnosť teda môže nastať, len keď
$x = z$ alebo $y = z$.

V~prvom prípade po dosadení $x = z$ do pôvodnej rovnice dostaneme $z-y =\sqrt{3}(z- y)$. Odtiaľ $z = y = x$.

V~druhom prípade, keď $y = z$, dôjdeme analogicky k~rovnakému výsledku.

\textit{Záver.} Riešením danej rovnice sú všetky trojice $(x, y, z) = (k, k, k)$, kde $k$ je ľubovoľné celé číslo.\\
\\
\kom Aj napriek tomu, že vzorové riešenie úlohy vyzerá zrozumiteľne, úloha riešiteľov krajských kôl potrápila (bola najhoršie hodnotenou úlohou daného krajského kola). Záludnosti sa ukrývajú vo vytyčovaní iracionálnych čísel a nie neznámych, vhodnej úprave rovnice a diskusii o~(i)racionalite oboch strán rovnice. \\
\\
}
