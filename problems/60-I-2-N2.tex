% Do not delete this line (pandoc magic!)

\problem{60-I-2-N2}{
Nech $x + 5y$ dáva zvyšok 1 po delení 7. Aký zvyšok po delení 7 dáva číslo $3x + 15y$? A číslo $4x + 13y$?
}{
\rieh Keďže $x + 5y = 7k + 1$ pre vhodné $k$, máme $3x + 15y = 3(7k + 1) = 7 \cdot 3k + 3$, čiže zvyšok je 3. Podobne $4x + 20y = 4(7k + 1) = 7 \cdot 4k + 4$, pritom číslo
$4x + 13y$ sa od $4x + 20y$ líši len o násobok 7, preto dáva rovnaký zvyšok.
}
