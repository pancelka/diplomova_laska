% Do not delete this line (pandoc magic!)

\problem{B-59-I-6}{
Reálne čísla $a$, $b$ majú túto vlastnosť: rovnica $x^2 -ax+b-1 = 0$ má v~množine reálnych čísel dva rôzne korene, ktorých rozdiel je kladným koreňom rovnice $x^2 - ax + b + 1 = 0$.
\begin{enumerate}[a)]
    \item Dokážte nerovnosť $b > 3$.
    \item Pomocou $b$ vyjadrite korene oboch rovníc.
\end{enumerate}
}{
\rieh  Označme $x_1$ menší a $x_2$ väčší koreň prvej rovnice. Potom platí $x_1 + x_2 = a$, $x_1 x_2 = b - 1$. Druhá rovnica má koreň $x_2 - x_1$, a keďže súčet oboch koreňov je $a$, musí byť druhý koreň $a - (x_2 - x_1 ) = x_1 + x_2 - x_2 + x_1 = 2x_1$. Súčin koreňov druhej rovnice je $(x_2 -x_1 )\cdot2x_1 = b+1$. Odtiaľ dostávame $b = -1+2x_1 x_2 -2x_1^2= -1+2(b-1)-2x_1^2$, a teda
\begin{equation} \label{eq:B59I6_1}
    b = 3 + 2x_1^3> 3,
\end{equation}
lebo z~rovnosti $x_1 = 0$ by vyplývalo $b + 1 = b - 1 = 0$.

Keďže $x_2 - x_1 > 0$ a $b + 1 > 0$, musí byť aj $x_1 > 0$; z~\ref{eq:B59I6_1} máme $x_1 =\sqrt{(b - 3)/2}$ a ďalej
$$x_2 =\frac{b-1}{x_1}=\frac{(b - 1)\sqrt{2}}{\sqrt{b-3}}.$$
Korene druhej rovnice sú potom
$$x_2 - x_1 = \frac{b+1}{} \ \ \ \ \text{a} \ \ \ \  2x_1=\sqrt{2(b - 3)}.$$
\\
\textbf{Iné riešenie*.} Korene prvej rovnice sú
$$x_1 = \frac{a -\sqrt{a^2 - 4b + 4}}{2}, \ \ \ \  x_2 =\frac{a +\sqrt{a^2 - 4b + 4}}{2},$$
pričom pre diskriminant máme
\begin{equation} \label{eq:B59I6_2}
    D = a^2 - 4(b - 1) > 0.
\end{equation}
Rozdiel koreňov $x_2 - x_1 =\sqrt{a^2 - 4b + 4}$ je koreňom druhej rovnice, a preto
\begin{equation} \label{eq:B59I6_3}
    \begin{aligned}
        a^2 - 4b + 4 - a \sqrt{a^2 - 4b + 4} + b + 1 &= 0,\\
a^2 - 3b + 5 &= a\sqrt{a 2 - 4b + 4},\\
a^4 + 2a^2 (5 - 3b) + (3b - 5)^2 &= a^4 - 4a^2 b + 4a^2,\\
(3b - 5)^2 &= a^2 (2b - 6).
    \end{aligned}
\end{equation}
Rovnosť  $a = 0$ nastáva práve vtedy, keď $3b - 5 = 0$; potom by ale neplatilo \ref{eq:B59I6_2}. Preto $a^2 > 0$, $(3b - 5)^2 > 0$, a teda aj $2b - 6 > 0$, čiže $b > 3$. Z \ref{eq:B59I6_2} a \ref{eq:B59I6_3} potom vyplýva $a > 0$, a teda $a = (3b - 5)/\sqrt{2(b - 3)}$; ďalej potom
\begin{align*}
x_1 &=\frac{1}{2}\bigg( \frac{3a-5}{\sqrt{2(b-3)}}-\sqrt{\frac{(3b-5)^2}{2(b-3)}}-4b+4\bigg)=\sqrt{\frac{b-3}{2}},\\
x_2 &=\frac{1}{2} \bigg( \frac{3a-5}{\sqrt{2(b-3)}}+\sqrt{\frac{(3b-5)^2}{2(b-3)}}-4b+4\bigg)=\frac{(b-1)\sqrt{2}}{\sqrt{b-3}}.
\end{align*}
Druhá rovnica má korene
\begin{align*}
x_3 &=\frac{a-\sqrt{a^2-4b-4}}{2}=\frac{b+1}{\sqrt{2(b-3)}}=x_2-x_1\\
x_4 &=\frac{a+\sqrt{a^2-4b-4}}{2}=\sqrt{2(b-3)}.
\end{align*}
\\
\kom Úloha sa dá vyriešiť relatívne \uv{netrikovo} vyjadrením koreňov prvej rovnice, dosadením ich rozdielu do druhej rovnice a odpovedajúcou diskusiou. Takýto prístup je síce zrozumiteľný, avšak dosť pracný. Ak študenti neprídu na prvý spôsob riešenia, považujeme za vhodné im ho ukázať ako dobrý príklad toho, ako nám použitie Viètovych vzorcov môže výraznej zjednodušiť výpočet.\\
\\
}
