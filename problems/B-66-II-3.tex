% Do not delete this line (pandoc magic!)

\problem{B-66-II-3}{}{
V rovine sú dané kružnice $k$ a $l$, ktoré sa pretínajú v bodoch $E$ a $F$. Dotyčnica ku kružnici $l$ zostrojená v bode $E$ pretína kružnicu $k$ v bode $H$ ($H \neq E$). Na oblúku $EH$ kružnice $k$, ktorý neobsahuje bod $F$, zvoľme bod $C$ ($E \neq C \neq H$) a priesečník priamky $CE$ s kružnicou $l$ označme $D$ ($D \neq E$). Dokážte, že trojuholníky $DEF$ a $CHF$
sú podobné.
}{
\rieh Z rovnosti obvodových uhlov nad tetivou $HF$ kružnice k vyplýva $|\ma HCF|= |\ma HEF|$. Uhol $HEF$ je zároveň úsekovým uhlom prislúchajúcim tetive $EF$ kružnice $l$, ktorý je však zhodný s obvodovým uhlom $EDF$ (obr.~\ref{fig:B66II3}). Celkovo tak platí
\begin{equation} \label{eq:B66II3_1}
    |\ma HCF| = |\ma HEF| = |\ma EDF|.
\end{equation}

\begin{figure}[h]
    \centering
    \includegraphics{images/B66II3\imagesuffix}
    \caption{}
    \label{fig:B66II3}
\end{figure}
Vzhľadom na to, že $CEFH$ je tetivový štvoruholník, je jeho vnútorný uhol pri vrchole $H$ zhodný s vonkajším uhlom pri jeho protiľahlom vrchole $E$. Platí teda
\begin{equation} \label{eq:B66II3_2}
    |\ma CHF| = |\ma DEF|.
\end{equation}
Z rovností \ref{eq:B66II3_1} a \ref{eq:B66II3_2} vyplýva na základe vety $uu$ podobnosť trojuholníkov $DEF$ a $CHF$. Tým je dôkaz hotový.\\
\\
\kom Úloha je relatívne jednoduchou aplikáciou poznatkov o stredových, obvodových a úsekových uhloch, preto dobe poslúži ako úvodná úloha seminára. Zároveň sa v úlohe vyskytuje spoločná tetiva dvoch kružníc, ktorá je prvkom mnohých geometrických úloh v kategórii B, takže je príjemné, že sa študenti s týmto prípadom zoznámia hneď na začiatku.
}
