% Do not delete this line (pandoc magic!)

\problem{58-II-2}{
V~pravouhlom trojuholníku $ABC$ označíme $P$ pätu výšky z~vrcholu $C$ na preponu $AB$ a $D, E$ stredy kružníc vpísaných postupne trojuholníkom $APC$, $CPB$. Dokážte, že stred
kružnice vpísanej trojuholníku $ABC$ je priesečníkom výšok trojuholníka $CDE$.
}{
\rieh V~pravouhlom trojuholníku $ABC$ s~preponou $AB$ označme $\alpha$ veľkosť vnútorného uhla pri vrchole $A$, zrejme potom platí $|\ma ACP| = 90^\circ -\alpha, |\ma PCB| = \alpha.$ Stred $D$ kružnice vpísanej trojuholníku $APC$ leží na osi uhla $PAC$, takže $|\ma DAC| = \frac{1}{2}\alpha$, a podobne aj $|\ma PCE| = \frac{1}{2}\alpha$. Odtiaľ pre veľkosť uhla $AUC$ v~trojuholníku $AUC$, pričom $U$ je priesečník polpriamok $AD$ a $CE$ (obr. 11), vychádza
$$|\ma AUC| = 180^\circ -\bigg(90^\circ -\alpha + \frac{1}{2}\alpha\bigg) -\frac{1}{2}\alpha = 90^\circ.$$
To znamená, že polpriamka $AD$ je kolmá na $CE$, úsečka $DU$ je teda výška v~trojuholníku $DEC$. Úplne rovnako zistíme, že aj polpriamka $BE$ (ktorá je zároveň osou uhla $ABC$) je kolmá na $CD$. Dostávame tak, že priesečník polpriamok $AD$ a $BE$, čo je stred kružnice vpísanej trojuholníku $ABC$, je zároveň aj priesečníkom výšok trojuholníka $DEC$.
\begin{center}
\includegraphics{images/58K21\imagesuffix}\\

Obr. 11
\end{center}
\textbf{Iné riešenie*.} Označme $F$ a $G$ zodpovedajúce priesečníky priamok $CD$ a $CE$ so stranou $AB$ (obr. 12). Podľa úlohy vyriešenej na seminári v~škole je trojuholník $CAG$
\begin{center}
\includegraphics{images/58K22\imagesuffix}\\

Obr. 12
\end{center}
rovnoramenný so základňou $CG$. Os $AD$ uhla $CAG$ rovnoramenného trojuholníka $CAG$ je tak aj jeho osou súmernosti, a je preto kolmá na základňu $CG$, teda aj na $CE$. Podobne zistíme, že aj trojuholník $CBF$ je rovnoramenný so základňou $CF$, takže os $BE$ uhla $FBC$ je kolmá na $CF$, teda aj na $CD$. Priesečník oboch osí $AD$ a $BE$ je tak nielen stredom kružnice vpísanej trojuholníku $ABC$, ale aj priesečníkom výšok trojuholníka $CDE$, čo sme mali dokázať.
}
