% Do not delete this line (pandoc magic!)

\problem{65-I-3}{}{
    Uvažujme výraz $$2x^2+y^2-2xy+2x+4.$$
    \begin{enumerate}[a)]

\item Nájdite všetky reálne čísla $x$ a $y$, pre ktoré daný výraz nadobúda svoju najmenšiu hodnotu.

\item Určte všetky dvojice celých nezáporných čísel $x$ a $y$, pre ktoré je hodnota daného výrazu rovná číslu 16.
\end{enumerate}
}{
\rieh Daný výraz $V (x, y)$ upravme podľa vzorcov pre $(A \pm B)^2$:$$
V(x, y) =  x^2 - 2xy + y^2 +  x^2+ 2x + 1 + 3 = (x - y)^2+ (x + 1)^2+ 3.$$

a) Prvé dva sčítance v~poslednom súčte sú druhé mocniny, majú teda nezáporné hodnoty. Minimum určite nastane v~prípade, keď pre niektoré $x$ a $y$ budú oba základy rovné nule (v~tom prípade pre inú dvojicu základov už bude hodnota výrazu $V (x, y)$ väčšia). Obe rovnosti $x - y = 0$, $x + 1 = 0$ súčasne naozaj nastanú, a to zrejme iba pre hodnoty $x = y = -1$. Dodajme (na to sa zadanie úlohy nepýta), že $V_{min} = V~(-1, -1)= 3$. Daný výraz tak nadobúda svoju najmenšiu hodnotu iba pre $x = y = -1$.\\

b) Podľa úpravy z~úvodu riešenia platí $$V (x, y) = 16 \Leftrightarrow (x -y)^2+ (x + 1)^2+ 3 = 16 \Leftrightarrow (x -y)^2+ (x + 1)^2= 13.$$
Oba sčítance $(x - y)^2$ a $(x + 1)^2$ sú (pre celé nezáporné čísla $x$ a $y$) z~množiny $\{0, 1, 4, 9, 16,\,\ldots \}$. Jeden preto zrejme musí byť 4 a druhý 9. Vzhľadom na predpoklad $x \geq 0$ je základ $x+1$ mocniny $(x+1)^2$ kladný, musí preto byť rovný 2 alebo 3 (a nie $-2$ či $-3$). V~prvom prípade,  t.\,j. pre $x = 1$, potom pre základ mocniny $(x - y)^2$ dostávame podmienku $1 - y = \pm $, teda $y = 1 \pm 3$, čiže $y = 4$ (hodnota $y = -2$ je zadaním časti b) vylúčená). V~druhom prípade, keď $x = 2$, dostaneme podobne z~rovnosti $x - y = 2 - y = \pm 2$ dve vyhovujúce hodnoty $y = 0$ a $y = 4$. Celkovo teda všetky hľadané dvojice $(x, y)$ sú $(1, 4), (2, 0)$ a $(2, 4)$.\\
\\
\kom Záverečná úloha stavia na poznatkoch z~predchádzajúcich úloh, avšak vyžaduje dodatočnú analýzu, preto sme ju zvolili ako vyvrcholenie prvého algebraického seminára.
}