% Do not delete this line (pandoc magic!)

\problem{66-II-4}{}{
Dokážte, že pre všetky kladné reálne čísla $a \leq b \leq c$ platí $$(-a + b + c)\bigg( \frac{1}{a}+\frac{1}{b}+\frac{1}{c}\bigg) \geq 3.$$
}{
\rieh  Nerovnosť vynásobíme kladným výrazom $abc$ a po roznásobení ju postupne (ekvivalentne) upravíme:
\begin{align*}
-a(bc + ac + ab) + b(bc + ac + ab) + c(bc + ac + ab) &\geq 3abc,\\
-abc - a^2c - a^2b + b^2c + abc + ab^2+ bc^2+ ac^2+ abc &\geq 3abc,\\
(b^2c - abc) + (bc^2 - abc) + (ac^2 - a^2c) + (ab^2 - a^2b) &\geq 0,\\
bc(b - a) + bc(c - a) + ac(c - a) + ab(b - a) &\geq 0.
\end{align*}
Vzhľadom na predpoklad $0 < a \leq b \leq c$ je výsledná, a teda aj pôvodná nerovnosť splnená.\\
\\
\textbf{Iné riešenie*.} Dokazovanú nerovnosť postupne upravíme, pričom využijeme známu nerovnosť $b/c + c/b \geq 2$, ktorá je pre kladné čísla $b, c$ ekvivalentná s nerovnosťou $(b - c)^2\geq 0$: $$(-a + b + c)\bigg( \frac{1}{a}+\frac{1}{b}+\frac{1}{c}\bigg) = 1+\bigg(\frac{b}{a}-\frac{a}{b}\bigg)+\bigg(\frac{c}{a}-\frac{a}{c}\bigg)+\bigg(\frac{b}{c}+\frac{c}{b}\bigg)\geq$$ $$
\geq 1+\frac{b^2-a^2}{ab}+\frac{c^2-a^2}{ac}+2\geq3,$$
pretože zrejme platí aj $a^2\leq b^2\leq c^2$.\\
\\
\textbf{Iné riešenie*.} Podľa predpokladov úlohy platia nerovnosti $-a + b + c \geq c$ a $\frac{1}{a}+\frac{1}{b}+\frac{1}{c}\geq \frac{2}{b}+\frac{1}{c}$.
Obe nerovnosti (s kladnými stranami) medzi sebou vynásobíme a získame tak $$(-a + b + c)\bigg(\frac{1}{a}+\frac{1}{b}+\frac{1}{c}\bigg) \geq c \bigg( \frac{2}{b}+\frac{1}{c}\bigg)=1 +\frac{2c}{b}\geq 3$$
pretože $c/b \geq 1$ podľa zadania.
\\
\\
\kom Ďalšia úloha, ktorú je možné rozlúsknuť spektrom rozličných prístupov. Ak študenti zvolia len cestu ekvivalentných úprav, ukážeme im aj riešenie, ktoré využíva nerovnosť $b/c+c/b \geq 2$ z predchádzajúceho seminára o nerovnostiach, rovnako ako riešenie pomocou vynásobenia nerovností medzi sebou. Takto dáme študentom príležitosť poznať aj iné prístupy, ktoré môžu byť užitočné pri ďalšom riešení úloh.\\
\\
%\\
%\\
%\kom Úvodná úloha slúži na opakovanie, pripomenutie naučených postupov a overenie toho, čo sa študenti doteraz naučili o nerovnostiach. Považujeme za vhodné ukázať všetky tri zmienené postupy riešenia, keďže ekvivalentné úpravy rovníc, využívanie známych rovností aj sčítanie dvoch a viac rovností sú všetko užitočné metódy, ktoré sa oplatí mať v našej riešiteľskej zásobe.
}