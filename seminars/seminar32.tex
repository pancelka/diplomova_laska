
\seminar{32}{Geometria VIII -- výpočtové úlohy}

\teachernote{
\subsection*{Ciele}
Precvičiť komplexnejšie úlohy zahŕňajúce geometrické výpočty


\subsection*{Úlohy a riešenia}
}



% Do not delete this line (pandoc magic!)

\problem{B-59-II-1}{
Kružnica $l(T; s)$ prechádza stredom kružnice $k(S; 2 cm)$. Kružnica $m(U; t)$ sa zvonka dotýka kružníc $k$ a $l$, pričom $US \perp ST$. Polomery $s$ a $t$ vyjadrené v centimetroch sú
celé čísla. Určte ich.
}{
\rieh Trojuholník $UST$ je pravouhlý. Jeho prepona $UT$ má dĺžku $s + t$, dĺžky odvesien sú $|US| = t + 2$, $|ST| = s$ \todo{fixni ma (obr. 1)}. Podľa Pytagorovej vety platí
$$(s + t)^2 = (t + 2)^2 + s^2.$$
Úpravami postupne dostávame
\begin{align*}
  s^2 + 2st + t^2 & = t^2 + 4t + 4 + s^2,\\
  st & = 2t + 2,\\
  t(s - 2) & = 2.  
\end{align*}
Čísla $t$ a $s - 2$ sú celé, preto $t$ musí byť deliteľom čísla $2$. Keďže $t$ je kladné, sú len dve možnosti; ak $t = 1$\,cm, tak $s = 4$\,cm, a ak $t = 2$\,cm, tak $s = 3$\,cm.

\todo{DOPLNIŤ Obr. 1}
}


% Do not delete this line (pandoc magic!)

\problem{B-66-S-2}{seminar32,geompoc,pytveta,netrgeo,skolskekolo}{
Na odvesnách $AC$ a $BC$ daného pravouhlého trojuholníka $ABC$ určte postupne body $K$ a $L$ tak, aby súčet
$$|AK|^2+ |KL|^2+ |LB|^2$$
nadobúdal najmenšiu možnú hodnotu a vyjadrite ju pomocou $c = |AB|$.
}{
\rieh V súlade s obr.~\ref{fig:B66S2} označme $x = |CL|$, $y = |CK|$, potom $|BL| = a - x$, a $|AK| = b - y$, pričom $a$, $b$ sú postupne dĺžky odvesien $BC$, $AC$.
\begin{figure}[h]
    \centering
    \includegraphics{images/B66S2\imagesuffix}
    \caption{}
    \label{fig:B66S2}
\end{figure}
Použitím Pytagorovej vety v pravouhlom trojuholníku $KLC$ dostaneme $|KL|^2= x^2 + y^2$, takže skúmaný súčet môžeme upraviť nasledujúcim spôsobom:
\begin{align*}
    |AK|^2+ |KL|^2+ |LB|^2 & = (b - y)^2+ x^2+ y^2+ (a - x)^2=\\
    & = 2x^2+ 2y^2 - 2ax - 2by + a^2+ b^2=\\
    & = 2\bigg(x-\frac{a}{2}\bigg)^2+ 2\bigg( y -\frac{b}{2}\bigg)^2+\frac{a^2 + b^2}{2}=\\
    & = 2\bigg(x -\frac{a}{2}\bigg)^2+ 2\bigg( y -\frac{b}{2}\bigg)^2+\frac{c^2}{2}.
\end{align*}
Vďaka nezápornosti druhých mocnín z toho vidíme, že skúmaný výraz nadobúda svoju najmenšiu hodnotu, konkrétne $\frac{1}{2}c$, práve vtedy, keď $x =\frac{1}{2}a$ a súčasne $y=\frac{1}{2}b$, teda práve vtedy, keď body $K$, $L$ sú postupne stredmi odvesien $AC$, $BC$ daného pravouhlého trojuholníka $ABC$.

\textit{Záver.} Najmenšia možná hodnota skúmaného súčtu je rovná $\frac{1}{1}c^2$. Túto hodnotu dostaneme práve vtedy, keď body $K$, $L$ budú postupne stredmi odvesien $AC$, $BC$ daného pravouhlého trojuholníka.
}


% Do not delete this line (pandoc magic!)

\problem{B-63-S-3}{
Na priamke $a$, na ktorej leží strana $BC$ trojuholníka $ABC$, sú dané body dotyku všetkých troch jemu pripísaných kružníc (body $B$ a $C$ nie sú známe). Nájdite na tejto priamke bod dotyku kružnice vpísanej. 
}{
V danom trojuholníku $ABC$ označme $X$, $Y$, $Z$ body dotyku vpísanej kružnice s jeho stranami a $x = |AY | = |AZ|$, $y = |BX| = |BZ|$, $z = |CX| = |CY|$ zhodné úseky dotyčníc k vpísanej kružnici z jednotlivých vrcholov (obr.~\ref{fig:B63S3_1}). Ak označíme\\
\begin{figure}[h]
    \centering
    \includegraphics{images/B63S3_1\imagesuffix}
    \caption{}
    \label{fig:B63S3_1}
\end{figure}
zvyčajným spôsobom $a$, $b$, $c$ dĺžky jednotlivých strán, platí
$$a = y + z, \ \ \ \  b = z + x, \ \ \ \ c = x + y.$$
Sčítaním týchto troch rovníc dostaneme (pomocou $s$ ako zvyčajne označujeme polovičný obvod trojuholníka)
$$2s = a + b + c = 2x + 2y + 2z,$$
takže nám vyjde
\begin{equation} \label{eq:B63S3}
    x + y + z = s, \ \ \ \ x = s - a,\ \ \ \ y = s - b, \ \ \ \ z = s - c.
\end{equation}
Pozrime sa teraz na pripísanú kružnicu trojuholníku $ABC$, ktorá sa dotýka jeho strany $BC$ v bode $P$ a polpriamok $AB$ a $AC$ v bodoch $R$ a $Q$ (obr.~\ref{fig:B63S3_2}). Zo zhodnosti úsekov príslušných dotyčníc k tejto kružnici máme
$$|AR| = |AQ|, \ \ \ \ |BR| = |BP|, \ \ \ \|CP| = |CQ|,$$
odkiaľ vychádza
\begin{align*}
    2|AR| = |AR| + |AQ| & = |AB| + |BR| + |AC| + |CQ| & =\\
& = |AB| + |BP| + |AC| + |CP| & = a + b + c = 2s,
\end{align*}
čiže $|AR| = |AQ| = s$. Z tejto rovnosti ale vyplýva, že $|BP| = |BR| = s - c$, čo je podľa \ref{eq:B63S3} zároveň dĺžka z úsečky $CX$, teda $|BP| = |CX|$. To znamená, že body $P$ a $X$ sú súmerne združené podľa stredu úsečky $BC$.
\begin{figure}[h]
    \centering
    \includegraphics{images/B63S3_2\imagesuffix}
    \caption{}
    \label{fig:B63S3_2}
\end{figure}
Analogicky by sme odvodili rovnosti $|BK| = s$ a $|CL| = s$ pre body dotyku $K$ a $L$ kružníc pripísaných stranám $CA$ a $AB$ (obr.~\ref{fig:B63S3_2}) trojuholníka $ABC$ s priamkou $a$. Z týchto posledných rovností však vidíme, že $|BL| = s - a = |CK|$, teda aj body $K$ a $L$ sú súmerne združené podľa stredu úsečky $BC$. Body $K$ a $L$ sú známe (z troch daných bodov na priamke sú to tie dva krajné), poznáme teda aj stred $S$ strany $BC$ (je to stred úsečky $KL$) a bod $X$ nájdeme ako obraz tretieho daného bodu $P$ v stredovej súmernosti podľa stredu úsečky $BC$.
}


% Do not delete this line (pandoc magic!)

\problem{B-65-I-3}{
V pravouhlom trojuholníku $ABC$ s preponou $AB$ a odvesnami dĺžok $|AC| = 4$\,cm a $|BC| = 3$\,cm ležia navzájom sa dotýkajúce kružnice $k_1(S_1; r_1 )$ a $k_2(S_2; r_2)$ tak, že $k_1$ sa dotýka strán $AB$ a $AC$, zatiaľ čo $k_2$ sa dotýka strán $AB$ a $BC$. Určte najmenšiu a najväčšiu možnú hodnotu polomeru $r_2$. 
}{
\rieh Majme také dve kružnice, ktoré spĺňajú predpoklady úlohy \todo{(obr. 1)}. Zrejme stred $S_1$ leží na osi uhla $BAC$ a stred $S_2$ na osi uhla $ABC$. Ďalej si uvedomme, že veľkosť\\
\\
\todo{DOPLNIŤ Obr. 1}\\
\\
polomeru $r_1$ kružnice $k_1$ je priamo úmerná dĺžke úsečky $AS_1$ a podobne veľkosť $r_2$ priamo úmerná dĺžke úsečky $BS_2$. Keď zväčšíme polomer jednej z kružníc, musí sa nutne polomer druhej kružnice zmenšiť. 

Kružnica $k_2$ nemôže mať polomer väčší ako najväčšia kružnica, ktorú možno do trojuholníka $ABC$ vpísať. Takou kružnicou je zrejme kružnica $k$ do trojuholníka $ABC$ vpísaná. A naopak najmenší polomer bude mať kružnica $k_2$, ak zvolíme $k_1 = k$. (Že v oboch opísaných prípadoch pre $k_2 = k$ aj pre $k_1 = k$ existuje príslušná \uv{vpísaná} kružnica $k_1$, resp. $k_2$, je vcelku zrejmé.)

Stačí teda vypočítať polomer $r$ kružnice $k$ do trojuholníka $ABC$ vpísanej a polomer kružnice $k_2$, ktorá sa dotýka kružnice $k$ a strán $AB$ a $BC$ daného trojuholníka.

Polomer $r$ vpísanej kružnice vypočítame napríklad zo vzorca $2S_{ABC} = ro$, pričom $S_{ABC}$ označuje obsah trojuholníka $ABC$ a $o$ jeho obvod.%\footnote{Iný postup využívajúci pravouhlosť trojuholníka $ABC$ je predmetom dopĺňajúcej úlohy.} 
Obsah daného pravouhlého trojuholníka $ABC$ s preponou $AB$ je pri zvyčajnom označení dĺžok strán rovný $\frac{1}{2}ab.$ Prepona v trojuholníku $ABC$ má (v centimetroch) podľa Pytagorovej vety veľkosť $c= \sqrt{a^2 + b^2}=\sqrt{3^2 + 4^2} = 5$. Maximálny polomer kružnice $k_2$ je teda 
$$r =\frac{2S_{ABC}}{o}=\frac{ab}{a+b+c}=\frac{3\cdot4}{3+4+5}= 1.$$

Pre výpočet polomeru $r_2$ kružnice $k_2$, ktorá sa dotýka kružnice $k$ a strán $AB$ a $BC$, označme $D$ a $E$ body, v ktorých sa kružnice $k$ a $k_2$ dotýkajú strany $AB$, a $F$, $G$ dotykové body kružnice k postupne so stranami $BC$ a $AC$ \todo{(obr. 2)}. Keďže daný trojuholník je \\
\\
\todo{DOPLNIŤ Obr. 2}\\
\\
pravouhlý, je $S_1FCG$ štvorec so stranou dĺžky $r = 1$, takže $|BF| = |BD| = 2$ a podľa Pytagorovej vety $|BS_1| =\sqrt{5}$. Z podobnosti pravouhlých trojuholníkov $BES_2$ a $BDS_1$ potom vyplýva
$$\frac{r_2}{|BS_2|}=\frac{r}{|BS_1|}, \ \ \ \ \text{čiže} \ \ \ \ \frac{r_2}{\sqrt{5}- r_2 - 1}=\frac{1}{\sqrt{5}}.$$
Po úprave tak pre hľadanú hodnotu neznámej $r_2$ dostaneme lineárnu rovnicu
$$r_2(\sqrt{5} + 1) =\sqrt{5}-1,$$
ktorú ešte zjednodušíme vynásobením $\sqrt{5}-1$. Zistíme tak, že najmenšia možná hodnota polomeru kružnice $k_2$ je rovná $$r_2 = \frac{3-\sqrt{5}}{2}.$$
}


%

\problem{B-61-II-3}{}{
Pravouhlému trojuholníku $ABC$ je vpísaná kružnica, ktorá sa dotýka prepony $AB$ v bode $K$. Úsečku $AK$ otočíme o $90^\circ$ do polohy $AP$ a úsečku $BK$ otočíme o $90^\circ$ do polohy $BQ$ tak, aby body $P$, $Q$ ležali v polrovine opačnej k polrovine $ABC$.
\begin{enumerate}[a)]
    \item Dokážte, že obsahy trojuholníkov $ABC$ a $PQK$ sú rovnaké.
    \item Dokážte, že obvod trojuholníka $ABC$ nie je väčší ako obvod trojuholníka $PQK$.Kedy nastane rovnosť obvodov?
\end{enumerate}
}{
\rieh a) Označme $S$ stred a $r$ polomer kružnice vpísanej trojuholníku $ABC$ a $L$, $M$ body dotyku tejto kružnice postupne so stranami $BC$, $CA$ (obr.~\ref{fig:B61II3}). Ak označíme $|AK| = x$, $|BK| = y$, tak $|AP| = |AM| = x$, $|KP| = x\sqrt{2}$, $|BQ| = |BL| = y$, $|KQ| = y\sqrt{2}$. Keďže oba uhly $AKP$, $BKQ$ majú veľkosť $45^\circ$, je trojuholník $PQK$ pravouhlý, takže jeho obsah je
$$S_{PQK} =\frac{x\sqrt{2}y\sqrt{2}}{2}=xy.$$

\begin{figure}[h]
    \centering
    \includegraphics{images/B61II3\imagesuffix}
    \caption{}
    \label{fig:B61II3}
\end{figure}
Štvoruholník $SLCM$ je štvorec so stranou dĺžky $r$ a $|AM| = x$, $|BL| = y$. Obsah trojuholníka $ABC$ je rovný súčtu obsahov trojuholníkov $ABS$, $BCS$ a $CAS$, teda
$$S_{ABC}=\frac{(x + y)r + (y + r)r + (x + r)r}{2}= (x + y + r)r.$$
Obsah trojuholníka $ABC$ je zároveň rovný
$$S_{ABC}=\frac{|AC| \cdot |BC|}{2}=\frac{(x + r)(y + r)}{2}=\frac{xy}{2}+\frac{(x + y + r)r}{2}=\frac{xy}{2}+\frac{S_{ABC}}{2}.$$
Odtiaľ dostávame $S_{ABC} = xy$, čiže $S_{ABC} = S_{PQK}$, čo sme mali dokázať.


b) V trojuholníku $ABC$ sú dĺžky strán $a = y + r$, $b = x + r$, $c = x + y$. Obvod trojuholníka $ABC$ je $a + b + |AB|$, obvod trojuholníka $PQK$ je $x\sqrt{2} + y\sqrt{2} + |PQ|$.

Zrejme platí $|AB| \leq |PQ|$ ($|AB|$ je vzdialenosťou rovnobežiek $AP$, $BQ$, (obr.~\ref{fig:B61II3}). Rovnosť nastane jedine v prípade $|AP| = |BQ|$, čiže $x = y$.
Ešte dokážeme, že $a + b \leq  x\sqrt{2} + y\sqrt{2}$, teda že $a + b \leq c\sqrt{2}$. Posledná nerovnosť je ekvivalentná s nerovnosťou, ktorú dostaneme jej umocnením na druhú, pretože obe jej strany sú kladné. Dostaneme tak $a^2 +b^2 +2ab \leq 2c^2$. Keďže v pravouhlom trojuholníku $ABC$ platí $a^2 + b^2 = c^2$, máme dokázať nerovnosť $2ab \leq a^2 + b^2$, ktorá je však ekvivalentná s nerovnosťou $0 \leq (a - b)^2$. Tá platí pre všetky reálne čísla $a$, $b$ a rovnosť v nej nastane jedine pre $a = b$, t. j. $x = y$.

Celkovo vidíme, že obvod trojuholníka $ABC$ je menší alebo rovný obsahu trojuholníka $PQK$ a rovnosť nastane práve vtedy, keď je pravouhlý trojuholník $ABC$ rovnoramenný.
}
\home{

\teachernote{
Keďže v nasledujúcom seminári je naplánované opakovanie, úlohou študentov bude si zbežne zopakovať, čomu sme sa posledných 9 mesiacov venovali. Zmyslom domácej práce nie je opätovné prepočítavanie všetkých príkladov, ale skôr získanie prehľadu a nadhľadu nad študovanými témami.
%Keďže v nasledujúcom seminári je naplánované opakovanie v réžii študentov, ich úlohou bude pripraviť si zhrnutie jednej zo štyroch oblastí, ktorými sme sa v seminári zaoberali (algebra, teória čísel, geometria, kombinatorika). Cieľom nie je znova prepočítavať všetky úlohy, ale zrozumiteľne zhrnúť kľúčové poznatky a stratégie, ktoré sme spolu so študentmi v seminároch používali. Študentov rozdelíme do štyroch skupín a každej z nich pridelíme jednu oblasť.
}
}