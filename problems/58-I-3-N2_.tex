% Do not delete this line (pandoc magic!)

\problem{58-I-3-N2 resp.54-C-I-5}{
Určte počet všetkých trojíc dvojciferných prirodzených čísel $a$, $b$, $c$, ktorých súčin $abc$ má zápis, v~ktorom sú všetky cifry rovnaké. Trojice líšiace sa len poradím čísel považujeme za rovnaké,  t.\,j. započítavame ich iba raz.
}{
\rieh Pre dvojmiestne čísla $a, b, c$ je súčin $abc$ číslo štvormiestne, alebo päťmiestne, alebo šesťmiestne. Ak sú teda všetky číslice čísla $abc$ rovné jednej číslici $k$, platí jedna z~rovností $abc = k~\cdot 1 111$, $abc = k~\cdot 11 111$ alebo $abc = k~\cdot 111 111$, $k \in \{1, 2, \ldots , 9\}$.

Čísla $1 111 = 11\cdot 101$ a $11111 = 41\cdot 271$ však majú vo svojom rozklade trojmiestne prvočísla, takže nemôžu byť súčinom dvojmiestnych čísel. Ostáva preto jediná možnosť:
$$ abc = k~\cdot 111 111 = k~\cdot 3 \cdot 7 \cdot 11 \cdot 13 \cdot 37.$$
Pozrime sa, ako môžu byť prvočísla 3, 7, 11, 13, 37 rozdelené medzi jednotlivé činitele $a, b, c$. Pretože súčiny $37 \cdot 3$ a $3 \cdot 7 \cdot 11$ sú väčšie ako 100, musí byť prvočíslo 37 samo ako jeden činiteľ a zvyšné štyri prvočísla 3, 7, 11, 13 musia byť rozdelené do dvojíc. Keďže aj súčin $11 \cdot 13$ je väčší ako 100, prichádzajú do úvahy iba rozdelenia na činitele $3 \cdot 11, 7 \cdot 13$ a 37, alebo na činitele $3 \cdot 13$, $7 \cdot 11$ a 37. K~týmto činiteľom ešte pripojíme možné činitele z~rozkladu číslice $k$ a dostaneme riešenia dvoch typov:
$$a = 33k_1, b = 91, c = 37k_2, \ \ \ \ \text {pričom} \ \ k_1 \in \{1, 2, 3\}, k_2 \in \{1, 2\},$$
$$a = 39k_1, b = 77, c = 37k_2,\ \ \ \ \text{pričom}\ \ k_1 \in \{1, 2\}, k_2 \in \{1, 2\},$$
Hľadaný počet trojíc čísel $a, b, c$ je teda $3 \cdot 2 + 2 \cdot 2 = 10$.\\
\\
\kom Posledná úloha seminára je zaujímavá myšlienkou, že čísla, ktoré majú všetky cifry rovnaké, sú násobkami čísel 11, 111, 1111, \ldots. Ďalšia analýza uplatní poznatky o~rozklade čísla na súčin prvočísel a je tak ďalším príkladom úlohy, v~ktorej musia študenti zapojiť vedomosti z~predchádzajúcich seminárov.
}
