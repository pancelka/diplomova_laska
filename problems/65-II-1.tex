% Do not delete this line (pandoc magic!)

\problem{65-II-1}
{Nájdite najmenšiu možnú hodnotu výrazu $$3x^2 - 12xy + y^4,$$
v~ktorom $x$ a $y$ sú ľubovoľné celé nezáporné čísla.
}{
\rieh Označme daný výraz $V$ a upravme ho dvojakým doplnením na štvorec: $$V = 3x^2 - 12xy + y^4= 3(x - 2y)^2 - 12y^2+ y^4= 3(x - 2y)^2+ (y^2 - 6)^2 - 36.$$
Zrejme platí $(x-2y)^2\geq0$, takže najmenšiu hodnotu výrazu V~pri pevnom $y$ dostaneme, keď položíme $x = 2y$. Ostáva preto nájsť najmenšiu možnú hodnotu mocniny $(y^2 - 6)^2$
s~nezápornou celočíselnou premennou $y$. Keďže $y^2 \in \{0, 1, 4, 9, 16, 25, \ldots\}$ a číslo 6 padne medzi čísla 4 a 9 tejto množiny, platí pre každé celé číslo $y$ nerovnosť $$(y^2 - 6)^
2\geq \mathrm{min} ((4 - 6)^2, (9 - 6)^2) = \mathrm{min}\{4, 9\} = 4.$$ Pre ľubovoľné celé čísla $x$ a $y$ tak dostávame odhad
$$V \geq 3 \cdot 0 + 4 - 36 = -32,$$
pritom rovnosť $V = -32$ nastáva pre $y = 2$ a $x = 2y = 4$.\\
\textit{Záver.} Hľadaná najmenšia možná hodnota daného výrazu je -32.\\
}