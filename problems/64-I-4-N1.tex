\problem{64-I-4-N1}{
Lichobežník $ABCD$ má základne s~dĺžkami $|AB|=a$ a $|CD|=C$ a jeho uhlopriečky sa pretínajú v~bode $U$. Aký je pomer obsahov trojuholníkov $ABU$ a $CDU$?
}{
\rie Trojuholníky $ABU$ a $CDU$ sú zrejme podobné ($|\ma BAU|=|\ma UCD|$, $|\ma ABU|=|\ma CDU|$, $|\ma AUB|=|\ma CUD|$, pretože prvé dve sú dvojice striedavých uhlov, posledné dva sú uhly vrcholové) s~koeficientom podobnosti $k=a/c$. Preto pre výšku $v_1$ na stranu $AB$ v~trojuholníku $ABU$ a výšku $v_2$ na stranu $CD$ v~trojuholníku $CDU$ platí $v_1/v_2=k$, resp. $v_1=kv_2=(av^2)/c$. Potom pre pomer obsahov trojuholníkov $ABU$ a $CDU$ máme
$$\frac{S_{ABU}}{S_{CDU}}=\frac{\frac{1}{2}av_1}{\frac{1}{2}cv_2}=\frac{a\frac{a}{cv_2}}{cv_2}=\frac{a^2}{c^2}.$$\\
\\
\textbf{Záverečný komentár} Na prvý pohľad by sa mohlo zdať, že študenti budú o(c)hromení množstvom nových poznatkov v~tomto seminári. Dúfame však, že sa tak nestane, keďže veľká väčšina obsahu by mala byť prinajmenšom povedomá, ak nie úplne zrozumiteľná. Seminár tiež patrí k~tým menej náročným, avšak je veľmi dôležitou prípravou pred tvrdšími orieškami.
}
