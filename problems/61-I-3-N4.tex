% Do not delete this line (pandoc magic!)

\problem{61-I-3-N4, resp. 50-II-1}{seminar08,nsdnsn,domacekolo}{
Nájdite všetky dvojice prirodzených čísel $a, b$, pre ktoré platí $a+b+[a, b]+(a, b) = 50$.
}{
\rieh Položme $a = ud$, $b = vd$, kde $d$ je najväčší spoločný deliteľ čísel $a, b$, prirodzené čísla $u, v$ sú nesúdeliteľné a $[a,b]=uvd$. Podľa zadania má platiť $ud+vd+uvd+d= 50$. Inak napísané, $(1 + u)(1 + v)d = 50$. Nájdime preto všetky rozklady čísla 50 na súčin troch prirodzených čísel $d, u+1, v+1$, z~ktorých posledné dve sú väčšie ako 1. Bez ujmy na všeobecnosti môžeme predpokladať, že $a \leq b$, tj. $u \leq v$. Dostaneme nasledujúce možnosti.
\begin{center}
\begin{tabular}{|c|c|c|c|c|c|c|}
\hline
$d$ & $u+1$ & $v+1$ & $u$ & $v$ & $a$ & $b$\\
\hline
1 & 2 & 25 & 1 & 24 & 1 & 24 \\
1 & 5 & 10 & 4 & 9 & 4 & 9\\
2 & 5 & 5 & 4 & 4 & 8 & 8 \\
5 & 2 & 5 & 1 & 4 & 5 & 20\\
\hline
\end{tabular}
\end{center}
V~prípade $d=2$ dostaneme $u=v=4$, to je však spor s~tým, že $u$ a $v$ sú nesúdeliteľné. Preto má úloha práve tri riešenia. \\
\\
\kom Úloha okrem vhodného zapísania čísel $a$, $b$ a $[a,b]$ vyžaduje ešte vhodnú úpravu rovnosti zo zadania, opäť tak kombinuje algebraické poznatky s~poznatkami z~oblasti teórie čísel.\\
\\
}
