% Do not delete this line (pandoc magic!)

\problem{B-59-I-5}{}{
Vnútri kratšieho oblúka $AB$ kružnice opísanej rovnostrannému trojuholníku $ABC$ je zvolený bod $D$. Tetiva $CD$ pretína stranu $AB$ v bode $E$. Dokážte, že trojuholník so stranami dĺžok $|AE|$, $|BE|$, $|CE|$ je podobný s trojuholníkom $ABD$.
}{
\rieh Veďme bodom $E$ rovnobežku so stranou $BC$ a označme $F$ jej priesečník so stranou $AC$. Trojuholník $AEF$ je rovnostranný, preto $|EF| = |AE|$ a tiež $|CF|= |BE|$. Trojuholník $FEC$ má teda dĺžky strán $|AE|$, $|BE|$, $|CE|$. Dokážeme, že je podobný s trojuholníkom $ABD$ \todo{doplniť (obr. 2)}.\\
\\
\todo{DOPLNIŤ Obr. 2}\\
\\
Uhly $ACD$ a $ABD$ sú obvodové nad tetivou $AD$, preto sú zhodné. Uhol $FEC$ je zhodný s uhlom $ECB$ (striedavé uhly) a ten je zhodný s obvodovým uhlom $DAB$. Podľa vety $uu$ sú teda trojuholníky $ECF$ a $ABD$ naozaj podobné.

\textbf{Iné riešenie*.} Obvodové uhly $DAB$ a $DCB$ sú zhodné, rovnako aj uhly $ADC$
a $ABC$, a preto sú trojuholníky $ADE$ a $CBE$ podobné. Odtiaľ vyplýva $|AE|/|AD| = |CE|/|CB| = |CE|/|AB|$. Analogicky aj trojuholníky $DEB$ a $AEC$ sú podobné, odkiaľ $|BE|/|BD| = |CE|/|AC| = |CE|/|AB|$. Z rovností $|AE|/|AD| = |CE|/|AB|= |BE|/|BD|$ vyplýva podobnosť trojuholníka s dĺžkami strán $|AE|$, $|CE|$, $|BE|$ s trojuholníkom $ABD$.
}
