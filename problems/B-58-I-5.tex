% Do not delete this line (pandoc magic!)

\problem{B-58-I-5}{
Trojuholníku $ABC$ je opísaná kružnica $k$. Os strany $AB$ pretne kružnicu $k$ v bode $K$, ktorý leží v polrovine opačnej k polrovine $ABC$. Osi strán $AC$ a $BC$ pretnú priamku $CK$ postupne v bodoch $P$ a $Q$. Dokážte, že trojuholníky $AKP$ a $KBQ$ sú zhodné.
}{
\rieh Označme $\alpha, \beta, \gamma$ zvyčajným spôsobom veľkosti vnútorných uhlov trojuholníka $ABC$ (obr.~\ref{fig:B58I5}). Bod $K$ leží na osi úsečky $AB$, preto $|AK| = |KB|$. Trojuholník $AKB$ je rovnoramenný so základňou $AB$, jeho vnútorné uhly pri vrcholoch $A$ a $B$ sú
\begin{figure}[h]
    \centering
    \includegraphics{images/B58I5\imagesuffix}
    \caption{}
    \label{fig:B58I5}
\end{figure}
teda zhodné. Podľa vety o obvodových uhloch sú zhodné aj uhly $BCK$ a $BAK$, resp.$ACK$ a $ABK$, preto sú zhodné aj uhly $BCK$ a $ACK$. Polpriamka $CK$ je teda osou uhla $ACB$: 
$$|\ma ACK| = |\ma BCK| = \frac{\gamma}{2}.$$
Keďže bod $P$ leží na osi strany $AC$, je trojuholník $ACP$ rovnoramenný a jeho vnútorné uhly pri základni $AC$ majú veľkosť $\frac{1}{2}\gamma$, takže jeho vonkajší uhol $APK$ pri vrchole $P$ má veľkosť $\frac{1}{2}\gamma + \frac{1}{2}\gamma = \gamma$. Rovnako z rovnoramenného trojuholníka $BCQ$  odvodíme, že aj veľkosť uhla $BQK$ je $\gamma$. Podľa vety o obvodových uhloch sú zhodné uhly $ABC$ a $AKC$, teda uhol $AKC$ (čiže uhol $AKP$) má veľkosť $\beta$ a -- celkom analogicky -- uhol $BKQ$ má veľkosť $\alpha$.

V každom z trojuholníkov $AKP$ a $BKQ$ už poznáme veľkosti dvoch vnútorných uhlov ($\beta$, $\gamma$, resp. $\alpha$, $\gamma$), takže vidíme, že zostávajúce uhly $KAP$ a $KBQ$ majú veľkosti$\alpha$, resp. $\beta$.

Z predošlého vyplýva, že trojuholníky $AKP$ a $KBQ$ sú zhodné podľa vety $usu$, lebo majú zhodné strany $AK$ a $KB$ aj obe dvojice k nim priľahlých vnútorných uhlov.

K uvedenému postupu dodajme, že výpočet uhlov $KAP$ a $KBQ$ cez uhly $APK$ a $BQK$ možno obísť takto: zhodnosť uhlov $KAP$ a $BAC$ (resp. $KBQ$ a $ABC$) vyplýva zo zhodnosti uhlov $KAB$ a $PAC$ (resp. $KBA$ a $QBC$).\\
\\
\kom Posledná úloha seminára pekne kombinuje vlastnosti uhlov a zhodnosť trojuholníkov, je tak dôstojným zakončením tohto geometrického stretnutia.
}
