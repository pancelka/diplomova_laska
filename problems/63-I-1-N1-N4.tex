% Do not delete this line (pandoc magic!)

\problem{63-I-1-N1-N4}{seminar04,vyrazy}{
a) Určte najmenšiu hodnotu výrazu $V = 5 + (x - 2)^2$, $x \in \RR$. Pre ktoré $x$ ju výraz nadobúda?

b) Určte najmenšiu možnú hodnotu výrazu $W = 9 - ab$, kde $a, b$ sú reálne čísla spĺňajúce podmienku $a + b = 6$. Pre ktoré hodnoty $a, b$ je $W$ minimálne?

c) Určte najmenšiu možnú hodnotu výrazu $Y = 12-ab$, kde $a, b$ sú reálne čísla spĺňajúce podmienku $a + b = 6$. Pre ktoré hodnoty $a, b$ je $Y$ minimálne?

d) Určte najväčšiu možnú hodnotu výrazu $K = 5 + ab$, kde $a, b$ sú reálne čísla spĺňajúce podmienku $a + b = 8$. Pre ktoré hodnoty $a, b$ je $K$ maximálne?
}{
\rie

a) Výraz $(x-2)^2$ je pre všetky reálne čísla $x$ nezáporný, jeho minimálna hodnota tak je 0, v~prípade $x=2$. Najmenšia hodnota výrazu $V$ je potom 5.

b) Z~podmienky vyjadríme premennú $a$ pomocou premennej $b$, teda $a=6-b$, dosadíme do $W$ a upravíme použitím vzorca $(A-B)^2$: $W=9-ab=9-(6-b)b=(b-3)^2$. Hodnota výrazu $W$ je tak vždy nezáporná a $W_{min}=0$, čo nastane pre $b=3$ a $a=3$.

c) Všimneme si, že $Y=W+3$ a keďže aj podmienka je v~tomto prípade rovnaká, platí $Y=(b-3)^2+3$. Potom $Y_{min}=3$ pre $a=3$ a $b=3$.

d) Podobne ako v~predchádzajúcich častiach, vyjadríme z~podmienky premennú $a$ pomocou premennej $b$: $a=8-b$ a dosadíme do výrazu $K$: $K = 5 + 8a - a^2= -(a - 4)^2+ 21$, kde sme využili tzv. úpravu na štvorec. Vidíme, že časť $-(a-4)^2$ je vždy nekladná, preto $K_{max}=21$ pre $a = b = 4$.\\

\kom V~častiach c) a d) predchádzajúcej úlohy sme využili tzv. úpravu výrazu na štvorec. Aj napriek tomu, že tento úkon je v~ŠVP zaradený až v~neskoršej časti školského roka (v~časti o~kvadratických rovniciach), považujeme za vhodné študentov s~metódou zoznámiť už teraz. Je totiž prirodzene spojená s~úpravami výrazov, jej pochopenie je možné aj bez širších znalostí riešenia kvadratických rovníc a v~úlohách MO nájde uplatnenie často.
}