\seminar{10}{Geometria I~-- základné poznatky}
\teachernote{
\subsection*{Ciele}
Zopakovať a upevniť základné poznatky z~planimetrie, ktoré by študenti mali mať zo základnej školy. Venovať sa vlastnostiam uhlov, trojuholníkov, štvoruholníkov a kružníc. Niektoré z~poznatkov odvodiť.

\subsubsection*{Úvodný komentár}

Keďže planimetria častokrát nie je súčasťou osnov 1. ročníka gymnázií, je potrebné poznatky žiakov z~tejto oblasti o~to starostlivejšie zopakovať. Geometrické úlohy majú veľmi často najhoršiu úspešnosť v~krajských kolách MO, čo môže mať viacero dôvodov. Nepopierateľne však študentom tréning pomôže, preto je geometrii v~priebehu roka venovaných 8 seminárov.

Zo zmienených dôvodov má preto tento seminár odlišnú štruktúru ako predchádzajúce -- viac ako riešeniu úloh z~olympiád sa venujeme opakovaniu základných vlastností uhlov, trojuholníkov, štvoruholníkov a kružníc, ktorých znalosti budú nenahraditeľné v~ďalších piatich geometrických seminároch. Spolu so študentmi tak vytvoríme základnú výbavu, ktorá im pomôže v~boji s~geometrickými záludnosťami.

Študenti by mali mať nasledujúce znalosti (voľne spracované podľa~\cite{kubat2000}):
\begin{itemize}
\item uhly
\begin{itemize}
\item chápať pojmy vrcholové, vedľajšie, súhlasné a striedavé uhly, vedieť nájsť dvojice takých uhlov a používať ich pri riešení úloh,
\end{itemize}
\item trojuholníky
\begin{itemize}
\item poznať základné vlastnosti strán a vnútorných uhlov trojuholníka: trojuholníková nerovnosť, súčet vnútorných uhlov,
\item vedieť popísať rozdiely medzi ostrouhlým, pravouhlým, tupouhlým, všeobecným, rovnoramenným a rovnostranným trojuholníkom,
\item chápať pojmy os uhla, os strany, výška, ťažnica, stredná priečka, kružnica vpísaná a opísaná trojuholníku a poznať ich vlastnosti,
\item poznať a vedieť používať vzorec na výpočet obsahu trojuholníka,
\item poznať a vhodne používať vety o~zhodnosti ($sss$, $sus$, $usu$, $Ssu$) a podobnosti ($sss$, $sus$, $uu$, $Ssu$) trojuholníkov,
\item poznať a používať Pytagorovu vetu pre pravouhlý trojuholník,
\end{itemize}
\item štvoruholníky
\begin{itemize}
\item vedieť popísať všeobecný štvoruholník a jeho špecifické prípady: rovnobežník, štvorec, obdĺžnik, kosoštvorec, kosodĺžnik, lichobežník,
\item poznať základné vzorce pre výpočet obsahu rôznych rovnobežníkov a lichobežníkov,
\item vedieť, že uhlopriečky v~pravouholníku a rovnobežníku sa polia a vedieť tento fakt využiť pri riešení úloh,
\end{itemize}
\item kružnice a kruhy
\begin{itemize}
\item chápať pojmy kružnica, kruh, kružnicový oblúk, dotyčnica, sečnica, tetiva, stredový a obvodový uhol,
\item poznať a vedieť používať Talesovu kružnicu,
\end{itemize}
\item riešenie konštrukčných úloh
\begin{itemize}
\item náčrt, rozbor, popis konštrukcie, diskusia o~počte riešení.
\end{itemize}
\end{itemize}
\kom Skôr než frontálny výklad je vhodné nechať skladať mozaiku vedomostí študentov. Ak pracujeme s~malou skupinou, môžeme o~vyššie spomenutých bodoch diskutovať všetci spoločne. Ak seminár navštevuje väčšie množstvo záujemcov o~matematiku, rozdelíme študentov na menšie skupiny, pričom každá spracuje poznatky o~zadanej neprázdnej podmnožine vyššie spomenutých oblastí. Tie si potom študenti navzájom odprezentujú, vedúci seminára nepresnosti vhodnými otázkami koriguje. Táto časť by mala zabrať približne polovicu, príp. dve tretiny času vyhradeného na seminár.
\\
\kom V~druhej polovici (až poslednej tretine seminára) niektoré zo základných tvrdení, ktoré budeme v~priebehu ďalších stretnutí využívať, dokážeme.\\
}

\subsection*{Úlohy a riešenia}

% Do not delete this line (pandoc magic!)

\problem{anonymous 1}{
Dokážte, že súčet veľkostí vnútorných uhlov ľubovoľného trojuholníka je $180^\circ$.
}{
\rie Veďme rovnobežku $XY$ so stranou $AB$ vrcholom $C$ trojuholníka $ABC$, tak že bod~$C$ leží medzi bodmi $X$ a $Y$ \todo{(obr.1)}. \\
\\
\todo{DOPLNIŤ Obr. 1}\\
\\
Potom $|\ma BAC|=|\ma ACX|$ a $|\ma ABC|=|\ma BCY|$, pretože ide o~dvojice striedavých uhlov. Keďže $|\ma ACX|+|\ma ACB|+ |\ma BCY|=180^\circ$, pretože uhol $XCY$ je priamy, platí aj $|\ma BAC|+|\ma ABC|+|\ma ACB|=180^\circ$.
}


% Do not delete this line (pandoc magic!)

\problem{66-I-3-N1}{seminar10,geomlah}{
Z~trojuholníkových nerovností medzi dĺžkami strán ľubovoľného trojuholníka odvoďte známe
pravidlo $\alpha < \beta \Rightarrow a < b$ o~porovnaní veľkostí vnútorných uhlov a dĺžok protiľahlých strán v~ľubovoľnom trojuholníku $ABC$.
}{
\rieh Ak je $\alpha  < \beta$, môžeme nájsť vnútorný bod $X$ strany $AC$, pre ktorý platí $|\ma ABX| = \alpha$, a teda $|AX| = |BX|$, takže z~trojuholníkovej nerovnosti $|BC| < |BX| + |XC|$ už vyplýva $a < b$.
\begin{figure}[h]
    \centering
    \includegraphics{images/66I3N1\imagesuffix}
    \caption{}
    \label{fig:66I3N1}
\end{figure}
}


% Do not delete this line (pandoc magic!)

\problem{63-I-4-N3}{
Dokážte vety:

a) Ak majú dva trojuholníky rovnakú výšku, potom pomer ich obsahov sa rovná pomeru dĺžok príslušných základní.

b) Ak majú dva trojuholníky zhodné základne, potom pomer ich obsahov sa rovná pomeru príslušných výšok.
}{
\rie a) Označme rovnakú výšku dvoch trojuholníkov $v$. V~trojuholníku $T_1$ je táto výškou na základňu $a_1$, v~trojuholníku $T_2$ na základňu $a_2$. Pomer obsahov týchto trojuholníkov je potom $$\frac{S_{T_1}}{S_{T_2}}=\frac{\frac{1}{2}a_1v}{\frac{1}{2}a_2v}=\frac{a_1}{a_2},$$ čo sme chceli dokázať.

b) Označme zhodnú základňu dvoch trojuholníkov $z$, v~trojuholníku $T_1$ je výška na túto základňu $v_1$, v~trojuholníku $T_2$ je výška na túto základňu $v_2$. Pomer obsahov trojuholníkov $T_1$ a $T_2$ je
$$\frac{S_{T_1}}{S_{T_2}}=\frac{\frac{1}{2}zv_1}{\frac{1}{2}zv_2}=\frac{v_1}{v_2},$$
čo je pomer príslušných výšok.\\
\\
}


% Do not delete this line (pandoc magic!)

\problem{61-I-5-N1}{
Pre všeobecný trojuholník $ABC$ so stranami $a$, $b$, $c$ a obsahom $S$ platí pre polomer $r$ vpísanej kružnice vzorec $r = 2S/(a + b + c)$. Dokážte.
}{
\rieh Stred $M$ vpísanej kružnice rozdeľuje uvažovaný trojuholník $ABC$ na tri menšie trojuholníky $BCM$, $ACM$, $ABM$ s~obsahmi $\frac{1}{2}ar$, $\frac{1}{2}br$, $\frac{1}{2}cr$, ktorých súčet je $S$, odkiaľ vyplýva dokazovaný vzorec.\\
\\
}


% Do not delete this line (pandoc magic!)

\problem{anonymous 2}{
Dokážte, že uhlopriečky v~rovnobežníku sa navzájom polia.
}{
\rie Označme $U$ priesečník uhlopriečok $AC$ a $BD$ rovnobežníka $ABCD$. Keďže uhly $ABD$ a $BDC$ sú striedavé, majú rovnakú veľkosť. Podobne uhly $BAC$ a $ACD$ sú rovnako veľké, pretože sú takisto dvojicou striedavých uhlov. Potom sú trojuholníky $ABU$ a $CDU$ zhodné, keďže sa zhodujú v~jednej strane $|AB|=|CD|$ a v~dvoch k~nej priľahlých uhloch. Preto aj $|AU|=|UC|$, $|BU|=|UD|$ a tvrdenie je dokázané.\\
\\
}


\problem{58-I-4-N1}{
Označme $U$ priesečník uhlopriečok daného konvexného štvoruholníka $ABCD$. Dokážte, že priamky $AB$ a $CD$ sú rovnobežné práve vtedy, keď trojuholníky $ADU$ a $BCU$ majú rovnaký obsah.
}{
\rie Rovnosť obsahov trojuholníkov $ADU$ a $BCU$ je ekvivalentná s~rovnosťou obsahov trojuholníkov $ABC$ a $ABD$ so spoločnou stranou $AB$, pretože $S_{ABC}=S_{ABU}+S_{BCU}$ a $S_{ABD}=S_{ABU}+S_{AUD}$. Trojuholníky $ABC$ a $ABD$ majú spoločnú základňu $AB$, takže ich obsahy budú rovnaké práve vtedy, ak výšky na túto stranu budú rovnaké, resp. ak body $C$ a $D$ budú od priamky $AB$ rovnako vzdialené. To nastane len v~prípade, ak body $C$ a $D$ ležia na priamke rovnobežnej s~priamkou $AB$, čo sme chceli dokázať.\\
\\
}


% Do not delete this line (pandoc magic!)

\problem{64-I-4-N1}{seminar10,geomlah}{
Lichobežník $ABCD$ má základne s~dĺžkami $|AB|=a$ a $|CD|=c$ a jeho uhlopriečky sa pretínajú v~bode $U$. Aký je pomer obsahov trojuholníkov $ABU$ a $CDU$?
}{
\rie
\begin{figure}[h]
    \centering
    \includegraphics{images/64I4N1\imagesuffix}
    \caption{}
    \label{fig:64I4N1}
\end{figure}
Trojuholníky $ABU$ a $CDU$ sú zrejme podobné ($|\ma BAU|=|\ma UCD|$, $|\ma ABU|=|\ma CDU|$, $|\ma AUB|=|\ma CUD|$, pretože prvé dve sú dvojice striedavých uhlov, posledné dva sú uhly vrcholové) s~koeficientom podobnosti $k=a/c$. Preto pre výšku $v_1$ na stranu $AB$ v~trojuholníku $ABU$ a výšku $v_2$ na stranu $CD$ v~trojuholníku $CDU$ platí $v_1/v_2=k$, resp. $v_1=kv_2=(av_2)/c$. Potom pre pomer obsahov trojuholníkov $ABU$ a $CDU$ máme
$$\frac{S_{ABU}}{S_{CDU}}=\frac{\frac{1}{2}av_1}{\frac{1}{2}cv_2}=\frac{a\frac{av_2}{c}}{cv_2}=\frac{a^2}{c^2}.$$\\
\\
\textbf{Záverečný komentár}\\ Na prvý pohľad by sa mohlo zdať, že študenti budú o(c)hromení množstvom nových poznatkov v~tomto seminári. Dúfame však, že sa tak nestane, keďže veľká väčšina obsahu by mala byť prinajmenšom povedomá, ak nie úplne zrozumiteľná. Seminár tiež patrí k~tým menej náročným, avšak je veľmi dôležitou prípravou pred tvrdšími orieškami.
}



\home{


% Do not delete this line (pandoc magic!)

\problem{58-I-2-D1}{}{
Nech $k$ je kružnica opísaná pravouhlému trojuholníku $ABC$ s~preponou $AB$ dĺžky $c$. Označme $S$ stred strany $AB$ a $D$ a $E$ priesečníky osí strán $BC$ a $AC$ s~jedným oblúkom $AB$ kružnice $k$. Vyjadrite obsah trojuholníka $DSE$ pomocou dĺžky prepony $c$.
}{
\rie Trojuholník $DSE$ je pravouhlý rovnoramenný s~pravým uhlom pri vrchole $S$, pretože odvesny $DS$ a $ES$ ležia na osiach navzájom kolmých strán. Odvesny majú dĺžku $\frac{c}{2}$, pretože sú to polomery kružnice opísanej trojuholníku $ABC$. Obsah trojuholníka $DSE$ je $\frac{1}{2}\cdot|DS|\cdot |DE|=\frac{1}{2}\cdot \frac{c}{2}\cdot\frac{c}{2}=\frac{c^2}{8}.$ \\
\\
}


% Do not delete this line (pandoc magic!)

\problem{58-I-2-D2}{
Vyjadrite obsah rovnoramenného lichobežníka $ABCD$ so základňami $AB$ a $CD$ pomocou dĺžok $a$, $c$ jeho základní a dĺžky $b$ jeho ramien.
}{
\rie Bez ujmy na všeobecnosti môžeme predpokladať, že $a>b$. Najprv vyjadríme výšku $v$ pomocou dĺžok základní a odvesien. Nech je $P$ päta výšky z~bodu $D$ na stranu $AB$. Potom $|AP|=(a-c)/2$. Použitím Pytagorovej vety v~pravouhlom trojuholníku $APD$ máme
$$\bigg(\frac{a-c}{2}\bigg)^2+v^2=b^2,$$
odkiaľ $v=\sqrt{b^2-(\frac{a-c}{2})^2}=\frac{1}{2}\sqrt{4b^2-(a-c)^2}$ a preto pre obsah lichobežníka dostávame $$S_{ABCD}=\frac{a+c}{2}\cdot v=\frac{1}{4}(a+c)\sqrt{4b^2-(a-c)^2}.$$
}


%% Do not delete this line (pandoc magic!)

\problem{anonymous 3}{
Použitím viet o~podobnosti trojuholníkov a Pytagorovej vety odvoďte Euklidove vety o~odvesne a o~výške pravouhlého trojuholníka.
}{
\rie Prehľadné odvodenie je možne nájsť v ~\cite{kadlecek1996}.
}

}

\teachernote{
\subsection*{Doplňujúce zdroje a materiály}
Ak študenti budú stále neistí v~používaní základných geometrických poznatkov, je možné ich odkázať na základoškolské učebnice geometrie, v~ktorých nájdu aj jednoduchšie príklady na precvičenie, príp. vhodným doplnkom geometrického vzdelania je aj publikácia ~\cite{kadlecek1996}.
}

