% Do not delete this line (pandoc magic!)

\problem{58-I-3-N1 resp. 56-C-S-1}{
Určte počet všetkých štvorciferných prirodzených čísel, ktoré sú deliteľné šiestimi a v~ktorých zápise sa vyskytujú práve dve jednotky.
}{
\rieh  Aby číslo bolo deliteľné šiestimi, musí byť párne a mať ciferný súčet deliteľný tromi. Označme teda $b$ číslicu na mieste jednotiek (tá musí byť párna, $b \in \{0, 2, 4, 6, 8\}$) a $a$ tú číslicu, ktorá je spolu s~číslicami 1, 1 ($a \neq 1$) na prvých troch miestach štvorciferného čísla, ktoré spĺňa požiadavky úlohy. Aby bol súčet číslic $a + 1 + 1 + b$ takého čísla deliteľný tromi, musí číslo $a + b$ dávať po delení tromi zvyšok 1. Pre $b \in \{0, 6\}$ tak máme pre $a$ možnosti $a \in {4, 7}$ $(a \neq 1)$, pre $b \in \{2, 8\}$ máme $a \in \{2, 5, 8\}$ a konečne pre $b = 4$ máme $a \in \{0, 3, 6, 9\}$. Pre každé zvolené $b$ a zodpovedajúce $a \neq 0$ sú zrejme tri možnosti, ako číslice 1, 1 a $a$ na prvých troch miestach usporiadať, to je spolu $(2 \cdot 2 + 2 \cdot 3 + 3) \cdot 3 = 39$ možností, pre $a = 0$ (keď $b = 4$) potom sú len dve možnosti (číslica nula nemôže byť prvá číslica štvorciferného čísla).

Celkom existuje 41 štvorciferných prirodzených čísel, ktoré spĺňajú podmienky.

Alternatívnym postupom je vypísanie všetkých možností na základe ciferného súčtu, ktorý musí byť deliteľný troma a zároveň sa končiť párnou cifrou.\\
\\
\kom Úloha využíva poznatky o~deliteľnosti, takže pekne nadväzuje na predchádzajúce semináre. Tiež je prvou úlohou o~cifernom zápise, v~riešení ktorej nevyužijeme rozvinutý zápis čísla, ale skôr intuitívne kombinatorické úvahy.

Ak sa študenti vyberú cestou vypisovania všetkých možných kombinácií, skúsime ich povzbudiť, aby ich úsilie bolo čo najsystematickejšie a efektívne, príp. prediskutujeme, či sa riešenie dá nájsť aj inou cestou. \\
\\
}
