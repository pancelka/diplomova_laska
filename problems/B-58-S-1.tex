% Do not delete this line (pandoc magic!)

\problem{B-58-S-1}{
V obore reálnych čísel riešte sústavu rovníc
\begin{align*}
    ax + y & = 2,\\
    x - y & = 2a,\\
    x + y & = 1
\end{align*}
s neznámymi $x, y$ a reálnym parametrom $a$. 
}{
\rieh Sčítaním druhej a tretej rovnice dostaneme $2x = 2a + 1$, odčítaním druhej rovnice od tretej $2y = -2a + 1$. Odtiaľ vyjadríme 
$$x = a+\frac{1}{2},\ \ \ \ y = -a + \frac{1}{2}\ \ \ \ (1)$$
a dosadíme do prvej rovnice pôvodnej sústavy. Po úprave dostaneme kvadratickú rovnicu
$$a^2 - \frac{1}{2}a - \frac{3}{2}= 0, \ \ \ \  (2)$$
ktorá má korene $a_1 = -1$ a $a_2 =\frac{3}{2}$. Pre každú z týchto dvoch (jediných možných) hodnôt parametra $a$ už ľahko stanovíme neznáme $x$ a $y$ dosadením do vzťahov \todo{fixni ma (1)}.

Daná sústava rovníc má riešenie iba pre dve hodnoty parametra $a$, jednak pre $a = -1$, keď je jej jediným riešením $(x, y) =\big(-\frac{1}{2}, \frac{3}{2}\big)$, jednak pre $a=\frac{3}{2}$, keď $(x, y) = (2, -1)$.

Skúška dosadením je jednoduchá, možno ju vynechať takýmto zdôvodnením: Sústava dvoch rovníc, ktorú sme dostali (a vyriešili) sčítaním a odčítaním druhej a tretej rovnice, je s dvojicou pôvodných rovníc ekvivalentná. Zostávajúca (prvá) rovnica sústavy je potom ekvivalentná s kvadratickou rovnicou \todo{fxni ma(2)}, ktorej riešením sme našli možné hodnoty parametra $a$.
}
