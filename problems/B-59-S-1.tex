% Do not delete this line (pandoc magic!)

\problem{B-59-S-1}{
Určte všetky hodnoty reálnych parametrov $p, q$, pre ktoré má každá z~rovníc
$$x(x - p) = 3 + q, \ \ \ \ x(x + p) = 3 - q$$
v~obore reálnych čísel dva rôzne korene, ktorých aritmetický priemer je jedným z~koreňov
zvyšnej rovnice.
}{
\rieh Z~Viètových vzťahov pre korene kvadratickej rovnice (ktoré vyplývajú z~rozkladu daného kvadratického trojčlena na súčin koreňových činiteľov) ľahko zistíme, že súčet koreňov prvej rovnice je $p$, takže ich aritmetický priemer je $\frac{1}{2}p$. Toto číslo má byť
koreňom druhej rovnice, preto
$$\frac{p}{2}\cdot \frac{3p}{2}= 3 - q. \ \ \ \ (1)$$
Podobne súčet koreňov druhej rovnice je $-p$, ich aritmetický priemer je $-\frac{1}{2}p$, a preto
$$-\frac{p}{2}\cdot \bigg(- \frac{3p}{2}\bigg)= 3 + q. \ \ \ \ (2)$$
Porovnaním oboch vzťahov (1) a (2) máme $3 - q = 3 + q$, čiže $q = 0$ a z~(1) potom vyjde $p = 2$ alebo $p = -2$.

Z~oboch nájdených riešení dostaneme tú istú dvojicu rovníc $x(x - 2) = 3$, $x(x + 2) = 3$. Korene prvej z~nich sú čísla $-1$ a $3$, ich aritmetický priemer je $1$. Korene druhej rovnice sú čísla $1$ a $-3$, ich aritmetický priemer je $-1$.
}
