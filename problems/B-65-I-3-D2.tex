% Do not delete this line (pandoc magic!)

\problem{B-65-I-3-D2}{}{
Polomer vpísanej kružnice trojuholníka $ABC$ je $r$. Zostrojme tri rôzne dotyčnice vpísanej kružnice rovnobežné so stranami trojuholníka. Polomery vpísaných kružníc troch malých \uv{odrezaných} trojuholníkov označme $r_A$, $r_B$, $r_C$ podľa vrcholov trojuholníka. Dokážte, že $r_A + r_B + r_C = r$.
}{
\rieh Z podobnosti malého trojuholníka k $ABC$ je $r_A/r = (v_a - 2r)/v_a$, pričom $v_a$ označuje veľkosť výšky z vrcholu $A$ v trojuholníku $ABC$. Podobné rovnice platia aj pre ostatné vrcholy, takže ostáva ukázať, že $1/v_a + 1/v_b + 1/v_c = 1/r$. Tu využijeme vzorec $ro = 2S = av_a = bv_b = cv_c$, pričom $o=a+b+c$.
}
