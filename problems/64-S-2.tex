% Do not delete this line (pandoc magic!)

\problem{64-S-2}{
Označme $K$ a $L$ postupne body strán $BC$ a $AC$ trojuholníka $ABC$, pre ktoré platí $|BK|= \frac{1}{3}|BC|$, $|AL| =\frac{1}{3}|AC|$. Nech $M$ je priesečník úsečiek $AK$ a $BL$. Vypočítajte pomer obsahov trojuholníkov $ABM$ a $ABC$.
}{
\rieh Označme $v$ výšku trojuholníka $ABC$ na stranu $AB$, $v_1$ výšku trojuholníka $ABM$ na stranu $AB$ a $v_2$ výšku trojuholníka $KLM$ na stranu $KL$ (obr.~\ref{fig:64S2}). Z~podobnosti trojuholníkov $LKC$ a $ABC$ (zaručenej vetou $sus$) vyplýva, že $|KL| =\frac{2}{3} |AB|$. Z~porovnania ich výšok zo spoločného vrcholu $C$ vidíme, že výška $v$ trojuholníka $ABC$ je rovná trojnásobku vzdialenosti priečky $KL$ od strany $AB$, teda $v = 3(v_1 +v_2)$. Keďže $AK$ a $BL$ sú priečky rovnobežiek $KL$ a $AB$, vyplýva zo zhodnosti prislúchajúcich striedavých uhlov podobnosť trojuholníkov $ABM$ a $KLM$.
\begin{figure}[h]
    \centering
    \includegraphics{images/64S2\imagesuffix}
    \caption{}
    \label{fig:64S2}
\end{figure}
Keďže $|KL| =\frac{2}{3}|AB|$, je tiež $v_2 =\frac{2}{3}v_1$, a preto $v_1 + v_2 =\frac{5}{3}v_1$, čiže
$$v = 3(v_1 + v_2) = 5v_1.$$
Trojuholníky $ABM$ a $ABC$ majú spoločnú stranu $AB$, preto ich obsahy sú v~pomere výšok na túto stranu, takže obsah trojuholníka $ABC$ je päťkrát väčší ako obsah trojuholníka $ABM$.\\
\\
\kom Ďalšia úloha, ktorá precvičuje rovnaké tvrdenie ako predchádzajúca. Pomery výšok je tentoraz potrebné určiť z~podobnosti trojuholníkov. Tu sa teda uplatnia znalosti precvičované na minulom seminárnom stretnutí. \\
\\
}
