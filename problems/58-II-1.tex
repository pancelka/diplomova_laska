% Do not delete this line (pandoc magic!)

\problem{58-II-1}{vyrazy,odhady,nerovnosti,krajskekolo}{
Uvažujme výraz $$V(x) =\frac{5x^4 - 4x^2 + 5}{x^4 + 1}.$$

a) Dokážte, že pre každé reálne číslo $x$platí $V (x) \geq 3$.

b) Nájdite najväčšiu hodnotu $V (x)$.

}{
\rie  Výraz $V$ je zrejme definovaný pre všetky reálne čísla $x$.

a) Keďže $x^4 +1 > 0$ pre každé $x$, nerovnosť $V (x) \geq 3$ je ekvivalentná s nerovnosťou $5x^4 - 4x^2 + 5 \geq 3(x^4 + 1)$, čiže $2x^4 - 4x^2 + 2 \geq 0$. Výraz na ľavej strane je rovný $2(x^2 - 1)^2$, takže je nezáporný pre každé $x$.

b) Využime nasledujúcu úpravu:
$$V (x) =\frac{5x^4 - 4x^2 + 5}{x^4 + 1}=\frac{5(x^4 +1)}{x^4 + 1}-\frac{4x^2}{x^4 + 1}=5-\frac{4x^2}{x^4 + 1}.$$
Keďže zlomok
$$\frac{4x^2}{x^4 + 1}$$
je vďaka párnym mocninám premennej $x$ pre ľubovoľné reálne číslo $x$ nezáporný, nadobúda výraz $V$ svoju najväčšiu hodnotu $V_{max}$ práve vtedy, keď
$$\frac{4x^2}{x^4 + 1}=0$$
teda práve vtedy, keď $x = 0$. Dostávame tak $V_{max} = V (0) = 5$.
}