\problem{62-I-2-N1}{
Dokážte, že pre ľubovoľné kladné čísla $a, b, c$ platí nerovnosť
$$\bigg(a +\frac{1}{b}\bigg)\bigg(b+\frac{1}{c}\bigg)\bigg(c+\frac{1}{a}\bigg)\geq 8$$
a zistite, kedy prechádza v~rovnosť.
}{
\rieh Ľavú stranu $L$ dokazovanej nerovnosti najskôr upravíme roznásobením a vzniknuté členy zoskupíme do súčtov dvojíc navzájom prevrátených výrazov:
\begin{multline*} L = \bigg(a +\frac{1}{b}\bigg)\bigg(b+\frac{1}{c}\bigg)\bigg(c+\frac{1}{a}\bigg) = \bigg(ab+ \frac{a}{c} + 1 +\frac{1}{bc}\bigg) \bigg(c +\frac{1}{a}\bigg)=\\ =\bigg( abc + \frac{1}{abc}\bigg)+\bigg( a+\frac{1}{a}\bigg)+ \bigg(b+\frac{1}{b}\bigg)+\bigg(c+\frac{1}{c}\bigg).\end{multline*}
Pretože pre $u > 0$ je $u+\frac{1}{u}\geq 2$, pričom rovnosť nastane práve vtedy, keď $u = 1$, pre výraz~$L$ platí $L \geq 2 + 2 + 2 + 2 = 8$, čo sme mali dokázať. Rovnosť $L = 8$ nastane práve vtedy, keď platí
$$ abc+\frac{1}{abc}=a+\frac{1}{a}=b+\frac{1}{b}=c+\frac{1}{c}=2$$
teda, ako sme už spomenuli, práve vtedy, keď $abc = a = b = c = 1$,  t.\,j. práve vtedy, keď $a = b = c = 1$.\\
\\
\kom Úloha sa dá riešiť využitím AG nerovnosti, tá však bude obsahom jedného z~ďalších seminárov, v~ktorom sa (okrem iného) k~tejto úlohe vrátime.\\
\\
}
