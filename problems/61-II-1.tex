% Do not delete this line (pandoc magic!)

\problem{61-II-1}{}{
Pre všetky reálne čísla $x, y, z$ také, že $x < y < z$, dokážte nerovnosť $$x^2 - y^2 + z^2> (x - y + z)^2.$$
}{
\rieh Aby sme mohli použiť vzorec $A^2 - B^2 = (A - B)(A + B)$, presuňme najskôr jeden z krajných členov ľavej strany, napríklad člen $z^2$, na pravú stranu:
\begin{align*}
x^2 - y^2 & > (x - y + z)^2-z^2,\\
(x - y)(x + y) & > (x - y + z - z)(x - y + z + z),\\
(x - y)(x + y) & > (x - y)(x - y + 2z).
\end{align*}
Keďže spoločný činiteľ $x - y$ oboch strán poslednej nerovnosti je podľa predpokladu úlohy číslo záporné, budeme s dôkazom hotoví, keď ukážeme, že zvyšné činitele spĺňajú opačnú nerovnosť $x + y < x - y + 2z$. Tá je však zrejme ekvivalentná s nerovnosťou $2y < 2z$, čiže $y < z$, ktorá podľa zadania úlohy naozaj platí.

\textbf{Iné riešenie*.} Podľa vzorca pre druhú mocninu trojčlena platí $$(x - y + z)^2= x^2+ y^2+ z^2 - 2xy + 2xz - 2yz.$$
Dosaďme to do pravej strany dokazovanej nerovnosti a urobme niekoľko ďalších ekvivalentných úprav:
\begin{align*}
x^2 - y^2+ z^2 &> x^2+ y^2+ z^2 - 2xy + 2xz - 2yz,\\
0 &> 2y^2 - 2xy + 2xz - 2yz,\\
0 &>  2y(y - x) + 2z(x - y),\\
0 &> 2(y - x)(y - z).
\end{align*}
Posledná nerovnosť už vyplýva z predpokladov úlohy, podľa ktorých je činiteľ $y - x$ kladný, zatiaľ čo činiteľ $y - z$ je záporný.
\\
\\
\kom Úloha sa dá vyriešiť jednoduchým použitím ekvivalentných úprav a diskusiou v závere, v ktorej je potrebné nezabudnúť na predpoklady z úvodu zadania. Ak študenti sami neprídu na dôkaz pomocou použitia vzorca $A^2-B^2$, je vhodné im ho ukázať, keďže tak budeme demonštrovať viacero odlišných prístupov k riešeniu úlohy. Zároveň úloha nevyžaduje špeciálne vedomosti a je tak príjemným prepojením tohto a minulého seminára o nerovnostiach. \\
\\
}
