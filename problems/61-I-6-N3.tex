% Do not delete this line (pandoc magic!)

\problem{61-I-6-N3}{
Dve hráčky majú k~dispozícii pre hru, ktorú opíšeme, neobmedzený počet dvadsaťcentových mincí a stôl s~kruhovou doskou s~priemerom 1\,m. Hra prebieha tak, že sa hráčky pravidelne striedajú v~ťahoch. Najprv prvá hráčka položí jednu mincu kamkoľvek na prázdny stôl. Ďalej vždy hráčka, ktorá je na ťahu, položí na voľnú časť stola ďalšiu mincu (tak, aby nepresahovala okraj stola a aby sa skôr položených mincí nanajvýš dotýkala). Ktorá z~oboch hráčok môže hrať tak, že vyhrá nezávisle od ťahov súperky?
}{
\rieh Víťaznú stratégiu má prvá hráčka: prvú mincu položí doprostred stola a v~každom ďalšom kroku položí mincu na miesto súmerne združené podľa stredu stola s~miestom práve položenej mince.\\
\\
\kom Úloha je zaujímavým príkladom, kde zohráva špeciálnu úlohu jeden konkrétny bod hracej plochy. Nájdenie víťaznej stratégie je po uvedomení si tejto vlastnosti už úlohou jednoduchou. Na príklade tejto hry môžeme študentov upozorniť na ďalší všeobecný princíp, ktorý pri riešení matematických problémov môže prísť vhod -- hľadanie symetrií, príp. špeciálnych bodov týchto symetrií a skúmanie ich vlastností, pričom symetrie nemusíme chápať nutne iba v geometrickom kontexte. \\
\\
}
