% Do not delete this line (pandoc magic!)

\problem{63-I-3-N4}{}{
Číslo $n$ je súčinom dvoch prvočísel. Ak zväčšíme každé z~nich o~1, ich súčin sa zväčší o~35. Určte číslo $n$.
}{
\rie Podobne ako v~predchádzajúcom prípade označme $p\leq q$ (nie nutne rôzne) prvočísla zo zadania a to prepíšme do tvaru rovnosti $(p+1)(q+1)=pq+35$. Po úprave dostávame $p+q=34$. Hľadáme teda dvojice prvočísel, ktorých súčet bude 34. Takými sú jedine 3 a 31, 5~a 29, 11 a 23, 17 a 17. Riešením úlohy je potom $n \in \{93, 145, 253, 289\}$.\\
\\
\kom Úvodné dve jednoduché úlohy majú prípravný charakter na úlohu nasledujúcu a sú skôr rozcvičkou, než náročnou aplikáciou vedomostí o~prvočíslach.\\
\\
}
