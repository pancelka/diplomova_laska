\seminar{34}

\subsection*{Téma}
Opakovanie II -- samostatné riešenie úloh

\subsection*{Ciele}
Samostatne overiť schopnosti študentov riešiť úlohy MO kategórie B (školské kolo).

\subsection*{Úvodný komentár.} Posledné seminárne stretnutie študenti strávia riešením troch úloh školského kola MO kategórie B. Môžeme využiť školské kolo použité v roku, keď seminár prebieha, príp. využiť niektorý zo starších ročníkov. My sme zvolili ročník 51, ktorý sa nám skladbou úloh zdal veľmi vhodný. Vzhľadom na časové možnosti študentov nebudeme vyžadovať popísaný postup riešenia, necháme študentov len samostatne pracovať bez akéhokoľvek napovedania. Na riešenie študentom ponecháme 90 minút. 

Vo zvyšnom čase so študentami prejdeme kľúčové myšlienky jednotlivých úloh, príp. ich vlastné nápady. Posledných niekoľko minút seminára je tiež vhodné venovať spätnej väzbe od študentov. Nezaškodí sa vrátiť k diskusii z prvého seminárneho stretnutia a zistiť, či seminár splnil očakávania, čo študenti na stretnutiach oceňovali a čo by, naopak, ešte privítali.


\subsection*{Úlohy a riešenia}

\problem{B-51-S-1}{
Určte reálne číslo $p$ tak, aby rovnica
$$x^2 + 4px + 5p^2 + 6p - 16 = 0$$
mala dva rôzne korene $x_1$, $x_2$ a aby súčet $x_1^2+ x_2^2$ bol čo najmenší.
}{
\rieh Pre korene $x_1$, $x_2$ danej kvadratickej rovnice (pokiaľ existujú) platí podľa Viètových vzťahov rovnosti
$$x_1 + x_2 =-4p \ \ \ \ \text{a} \ \ \ \  x_1x_2 = 5p^2 + 6p - 16,$$
z ktorých vypočítame skúmaný súčet
\begin{align*}
x_1^2 + x_2^2 &  = (x_1 + x_2)^2 - 2x_1x_2 = (-4p)^2 - 2(5p^2 + 6p - 16) =\\
& = 6p^2 - 12p + 32 = 6(p- 1)^2 + 26.
\end{align*}
Odtiaľ vyplýva nerovnosť $x_1^2 + x_2^2 = 26$, pritom rovnosť môže nastať, len keď $p = 1$. Zistíme preto, či pre $p = 1$ má daná rovnica skutočne dve rôzne riešenia. Ide o rovnicu $x^2 + 4x - 5 = 0$ s koreňmi $x_1 = -5$ a $x_2 = 1$. Tým je úloha vyriešená.

Dodajme, že väčšina riešiteľov pravdepodobne najprv zistí, pre ktoré $p$ má daná rovnica dva rôzne korene. Pretože pre jej diskriminant $D$ platí
$$D = (4p)^2 - 4(5p^2 + 6p - 16) = -4p^2 - 24p + 64 = -4(p + 8)(p - 2),$$
sú také p práve čísla z intervalu $(-8, 2)$.\\
\textit{Odpoveď.} Minimálna hodnota súčtu $x_1^2+x_2^2$ (rovná 26) zodpovedá jedinému číslu $p = 1$.
}

\problem{B-51-S-2}{
Vnútri strán $BC$, $CA$, $AB$ daného ostrouhlého trojuholníka $ABC$ sú po rade vybrané body $X$, $Y$ a $Z$ tak, že každému zo štvoruholníkov $ABXY$, $BCYZ$ a $CAZX$ sa dá opísať kružnica. Dokážte, že body $X$, $Y$, $Z$ sú päty výšok trojuholníka $ABC$.
}{
\rieh V tetivovom štvoruholníku $ABXY$ označme $\varphi = |\ma AXB| = |\ma AYB|$ veľkosť oboch zhodných obvodových uhlov nad spoločnou tetivou $AB$ (obr.~\ref{fig:B51S2}). Podobne
\begin{figure}[h]
    \centering
    \includegraphics{images/B51S2\imagesuffix}
    \caption{}
    \label{fig:B51S2}
\end{figure}
označme $\psi = |\ma BZC| = |\ma BYC|$ a $\omega = |\ma CXA| = |\ma CZA|$ veľkosti zhodných obvodových uhlov nad tetivami $BC$ a $CA$ v tetivových štvoruholníkoch $BCYZ$ a $CAZX$. Keď zapíšeme postupne rovnosti pre každú z troch dvojíc vyznačených susedných uhlov pri vrcholoch $X$, $Y$ a $Z$, dostaneme pre neznáme veľkosti $\varphi$, $\psi$ a $\omega$ sústavu troch lineárnych rovníc
\begin{align*}
\varphi + \psi & = \pi, \\
\psi + \omega & = \pi, \\
\omega + \varphi & = \pi,
\end{align*}
ktorá má jediné riešenie $\varphi = \psi = \omega = \pi/2$, čo jednoducho zistíme napr. odčítaním ľubovoľných dvoch rovníc a dosadením. Tým je tvrdenie úlohy dokázané. 

\textit{Poznámka.} Ak sú naopak body $X$, $Y$ a $Z$ päty výšok trojuholníka $ABC$, sú štvoruholníky $ABXY$, $BCYZ$ a $CAZX$ tetivové podľa Tálesovej vety.
}

\problem{B-51-S-3}{
Na tabuli sú napísané čísla $1, 2, \ldots, 17$. Čísla postupne zotierame, a to tak, že z doposiaľ nezotretých čísel zvolíme ľubovoľné číslo $k$ a zotrieme všetky tie čísla na tabuli, ktoré delia číslo $k+17$. Dokážte, že opakovaním tejto procedúry sa nám nepodarí zotrieť všetky čísla.
}{
\rieh Pretože pre zvolené číslo $k$ vždy platí $18 \leq k + 17 \leq 34$ a medzi číslami $18, 19, \ldots , 34$ má každé z čísel $12, 13, \ldots , 17$ iba jeden násobok (konkrétne dvojnásobok), ľubovoľné číslo $m \in \{12, 13, \ldots , 17\}$ zotrieme iba pri voľbe jediného čísla $k$ (pri ktorom $k + 17 = 2m$). Napríklad číslo 15 zotrieme iba voľbou $k = 13$, číslo 13 iba voľbou $k = 9$. Na zotretie oboch čísel 15 a 13 teda musíme niekedy vybrať k$ = 13$ a niekedy neskôr $k = 9$. Potom ale v okamihu výberu čísla $k = 9$ je už zotreté ako číslo 10 (zotreli sme ho najneskôr pri výbere $k = 13$), tak číslo 1 (to sme zotreli hneď pri prvom výbere). Číslo $k + 17$ je deliteľné deviatimi iba pri výberoch $k = 1$ a $k = 10$, pri žiadnom ďalšom výbere už preto nezotrieme číslo 9. Dokázali sme, že opakovaním danej procedúry nemožno zotrieť všetky tri čísla 15, 13 a 9, tým skôr nemožno zotrieť všetky čísla od 1 do 17.

\textbf{Iné riešenie*.} Pripusťme, že všetky čísla možno zotrieť po $n$ výberoch čísla $k$ (spojených so zotieraním všetkých deliteľov čísla $k+ 17$) a že každým výberom sa niečo zotrie (inak je taký výber zbytočný). Posledné o. i. znamená, že každé číslo je vybrané najviac raz. Zrejme $n > 1$ a pre posledné vybrané číslo $k_n$ musí platiť $k_n\mid (k_n + 17)$, t. j. $k_n = 17$ (možnosť $k_n = 1$ je vylúčená tým, že číslo 1 je zotreté hneď pri prvom výbere). Pred posledným výberom sú na tabuli len delitele čísla 34, teda okrem čísla 17 prípadne číslo 2. Keby tam číslo 2 nebolo, muselo by opäť platiť $k_{n-1} \mid (k_{n-1} + 17)$, čo už možné nie je. Preto nutne $k{n-1} = 2$. Taká voľba je ale zbytočná, pretože číslo $2+17$ je prvočíslo.
}
