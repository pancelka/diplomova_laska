% Do not delete this line (pandoc magic!)

 \problem{59-II-3}{
Daná je kružnica $k$ so stredom $S$. Kružnica $l$ má väčší polomer ako kružnica $k$, prechádza jej stredom a pretína ju v~bodoch $M$ a $N$. Priamka, ktorá prechádza bodom $N$ a je rovnobežná s~priamkou $MS$, vytína na kružniciach tetivy $NP$ a $NQ$. Dokážte, že trojuholník $MPQ$ je rovnoramenný.
}{
\rieh Polomer kružnice $k$ označme $r$. Označenie vrcholov $P$, $Q$ v~trojuholníku $MPQ$ nie je dôležité, preto bez ujmy na všeobecnosti označme $P$ ten z~bodov priamky vedenej bodom $N$ rovnobežne s~priamkou $MS$, ktorý leží na kružnici $k$. Bod $Q$ potom leží na kružnici $l$ a štvoruholník $NQMS$ je lichobežník vpísaný do kružnice $l$ (\ref{fig:59II3_1}). Je teda rovnoramenný s~ramenami $MQ$ a $NS$ dĺžky $r$. Navyše aj úsečky $SP$ a $SM$ majú dĺžku $r$. Z~rovnoramenného trojuholníka $NPS$ a rovnoramenného lichobežníka $NQMS$ vyplýva rovnosť uhlov $|\ma SPN| = |\ma SNP| = |\ma MQP|$. Priečka $PQ$ teda pretína priamky $SP$ a $MQ$ pod rovnako veľkými uhlami, a preto (podľa vety o~súhlasných uhloch) sú priamky $SP$ a $MQ$ rovnobežné. Štvoruholník $PQMS$ je teda rovnobežník, a keďže $|SM| = |SP| = r$, je to dokonca kosoštvorec. Odtiaľ je už zrejmé, že trojuholník $MPQ$ je rovnoramenný s~ramenami $PQ$ a $MQ$ dĺžky $r$.
\begin{figure}[h]
    \centering
    \includegraphics{images/59K31\imagesuffix}
    \caption{}
    \label{fig:59II3_1}
\end{figure}
\\
\textit{Poznámka.} Existencia tetív $NP$ a $NQ$ v~zadaní je zaručená vďaka predpokladu, že kružnica $l$ má väčší polomer ako kružnica $k$. Ak označíme $C$ stred úsečky $SM$ a $E$ ten priesečník kružnice $k$ s~osou úsečky $SM$, ktorý leží v~polrovine $SMO$, bude stred O~kružnice $l$ ležať na polpriamke $CE$ až za bodom $E$ (\ref{fig:59II3_2}). Ďalší priesečník $N$ oboch
\begin{figure}[h]
    \centering
    \includegraphics{images/59K32\imagesuffix}
    \caption{}
    \label{fig:59II3_2}
\end{figure}
kružníc preto padne do pásu medzi rovnobežkami $SM$ a $N_0 E$ v~polrovine $OCS$, pričom $N_0$ je štvrtý vrchol kosoštvorca s~vrcholmi $S$, $M$, $E$. Na to stačí ukázať, že kružnica $l$ pretne polpriamku $EN_0$ až za bodom $N_0$, teda že jej polomer $OS$ je väčší ako dĺžka úsečky $ON_0$. Toto porovnanie dvoch strán trojuholníka $OSN_0$ jednoducho vyplýva z~porovnania jeho vnútorných uhlov: uhol pri vrchole $N_0$ je najväčší, lebo oba uhly pri protiľahlej strane $OS$ sú menšie ako $60^\circ$ (trojuholník $ESN_0$ je rovnostranný). Ľahko nahliadneme, že každá z~rovnobežiek uvedeného pásu pretína každú z~oboch kružníc v~dvoch bodoch (vždy súmerne združených podľa príslušnej osi kolmej na $SM$). Tým je dokázaná nielen existencia oboch tetív $NP$ a $NQ$, ale aj to, že ich krajné body $P$ a $Q$ ležia na rovnakej strane od bodu $N$ (ako na \ref{fig:59II3_1}), lebo oba body zrejme ležia v~polrovine opačnej k~spomenutej polrovine $OCS$.\\
\\
\kom Diskusia v~poznámke je len zaujímavým doplnkom úlohy, existencia tetív je totiž predpokladom zadania a nie je nutné ju dokazovať. Úloha využíva úvahu, že lichobežník, ktorého základne sú rovnobežné tetivy danej kružnice, je rovnoramenný, ktorá môže byť pre študentov zaujímavým uvedomením.\\
}