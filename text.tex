\newcounter{seminar}
\newcounter{problem}
\newcommand{\seminar}[1]{
  \setcounter{seminar}{#1}
  \setcounter{problem}{0}
  \subsection*{Seminár #1}
}

% LaTeX header setting the web/thesis commands right
% DOCUMENT COMPOSITION DIRECTIVES (PDF)

% Use PDF images
\newcommand{\imagesuffix}{}

% Do not display download links
\newcommand{\weblinks}[1]{}

% Display problems + solutions
\newcommand{\problem}[3]{
  \stepcounter{problem}
  \begin{tcolorbox}[breakable,notitle,boxrule=0pt,colback=light-gray,colframe=light-gray]
    \textbf{Úloha \theseminar .\theproblem.}
    [#1] #2
  \end{tcolorbox}
  \textbf{Riešenie \theseminar .\theproblem.} #3
}


\HlavickaUvod
\setcounter{page}{1}
\pagenumbering{arabic}

\chapter*{Úvod}
\label{chap:intro}
\addcontentsline{toc}{chapter}{\nameref{chap:intro}}


Diplomová práca sa venuje tvorbe prezenčného matematického seminára pre študentov prvého ročníka so záujmom o matematiku. Práca má dve časti. V prvej z nich zasadíme seminár do širšieho kontextu matematického vzdelávania stredoškolákov. Podávame stručný prehľad mimoškolských aktivít, do ktorých sa študenti so záujmom o matematiku môžu zapojiť. Sem bezpochyby patrí najstaršia predmetová olympiáda -- matematická olympiáda, rozličné korešpondenčné semináre, tábory, sústredenia a letné školy s matematickou tematikou, ako aj jednodňové podujatia, či už sú to súťaže alebo (ma)tematicky zamerané semináre. Pri tvorbe obsahov seminárnych stretnutí sme potom vychádzali zo školského vzdelávacieho programu pre prvý ročník gymnázií (ktorý vychádza z rámcového vzdelávacieho programu, je však hmatateľnejším vodidlom k tomu, aké znalosti a zručnosti by študenti mali mať) a analýzy úloh matematickej olympiády kategórie C.

Na základe poznatkov z prvej kapitoly sme potom vytvorili osnovy seminárov pre študentov so záujmom o ďalšie matematické vzdelávanie. Semináre sú naplánované s týždennou frekvenciou, pričom jedno seminárne stretnutie má trvať približne 120\,minút. V hlavnom texte ponúkame návrh úloh, ktorými sa študenti pod vedením vedúceho seminára majú zaoberať, sprievodné komentáre aj domáce práce. Ďalej sú obsahom stretnutí tímové matematické súťaže a zaradené sú aj semináre zamerané na analýzu úloh aktuálneho ročníka matematickej olympiády.

Práca bola napísaná s využitím systému \LaTeX, obrázky sme prevzali zo verejne dostupných vzorových riešení úloh MO.

%\textbf{Matematický seminář pro talentované studenty}\\
%Zadání: Studentka sestaví podrobnou osnovu včetně řešených příkladů a cvičení pro středoškolský seminář připravující studenty na matematické soutěže (zejména matematickou olympiádu). Obsah semináře zohlední RVP pro gymnázia i obvyklý průběh studia, stejně jako typické úlohy řešené v~matematických soutěžích. Dle zájmu studentky je možné práci pojmout buď jako přípravu podkladů pro prezenční seminář nebo jako přípravu virtuálního webového semináře.

\cleardoublepage

\renewcommand{\chaptermark}[1]{\markboth{\thechapter. #1}{}}
\renewcommand{\sectionmark}[1]{\markright{\thesection. #1}{}}
\HlavickaKapitoly
\chapter{Širší kontext práce}
\section{Príležitosti pre študentov so záujmom o~matematiku}

Stredoškoláci, ktorých matematika zaujíma a láka ich odhaľovanie matematických tajomstiev aj za hranicami 3-4 hodín bežnej školskej matematiky, majú počas svojich stredoškolských štúdií veľké množstvo príležitostí. Cibriť si svoj matematický um môžu v~rôznych dlhodobých či jednorázových súťažiach, seminároch či letných školách. V~tejto krátkej kapitole uvedieme stručný prehľad možností, ktoré sa študentom slovenských a českých stredných škôl ponúkajú. Predstavíme matematickú olympiádu, korešpondenčné semináre a ďalšie súťaže. Nekladieme si však za cieľ podať úplný a vyčerpávajúci prehľad všetkých školských aj mimoškolských matematických aktivít spolu s~ich historickým vývojom, ale skôr predstaviť čitateľovi mozaiku matematických príležitostí pre stredoškolákov. Pre záujemcov o~hlbší pohľad do problematiky európskych matematických súťaží môže byť zaujímavým čítaním~\cite{huv},  o~vývoji práce s~matematickými talentmi pútavo pojednáva Svrček~\cite{svrcek2014} a o~histórii matematickej olympiády na Slovensku a v~Českej republike sa nemálo dozvieme z~\cite{dos}.


\subsection*{Matematická olympiáda}

Matematická olympiáda je najstaršou predmetovou olympiádou a je zároveň \uv{kráľovnou} matematických súťaží u~nás. Prvýkrát sa matematická olympiáda konala v~školskom roku 1951/52 a teda v~školskom roku 2017/18 sa mali študenti možnosť zúčastniť už 67.\,ročníka.

Úlohy matematickej olympiády sú autorské a náročnejšie ako typické školské úlohy. Často vyžadujú okrem dobrého zvládnutia školských poznatkov aj istý matematický cvik, vo vyšších kategóriách je na vyriešenie niektorých úloh nezriedka potrebné naštudovanie pasáží, ktoré nie sú bežnou súčasťou osnov. Okrem samotného výsledku úlohy je veľmi dôležitý postup, ktorým sa študent k~riešeniu dopracoval a jeho zdôvodnenie. Účastníci olympiády sa tak trénujú nielen v~riešení matematických úloh, ale aj v~prehľadom a zrozumiteľnom zápise ich riešenia.

V~súčasnosti študenti súťažia v~ôsmich rôznych kategóriách, ktoré sú určené pre žiakov druhého stupňa ZŠ a študentov SŠ, kategórie Z5, Z6, Z7, Z8, Z9, sú po rade určené žiakom 5., 6., 7., 8. a 9. ročníka, kategórie C, B, A sú pripravené pre študentov 1., 2. a 3.-4. ročníkov gymnázií a stredných odborných škôl a zodpovedajúcich ročníkov viacročných gymnázií. Študenti sa okrem svojej príslušnej kategórie môžu tiež zúčastniť olympiády v~kategórii vyššej. Všetky kategórie začínajú domácim kolom, základoškolákov potom preverí kolo okresné, kde súťaž pre kategórie Z5-Z8 končí. Kategória Z9 je zavŕšená kolom krajským. Stredoškolské kategórie pokračujú po domácom kole ešte školským a krajským, v~prípade kategórie A~sa najlepší riešitelia z~celej republiky stretnú na kole celoštátnom. Šestica najúspešnejších riešiteľov celoštátneho kola postupuje na Medzinárodnú matematickú olympiádu, kde si meria sily so stredoškolákmi z~celého sveta.

\subsection*{Korešpondenčné semináre}

Korešpondenčné semináre sú ďalším spôsobom, akým si študenti môžu rozširovať a trénovať svoje (nielen) matematické znalosti a skúsenosti. Obsahovo sú matematické semináre podobné matematickej olympiáde, pretože študenti sa taktiež venujú úlohám vyššej ako školskej náročnosti, ktorých riešenia spolu s~postupom a zdôvodnením prehľadne spisujú. Korešpondenčné semináre nie sú len doménou matematikov, existujú totiž aj semináre z~programovania, fyziky, biológie či ekológie.

Aj keď sa jednotlivé matematické semináre od seba v~detailoch odlišujú, ich priebeh veľmi podobný. Organizátori zverejnia zadania úloh a študenti majú niekoľko týždňov na ich vyriešenie. Úlohy sa potom posielajú organizátorom, ktorí ich opravia, ohodnotia bodmi, pridajú komentáre k~jednotlivým riešeniam a pošlú ich naspäť študentom. Tento cyklus sa opakuje niekoľkokrát ročne, pričom sérií býva 4-8, v~každej z~nich na študentov čaká 4-9 úloh. Neoddeliteľnou súčasťou korešpondenčných seminárov sú sústredenia pre niekoľko desiatok najlepších riešiteľov, ktoré študentov zvyčajne okrem matematických vedomostí obohatia aj o~nové priateľstvá a ďalšie zážitky.

V~Českej republike a na Slovensku existuje matematických korešpondenčných seminárov niekoľko, nebýva výnimkou, že slovenskí študenti riešia aj semináre české a naopak. Uvedieme semináre, ktoré môžu byť zaujímavé pre stredoškolských študentov.

\subsubsection{České korešpondenčné semináre}
\begin{itemize}
\item \textit{PraSe} (\textit{MKS}) -- \textit{Pražský korešpondenční seminář} organizovaný Matematicko-fyzikálnou fakultou  Karlovej univerzity v~Prahe, osem sérií úloh rozdelených na jesennú a jarnú časť, väčšina sérií obsahuje 8 úloh, okrem toho ešte tematická séria zameraná na istú oblasť matematiky, dve sústredenia ročne. Podrobnejšie informácie je možné nájsť na [\url{https://mks.mff.cuni.cz/info/pravidla.php}]
\item \textit{BrKoS} -- \textit{Brněnský korešpondenční seminář} organizovaný pod záštitou Prírodovedeckej fakulty Masarykovej univerzity v~Brne, šesť sérií po siedmich úlohách, jedno jesenné sústredenie. Viac informácií je zverejenených na \cite{brkos}
\item \textit{M\&M} -- organizovaný opäť Matematicko-fyzikálnou fakultou Karlovej univerzity v~Prahe, typovo sa líši od predchádzajúcich dvoch: študenti majú okrem riešenia zadaných úloh možnosť bádať nad zadanými témami, písať články a reagovať na články rovesníkov. Pre najlepších riešiteľov je taktiež pripravené sústredenie. Viac informácií je dostupných na~\cite{mam}.
\end{itemize}

\subsubsection{Slovenské korešpondenčné semináre}
\begin{itemize}
\item \textit{KMS} -- \textit{Korešpondenčný matematický seminár} organizovaný neziskovou organizáciou Trojsten, zimná a letná časť, každá pozostávajúca z~troch sérií desiatich úloh, dve vekové kategórie, pre ktoré sú na konci každej časti organizované samostatné sústredenia. Podrobnejší prehľad na \cite{kms}.
\item \textit{STROM} -- \textit{Seminár talentovaných riešiteľov obľubujúcich matematiku} organizovaný združením \textit{STROM}, takisto dve časti po tri série úloh a dve sústrednia. Ďalšie informácie sa nachádzajú na \cite{strom}.
\end{itemize}

\subsubsection*{Česko-slovenský seminár}
\begin{itemize}
\item \textit{iKS} -- \textit{Medzinárodný korešpondenčný seminár} určený pre pokročilých riešiteľov, obsahuje ťažké úlohy, a tomu odpovedá aj počet riešiteľov, ktorý je rádovo menší ako v~ostatných zmienených seminároch.
\end{itemize}

Okrem archívu súťažných úloh a ich riešení väčšina seminárov zverejňuje ďalšie zaujímavé články z~rôznych oblastí matematiky, materiály zo sústredení a doplňujúce úlohy. Webové stránky týchto seminárov sa tak stávajú takmer nevyčerpateľnou zásobárňou inšpirácie a priestoru na precvičovanie riešiteľských zručností.


\subsection*{Krátkodobé súťaže}

Okrem vyššie spomínaných dlhodobejších súťaží sa študenti môžu zapojiť do ďalších matematických akcií. Jednoznačne najmasovejšou\footnote{V roku 2018 sa súťaže zúčastnilo viac ako $22000$ stredoškolákov.} z~nich je \textit{Matematický klokan}, ktorý je pripravovaný pre študentov základných a stredných škôl. Na rozdiel od predchádzajúcich súťaží tu študenti riešia menej náročné, no stále originálne úlohy a dôležitý je len správny výsledok, ktorý študenti vyberajú z~piatich ponúknutých možností.

Jednou z~mála súťaží, ktorá nie je určená pre jednotlivcov, je \textit{Náboj}. V~\textit{Náboji} riešia 5-členné družstvá po dobu dvoch hodín úlohy, ktoré \uv{si vyžadujú istú dávku invencie a dôvtipu}. Na začiatku súťaže dostanú družstvá šesticu úloh, po vyriešení ktorejkoľvek z~nich výmenou za ňu získa novú úlohu. V~roku 2018 sa súťaž, pripravovaná pod záštitou viacerých univerzít a korešpondenčných seminárov, uskutočnila dokonca v~15 mestách celej Európy.

\subsection*{Letné školy a tábory}

Študentom, ktorým by sa letné prázdniny mohli zdať málo matematicky intenzívne, vychádzajú v~ústrety rôzne letné školy a tábory, na ktoré sa môžu prihlásiť aj bez toho, aby predtým riešili korešpondenčné či iné súťaže. Príkladmi sú \textit{Letná škola matematiky a fyziky}, \textit{Letný tábor Trojstenu}, \textit{Letná škola Trojstenu}, sústredenie MOFO pre riešiteľov matematickej a fyzikálnej olympiády či \textit{Letné študentské sústredenie TNC}. Na rozdiel od predchádzajúcich aktivít nie je počas letných táborov a škôl kladený dôraz len na získavanie nových poznatkov z~matematiky (a prípadne fyziky), ale aj na netradičné hry, športové a nešportové aktivity a upevňovanie priateľstiev medzi účastníkmi. Vlastné skúsenosti autorky sa môžu len pridať k~hlasom, ktoré tvrdia, že sústredenia a tábory sú zážitkami na celý život.


%\textit{Kromě populárních korespondenčních seminářů jsou v České republice pořádány další neméně zajímavé matematické soutěže. V mnoha případech opět ve spolupráci se Slovenskem. Za zmínku jistě stojí matematická soutěž Náboj, viz. např. [64] nebo Pražská střela, viz např. [76]. Uvedené matematické soutěže jsou známé především pro svou dynamičnost a živou atmosféru. Náboj je mezinárodní matematickou soutěží, které se účastní pětičlenné týmy středoškoláků z různých škol České a Slovenské republiky. Soutěž probíhá současně v Praze, Bratislavě, Opavě a Košicích, a to ve dvou soutěžních kategoriích, přičemž v kategorii Juniorů mohou soutěžit pouze týmy, jejichž členové jsou žáky nejvýše 2. ročníku střední školy. Jinak je tomu v kategorii Seniorů, kde mohou soutěžit týmy středoškolá ků v libovolném složení. Celá soutěž trvá pouhé 2 hodiny a soutěžící týmy přistupují k řešení 6 soutěžních úloh. Týmy středoškoláků řeší zadané soutěžní úlohy postupně, přičemž zadání další soutěžní úlohy obdrží až po odevzdání správného řešení úlohy předchozí. Náboj je matematickou soutěží s netradičními soutěžními úlohami, vyžadujícími jistou dávku invence a důvtipu, jak tvrdí sami organizátoři této soutěže. Další matematickou soutěží, které se mohou čeští středoškoláci (konkrétně žáci 3. ročníků) účastnit je Moravskoslezský matematický šampionát, viz např. [75], pořádaný v prostorách Wichterlova gymnázia v Ostravě – Porubě. Jedná se o jednodenní matematickou soutěž. Soutěž probíhá každoročně, vždy v říjnu, a v roce 2012 proběhne již 10. ročník této ne příliš známé matematické soutěže. Soutěžící řeší 5 zajímavých úloh, přičemž v úlohách je kladen důraz na uplatnění matematiky v praktickém životě. V České republice existuje i matematická soutěž určená žákům středních odborných škol. Jde o Celostátní matematickou soutěž žáků SOŠ, SPŠ, ISŠ, SOU, OU, …. Je organizována JČMF, viz např. [77], OA a VOŠ Valašské Meziříčí ve spolupráci s dalšími 21 odbornými školami z celé České republiky, a to v 7 kategoriích podle typu školy, viz např. [65]. Navíc jsou pro jednotlivé kategorie vymezeny i tematické okruhy, což není v matematických soutěžích obvyklé. Poslední zmínka bude patřit účasti českých středoškoláků na mezinárodních soutěžích. Pro přehlednost je v textu uveden pouze výčet mezinárodních matematických soutěží s českou participací, detailněji jsou jednotlivé matematické soutěže popsány ve 4. kapitole. }


\section{Bližšia špecifikácia zadania práce}

Všeobecné zadanie práce bolo po diskusii s~vedúcim práce konkretizované nasledujúcim spôsobom. Obsahom diplomovej práce budú podklady pre prezenčný seminár určený študentom prvého ročníka gymnázia so záujmom o~matematiku. Neprázdnu podmnožinu účastníkov tak pravdepodobne budú tvoriť študenti na matematiku nadaní, cieľom však nie je vytvoriť seminár, ktorý by malú skupinu študentov pripravoval na úspešnú účasť na Medzinárodnej matematickej olympiáde už v~takomto mladom veku. Túto úlohu spĺňa nemalé množstvo existujúcich aktivít (bohatý a výstižný prehľad je možné nájsť v~\cite{svrcek2014}) a tento text nemá ambíciu byť ďalším z~nich. Omnoho dôležitejším predpokladom pre účasť na seminári je tak záujem študenta o matematiku, viac než jeho objektívne ohodnotenie nadanie.

Práca si kladie za cieľ vytvoriť súbor seminárov obsahujúci dostatok úloh a problémov k~samostatnému precvičovaniu, ktorý by bol prístupný širšiemu spektru študentov, nielen úzkej matematickej špičke. Dúfame, že si tak nájde cestu k väčšiemu okruhu študentov a učiteľov. V~neposlednom rade pri špecifikácii zadania zavážil fakt, že autorka má k~zvolenej vekovej kategórii blízko, či už ako organizátorka matematických seminárov a sústredení pre žiakov druhého stupňa základnej školy, alebo začínajúca učiteľka v~nižších triedach gymnázia.

\section*{Východiská pri tvorbe osnov}
Obsah seminárnych stretnutí sa opiera najmä o~dva kľúčové aspekty -- matematickú olympiádu kategórie C a školský vzdelávací program (ŠVP) pre gymnáziá.

\subsection*{Školský vzdelávací program}

Pre potreby tejto práce sme vychádzali zo ŠVP publikovaného na webových stránkach Gymnázia Brno, třída Kapitána Jaroše.

Školský vzdelávací program pre prvý ročník gymnáziálneho vzdelávania sa svojim zameraním zásadne nelíši medzi matematickou a nematematickou triedou a zahŕňa štyri oblasti, ktoré sú podrobnejšie popísané v~nasledujúcej tabuľke.

\begin{longtable}{ p{.48\textwidth} p{.48\textwidth}} \toprule
\multicolumn{2}{p{\textwidth}}{Algebraické výrazy, mocniny a odmocniny} \\ \midrule
  Žiak & Učivo \\ \vspace{-10pt}
  \begin{itemize}
    \item efektívne upravuje výrazy s~premennými, určuje definičný obor výrazov
    \item rozkladá mnohočleny na súčin vynímaním a použitím vzorcov
    \item uskutočňuje operácie s~mocninami a odmocninami, upravuje číselné výrazy
  \end{itemize} &  \vspace{-10pt}
  \begin{itemize}
    \item mnohočleny, lomené výrazy
    \item mocniny s~prirodzeným, celým a racionálnym exponentom, druhá a tretia odmocnina
    \item výrazy s~mocninami a odmocninami
  \end{itemize} \\ \midrule
  \multicolumn{2}{p{\textwidth}}{Teória množín, výroková logika, matematické vety a dôkazy} \\ \midrule
  Žiak & Učivo \\ \vspace{-10pt}
  \begin{itemize}
    \item uskutočňuje správne operácie s~množinami, množiny využíva pri riešení úloh
    \item pracuje správne s~výrokmi, používa správne logické spojky a kvantifikátory
    \item presne formuluje svoje myšlienky a zrozumiteľne sa vyjadruje
    \item rozumie logickej stavbe matematickej vety
    \item vhodnými metódami uskutočňuje dôkazy jednoduchých matematických viet
  \end{itemize} & \vspace{-10pt}
  \begin{itemize}
    \item množiny, operácie s~množinami (zjednotenie, prienik, rozdiel množín, doplnok množiny v~množine, podmnožina, rovnosť množín, Vennove diagramy, de Morganove pravidlá)
    \item výroky, negácie, kvantifikátory, logické spojky (konjunkcia, alternatíva, implikácia, ekvivalencia), výrokové formuly, tautológia, obmena a obrátenie implikácie
    \item definícia, veta, dôkaz
    \item priamy dôkaz, nepriamy dôkaz, dôkaz sporom
  \end{itemize} \\ \midrule
\multicolumn{2}{p{\textwidth}}{Teória čísel} \\ \midrule
  Žiak & Učivo\\ \vspace{-10pt}
  \begin{itemize}
    \item vysvetlí vzťahy medzi číselnými obormi $\NN, \ZZ, \QQ, \QQ'_{\RR}, \RR$
    \item používa vlastnosti deliteľnosti prirodzených čísel
    \item operuje s~intervalmi, aplikuje geometrický význam absolútnej hodnoty,
    \item odhaduje výsledky numerických výpočtov a efektívne ich uskutočňuje, účelne používa kalkulačku
  \end{itemize} & \vspace{-10pt}
  \begin{itemize}
    \item číslo, premenná
    \item číselné obory $\NN, \ZZ, \QQ, \QQ'_{\RR}, \RR$
    \item prirodzené čísla, deliteľnosť ($a$ delí $b$, najväčší spoločný deliteľ, najmenší spoločný násobok, čísla súdeliteľné a nesúdeliteľné, prvočísla a zložené čísla, základná veta aritmetiky)
    \item celé čísla
    \item racionálne čísla
    \item reálne čísla, intervaly, absolútna hodnota
  \end{itemize}\\ \midrule
\multicolumn{2}{p{\textwidth}}{Rovnice a nerovnice} \\ \midrule
  Žiak & Učivo \\ \vspace{-10pt}
  \begin{itemize}
    \item rieši lineárne a kvadratické rovnice, nerovnice a ich sústavy, v~jednoduchých prípadoch diskutuje riešiteľnosť alebo počet riešení
    \item rozlišuje ekvivalentné a neekvivalentné úpravy, zdôvodní, kedy je skúška nutnou súčasťou riešenia
    \item geometricky interpretuje číselné, algebraické a funkčné vzťahy, graficky znázorňuje riešenia rovníc, nerovníc a ich sústav
    \item analyzuje a rieši problémy, v~ktorých aplikuje riešenie lineárnych a kvadratických rovníc a ich sústav
  \end{itemize} & \vspace{-10pt}
  \begin{itemize}
    \item lineárne rovnice a nerovnice
    \item kvadratická rovnica (diskriminant, vzťahy medzi koreňmi a koeficientami, rozklad kvadratického trojčlenu, doplnenie na štvorec), kvadratická nerovnica
    \item rovnice a nerovnice v~súčinovom a podielovom tvare
    \item rovnice s~neznámou v~menovateli a pod odmocninou
    \item lineárna a kvadratická rovnica s~parametrom
    \item kartézsky súčin, binárna relácia a ich grafy
    \item sústavy lineárnych rovníc a nerovníc
  \end{itemize}\\ \bottomrule
\end{longtable}

\subsection*{Matematická olympiáda kategórie C}

Ďalším oporným pilierom boli zadania matematických olympiád kategórie C. Podrobne sme preskúmali úlohy všetkých kôl posledných 10 ročníkov MO (počínajúc 66.\,ročníkom v~školskom roku 2016/2017 a končiac 57.\,ročníkom v~školskom roku 2007/2008) a úlohy v~nich sme rozdelili do štyroch kategórií nasledovne.


\begin{itemize}
\item Algebraické výrazy, rovnice, sústavy rovníc, nerovnosti
\begin{itemize}
\item úpravy výrazov využívajúce identity,
\item slovné úlohy vedúce na riešenie rovníc,
\item sústavy rovníc,
\item rovnice s~parametrami,
\item nerovnosti.
\end{itemize}
\item Teória čísel
\begin{itemize}
\item deliteľnosť, rozklad na prvočísla,
\item najmenší spoločný násobok, najväčší spoločný deliteľ,
\item ciferné zápisy čísel,
\item lomené výrazy,
\item zvyškové triedy.
\end{itemize}
\item Geometria
\begin{itemize}
\item úlohy využívajúce podobnosť trojuholníkov,
\item úlohy využívajúce Pytagorovu vetu,
\item úlohy o obsahoch rovinných útvarov,
\item úlohy o kružniciach vpísaných a opísaných trojuholníku,
\item netradičné úlohy.
\end{itemize}
\item Rôzne (najčastejšie kombinatorika)
\begin{itemize}
\item úlohy využívajúce mriežky a šachovnice,
\item hľadanie víťaznej stratégie,
\item úlohy využívajúce Dirichletov princíp,
\item úlohy o~počte známych (teória grafov),
\item logické úlohy nevyžadujúce žiadne špeciálne vedomosti.
\end{itemize}
\end{itemize}


Toto rozdelenie ďalej zohráva úlohu v~zaraďovaní obsahov seminárov -- v~nich sa budeme postupne venovať všetkým štyrom vyššie spomenutým oblastiam. Zároveň však krajské kolo kategórie C prebieha v~prvej polovici apríla, preto je zmysluplné využiť zostávajúcu časť školského roka na pozvoľné začatie prípravy na úlohy MO kategórie~B, prípadne ďalšie matematické zaujímavosti.

\section{Doplňujúce zdroje, materiály a inšpirácia}

Aj napriek tomu, že prevažná väčšina seminárov sa opiera najmä o úlohy matematickej olympiády z minulých rokov, vychádzali sme pri tvorbe obsahu z ďalších zdrojov.

Výborným sprievodcom do sveta riešenia matematických problémov je~\cite{holton2010}, ktorý veľmi pútavou formou zoznamuje čitateľa s rôznymi oblasťami stredoškolskej olympiádnej matematiky a metódami riešenia problémov. Kniha je plná veľkého množstva cvičení a úloh, všetky majú v publikácii aj riešenia.

Podobne zaujímavým počinom je aj~\cite{zeitz2007}, ktorý však obsahuje aj matematiku vysokoškolskej úrovne. Obe publikácie sú písané v angličtine, český ani slovenský preklad zatiaľ nevyšiel, no ak je čitateľ vybavený angličtinou aspoň na základnej úrovni, nemali by tieto dve knihy ujsť jeho pozornosti.

V českých a slovenských kruhoch MO sú veľmi populárne knihy z edície \textit{Škola mladých matematikov}, ktorá začala vychádzať v roku 1961 na podnet Ústredného výboru MO. Edícia obsahuje 61 knižočiek na rozličné témy rozširujúce a prehlbujúce matematické znalosti stredoškolákov. Nie všetky knižky sú však v celom svojom obsahu prístupné mladším žiakom gymnázií, keďže mnohé sa venujú aj pokročilejším partiám matematiky. Užívateľský príjemný je fakt, že všetky knihy sú dostupné na TODO

Poslednou, ale určite nie najmenej dôležitou dvojicou kníh je~\cite{herman2004} a \cite{herman2011} , ktoré ponúkajú hlboký a precízny pohľad na riešenie úloh z oblasti rovníc, nerovností, teórie čísel a kombinatoriky.



\section{Celoročný koncept seminára}

Pri tvorbe a zaradení jednotlivých seminárov sme okrem ŠVP pre gymnáziá a typických úloh MO brali do úvahy aj rytmus školského roka, ktorý ovplyvňujú najmä termíny prázdnin a v~prípade matematicky zameraných študentov aj termíny jednotlivých kôl MO. V~ďalších odsekoch tak vysvetlíme, ako sme všetky uvedené skutočnosti pretavili do osnovy obsahu seminára na jeden školský rok.

Školský rok má približne 40 týždňov (10 mesiacov $\times$ 4 týždne). Predpokladáme, že seminárne stretnutia začnú v~polovici septembra, keďže prvých pár týždňov sa časové plány študentov aj učiteľov ešte len ustaľujú. Podobne, posledné dva júnové týždne bývajú v~školách často nepravidelné, pretkané školskými výletmi a ďalšími školskými aktivitami. Študentov tiež potešia vianočné a jarné, týždeň trvajúce, prázdnin, kedy sa seminár neuskutoční. Okrem toho sme v máji zaradili len tri stretnutia, keďže v tomto období je na mnohých školách režim, vzhľadom na maturitné skúšky, nepravidelný.

Celkovo teda v~našom pláne počítame s~33 stretnutiami počas celého školského roka, pričom si autorka práce uvedomuje, že tento počet stretnutí nie je univerzálne použiteľný každým učiteľom v~každom školskom roku.

Prirodzenými míľnikmi, okrem väčších prázdninových úsekov, sú tiež termíny späté s~jednotlivými kolami MO, v~našom prípade budeme brať ohľad najmä na kategóriu C. Riešenia domáceho kola musia študenti spravidla odovzdať najneskôr v~prvej polovici januára. Približne desať dní po tomto termíne sa koná kolo školské a sezóna je zavŕšená krajským kolom, ktoré študentov zvyčajne potrápi v~prvej polovici apríla.

Za vhodné tiež považujeme zmieniť, že žiadne kolo žiadneho ročníka MO sa nezaobíde bez geometrickej úlohy. Tie tak tvoria minimálne štvrtinu až polovicu všetkých úloh v~danom ročníku. ŠVP pre gymnáziá však geometriu v~prvom ročníku neobsahuje. Pohľad na olympiádne úlohy nám prezradí, že úlohy kategórie C nevyžadujú žiadne špeciálne geometrické vedomosti, je však vhodné so študentami osviežiť ich poznatky zo základnej školy a riešenie úloh z~geometrie postupne trénovať.

Seminár je rozdelený na tri hlavné časti -- pred termínom domáceho kola, medzi domácim a krajským kolom a po krajskom kole. Prvé dve časti sú usporiadané tak, aby sme sa v~každej z~nich dotkli väčšiny zo štyroch zmienených typov úloh MO, teda v~každom z~dvoch blokov sa budeme postupne venovať algebraickým výrazom a rovniciam (úpravy výrazov, riešenie (systémov) rovníc a nerovností), teórii čísel (deliteľnosť, prvočísla, ciferné zápisy) a geometrii, v~druhom bloku seminár obsahuje aj stretnutia zamerané na kombinatorické úlohy. V~treťom bloku sme zaradili dva algebraické semináre zamerané na to, čo by približne v rovnakom čase mali preberať študenti na vyučovacích hodinách, matematickú hru a opakovanie pod taktovkou samotných študentov.

Možnosťou, ktorú sme pri zaraďovaní obsahu jednotlivých stretnutí zvažovali, bolo rozdelenie celého školského roka do štyroch veľkých blokov, ktoré by zodpovedali  štyrom oblastiam úloh MO. Tomuto variantu sme však nakoniec prednosť nedali, pričom hlavným dôvodom bolo, že sme chceli študentov pripraviť na čo najširšie spektrum úloh už pred domácim a školským kolom. Zároveň veríme, že vracanie sa k~už načatým témam pomôže študentom lepšie si ich upevniť, preto sa nám rozdelenie popísané vyššie zdalo vhodnejšie.

\subsubsection*{Stručný prehľad obsahu seminárov}

V~tabuľke nižšie uvádzame konkrétne zamerania jednotlivých seminárov, ktoré vznikli na základe zohľadnenia poznatkov z~predchádzajúcich štyroch odsekov. Tento stručný prehľad má poslúžiť ako počiatočná orientácia v~kontexte celého školského roka. Konkrétnejší popis obsahu, spolu s~úlohami, komentármi a domácou prácou je obsahom nasledujúcej kapitoly.

\begin{longtable}{ p {0.04\textwidth} p{.35\textwidth} p{.5\textwidth}} \toprule
\textbf{September} & &\\ \midrule
1. & Úvod do seminára, \textit{Matematico} & Základné informácie,  očakávania, matematická hra.\\
2. & Ako riešiť &  Všeobecná diskusia o~spôsobe riešenia úloh/problémov v~matematike.\\ \midrule
\textbf{Október} & &\\ \midrule
3. & Algebraické výrazy a rovnice I~& Opakovanie vedomostí zo ZŠ, práca s~výraz\-mi.\\
4. & Algebraické výrazy a rovnice II & Jednoduchšie nerovnosti. \\
5. & Algebraické výrazy a rovnice III & Rovnice a systémy rovníc.\\
6. & Teória čísel I& Opakovanie vedomostí zo ZŠ, deliteľnosť.\\ \midrule
\textbf{November} & &  \\ \midrule
7. &Teória čísel II & Najmenší spoločný násobok, najväčší spoločný deliteľ. \\
8. & Teória čísel III & Ciferné zápisy čísel.\\
9. & Geometria I & Opakovanie poznatkov zo ZŠ, správne riešenie geometrickej úlohy. \\
10. & Geometria II & Úlohy využívajúce podobnosť trojuholníkov a Pytagorovu vetu.\\ \midrule
\textbf{December} &  &\\ \midrule
11. & Geometria III & Úlohy o~obsahoch.\\
12. & Geometria IV & Úlohy o~kružnici vpísanej a opísanej trojuholníku.\\
13. & \textit{Náboj} & Vianočná tímová súťaž v počítaní úloh.\\ \midrule
\textbf{Január} & & \\ \midrule
14. & Domáce kolo MO & Analýza úloh domáceho kola.\\
15. & Konzultačný seminár & Konzultácia nejasností pred školským kolom MO. \\
16. & Školské kolo MO & Analýza úloh školského kola.\\
17. & Algebraické výrazy a rovnice IV& Nerovnosti -- pokračovanie. \\ \midrule
\textbf{Február} & &  \\ \midrule
18. & Algebraické výrazy a rovnice V. & Zložitejšie rovnice a sústavy rovníc\\
19. & Teória čísel IV & Úlohy o~prvočíslach.\\
20. & Teória čísel V & Miš-maš.\\ \midrule
\textbf{Marec} & &\\ \midrule
21. & Geometria V& Úlohy o štvrouholníkoch. \\
22. & Geometria VI & Netradičné geometrické úlohy.\\
23. & Súťaž \textit{Náboj} & Účasť na medzinárodnej súťaži tímov. \\
24. & Rôzne I & Úlohy o mriežkach a šachovniciach.\\ \midrule
\textbf{Apríl} & & \\ \midrule
25. & Rôzne II & Hľadanie víťaznej stratégie, logické úlohy.\\
26. & Krajské kolo MO & Analýza úloh krajského kola MO.\\
27. & Hra SET& Matematická hra, úlohy využívajúce kartičky z hry. \\
28. & Súťaž \textit{Náboj} & Tímová súťaž v~riešení úloh.\\ \midrule
\textbf{Máj} & &\\ \midrule
29. & Algebraické výrazy a rovnice VI & Rovnice a sústavy rovníc s~parametrom.\\
30. & Algebraické výrazy a rovnice VII & Úlohy o~kvadratických rovniciach využívajúce vzťahy medzi koreňmi.\\
31. & Opakovanie I & Opakovanie celkov Algebra a Teória čísel. \\ \midrule
\textbf{Jún} & & \\ \midrule
32. & Opakovanie II  & Opakovanie celkov Geometria a Kombinatorika.\\
33. & MO kategórie B & Záverečné riešenie úloh, spätná väzba, uzavretie.\\ \bottomrule
\end{longtable}

Záverom prvej kapitoly ešte poznamenajme, že veľkú väčšinu úloh v seminári tvoria úlohy prevzaté z rôznych kôl MO. V ďalšom texte sú tieto úlohy jasne označené v tradičnom formáte značenia úloh MO, kde prvé číslo je príslušný ročník MO, ďalej kolo (I pre kolo domáce, S pre kolo školské a II pre kolo krajské) a číslo úlohy. Ak za číslom úlohy nasleduje ešte ďalší identifikátor, znamená to, že ide o príslušnú úlohu návodnú (N) alebo doplňujúcu (D). Taktiež riešenia sú takmer vždy prevzaté z oficiálnych riešení (ktoré sú spolu zo zadaniami voľne prístupné na stránkach MO ) Prevzaté alebo len mierne upravené riešenia sú označené hviezdičkou \textbf{*}, pôvodné riešenia hviezdičku nemajú.


\chapter{Osnovy seminárnych stretnutí}

\seminar{1}

\weblinks{Na stiahnutie: \href{pdf/seminar01-teacher.pdf}{učiteľská verzia}, \href{pdf/seminar01-student.pdf}{študentská verzia}}

\subsection*{Téma}
Úvod do seminára, očakávania, \textit{Matematico}

\subsection*{Ciele}
Zoznámiť študentov s povahou seminára a plánom na školský rok, zoznámiť sa so študentami, motivovať na začiatok matematickou hrou.

\subsection*{Priebeh}

Keďže ide o prvý seminár, oboznámime študentov s tým, čo môžu v priebehu roka od stretnutí očakávať: aká bude forma seminárov a čo bude ich obsahom. Zároveň považujeme za vhodné porozprávať sa so študentmi o ich motivácii -- čo ich na seminár privádza a čo od neho očakávajú. Takáto informácia nám potom môže poslúžiť pri plánovaní alebo prispôsobovaní obsahu konkrétnej skupine, s ktorou budeme pracovať. Napríklad ak sa seminára budú účastniť v drvivej väčšine študenti s ambíciami na úspešné umiestnenie na Medzinárodnej matematickej olympiáde, je možné hravejšie semináre nahradiť náročnejšími matematickými partiami.

Po tomto úvode zaradíme matematicko-logickú hru \textit{Matematico}, ktorá študentov otestuje v rýchlom a optimálnom rozhodovaní sa.

\subsubsection*{Pravidlá}

Každý študent dostane tabuľku pozostávajúcu z $5\times5$ štvorčekov, do ktorej si bude zapisovať čísla, ktoré bude vedúci seminára postupne vyťahovať z balíčka. Balíček obsahuje 52 čísel, každé z čísel $1-13$ sa v balíčku nachádza štyrikrát. Študenti majú vždy 7 sekúnd na to, aby číslo zapísali do práve jedného voľného políčka v tabuľke. Po vyplnení všetkých políčok si študenti spočítajú body podľa kľúča v tabuľke \ref{fujky}.
\begin{table}
\begin{tabular}{l l l}
Číselná kombinácia & v riadku alebo stĺpci & na uhlopriečke \\
\hline
dve zhodné čísla & 10 & 20\\
dva páry zhodných čísel & 20 & 40\\
tri zhodné čísla & 40 & 50 \\
tri zhodné čísla a dve zhodné čísla & 80 & 90 \\
štyri zhodné čísla & 160 & 170 \\
päť za sebou idúcich čísel & 50 & 60 \\
tri jednotky a dve trinástky & 100 & 110 \\
\end{tabular}
\caption{Body v hre \textit{Matematico}} \label{fujky}
\end{table}

Jednotlivé výsledky študentov píšeme na tabuľu, aby bolo možné vidieť rozsah nahraných bodov. Aby si sa študenti s hrou poriadne zoznámili, je vhodné zahrať aspoň dve alebo tri kolá. Po nich povzbudíme študentov, aby so spolužiakmi prediskutovali stratégiu, ktorú počas hry používajú, čo sa im overilo a čo nie, príp. podľa akého kľúča zapisujú čísla do tabuľky a či stratégiu v priebehu hry menia. Po výmene skúseností opäť odohráme dve alebo tri kolá a sledujeme, či sa priemerný počet nahraných bodov zvýšil.

Po poslednom kole vyzveme študentov, aby sa použitím čísel, ktoré boli v tomto kole vytiahnuté z balíčka, snažili maximalizovať bodový zisk, t.j. daných 25 čísel majú usporiadať do tabuľky tak, aby získali čo najviac bodov. Získané počty bodov pripíšeme na tabuľu porovnáme s predchádzajúcim kolom.

Zaujímavým pozorovaním je, že rozptyl bodov v takomto modifikovanom kole je značne menší ako v troch pôvodných. Môžeme sa so žiakmi porozprávať o dôvodoch, ktoré k tomu môžu viesť.

Na záver seminára môžeme študentov nechať maximalizovať bodový zisk použitím ľubovoľných čísel v ponuke a vyhlásiť súťaž o bonus. \\

Hra je dostupná aj online na \url{http://yetty.github.io/Matematico/}, prípadne na \url{http://matematico.cz/}

\subsubsection*{Variácie}

Študenti môžu \textit{Matematico} hrať aj v dvojiciach. Takéto usporiadanie ponúka možnosť diskutovať rozhodnutia, ktoré študenti robia a trénuje ich argumentačné a vysvetľovacie schopnosti.

\textbf{Záverečný komentár}
Prvý seminár je časovo aj obsahovo menej hutný ako semináre nasledujúce, nie je to však nijako na škodu, keďže jeho zámer bol viac organizačný a informatívny, než matematický.

\seminar{2}

\weblinks{Na stiahnutie: \href{pdf/seminar02-teacher.pdf}{učiteľská verzia}, \href{pdf/seminar02-student.pdf}{študentská verzia}}

\subsection*{Téma}
Prístup k riešeniu matematických úloh, typy dôkazov

\subsection*{Ciele}
Prediskutovať so študentmi rôzne prístupy k riešeniu neznámych problémov, zopakovať a/alebo zoznámiť s typmi dôkazov používaných v matematike.

\subsection*{Priebeh}

Spoločne so študentami prečítame preriešime prvú kapitolu z~\cite{holton2010}

\seminar{3}{Prístup k riešeniu matematických úloh, typy dôkazov}

\teachernote{
\subsection*{Ciele}
Prediskutovať so študentmi rôzne prístupy k riešeniu neznámych problémov, zopakovať a/alebo zoznámiť s typmi dôkazov používaných v matematike.

\subsection*{Priebeh}

Seminár prebehne formou štrukturovanej diskusie, v ktorej so študentami rozoberieme korektný spôsob riešenia matematických úloh a problémov, zamyslíme sa nad tým, čo musí správne riešenie obsahovať a čoho sa, naopak, vyvarovať a vyzbrojíme študentov základnými stratégiami, ktoré môžu pri riešení úloh využiť.

\subsubsection*{Typy úloh}

Pri riešení matematických problémov sa stretávame s dvoma základnými kategóriami: buď je úlohou dokázať (príp. vyvrátiť) dané tvrdenie, alebo nájsť objekty (čísla, tvary, výrazy, množiny bodov), ktoré vyhovujú zadanými podmienkam. Niekedy je úlohou nájsť aspoň jeden vhodný príklad, niekedy sa musia riešitelia popasovať s nájdením všetkých objektov majúcich vhodné vlastnosti.

\subsubsection*{Fázy riešenia}

Podľa \cite{holton2010} má riešenie problémov nasledujúce fázy.
\begin{enumerate}[a)]
\item Prečítanie a porozumenie.
\item Kľúčové slová.
\item Panika.
\item Systém.
\item Vzorce.
\item Odhad.
\item Matematická technika.
\item Vysvetlenie.
\item Zovšeobecnenie.
\end{enumerate}

V prípade úloh MO sa častokrát k poslednému kroku nedostaneme, považujeme však za vhodné ho zmieniť, keďže sa dotýka všeobecnejšieho matematického myslenia, ku ktorému sa študentov pravdepodobne snažíme viesť.

\subsubsection*{Správne riešenie má$\ldots$}

V tejto časti seminára so študentmi prečítame zopár odsekov o riešení úloh z \cite{kms} a pozrieme na sa dve nesprávne riešenia úlohy o lodiach. \\
\\
Nasledujúci text vo zvyšku tejto časti je (takmer) doslovným výťahom z rozsiahlejšieho textu \uv{Ako riešiť}, dostupnom na \cite{kms}.

Vyriešiť úlohu neznamená len nájsť výsledok. Treba taktiež dokázať, že nájdený výsledok je správny. ($\ldots$) Pri riešení úloh je dôležité vedieť, kedy mám už úlohu dokončenú so všetkým, čo treba, a kedy mi k nej ešte niečo chýba.

Opisuj svoje úvahy všeobecne. Veľa úloh ($\ldots$) je formulovaných všeobecne. Treba v nich napríklad niečo dokázať pre všetky prirodzené čísla $n$ (alebo niečo obdobné). Bežným postupom, ako prísť na riešenie takýchto úloh, je skúšať ich vyriešiť postupne pre $n$ rovné $1, 2, 3,\,\ldots$ objaviť pri tom spoločné princípy a zovšeobecniť ich. Avšak nestačí do riešenia napísať, ako sa úloha rieši pre niekoľko malých hodnôt a zvyšok odbiť slovami a tak ďalej. Trénuj si pri riešení úloh všeobecné vyjadrenie, príde ti vhod počas strednej školy a neskôr aj počas vysokej.

Dokazuj veci poriadne, nie intuitívne. Častým nedostatkom riešení býva nedostatočné zdôvodnenie tvrdení, resp. zdôvodnenie len na intuitívnej úrovni. Veľmi častými pojmami, ktoré signalizujú intuitívne dokazovanie, sú tvrdenia: \uv{Robíme to optimálne a preto to je najlepšie, ako sa dá.} \uv{Zoberieme si najhorší možný prípad\ldots} \uv{Ak to spravíme inak, tak si len pohoršíme.} Aby takéto pojmy mali význam v dôkazoch, musia byť riadne podložené matematickými pojmami. Väčšinou nám však len pomôžu na odhalenie správneho výsledku, ktorý potom už riadne bez nich dokážeme.

Príklad dôkazu, ktorý je založený na intuícii a nie je formálne správny, nasleduje.\\
\\
\problem{Lode}{
Na pláne $7\times 7$ hráme hru lode. Nachádza sa na ňom jedna loď $2\times3$. Môžeme vystreliť na ľubovoľné políčko plánu, a ak loď zasiahneme, hra končí. Ak nie, strieľame znova. Určte najmenší počet výstrelov, ktoré potrebujeme, aby sme s istotou loď zasiahli.}
{\textbf{Pokus o riešenie 1*.}
Naším cieľom je strieľať čo najoptimálnejšie, aby sme použili čo najmenej výstrelov. Neoplatí sa nám preto strieľať na políčka pri sebe, lebo výstrely pri sebe pokryjú menej políčok ako výstrely ďalej od seba. Nesmieme však strieľať moc ďaleko od seba. Ak medzi dvoma výstrelmi sú aspoň dve políčka už sa tam môže zmestiť loď. Najlepšie teda bude, keď medzi susednými výstrelmi bude jedno voľné políčko. To vieme dosiahnuť vystrelením na políčka ako je znázornené na obrázku.
\begin{figure}[h]
    \centering
    \includegraphics[width=0.35\textwidth]{images/lode1\imagesuffix}
    \caption{}
    \label{fig:lode1}
\end{figure}
Na to, aby sme s istotou zasiahli loď, potrebujeme najmenej 9 výstrelov. Na menej výstrelov to nie je možné, pretože sme strieľali najoptimálnejšie.\\
\\
\textit{Rozbor riešenia*.} Na prvý pohľad toto riešenie môže vyzerať výborne, avšak riešenie má vážny nedostatok - chýba mu zdôvodnenie, prečo nestačí menej ako 9 výstrelov. Ale prečo? Veď sa v riešení spomína, že menej výstrelov nestačí. Objavuje sa tu aj ďalší problém. Uvedené zdôvodnenia sú len veľmi vágne a len na intuitívnej úrovni. Poďme si to postupne rozobrať.

\uv{Neoplatí sa nám preto strieľať na políčka pri seba...} Slovíčko \uv{neoplatí} je znamením intuitívnych dôkazov. Aby malo v dôkaze význam, musí byť matematicky podložené. Ako je podložené tu? Riešiteľ síce opisuje, že výstrely pri sebe pokryjú menej políčok ako výstrely ďalej od seba, ale tu už matematické podklady končia. Riešiteľ nepíše, čo to znamená pokryť políčko. Znamená to, že doňho nesmie zasahovať loď, nesmie byť v ňom ľavý horný roh lode alebo niečo iné? Taktiež nezdôvodňuje, prečo počet pokrytých políčok bude menší. Patrilo by sa sem uviesť nejaký výpočet, že naozaj ten počet (tak ako ho máme definovaný) vyjde menší.

Ďalším problémom je, že riešiteľ sa pozerá na polohu iba dvoch políčok. Vo všeobecnosti neplatí, že lokálne najoptimálnejšie riešenie je najoptimálnejšie aj globálne. Môže sa nám stať (zatiaľ teoreticky), že dva výstrely nebudú umiestnené optimálne, ale to, čo pri nich stratíme, môžeme získať pri iných dvojiciach a môže nás to doviesť k ostro lepšiemu rozloženiu výstrelov ako keby sme sa snažili všetko spraviť \uv{najoptimálnejšie} na úrovni susedných výstrelov.

Uvedené riešenie teda iba intuitívne vysvetľuje, prečo je 9 najmenší počet výstrelov. Použitý postup je pre nás dobrý, keď sa snažíme nájsť políčka kam strieľať. Avšak z formálnej stránky to hodnote riešenia nepridáva a riešenie má rovnakú hodnotu, ako keby sme len v ňom uviedli obrázok s vetou, že 9 výstrelov stačí.

Z hľadiska formulácie riešenia je problém, že riešenie v sebe mieša dve časti: konštrukciu, že 9 výstrelov stačí a dôkaz, že menej nestačí. Síce toto nie je chyba, ktorá by sama o sebe zapríčinila stratu bodov. Ale rozčlenenie riešenia na dve časti vám pomôže si ujasniť, že každá z týchto častí je vyriešená úplne a nie len intuitívne.\\

Čo v tejto fáze riešenia? Rozhodne nie sme spokojní s tým, že úlohu máme. Keď máme pocit, že lepšie to už nejde, pustíme sa do dokazovania, že je to naozaj tak. Častokrát si táto fáza vyžaduje pozerať sa na úlohu z inej strany. To bude zrejme potrebné aj v našom prípade. Jednou našou možnosťou síce je matematicky podložiť naše úvahy, ale čaká nás niekoľko problémov. Ide hlavne o to, že sa chceme pozerať na to, ako musia byť umiestnené dva výstrely, čo nemusí stačiť, ako sme spomenuli vyššie.\\
\\
\textbf{Pokus o riešenie 2*.} Na nasledujúcom obrázku vidíme 8 obdĺžnikov $2\times3$ rozmiestnených na plániku, ktoré sa neprekrývajú.

\begin{figure}[h]
    \centering
    \includegraphics[width=0.35\textwidth]{images/lode2\imagesuffix}
    \caption{}
    \label{fig:lode2}
\end{figure}

Ak by sme mali menej ako 8 výstrelov, tak jeden z ôsmich obdĺžnikov nezasiahneme a zrovna tam sa môže nachádzať loďka. Potrebujeme teda do každého obdĺžnika vystreliť jeden výstrel. Musíme ich umiestniť zároveň tak, aby sa medzi ne nezmestila žiadna iná loď. Na to, aby sme s istotou zasiahli loď, potrebujeme najmenej 8 výstrelov.\\
\\
\textit{Rozbor riešenia*.} Toto riešenie taktiež nie je úplné. Riešiteľ v ňom ukázal len, že potrebuje aspoň 8 výstrelov. Neukázal, že 8 výstrelov stačí. Hovorí iba v teoretickej rovine, že ak sa mu podarí tých 8 výstrelov vhodne umiestniť, tak budú stačiť. Ale je to naozaj možné? Čo ak z nejakého iného dôvodu 8 výstrelov stačiť nebude.\\
\\
\textit{Spoločný rozbor*.} Vidíme, že riešenia 1 a 2 majú rôzne výsledky, jedno tvrdí, že najmenej výstrelov je 9 a druhé 8. Jedno riešenie je teda nesprávne, ale ktoré? Ak sme pri riešení tejto úlohy v stave, že máme všetko, čo tieto dve riešenia spolu, tak vieme, že správnym riešením úlohy je 8 alebo 9 výstrelov. Pre dokončenie úlohy potrebujeme buď nájsť 8 políčok, na ktoré nám stačí vystreliť, alebo dokázať, že 8 výstrelov nestačí.

Na prekvapenie (možno nie pre všetkých je to prekvapenie) naozaj stačí 8 výstrelov. Všetky intuitívne argumenty v riešení 1 sú teda nesprávne. Na riešení uvedenom nižšie si môžete všimnúť, že nie je pravda, že sa neoplatí strieľať blízko seba. Stalo sa tu, pred čím sme varovali. Tým, že sme na niektorých miestach strieľali „neoptimálne“ zlepšilo situáciu inde a v konečnom dôsledku sme dosiahli lepšie riešenie.\\
\\
\textbf{Vzorové riešenie*.} Ukážeme, že 8 je najmenší počet výstrelov, ktorý potrebujeme, aby sme s istotou zasiahli loď.

\begin{figure}[h]
    \centering
    \includegraphics[width=0.6\textwidth]{images/lode3\imagesuffix}
    \caption{}
    \label{fig:lode3}
\end{figure}


Na obrázku \ref{fig:lode3} vľavo vidíme, že môžeme na plán umiestniť 8 neprekrývajúcich sa obdĺžnikov $2\times 3$ (stredné políčko ostane prázdne). Aby sme s istotou zasiahli loď, musíme zasiahnuť aspoň jedno políčko v každom z ôsmich vyznačených obdĺžnikov, preto potrebujeme aspoň 8 výstrelov.

Na obrázku vpravo je uvedený príklad výberu ôsmych políčok, na ktoré stačí vystreliť, aby sa už mimo nich nedala na plán umiestniť žiadne loď $2\times3$. Preto týchto 8 výstrelov k zasiahnutiu lode vždy stačí.\\
\\
\textit{Rozbor vzorového riešenia*.} Táto úloha je pekným príkladom toho, ako sa to, čo je napísané v riešení, líši od toho, čím sme prechádzali pri riešení úlohy. Hoci je samotné riešenie úlohy krátke, úlohu sme veru krátko neriešili. Väčšina riešiteľov začne s deviatimi výstrelmi ako v pokuse 1 a potom sa postupne cez rôzne vylepšenia strieľania, rôzne pokusy o 8 výstrelov a ďalšie iné úvahy dostane k riešeniu s ôsmimi výstrelmi.

Všimnime si ďalej členenie riešenia. Najprv uvedieme, čo chceme ukázať (táto časť môže byť aj na konci) a vo zvyšku riešenia dokážeme, že naše riešenie je správne. Riešenie je pekne rozčlenené na dve časti podľa dvoch bodov, čo potrebujeme pri takejto úlohe ukázať (môžu byť aj vymenené). V prvej časti ukážeme, že potrebujeme aspoň 8 výstrelov. V druhej ukážeme, že 8 výstrelov nám naozaj stačí. Nie je tu žiadne premiešavanie týchto dvoch častí.\\
\\
\kom V tejto časti seminára je možné so študentmi text vyššie spoločne prečítať a analyzovať. Zaujímavejšou a prínosnejšou sa však zdá byť možnosť, kedy študentom predostrieme vyššie spomenuté nesprávne pokusy o riešenie a spoločne budeme hľadať problematické miesta a diskutovať, ako ich vylepšiť.
}

%Stačí ukázať správnosť riešenia
%Pri spisovaní svojich riešení môžeš byť zvyknutý písať všetky svoje myšlienkové pochody, ktoré Ťa priviedli k riešeniu. To, čo v KMS hodnotíme, je správnosť riešení. Pri niektorých úlohách je to len malá časť zo všetkých úvah, ktoré pri riešení urobíš (napr. pri úlohe Lode).

%Keď vyriešiš úlohu, tak predtým, ako sa spustíš do jej spisovania, premysli si, čo všetko potrebuješ na to, aby si zdôvodnil správnosť svojho riešenia. Stačí, keď napíšeš len to. Takéto premyslenie ti pomôže skontrolovať, či si nezabudol na dôležitú úvahu. Taktiež, keď riešenie okrešeš o zbytočné veci, ušetríš si čas, narobíš menej chýb a môžeš sústrediť svoju pozornosť na to podstatné. (Ako bonus ešte spravíš radosť nám opravovateľom.)

%Naučiť sa správne spisovať riešenia chce veľa cviku. Ak si nie si istý, čo všetko treba v riešení a čo nie, tým, že toho do riešenia napíšeš viac, body nestratíš.

%Úlohy, kde hľadáme výsledky
%Cieľom mnohých úloh je nájsť čísla, ktoré vyhovujú nejakým podmienkam. Výsledkom tejto úlohy je nejaká množina čísel, či už jedno, dve, kľudne aj nekonečne veľa.% Príkladom úlohy tohto typu je úloha Mince.
%Aby sme ukázali, že náš výsledok je správny, potrebujeme ukázať dve veci:

%Každé číslo z nášho výsledku vyhovuje podmienkam zo zadania (hovorí sa tomu aj skúška správnosti).
%Žiadne iné čísla podmienkam zo zadania nevyhovujú.
%V niektorých úlohách je možné tieto dve veci ukázať naraz, napr. pomocou ekvivalentných úprav pri riešení rovníc. %Avšak ak si nie si istý/-á, napíš do svojho riešenia každú časť zvlášť.

%V takýchto úlohách nemusíme hľadať vždy čísla, ale aj funkcie, tabuľky, body a iné veci. To však nič nemení na spomenutých dvoch veciach, ktoré musia byť obsiahnuté v úplnom riešení.

%Úlohy, kde hľadáme niečo najmenšie / najväčšie
%Ďalšiu skupinu úloh tvoria tie, kde je potrebné nájsť najmenšie číslo, najmenší rozmer tabuľky, počet ľudí v skupine, pre ktorý je niečo možné. Príkladom úlohy tohto typu je úloha Lode.

%Rovnako ako v predchádzajúcom type úloh, aj tu treba ukázať rovnaké dve veci. Ak chceme ukázať, že najmenšie číslo n, ktoré vyhovuje zadaniu je k, potrebujeme ukázať:

%Pre $n=k$ je možné splniť zadanie (tzv. konštrukcia).
%Pre $n<k$ nie je možné splniť zadanie (tzv. dolný odhad).
%Vo väčšine úloh tohto typu nejdú tieto dve veci zlúčiť do jednej. Treba si dať pozor, že ukázať druhý bod vo všeobecnosti neznamená ukázať, že zadanie nemožno splniť $n=k-1$.

%Ukázať iba jeden zo spomenutých bodov nestačí:

%Ukážeme iba, že pre nejaké $n=a$ zadanie ide splniť (konštrukciu). To znamená, že hľadané najmenšie n môže byť najviac a. Aby sme úlohu doriešili, potrebujeme nájsť dôkaz, prečo zadanie nemožno splniť pre $n<a$ (pomocou dolného odhadu dokázať, že neexistuje lepšia konštrukcia).
%Ukážeme iba, že pre $n<a$ zadanie nemožno splniť (dolný odhad). Vtedy vieme, že hľadané najmenšie n je aspoň a. Pre dokončenie úlohy potrebujeme nájsť konštrukciu splnenia zadania pre $n=a$ (pomocou konštrukcie dokázať, že neexistuje lepší dolný odhad).
%V praxi to vyzerá tak, že postupne hľadáme lepšie konštrukcie a lepšie dolné odhady až sa nám naraz stretnú na jednej hodnote a riešenie je hotové.

%Pre úlohy, kde hľadáme niečo najväčšie, platia obdobné zásady.


\subsubsection*{Stratégie na začiatok}

Častokrát sa stane, že úloha, pred ktorú sú študenti postavení, je náročná a nie je zrejmé, akým spôsobom sa pustiť do jej riešenia. Preto považujeme za vhodné predstaviť študentom štvoricu stratégií, ktoré im môžu pomôcť v začiatkoch riešenia problému.

\color{red} (Doplniť popis stratégií podľa \cite{zeitz2007}.)

\begin{enumerate}
\item Zorientuj sa.
\item Vyhrň si rukávy a pusť sa do práce.
\item Predposledný krok.
\item Zjednodušuj.
\end{enumerate}

\color{black}
\subsubsection*{Typy dôkazov}
\todo{Doplniť prehľad dôkazov podľa \cite{polak2014}.}

\begin{enumerate}
\item Priamy dôkaz. \todo{Úloha \cite{zeitz2007}: Ktoré prirodzené čísla majú nepárny počet deliteľov? Komentár o použitých stratégiách.}
\item Dôkaz sporom. \todo{$\sqrt{2}$ je iracionálne číslo.}
\item Dôkaz použitím matematickej indukcie.
\end{enumerate}
}


\seminar{4}

\subsection*{Téma}
Algebraické výrazy, rovnice a nerovnosti II -- nerovnosti

\teachernote{
\subsection*{Ciele}
Zoznámiť študentov so základnými metódami pri dokazovaní nerovností a nerovnosťou $a+\frac{1}{a}\geq 2$, ktorá platí pre každé kladné reálne číslo $a$.

\subsubsection*{Úvodný komentár}
Dokazovanie nerovností nie je bežným obsahom základoškolskej, príp. gymnaziálnej výuky, keďže študenti sa stretávajú prevažne s~cvičeniami a problémami, kde je ich úlohou riešiť (lineárne) nerovnice. Dokazovanie nerovností je však častou súčasťou všetkých kôl MO, preto považujeme za vhodné tieto typy úloh so študentami precvičovať. Keďže je tento seminár jedným z~dvoch, ktoré sú na nerovnosti zamerané, budeme sa v~ňom zaoberať jednoduchšími úlohami. Študenti si tak osvoja základné postupy, ktoré im neskôr (snáď) poslúžia pri úlohách zložitejších, zaradených do seminára v~budúcnosti.

}
\subsection*{Úlohy a riešenia}

% Do not delete this line (pandoc magic!)

\problem{58-S-1}{seminar05,nerovnosti}{
Dokážte, že pre ľubovoľné nezáporné čísla $a, b, c$ platí $$(a + bc)(b + ac) \geq ab(c + 1)^2.$$
Zistite, kedy nastane rovnosť.
}{
\rieh Roznásobením a ďalšími ekvivalentnými úpravami dostaneme
\begin{align*}
ab + b^2 c + a^2 c + abc^2 &\geq abc^2 + 2abc + ab,\\
b^2 c + a^2 c &\geq 2abc,\\
(a - b)^2 c &\geq 0.
\end{align*}
Podľa zadania platí $c \geq 0$ a druhá mocnina reálneho čísla $a-b$ je tiež nezáporná, takže je nezáporná aj ľavá strana upravenej nerovnosti. Rovnosť v~tejto (a rovnako aj v~pôvodnej nerovnosti) nastane práve vtedy, keď $a - b = 0$ alebo $c = 0$, teda práve vtedy, keď je splnená aspoň jedna z~podmienok $a = b$, $c = 0$.\\
\\
\kom Úloha demonštruje jeden zo základných spôsobov dokazovania nerovností: úpravu výrazu na jednej strane nerovnosti na tvar, o~ktorom s~určitosťou vieme, že je nezáporný/nekladný a jeho porovnanie s~nulou. Taktiež si študenti precvičia ekvivalentné úpravy nerovností a úpravy výrazov do tvaru súčinu.\\
\\
}


% Do not delete this line (pandoc magic!)

\problem{66-I-1-N1}{}{
Dokážte, že pre ľubovoľné reálne čísla $x$, $y$ a $z$ platia nerovnosti
\begin{enumerate}[a)]
\item $2xyz \leq x^2+ y^2z^2$,
\item $(x^2-y^2)^2\geq 4xy(x - y)^2$.
\end{enumerate}
}{
\rie a) Prevedieme výraz $2xyz$ na pravú stranu nerovnosti a upravíme pomocou vzorca $A^2-2AB-B^2=(A-B)^2$ na tvar $0 \leq (x - yz)^2$, ktorý je pravdivým výrokom, keďže druhá mocnina ľubovoľného výrazu je vždy nezáporná.\\

b) Výraz z~pravej strany nerovnosti prevedieme na opačnú stranu a upravíme nasledujúcim spôsobom:
\begin{align*}
((x-y)(x+y))^2-4xy(x-y)^2 &\geq 0,\\
(x-y)^2(x+y)^2-4xy(x-y)^2 &\geq 0,\\
(x-y)^2((x+y)^2-4xy) &\geq 0,\\
(x-y)^2(x+2xy+y^2-4xy) &\geq 0,\\
(x-y)^4 &\geq 0.
\end{align*}
Posledná nerovnosť je zrejme pravdivým tvrdením a pôvodná nerovnosť je tak dokázaná.\\
\\
\kom Úloha neprináša žiadny nový princíp, je však dobrým tréningom práce s~upravovaním výrazov, podobne ako úloha nasledujúca.\\
\\
}


% Do not delete this line (pandoc magic!)

\problem{66-I-1-N2}{seminar05,nerovnosti}{
Dokážte, že pre ľubovoľné kladné čísla $a$, $b$ platí nerovnosť $$\frac{a}{b^2}+ \frac{b}{a^2}\geq \frac{1}{a} + \frac{1}{b}.$$
}{
\rieh Nerovnosť zo zadania ekvivalentne upravíme. Vynásobíme celú nerovnosť kladným výrazom $a^2b^2$. Ľavú stranu $a^3+b^3$ upravíme na súčin pomocou vzorca $a^3+b^3=(a+b)(a^2-ab+b^2)$, pravú stranu $ab^2+a^2b$ upravíme na súčin vyňatím výrazu $ab$ na tvar $ab(a+b)$. Dostaneme tak nerovnosť $(a+b)(a^2-ab+b^2)\geq ab(a+b)$. Tá po vydelení kladným výrazom  $a + b$ a úprave na súčin dostane tvar $(a - b)^2\geq 0$, ktorý je zrejme pravdivým tvrdením. \\
\\
\kom Úloha využíva rovnaký princíp ako prechádzajúce dve. Prvýkrát však pri úprave využívame násobenie a delenie výrazmi. Tým sa z~úlohy stáva dobrá príležitosť na pripomenutie faktu, že pri úprave nerovností musíme brať do úvahy (ne)zápornosť výrazov, ktoré pri takýchto úkonoch využívame.\\
\\
\kom Ďalším z~užitočných nástrojov pri dokazovaní nerovností je znalosť nerovnosti $u+\frac{1}{u}\geq 2$ pre každé kladné reálne číslo $u$, pričom táto nerovnosť prechádza v~rovnosť len pre $u=1$. Dokázanie tohto faktu nie je zložité: vynásobením celej nerovnosti $u$, prevedením všetkých členov na jednu stranu dostávame $(u-1)^2\geq 0$, čo je pravdivé tvrdenie. Nasledujúce úlohy sú zaradené ako tréning uplatnenia tejto nerovnosti.\\
\\
}


\problem{62-I-2-N1}{
Dokážte, že pre ľubovoľné kladné čísla $a, b, c$ platí nerovnosť
$$\bigg(a +\frac{1}{b}\bigg)\bigg(b+\frac{1}{c}\bigg)\bigg(c+\frac{1}{a}\bigg)\geq 8$$
a zistite, kedy prechádza v~rovnosť.
}{
\rieh Ľavú stranu $L$ dokazovanej nerovnosti najskôr upravíme roznásobením a vzniknuté členy zoskupíme do súčtov dvojíc navzájom prevrátených výrazov:
\begin{multline*} L = \bigg(a +\frac{1}{b}\bigg)\bigg(b+\frac{1}{c}\bigg)\bigg(c+\frac{1}{a}\bigg) = \bigg(ab+ \frac{a}{c} + 1 +\frac{1}{bc}\bigg) \bigg(c +\frac{1}{a}\bigg)=\\ =\bigg( abc + \frac{1}{abc}\bigg)+\bigg( a+\frac{1}{a}\bigg)+ \bigg(b+\frac{1}{b}\bigg)+\bigg(c+\frac{1}{c}\bigg).\end{multline*}
Pretože pre $u > 0$ je $u+\frac{1}{u}\geq 2$, pričom rovnosť nastane práve vtedy, keď $u = 1$, pre výraz~$L$ platí $L \geq 2 + 2 + 2 + 2 = 8$, čo sme mali dokázať. Rovnosť $L = 8$ nastane práve vtedy, keď platí
$$ abc+\frac{1}{abc}=a+\frac{1}{a}=b+\frac{1}{b}=c+\frac{1}{c}=2$$
teda, ako sme už spomenuli, práve vtedy, keď $abc = a = b = c = 1$,  t.\,j. práve vtedy, keď $a = b = c = 1$.\\
\\
\kom Úloha sa dá riešiť využitím AG nerovnosti, tá však bude obsahom jedného z~ďalších seminárov, v~ktorom sa (okrem iného) k~tejto úlohe vrátime.\\
\\
}


% Do not delete this line (pandoc magic!)

\problem{66-I-1}{}{
Dokážte, že pre ľubovoľné reálne číslo a platí nerovnosť $$a^2+\frac{1}{a^2-a+1}\geq a+1.$$ Určte, kedy nastáva rovnosť.
}{
\rieh Úpravou dvojčlena $a^2 - a$ doplnením na štvorec a využitím faktu že druhá mocnina reálneho čísla je nezáporná ukážeme, že menovateľ zlomku v~nerovnosti je kladný:
$$a^2-a+1=\bigg(a^2-a+\frac{1}{4}\bigg) +\frac{3}{4}=\bigg(a-\frac{1}{2}\bigg)^2+\frac{3}{4}\geq \frac{3}{4}>0.$$
Ak ním teda obe strany dokazovanej nerovnosti vynásobíme, dostaneme ekvivalentnú nerovnosť
$$a^2(a^2-a+1)+1\geq (a+1)(a^2-a+1).$$
Po roznásobení a zlúčení rovnakých mocnín a dôjdeme k~nerovnosti
$$ a^4-2a^3+a^2\geq 0,$$
ktorá však platí, pretože jej ľavá strana má rozklad $a^2 (a - 1)^2$ s~nezápornými činiteľmi $a^2$ a $(a - 1)^2$. Tým je pôvodná nerovnosť pre každé reálne číslo a dokázaná. Zároveň sme zistili, že rovnosť vo výslednej, a teda aj v~pôvodnej nerovnosti nastane práve vtedy, keď platí $a^2 (a - 1)^2 = 0$, teda jedine vtedy, keď $a = 0$ alebo $a = 1$.\\
\\
\textbf{Iné riešenie*.} Danú nerovnosť môžeme prepísať na tvar $$ (a^2 - a + 1) + \frac{1}{a^2-a+1}\geq 2 \ \ \ \ \text{čiže} \ \ \ \  u~+\frac{1}{u}\geq 2,$$ pričom $u = a^2 -a + 1$. Využitím faktu, že posledná nerovnosť platí pre každé kladné
reálne číslo $u$ a že prechádza v~rovnosť jedine pre $u = 1$.

Na dôkaz pôvodnej nerovnosti ostáva už len overiť, že výraz $u = a^2 - a + 1$ je kladný pre každé reálne číslo $a$. To možno spraviť rovnako ako v~prvom riešení, alebo prepísať nerovnosť $a^2 - a + 1 > 0$ na tvar $$a(a -1) > -1$$ a uskutočniť krátku diskusiu: Posledná nerovnosť platí ako pre každé $a \geq 1$, tak pre každé $a\leq 0$, lebo v~oboch prípadoch máme dokonca $a(a - 1) \geq 0$; pre zvyšné hodnoty $a$, teda pre $a \in (0, 1)$, je súčin $a(a - 1)$ síce záporný, avšak určite väčší ako $-1$, pretože oba činitele $a$, $a - 1$ majú absolútnu hodnotu menšiu ako 1. Prepísaná nerovnosť je
tak dokázaná pre každé reálne číslo $a$, a tým je podmienka pre použitie nerovnosti $u + \frac{1}{u} \geq 2$ pre $u = a^2 + a + 1$ overená.

Ako sme už uviedli, rovnosť $u + \frac{1}{u} = 2$ nastane jedine pre $u = 1$. Pre rovnosť v~nerovnosti zo zadania úlohy tak dostávame podmienku $a^2 -a+1 = 1$, čiže $a(a-1)= 0$, čo je splnené iba pre $a = 0$ a pre $a = 1$.\\
\\
\kom Úloha využíva spojenie viacerých poznatkov -- faktu, že druhá mocnina akéhokoľvek reálneho čísla je nezáporná, úpravu na štvorec, ekvivalentné úpravy nerovností a tiež známu nerovnosť $u+\frac{1}{u} \geq 2$ pre každé kladné reálne $u$. Je síce náročnejšia ako úlohy, ktorými sme sa doteraz zaoberali, ale považujeme ju za vhodnú ilustráciu toho, ako nám rozšírený arzenál metód pomôže v~úspešnom zvládnutí zložitejších problémov. Úloha tiež demonštruje, že k~správnemu riešeniu častokrát vedú viaceré cesty.\\
\\
}


% Do not delete this line (pandoc magic!)

\problem{59-I-5}{seminar05,nertaz,domacekolo}{
Dokážte, že pre ľubovoľné kladné reálne čísla $a, b$ platí
$$ \sqrt{ab}\leq \frac{2(a^2+3ab+b^2)}{5(a+b)}\leq \frac{a+b}{2},$$
a pre každú z~oboch nerovností zistite, kedy prechádza na rovnosť.
}{
\rieh Pravá nerovnosť je ekvivalentná s~nerovnosťou
$$ 4(a^2 + 3ab + b^2 ) \leq 5(a + b)^2,$$
ktorú možno ekvivalentne upraviť na nerovnosť $a^2 + b^2 - 2ab = (a - b)^2 \geq 0$. Tá je splnená vždy a rovnosť v~nej nastane práve vtedy, keď $a = b$.

Z~ľavej nerovnosti odstránime zlomky a umocníme ju na druhú,
\begin{align*}
25ab(a^2 + 2ab + b^2) &\leq 4(a^4 + 9a^2 b^2 + b^4 + 6a^3 b + 6ab^3 + 2a^2 b^2),\\
25ab(a^2 + b^2 ) + 50a^2 b^2 &\leq 4a^4 + 4b^4 + 44a^2 b^2 + 24ab(a^2 + b^2 ),
\end{align*}
takže po úprave dostaneme ekvivalentnú nerovnosť
$$4a^4 + 4b^4 - 6a^2 b^2 \geq ab(a^2 + b^2 ).$$
Po odčítaní výrazu $2a^2 b^2$ od oboch strán nerovnosti sa nám podarí na oboch stranách použiť úpravu na štvorec. Dostaneme tak (opäť ekvivalentnú) nerovnosť $$ 4(a^2 - b^2 )^2 \geq ab(a - b)^2.$$
Rozdiel štvorcov v~zátvorke na ľavej strane ešte rozložíme na súčin a vzťah upravíme
na tvar $4(a - b)^2 (a + b)^2 \geq ab(a - b)^2$.

Ak $a = b$, platí rovnosť. Ak $a \neq b$, môžeme poslednú nerovnosť vydeliť kladným výrazom $(a - b)^2$ a dostaneme tak nerovnosť $4(a + b)^2 \geq ab$, čiže $4a^2 + 4b^2 + 7ab \geq 0$. Ľavá strana tejto nerovnosti je vždy kladná, preto vyšetrovaná nerovnosť platí pre všetky kladné čísla $a, b$, pričom rovnosť v~nej nastane práve vtedy, keď $a = b$.\\
\\
\kom Táto úloha prvýkrát prináša sústavu nerovností a je vhodné so študentmi zopakovať, ako k~dokazovaniu sústav nerovností pristupujeme: musíme dokázať riešenie každej nerovnosti zvlášť. V~priebehu riešenia opäť využijeme úpravu na štvorec a nezápornosť druhej mocniny reálneho čísla. Úloha sa dá riešiť ešte iným spôsobom, ten si však ukážeme v~ďalšom seminári zameranom na nerovnosti.
}




\subsection*{Domáca práca}

\problem{62-I-2-N1}{
Dokážte, že pre ľubovoľné kladné čísla $a, b, c$ platí nerovnosť
$$\bigg(a +\frac{1}{b}\bigg)\bigg(b+\frac{1}{c}\bigg)\bigg(c+\frac{1}{a}\bigg)\geq 8$$
a zistite, kedy prechádza v~rovnosť.
}{
\rieh Ľavú stranu $L$ dokazovanej nerovnosti najskôr upravíme roznásobením a vzniknuté členy zoskupíme do súčtov dvojíc navzájom prevrátených výrazov:
\begin{multline*} L = \bigg(a +\frac{1}{b}\bigg)\bigg(b+\frac{1}{c}\bigg)\bigg(c+\frac{1}{a}\bigg) = \bigg(ab+ \frac{a}{c} + 1 +\frac{1}{bc}\bigg) \bigg(c +\frac{1}{a}\bigg)=\\ =\bigg( abc + \frac{1}{abc}\bigg)+\bigg( a+\frac{1}{a}\bigg)+ \bigg(b+\frac{1}{b}\bigg)+\bigg(c+\frac{1}{c}\bigg).\end{multline*}
Pretože pre $u > 0$ je $u+\frac{1}{u}\geq 2$, pričom rovnosť nastane práve vtedy, keď $u = 1$, pre výraz~$L$ platí $L \geq 2 + 2 + 2 + 2 = 8$, čo sme mali dokázať. Rovnosť $L = 8$ nastane práve vtedy, keď platí
$$ abc+\frac{1}{abc}=a+\frac{1}{a}=b+\frac{1}{b}=c+\frac{1}{c}=2$$
teda, ako sme už spomenuli, práve vtedy, keď $abc = a = b = c = 1$,  t.\,j. práve vtedy, keď $a = b = c = 1$.\\
\\
\kom Úloha sa dá riešiť využitím AG nerovnosti, tá však bude obsahom jedného z~ďalších seminárov, v~ktorom sa (okrem iného) k~tejto úlohe vrátime.\\
\\
}


% Do not delete this line (pandoc magic!)

\problem{62-I-2-D2}{nerovnosti,domacekolo,doplnujuca}{
Dokážte, že pre ľubovoľné rôzne kladné čísla $a, b$ platí
$$ \frac{a+b}{2}<\frac{2(a^2 +ab+b^2)}{3(a+b)}<\sqrt{\frac{a^2+b^2}{2}}.$$
}{
\rieh 58-C-I-6
}

% Do not delete this line (pandoc magic!)

\problem{66-I-1-D3, resp. 58-I-6}{seminar05,nertaz,domacekolo}{
Dokážte, že pre ľubovoľné rôzne kladné čísla $a, b$ platí
$$\frac{a+b}{2}<\frac{2(a^2 + ab + b^2 )}{3(a+b)}<\sqrt{\frac{a^2+b^2}{2}}.$$
}{
\rieh Ľavú nerovnosť dokážeme ekvivalentnými úpravami:
\begin{align*}
\frac{a+b}{2}&<\frac{2(a^2 + ab + b^2 )}{3(a+b)}, \ \ | \cdot 6(a+b)\\
3(a+b)^2&<4(a^2+ab+b^2),\\
0&<(a-b)^2.
\end{align*}
Posledná nerovnosť vzhľadom na predpoklad $a\neq b$ platí. Aj pravú nerovnosť zo zadania budeme ekvivalentne upravovať, začneme umocnením každej strany na druhú:
\begin{align*}
\frac{4(a^2 + ab + b^2 )^2}{9(a + b)^2}&<\frac{a^2 + b^2}{2}, \ \ | \cdot 18(a + b)^2\\
8(a^2 + ab + b^2 )^2 &< 9(a^2 + b^2 )(a + b)^2,\\
8(a^4 + b^4 + 2a^3 b + 2ab^3 + 3a^2 b^2 ) &< 9(a^4 + b^4 + 2a^3 b + 2ab^3 + 2a^2 b^2 ),\\
6a^2 b^2 &< a^4 + b^4 + 2a^3 b + 2ab^3.
\end{align*}
Posledná nerovnosť je súčtom nerovností $2a^2 b^2 < a^4 + b^4$ a $4a^2 b^2 < 2a^3 b + 2ab^3$, ktoré obe platia, lebo po presune členov z~ľavých strán na pravé dostaneme po rozklade už zrejmé nerovnosti $0 < (a^2- b^2)^2$, resp. $0 < 2ab(a - b)^2$.
}



\teachernote{
\subsection*{Doplňujúce zdroje a materiály}
Publikácií a článkov zaoberajúcich sa dokazovaním nerovností existuje veľké množstvo. Ak by študenti mali záujem o~širšie štúdium tejto problematiky, na úvod je vhodné odporučiť im napr. publikácie~\cite{bocek1994} alebo [YY]TODO.
}

\section*{Seminár 5}

\weblinks{Na stiahnutie: \href{pdf/seminar05-teacher.pdf}{učiteľská verzia}, \href{pdf/seminar05-student.pdf}{študentská verzia}}

\subsection*{Téma}
Algebraické výrazy, rovnice a nerovnosti III -- lineárne rovnice a sústavy lineárnych rovníc

\subsection*{Ciele}
Upevniť poznatky o~riešení lineárnych rovníc a sústav lineárnych rovníc.

\subsection*{Úlohy a riešenia}

\textbf{Úvodný komentár.} Seminárne stretnutie je zamerané na prácu s~rovnicami a úlohami, ktoré na rovnice, príp. sústavy rovníc vedú. Ide o~jedno z~dvoch stretnutí (ďalším je Seminár 19), v~tejto chvíli sa zameriame na jednoduchšie (priamočiarejšie) úlohy, v~pokračovaní sa potom budeme zaoberať zložitejšími rovnicami a sústavami.\\
\\
\begin{tcolorbox}[breakable,notitle,boxrule=0pt,colback=light-gray,colframe=light-gray]\ul{5.1} [60-I-1-N1] Máme tri čísla so súčtom 2010, pričom každé z~nich je aritmetickým priemerom zvyšných dvoch. Aké sú to čísla?

\end{tcolorbox}

\rie Označme hľadané čísla $a,b$ a $c$. Podľa zadania platí
\begin{align*}
\frac{a+b}{2} &=c,\\
\frac{b+c}{2} &=a,\\
\frac{c+a}{2} &=b.
\end{align*}
Riešením sústavy dostávame $a=b=c$ a z~podmienky $a+b+c=2010$ potom jediné riešenie úlohy $a=b=c=670$.\\
\\
\begin{tcolorbox}[breakable,notitle,boxrule=0pt,colback=light-gray,colframe=light-gray]\ul{5.2} [60-I-1-N2] Máme tri čísla, o~ktorých vieme, že každé z~nich je aritmetickým priemerom niektorých dvoch z~našich troch čísel. Dokážte, že naše tri čísla sú rovnaké.

\end{tcolorbox}

\rieh Predpokladajme, že niektoré z~našich čísel je priemerom seba a iného z~našich čísel. Potom ich vieme označiť $a, b, c$ tak, že $a = (a + b)/2$. Z~tejto rovnosti vyplýva $a = b$. Číslo $c$ je buď priemerom čísel $a$ a $b$, z~čoho hneď máme, že je týmto číslam rovné, alebo je priemerom seba a niektorého z~čísel $a, b$, čiže $c = (c + a)/2$, z~toho opäť dostaneme $c = a = b$. Ak každé z~našich čísel je aritmetickým priemerom zvyšných dvoch, riešime predošlú úlohu.\\
\\
\begin{tcolorbox}[breakable,notitle,boxrule=0pt,colback=light-gray,colframe=light-gray]\ul{5.3} [60-I-1]
Lucia napísala na tabuľu dve nenulové čísla. Potom medzi ne postupne vkladala znamienka plus, mínus, krát a delené a všetky štyri príklady správne vypočítala. Medzi výsledkami boli iba dve rôzne hodnoty. Aké dve čísla mohla Lucia na tabuľu napísať?

\end{tcolorbox}

\rieh Označme hľadané čísla $a, b$. Keďže $b \neq 0$ nutne $a + b \neq a - b$. Každé z~čísel $a \cdot b, a : b$ je rovné buď $a + b$, alebo $a - b$. Stačí teda rozobrať štyri prípady a v~každom z~nich vyriešiť sústavu rovníc. Ukážeme si však rýchlejší postup.
Ak by platilo
$$ a + b = a \cdot b \ \mathrm{a} \ a - b = a : b \ \ \  \mathrm{alebo} \ \ \  a + b = a : b \ \mathrm{a} \ a - b = a \cdot b,$$
vynásobením rovností by sme v~oboch prípadoch dostali $a^2 - b^2 = a^2$ , čo je v~spore s~$b\neq 0$. Preto sú čísla $a \cdot b$ a $a : b$ buď obe rovné $a + b$ alebo obe rovné $a - b$. Tak či tak musí platiť $a \cdot b = a : b$, odkiaľ po úprave $a(b^2 - 1) = 0$. Keďže $a\neq 0$, nutne $b \in \{ 1, - 1 \}$. Ale ak $b = 1$, tak štyri výsledky sú postupne $a + 1$, $a - 1$, $a$, $a$, čo sú pre každé $a$ až tri rôzne hodnoty. Pre $b = - 1$ máme výsledky $a - 1$, $a + 1$, $- a$, $- a$. Dva rôzne výsledky to budú práve vtedy, keď $a - 1 = - a$ alebo $a + 1 = - a$. V~prvom prípade dostávame $a =\frac{1}{2}$, v~druhom $a = - \frac{1}{2}$.
Lucia mohla na začiatku na tabuľu napísať buď čísla $\frac{1}{2}$ a $- 1$, alebo čísla $-\frac{1}{2}$ a $- 1$.\\
\\
\kom Ak sa študenti rozhodnú riešiť sústavy rovníc pre spomínané štyri prípady, je to v~poriadku, na záver by však bolo inšpiratívne ukázať im aj rýchlejší postup.\\
\\
\begin{tcolorbox}[breakable,notitle,boxrule=0pt,colback=light-gray,colframe=light-gray]\ul{5.4} [60-II-1]
Na tabuli sú napísané práve tri (nie nutne rôzne) reálne čísla. Vieme, že súčet ľubovoľných dvoch z~nich je tam napísaný tiež. Určte všetky trojice takých čísel.

\end{tcolorbox}

\rieh Označme čísla napísané na tabuli $a, b, c$. Súčet $a + b$ sa tiež nachádza na tabuli, je teda rovný jednému z~čísel $a, b, c$. Keby $a + b$ bolo rovné $a$ alebo $b$, bola by na tabuli aspoň jedna nula. Rozoberieme preto tri prípady podľa počtu núl napísaných na tabuli.

Ak sú na tabuli aspoň dve nuly, ľahko sa presvedčíme, že súčet každých dvoch čísel z~tabule je tam tiež. Dostávame, že trojica $t, 0, 0$ je pre ľubovoľné reálne číslo $t$ riešením úlohy.

Ak je na tabuli práve jedna nula, je tam trojica $a, b, 0,$ pričom $a$ aj $b$ sú nenulové čísla. Súčet $a + b$ teda nie je rovný ani $a$, ani $b$, musí preto byť rovný 0. Dostávame tak ďalšiu trojicu $t, -t, 0$, ktorá je riešením úlohy pre ľubovoľné reálne číslo $t$.
Ak na tabuli nie je ani jedna nula, súčet $a + b$ nie je rovný ani $a$, ani $b$, preto $a + b = c$. Z~rovnakých dôvodov je $b + c = a$ a $c + a = b$. Dostali sme sústavu troch lineárnych rovníc s~neznámymi $a, b, c$, ktorú môžeme vyriešiť. Avšak hneď z~prvých dvoch rovníc po dosadení vyjde $b + (a + b) = a$, čiže $b = 0$. To je v~spore s~tým, že na tabuli žiadna nula nie je.

\textit{Záver.} Úlohe vyhovujú trojice $t, 0, 0$ a $t, -t, 0$ pre ľubovoľné reálne číslo $t$ a žiadne iné.\\
\\
\begin{tcolorbox}[breakable,notitle,boxrule=0pt,colback=light-gray,colframe=light-gray]\ul{5.5} [60-S-1]
Po okruhu behajú dvaja atléti, každý inou konštantnou rýchlosťou. Keď bežia opačnými smermi, stretávajú sa každých 10 minút, keď bežia rovnakým smerom, stretávajú sa každých 40 minút. Za aký čas zabehne okruh rýchlejší atlét?

\end{tcolorbox}

\rieh Označme rýchlosti bežcov $v_1$ a $v_2$ tak, že $v_1 > v_2$ (rýchlosti udávame v~okruhoch za minútu). Predstavme si, že atléti vyštartujú z~rovnakého miesta, ale opačným smerom. V~okamihu ich ďalšieho stretnutia po 10 minútach bude súčet dĺžok oboch prebehnutých úsekov zodpovedať presne dĺžke jedného okruhu, teda
$10v_1 +10v_2= 1$.
Ak bežia atléti z~rovnakého miesta rovnakým smerom, dôjde k~ďalšiemu stretnutiu, akonáhle rýchlejší atlét zabehne o~jeden okruh viac ako pomalší. Preto $40v_1 -40v_2 = 1$.
Dostali sme sústavu dvoch lineárnych rovníc s~neznámymi $v_1 , v_2$:
\begin{align*}
10v_1 + 10v_2 &= 1,\\
40v_1 - 40v_2 &= 1,
\end{align*}
ktorú vyriešime napríklad tak, že k~štvornásobku prvej rovnice pripočítame druhú, čím dostaneme $80v_1 = 5$, čiže $v_1 =\frac{1}{16}$. Zaujíma nás, ako dlho trvá rýchlejšiemu bežcovi prebehnúť jeden okruh, teda hodnota podielu $1/v_1$ . Po dosadení vypočítanej hodnoty $v_1$ dostaneme odpoveď: 16 minút.\\
\\
\textit{Poznámka.} Úlohu možno riešiť aj úvahou: za 40 minút ubehnú atléti spolu 4 okruhy (to vyplýva z~prvej podmienky), pritom rýchlejší o~1 okruh viac ako pomalší (to vyplýva z~druhej podmienky). To teda znamená, že prvý za uvedenú dobu ubehne 2,5 okruhu a druhý 1,5 okruhu, takže rýchlejší ubehne jeden okruh za 40/2,5 = 16 minút.\\
\\
\kom V~tomto prípade vyžaduje netriviálne úsilie správne zostavenie sústavy rovníc tak, aby skutočne zodpovedala zadaniu. Jej vyriešenie potom už zložité nie je. Za zmienku stojí, že úloha je vhodným príkladom situácie, v~ktorej si zmysluplnosť výsledku môžeme aspoň približne overiť (záporné rýchlosti, rýchlosti väčšie ako rýchlosť svetla).\\
\\
\begin{tcolorbox}[breakable,notitle,boxrule=0pt,colback=light-gray,colframe=light-gray]\ul{5.6} [66-I-4-N1] Určte všetky dvojčleny $P (x) = ax + b$, pre ktoré platí $P(2) = 3$ a $P (3) = 2$.

\end{tcolorbox}

\rie Z~podmienok zo zadania zostavíme dve rovnice s~dvomi neznámymi $a$ a $b$:
\begin{align*}
P (2) &= 2a + b = 3,\\
P (3) &= 3a + b = 2.
\end{align*}
Odčítaním prvej rovnice od druhej ihneď dostávame $a = -1$, dosadením tejto hodnoty do jednej z~podmienok potom máme $b= 5$. Sústava má práve jedno riešenie, a preto zadaniu vyhovuje jediný dvojčlen $P(x)=-x+5$.\\
\\
\begin{tcolorbox}[breakable,notitle,boxrule=0pt,colback=light-gray,colframe=light-gray]\ul{5.7} [66-I-4-N2] Určte všetky trojčleny $P (x) = ax^2+ bx + c$, pre ktoré platí $P (1) = 4$, $P (2) = 9$ a $P (3) = 18$.

\end{tcolorbox}

\rie Podobne, ako v~predchádzajúcej úlohe, zostavíme z~podmienok sústavu troch lineárnych rovníc s~tromi neznámymi $a$, $b$ a $c$:
\begin{align*}
P(1) &= a + b + c = 4, \\
P(2) &= 4a + 2b + c = 9, \\
P(3) &= 9a + 3b + c = 18.
\end{align*}
Sústava má opäť jediné riešenie $a = 2, b = -1, c = 3$, a preto existuje práve jeden trojčlen vyhovujúci zadaniu: $P(x)=2x^2-x+3$.\\
\\
\kom Predchádzajúce dve jednoduchšie úlohy majú prípravný charakter na nasledujúcu úlo\-hu a domácu prácu. Študenti si prostredníctvom nich zopakujú metódy riešenia sústav rovníc s~viacerými neznámymi. Tieto metódy by študentom mali byť známe zo ZŠ, ak však zistíme, že ich používanie nie je až také samozrejmé, je vhodné zaradiť niekoľko jednoduchších úloh, napr. z~\cite{kubat2000}.\\

\begin{tcolorbox}[breakable,notitle,boxrule=0pt,colback=light-gray,colframe=light-gray]\ul{5.8} [66-I-4-N3] Určte všetky dvojčleny $P (x) = ax+b$ s~celočíselnými koeficientmi $a$ a $b$, pre ktoré platí $P (1) < P (2)$ a $P (1)^2+ P(2)^2= 5$.

\end{tcolorbox}

\rieh Keďže $a$ a $b$ sú podľa zadania celé čísla, budú celými číslami aj hodnoty $P(1)$ a $P(2)$. Preto hľadáme, akými spôsobmi sa dá číslo 5 zapísať ako súčet dvoch druhých mocnín celých čísel. Ak neberieme ohľad na poradie sčítancov, je taký spôsob jediný: $5 = (\pm 1)^2+ (\pm 2)^2$. Zároveň vieme, že $P(1)<P(2)$, preto dvojicu $(P(1), P(2))$ tvoria niektoré z~nasledujúcich štyroch možností: $(1, 2), (-1, 2), (-2, -1), (-2, 1)$. Každá z~týchto štyroch dvojíc podmienok potom vedie k~sústave dvoch rovníc s~dvomi neznámymi, takže dostávame objemnejšiu variáciu prvej úlohy tohto seminára. Vyriešením systémov získame 4 vyhovujúce dvojčleny $x + 0$, $3x - 4$, $x - 3$ a $3x - 5$.\\
\\
\kom Úloha využíva takmer rovnaký princíp ako prvé dve seminárne úlohy, vyžaduje však dodatočnú analýzu plynúcu z~poslednej podmienky, čo úlohe pridáva na náročnosti. Poslednou úlohou tejto gradovanej série je domáca práca, ktorej analýza povedie k~riešeniu niekoľkých sústav troch rovníc s~tromi neznámymi.\\
\\
\begin{tcolorbox}[breakable,notitle,boxrule=0pt,colback=light-gray,colframe=light-gray]\ul{5.9} [66-I-4-D2] Koeficienty $a, b, c$ trojčlena $P (x) = ax^2+ bx + c$ sú reálne čísla, pritom každá z~troch jeho hodnôt $P (1), P (2)$ a $P (3)$ je celým číslom. Vyplýva z~toho, že aj čísla $a, b, c$ sú celé, alebo je nutne celé aspoň niektoré z~nich (ktoré)?

\end{tcolorbox}

\rieh Nevyplýva. Uvážme príklad trojčlena $P (x) =\frac{1}{2}x^2+\frac{1}{2}x+1$: z~vyjadrenia $P (x) =\frac{1}{2}x(x + 1) + 1$ vyplýva, že $P (x)$ je celým číslom pre každé celé $x$, pretože súčin $x(x + 1)$ je vtedy deliteľný dvoma. Vo všeobecnej situácii je iba koeficient $c$ nutne celé číslo; vyplýva to z~vyjadrenia $c = P (0) = 3P (1) - 3P (2) + P (3)$.\\
\\
\kom Úloha je zaujímavá tým, že cesta k~riešeniu je tentoraz menej priamočiara a študenti pravdepodobne prídu na viac rôznych príkladov mnohočlenov s~neceločíselnými koeficientami, ktoré dané podmienky spĺňajú. Zaujímavá bude tiež pravdepodobne diskusia nad zdôvodnením, ktoré z~koeficientov nutne celočíselné byť musia.\\
\\
\begin{tcolorbox}[breakable,notitle,boxrule=0pt,colback=light-gray,colframe=light-gray]\ul{5.10} [59-S-3] Nájdite všetky dvojice nezáporných celých čísel $a$, $b$, pre ktoré platí
$$a^2 + b + 2 = a + b^2.$$

\end{tcolorbox}

\rieh Rovnicu prepíšeme na tvar $2 = (b^2 -a^2 )-(b-a)$, z~ktorého po využití vzťahu pre rozdiel štvorcov a následnom vyňatí výrazu $b - a$ dostaneme $2 = (b - a)(a + b - 1)$.
Keďže 2 je prvočíslo, máme pre uvedený súčin nasledujúce štyri možnosti:

\begin{enumerate}[a)]
\item $b - a = 1$ a $a + b - 1 = 2$, potom $a = 1$ a $b = 2.$
\item  $b - a = 2$ a $a + b - 1 = 1$, potom $a = 0$ a $b = 2$.
\item $b - a = -1$ a $a + b - 1 = -2$. Druhú rovnicu možno prepísať na tvar $a + b = -1$, z~ktorého vidíme, že rovnosť nenastane pre žiadnu dvojicu nezáporných celých čísel.
\item  $b - a = -2$ a $a + b - 1 = -1$. Druhú rovnicu možno prepísať na tvar $a + b = 0$, z~ktorého vidíme, že vyhovuje jediná dvojica nezáporných celých čísel $a = b = 0$, ktorá však nevyhovuje prvej rovnici.
\end{enumerate}

\textit{Záver.} Úloha má dve riešenia: Buď $a = 1$ a $b = 2$, alebo $a = 0$ a $b = 2$.

\textit{Poznámka.} Namiesto rozboru štyroch možností môžeme začať úvahou, že nulové čísla $a, b$ nie sú riešením úlohy, takže $a + b - 1 = 0$, a teda aj $b - a = 0$. Stačí teda uvažovať iba možnosti a) a b).\\
\\
\textbf{Iné riešenie.} Rovnicu upravíme na tvar $2 = (b^2 - b) - (a^2 - a)$, resp. na tvar $2 = b(b - 1)-a(a-1)$. Z~nasledujúcej tabuľky a tvaru čísel $x^2 -x = x(x-1)$ je zrejmé, že rozdiely medzi susednými hodnotami výrazov $x(x - 1)$ rastú s~rastúcim $x$ (ľahko sa o~tom presvedčíme výpočtom: $(x + 1)x - x(x - 1) = 2x$).
\begin{center}
\begin{tabular}{|c|c|c|c|c|c|c|c|}
\hline
$x$ & 0 & 1 & 2 & 3 & 4 & 5 & \ldots \\
\hline
$x(x-1)$ & 0 & 0 & 2 & 6 & 12 & 20 & \ldots\\
\hline
\end{tabular}
\end{center}
Môže teda platiť iba $b^2 - b = 2$ a $a^2 - a = 0$. Odtiaľ $a \in \{0, 1\}$ a $b = 2$. Riešením úlohy sú teda dve dvojice nezáporných celých čísel: $a = 0, b = 2$ a $a = 1, b = 2$.\\
\\
\kom Úloha je (okrem iného) zaujímavá tým, že poskytuje priestor na rôznorodé prístupy k~riešeniu a môže byť dobrým podnetom na vzájomné vysvetľovanie riešení medzi študentmi. Kľúčovým prvkom riešenia je úprava rovnice na vhodný tvar -- na tomto mieste je študentom vhodné pripomenúť, že zručnosť a dôvtip pri manipulácii s~algebraickými výrazmi nájdu uplatenie v~širokom spektre problémov, nielen na prvom seminárnom stretnutí, ktoré na túto problematiku bolo zamerané.

\subsection*{Domáca práca}
\begin{tcolorbox}[breakable,notitle,boxrule=0pt,colback=light-gray,colframe=light-gray]\ul{5.11} [66-I-4]
Nájdite všetky trojčleny $P(x)=ax^2+bx+c$ s~celočíselnými koeficientami $a, b, c$, pre ktoré platí $P(1) < P(2) < P(3)$ a zároveň $$(P(1))^2+ (P(2))^2+ (P(3))^2= 22.$$
\end{tcolorbox}

\rie Keďže $a, b, c$ sú podľa zadania celé čísla, sú také aj hodnoty $P(1), P(2)$ a $P(3)$. Ich druhé mocniny, čiže čísla  $P(1)^2 , P(2)^2$ a $P(3)^2$, sú preto druhými mocninami
celých čísel, teda tri (nie nutne rôzne) čísla z~množiny $\{0, 1, 4, 9, 16, 25, . . .\}$. Ich súčet je podľa zadania rovný 22, takže každý z~troch sčítancov je menší ako šieste možné číslo 25. Akými spôsobmi možno vôbec zostaviť súčet 22 z~troch čísel vybraných z~množiny  $\{0, 1, 4, 9, 16\}$?
Systematickým rozborom rýchlo zistíme, že rozklad čísla 22 na súčet troch druhých mocnín je (až na poradie sčítancov) iba jeden, a to $22 = 4+9+9$. Dve z~čísel $P(1), P(2)$ a $P(3)$ majú teda absolútnu hodnotu 3 a tretie 2, a keďže
$P(1) < P(2) < P(3)$, musí nutne platiť $P(1) = -3$, $P(3) = 3$ a $P(2) \in \{-2, 2\}$. Pre každú z~oboch vyhovujúcich
trojíc $(P(1), P(2), P(3)) = (-3, -2, 3)$ a $(P(1), P(2), P(3)) = (-3, 2, 3)$ určíme koeficienty $a, b, c$ príslušného trojčlena $P(x)$ tak, že nájdené hodnoty dosadíme do pravých strán rovníc
\begin{align*}
a + b + c &= P(1),\\
4a + 2b + c &= P(2),\\
9a + 3b + c &= P(3)
\end{align*}
a výslednú sústavu troch rovníc s~neznámymi $a, b, c$ vyriešime. Tento jednoduchý výpočet tu vynecháme, v~oboch prípadoch vyjdú celočíselné trojice $(a, b, c)$, ktoré zapíšeme rovno ako koeficienty trojčlenov, ktoré sú jedinými dvoma riešeniami danej úlohy:
$$P_1(x) = 2x^2 -5x \ \ \ \  \text{a} \ \ \ \ P_2 (x) = -2x^2+ 11x -12.$$
\begin{tcolorbox}[breakable,notitle,boxrule=0pt,colback=light-gray,colframe=light-gray]\ul{5.12} [62-II-3]
Nájdite všetky dvojice celých kladných čísel $a$ a $b$, pre ktoré je číslo $a^2 +b$ o~62 väčšie
ako číslo $b^2 + a$.

\end{tcolorbox}

\rie  Zadanie zapíšeme rovnosťou, ktorej pravú stranu rovno upravíme na súčin: $$62 = (a^2+ b) - (b^2
+ a) = (a^2 - b^2)- (a - b) = (a - b)(a + b - 1).$$
Súčin celých čísel $u = a - b$ a $v = a + b - 1$ je teda rovný súčinu dvoch prvočísel $2 \cdot 31$.
Keďže $v \geq 1 + 1 - 1 = 1$, je nutne aj číslo $u$ kladné a zrejme $u < v$, takže $(u, v)$ je jedna
z~dvojíc $(1, 62)$ alebo $(2, 31)$. Ak vyjadríme naopak $a, b$ pomocou $u, v$, dostaneme $$a =\frac{u+v+1}{2} \ \ \ \  \textrm{a} \ \ \ \  b=\frac{v-u+1}{2}.$$ Pre $(u, v) = (1, 62)$ tak dostávame riešenie $(a, b) = (32, 31)$, dvojici $(u, v) = (2, 31)$ zodpovedá druhé riešenie $(a, b) = (17, 15)$. Iné riešenia úloha nemá.\\
\\
\kom Úloha je veľmi podobná tej, ktorou sme sa zaoberali na stretnutí, slúži tak na overenie toho, či si študenti princíp riešenia osvojili. Zároveň ale zadanie nie je zapísané priamo rovnosťou, takže úloha precvičí aj schopnosť transformovať slovný text na matematický zápis.\\
\\
\begin{tcolorbox}[breakable,notitle,boxrule=0pt,colback=light-gray,colframe=light-gray]\ul{5.13} [60-I-1-D1] Nech $n$ je prirodzené číslo väčšie ako 2. Máme $n$ čísel so súčtom $n$, pričom každé z~nich je aritmetickým priemerom ostatných čísel. Aké sú to čísla?

\end{tcolorbox}

\rieh Usporiadajme si naše čísla podľa veľkosti, nech $x_1 \leq x_2 \leq \ldots \leq x_n$. Aritmetický priemer skupiny čísel je aspoň taký, ako najmenšie z~nich. Aritmetický priemer čísel $x_2, x_3,\ldots , x_n$ je preto aspoň $x_2$, a je rovný $x_1$ len v~prípade, že žiadne z~čísel $x_3, \ldots , x_n$ nie je väčšie ako $x_2$. Z~toho hneď dostávame, že všetky naše čísla musia byť rovnaké a teda rovné 1.


\section{November}
\seminar{6}{Algebraické výrazy, rovnice a nerovnosti III -- lineárne rovnice a sústavy lineárnych rovníc}

\teachernote{
\subsection*{Ciele}
Upevniť poznatky o~riešení lineárnych rovníc a sústav lineárnych rovníc.

\subsubsection*{Úvodný komentár}

SSeminárne stretnutie je zamerané na prácu s~rovnicami a úlohami, ktoré na rovnice, príp. sústavy rovníc vedú. Ide o~jedno z~dvoch stretnutí (ďalším je Seminár 19), v~tejto chvíli sa zameriame na jednoduchšie (priamočiarejšie) úlohy, v~pokračovaní sa potom budeme zaoberať zložitejšími rovnicami a sústavami.

}
\subsection*{Úlohy a riešenia}



% Do not delete this line (pandoc magic!)

\problem{60-I-1-N1}{seminar06,rovnice,sustavy}{
Máme tri čísla so súčtom 2010, pričom každé z~nich je aritmetickým priemerom zvyšných dvoch. Aké sú to čísla?
}{
\rie Označme hľadané čísla $a,b$ a $c$. Podľa zadania platí
\begin{align*}
\frac{a+b}{2} &=c,\\
\frac{b+c}{2} &=a,\\
\frac{c+a}{2} &=b.
\end{align*}
Riešením sústavy dostávame $a=b=c$ a z~podmienky $a+b+c=2010$ potom jediné riešenie úlohy $a=b=c=670$.\\

}


% Do not delete this line (pandoc magic!)

\problem{60-I-1-N2}{}{
Máme tri čísla, o~ktorých vieme, že každé z~nich je aritmetickým priemerom niektorých dvoch z~našich troch čísel. Dokážte, že naše tri čísla sú rovnaké.
}{
\rieh Predpokladajme, že niektoré z~našich čísel je priemerom seba a iného z~našich čísel. Potom ich vieme označiť $a, b, c$ tak, že $a = (a + b)/2$. Z~tejto rovnosti vyplýva $a = b$. Číslo $c$ je buď priemerom čísel $a$ a $b$, z~čoho hneď máme, že je týmto číslam rovné, alebo je priemerom seba a niektorého z~čísel $a, b$, čiže $c = (c + a)/2$, z~toho opäť dostaneme $c = a = b$. Ak každé z~našich čísel je aritmetickým priemerom zvyšných dvoch, riešime predošlú úlohu.\\
\\
}


\problem{60-I-1}{
Lucia napísala na tabuľu dve nenulové čísla. Potom medzi ne postupne vkladala znamienka plus, mínus, krát a delené a všetky štyri príklady správne vypočítala. Medzi výsledkami boli iba dve rôzne hodnoty. Aké dve čísla mohla Lucia na tabuľu napísať?
}{
\rieh Označme hľadané čísla $a, b$. Keďže $b \neq 0$ nutne $a + b \neq a - b$. Každé z~čísel $a \cdot b, a : b$ je rovné buď $a + b$, alebo $a - b$. Stačí teda rozobrať štyri prípady a v~každom z~nich vyriešiť sústavu rovníc. Ukážeme si však rýchlejší postup.
Ak by platilo
$$ a + b = a \cdot b \ \mathrm{a} \ a - b = a : b \ \ \  \mathrm{alebo} \ \ \  a + b = a : b \ \mathrm{a} \ a - b = a \cdot b,$$
vynásobením rovností by sme v~oboch prípadoch dostali $a^2 - b^2 = a^2$ , čo je v~spore s~$b\neq 0$. Preto sú čísla $a \cdot b$ a $a : b$ buď obe rovné $a + b$ alebo obe rovné $a - b$. Tak či tak musí platiť $a \cdot b = a : b$, odkiaľ po úprave $a(b^2 - 1) = 0$. Keďže $a\neq 0$, nutne $b \in \{ 1, - 1 \}$. Ale ak $b = 1$, tak štyri výsledky sú postupne $a + 1$, $a - 1$, $a$, $a$, čo sú pre každé $a$ až tri rôzne hodnoty. Pre $b = - 1$ máme výsledky $a - 1$, $a + 1$, $- a$, $- a$. Dva rôzne výsledky to budú práve vtedy, keď $a - 1 = - a$ alebo $a + 1 = - a$. V~prvom prípade dostávame $a =\frac{1}{2}$, v~druhom $a = - \frac{1}{2}$.
Lucia mohla na začiatku na tabuľu napísať buď čísla $\frac{1}{2}$ a $- 1$, alebo čísla $-\frac{1}{2}$ a $- 1$.\\
\\
\kom Ak sa študenti rozhodnú riešiť sústavy rovníc pre spomínané štyri prípady, je to v~poriadku, na záver by však bolo inšpiratívne ukázať im aj rýchlejší postup.\\
\\
}


\problem{60-II-1}{
Na tabuli sú napísané práve tri (nie nutne rôzne) reálne čísla. Vieme, že súčet ľubovoľných dvoch z~nich je tam napísaný tiež. Určte všetky trojice takých čísel.
}{
\rieh Označme čísla napísané na tabuli $a, b, c$. Súčet $a + b$ sa tiež nachádza na tabuli, je teda rovný jednému z~čísel $a, b, c$. Keby $a + b$ bolo rovné $a$ alebo $b$, bola by na tabuli aspoň jedna nula. Rozoberieme preto tri prípady podľa počtu núl napísaných na tabuli.

Ak sú na tabuli aspoň dve nuly, ľahko sa presvedčíme, že súčet každých dvoch čísel z~tabule je tam tiež. Dostávame, že trojica $t, 0, 0$ je pre ľubovoľné reálne číslo $t$ riešením úlohy.

Ak je na tabuli práve jedna nula, je tam trojica $a, b, 0,$ pričom $a$ aj $b$ sú nenulové čísla. Súčet $a + b$ teda nie je rovný ani $a$, ani $b$, musí preto byť rovný 0. Dostávame tak ďalšiu trojicu $t, -t, 0$, ktorá je riešením úlohy pre ľubovoľné reálne číslo $t$.
Ak na tabuli nie je ani jedna nula, súčet $a + b$ nie je rovný ani $a$, ani $b$, preto $a + b = c$. Z~rovnakých dôvodov je $b + c = a$ a $c + a = b$. Dostali sme sústavu troch lineárnych rovníc s~neznámymi $a, b, c$, ktorú môžeme vyriešiť. Avšak hneď z~prvých dvoch rovníc po dosadení vyjde $b + (a + b) = a$, čiže $b = 0$. To je v~spore s~tým, že na tabuli žiadna nula nie je.

\textit{Záver.} Úlohe vyhovujú trojice $t, 0, 0$ a $t, -t, 0$ pre ľubovoľné reálne číslo $t$ a žiadne iné.\\
\\
}


\problem{60-S-1}{
Po okruhu behajú dvaja atléti, každý inou konštantnou rýchlosťou. Keď bežia opačnými smermi, stretávajú sa každých 10 minút, keď bežia rovnakým smerom, stretávajú sa každých 40 minút. Za aký čas zabehne okruh rýchlejší atlét?
}{
\rieh Označme rýchlosti bežcov $v_1$ a $v_2$ tak, že $v_1 > v_2$ (rýchlosti udávame v~okruhoch za minútu). Predstavme si, že atléti vyštartujú z~rovnakého miesta, ale opačným smerom. V~okamihu ich ďalšieho stretnutia po 10 minútach bude súčet dĺžok oboch prebehnutých úsekov zodpovedať presne dĺžke jedného okruhu, teda
$10v_1 +10v_2= 1$.
Ak bežia atléti z~rovnakého miesta rovnakým smerom, dôjde k~ďalšiemu stretnutiu, akonáhle rýchlejší atlét zabehne o~jeden okruh viac ako pomalší. Preto $40v_1 -40v_2 = 1$.
Dostali sme sústavu dvoch lineárnych rovníc s~neznámymi $v_1 , v_2$:
\begin{align*}
10v_1 + 10v_2 &= 1,\\
40v_1 - 40v_2 &= 1,
\end{align*}
ktorú vyriešime napríklad tak, že k~štvornásobku prvej rovnice pripočítame druhú, čím dostaneme $80v_1 = 5$, čiže $v_1 =\frac{1}{16}$. Zaujíma nás, ako dlho trvá rýchlejšiemu bežcovi prebehnúť jeden okruh, teda hodnota podielu $1/v_1$ . Po dosadení vypočítanej hodnoty $v_1$ dostaneme odpoveď: 16 minút.\\
\\
\textit{Poznámka.} Úlohu možno riešiť aj úvahou: za 40 minút ubehnú atléti spolu 4 okruhy (to vyplýva z~prvej podmienky), pritom rýchlejší o~1 okruh viac ako pomalší (to vyplýva z~druhej podmienky). To teda znamená, že prvý za uvedenú dobu ubehne 2,5 okruhu a druhý 1,5 okruhu, takže rýchlejší ubehne jeden okruh za 40/2,5 = 16 minút.\\
\\
\kom V~tomto prípade vyžaduje netriviálne úsilie správne zostavenie sústavy rovníc tak, aby skutočne zodpovedala zadaniu. Jej vyriešenie potom už zložité nie je. Za zmienku stojí, že úloha je vhodným príkladom situácie, v~ktorej si zmysluplnosť výsledku môžeme aspoň približne overiť (záporné rýchlosti, rýchlosti väčšie ako rýchlosť svetla).\\
\\
}


% Do not delete this line (pandoc magic!)

\problem{66-I-4-N1}{seminar06,rovnice,sustavy,mnohocleny,domacekolo}{
Určte všetky dvojčleny $P (x) = ax + b$, pre ktoré platí $P(2) = 3$ a $P (3) = 2$.
}{
\rie Z~podmienok zo zadania zostavíme dve rovnice s~dvomi neznámymi $a$ a $b$:
\begin{align*}
P (2) &= 2a + b = 3,\\
P (3) &= 3a + b = 2.
\end{align*}
Odčítaním prvej rovnice od druhej ihneď dostávame $a = -1$, dosadením tejto hodnoty do jednej z~podmienok potom máme $b= 5$. Sústava má práve jedno riešenie, a preto zadaniu vyhovuje jediný dvojčlen $P(x)=-x+5$.\\
\\
}


% Do not delete this line (pandoc magic!)

\problem{66-I-4-N2}{}{
Určte všetky trojčleny $P (x) = ax^2+ bx + c$, pre ktoré platí $P (1) = 4$, $P (2) = 9$ a $P (3) = 18$.
}{
\rie Podobne, ako v~predchádzajúcej úlohe, zostavíme z~podmienok sústavu troch lineárnych rovníc s~tromi neznámymi $a$, $b$ a $c$:
\begin{align*}
P(1) &= a + b + c = 4, \\
P(2) &= 4a + 2b + c = 9, \\
P(3) &= 9a + 3b + c = 18.
\end{align*}
Sústava má opäť jediné riešenie $a = 2, b = -1, c = 3$, a preto existuje práve jeden trojčlen vyhovujúci zadaniu: $P(x)=2x^2-x+3$.\\
\\
\kom Predchádzajúce dve jednoduchšie úlohy majú prípravný charakter na nasledujúcu úlo\-hu a domácu prácu. Študenti si prostredníctvom nich zopakujú metódy riešenia sústav rovníc s~viacerými neznámymi. Tieto metódy by študentom mali byť známe zo ZŠ, ak však zistíme, že ich používanie nie je až také samozrejmé, je vhodné zaradiť niekoľko jednoduchších úloh, napr. z~\cite{kubat2000}.\\
}


% Do not delete this line (pandoc magic!)

\problem{66-I-4-N3}{seminar06,rovnice,sustavy,mnohocleny,domacekolo}{
Určte všetky dvojčleny $P (x) = ax+b$ s~celočíselnými koeficientmi $a$ a $b$, pre ktoré platí $P (1) < P (2)$ a $P (1)^2+ P(2)^2= 5$.
}{
\rieh Keďže $a$ a $b$ sú podľa zadania celé čísla, budú celými číslami aj hodnoty $P(1)$ a $P(2)$. Preto hľadáme, akými spôsobmi sa dá číslo 5 zapísať ako súčet dvoch druhých mocnín celých čísel. Ak neberieme ohľad na poradie sčítancov, je taký spôsob jediný: $5 = (\pm 1)^2+ (\pm 2)^2$. Zároveň vieme, že $P(1)<P(2)$, preto dvojicu $(P(1), P(2))$ tvoria niektoré z~nasledujúcich štyroch možností: $(1, 2), (-1, 2), (-2, -1), (-2, 1)$. Každá z~týchto štyroch dvojíc podmienok potom vedie k~sústave dvoch rovníc s~dvomi neznámymi, takže dostávame objemnejšiu variáciu prvej úlohy tohto seminára. Vyriešením systémov získame 4 vyhovujúce dvojčleny $x + 0$, $3x - 4$, $x - 3$ a $3x - 5$.\\
\\
\kom Úloha využíva takmer rovnaký princíp ako prvé dve seminárne úlohy, vyžaduje však dodatočnú analýzu plynúcu z~poslednej podmienky, čo úlohe pridáva na náročnosti. Poslednou úlohou tejto gradovanej série je domáca práca, ktorej analýza povedie k~riešeniu niekoľkých sústav troch rovníc s~tromi neznámymi.\\
\\
}


% Do not delete this line (pandoc magic!)

\problem{66-I-4-D2}{seminar06,rovnice,mnohocleny,domacekolo}{
Koeficienty $a, b, c$ trojčlena $P (x) = ax^2+ bx + c$ sú reálne čísla, pritom každá z~troch jeho hodnôt $P (1), P (2)$ a $P (3)$ je celým číslom. Vyplýva z~toho, že aj čísla $a, b, c$ sú celé, alebo je nutne celé aspoň niektoré z~nich (ktoré)?
}{
\rieh Nevyplýva. Uvážme príklad trojčlena $P (x) =\frac{1}{2}x^2+\frac{1}{2}x+1$: z~vyjadrenia $P (x) =\frac{1}{2}x(x + 1) + 1$ vyplýva, že $P (x)$ je celým číslom pre každé celé $x$, pretože súčin $x(x + 1)$ je vtedy deliteľný dvoma. Vo všeobecnej situácii je iba koeficient $c$ nutne celé číslo; vyplýva to z~vyjadrenia $c = P (0) = 3P (1) - 3P (2) + P (3)$.\\
\\
\kom Úloha je zaujímavá tým, že cesta k~riešeniu je tentoraz menej priamočiara a študenti pravdepodobne prídu na viac rôznych príkladov mnohočlenov s~neceločíselnými koeficientami, ktoré dané podmienky spĺňajú. Zaujímavá bude tiež pravdepodobne diskusia nad zdôvodnením, ktoré z~koeficientov nutne celočíselné byť musia.\\
\\
}


\problem{59-S-3}{
Nájdite všetky dvojice nezáporných celých čísel $a$, $b$, pre ktoré platí
$$a^2 + b + 2 = a + b^2.$$
}{
\rieh Rovnicu prepíšeme na tvar $2 = (b^2 -a^2 )-(b-a)$, z~ktorého po využití vzťahu pre rozdiel štvorcov a následnom vyňatí výrazu $b - a$ dostaneme $2 = (b - a)(a + b - 1)$.
Keďže 2 je prvočíslo, máme pre uvedený súčin nasledujúce štyri možnosti:

\begin{enumerate}[a)]
\item $b - a = 1$ a $a + b - 1 = 2$, potom $a = 1$ a $b = 2.$
\item  $b - a = 2$ a $a + b - 1 = 1$, potom $a = 0$ a $b = 2$.
\item $b - a = -1$ a $a + b - 1 = -2$. Druhú rovnicu možno prepísať na tvar $a + b = -1$, z~ktorého vidíme, že rovnosť nenastane pre žiadnu dvojicu nezáporných celých čísel.
\item  $b - a = -2$ a $a + b - 1 = -1$. Druhú rovnicu možno prepísať na tvar $a + b = 0$, z~ktorého vidíme, že vyhovuje jediná dvojica nezáporných celých čísel $a = b = 0$, ktorá však nevyhovuje prvej rovnici.
\end{enumerate}

\textit{Záver.} Úloha má dve riešenia: Buď $a = 1$ a $b = 2$, alebo $a = 0$ a $b = 2$.

\textit{Poznámka.} Namiesto rozboru štyroch možností môžeme začať úvahou, že nulové čísla $a, b$ nie sú riešením úlohy, takže $a + b - 1 = 0$, a teda aj $b - a = 0$. Stačí teda uvažovať iba možnosti a) a b).\\
\\
\textbf{Iné riešenie.} Rovnicu upravíme na tvar $2 = (b^2 - b) - (a^2 - a)$, resp. na tvar $2 = b(b - 1)-a(a-1)$. Z~nasledujúcej tabuľky a tvaru čísel $x^2 -x = x(x-1)$ je zrejmé, že rozdiely medzi susednými hodnotami výrazov $x(x - 1)$ rastú s~rastúcim $x$ (ľahko sa o~tom presvedčíme výpočtom: $(x + 1)x - x(x - 1) = 2x$).
\begin{center}
\begin{tabular}{|c|c|c|c|c|c|c|c|}
\hline
$x$ & 0 & 1 & 2 & 3 & 4 & 5 & \ldots \\
\hline
$x(x-1)$ & 0 & 0 & 2 & 6 & 12 & 20 & \ldots\\
\hline
\end{tabular}
\end{center}
Môže teda platiť iba $b^2 - b = 2$ a $a^2 - a = 0$. Odtiaľ $a \in \{0, 1\}$ a $b = 2$. Riešením úlohy sú teda dve dvojice nezáporných celých čísel: $a = 0, b = 2$ a $a = 1, b = 2$.\\
\\
\kom Úloha je (okrem iného) zaujímavá tým, že poskytuje priestor na rôznorodé prístupy k~riešeniu a môže byť dobrým podnetom na vzájomné vysvetľovanie riešení medzi študentmi. Kľúčovým prvkom riešenia je úprava rovnice na vhodný tvar -- na tomto mieste je študentom vhodné pripomenúť, že zručnosť a dôvtip pri manipulácii s~algebraickými výrazmi nájdu uplatenie v~širokom spektre problémov, nielen na prvom seminárnom stretnutí, ktoré na túto problematiku bolo zamerané.
}



\home{
\subsection*{Domáca práca}

% Do not delete this line (pandoc magic!)

\problem{66-I-4}{seminar06,rovnice,sustavy,mnohocleny}{
Nájdite všetky trojčleny $P(x)=ax^2+bx+c$ s~celočíselnými koeficientami $a, b, c$, pre ktoré platí $P(1) < P(2) < P(3)$ a zároveň $$(P(1))^2+ (P(2))^2+ (P(3))^2= 22.$$
}{
\rieh Keďže $a, b, c$ sú podľa zadania celé čísla, sú také aj hodnoty $P(1), P(2)$ a $P(3)$. Ich druhé mocniny, čiže čísla  $P(1)^2 , P(2)^2$ a $P(3)^2$, sú preto druhými mocninami
celých čísel, teda tri (nie nutne rôzne) čísla z~množiny $\{0, 1, 4, 9, 16, 25, . . .\}$. Ich súčet je podľa zadania rovný 22, takže každý z~troch sčítancov je menší ako šieste možné číslo 25. Akými spôsobmi možno vôbec zostaviť súčet 22 z~troch čísel vybraných z~množiny  $\{0, 1, 4, 9, 16\}$?
Systematickým rozborom rýchlo zistíme, že rozklad čísla 22 na súčet troch druhých mocnín je (až na poradie sčítancov) iba jeden, a to $22 = 4+9+9$. Dve z~čísel $P(1), P(2)$ a $P(3)$ majú teda absolútnu hodnotu 3 a tretie 2, a keďže
$P(1) < P(2) < P(3)$, musí nutne platiť $P(1) = -3$, $P(3) = 3$ a $P(2) \in \{-2, 2\}$. Pre každú z~oboch vyhovujúcich
trojíc $(P(1), P(2), P(3)) = (-3, -2, 3)$ a $(P(1), P(2), P(3)) = (-3, 2, 3)$ určíme koeficienty $a, b, c$ príslušného trojčlena $P(x)$ tak, že nájdené hodnoty dosadíme do pravých strán rovníc
\begin{align*}
a + b + c &= P(1),\\
4a + 2b + c &= P(2),\\
9a + 3b + c &= P(3)
\end{align*}
a výslednú sústavu troch rovníc s~neznámymi $a, b, c$ vyriešime. Tento jednoduchý výpočet tu vynecháme, v~oboch prípadoch vyjdú celočíselné trojice $(a, b, c)$, ktoré zapíšeme rovno ako koeficienty trojčlenov, ktoré sú jedinými dvoma riešeniami danej úlohy:
$$P_1(x) = 2x^2 -5x \ \ \ \  \text{a} \ \ \ \ P_2 (x) = -2x^2+ 11x -12.$$
}


% Do not delete this line (pandoc magic!)

\problem{62-II-3}{
Nájdite všetky dvojice celých kladných čísel $a$ a $b$, pre ktoré je číslo $a^2 +b$ o~62 väčšie
ako číslo $b^2 + a$.
}{
\rieh  Zadanie zapíšeme rovnosťou, ktorej pravú stranu rovno upravíme na súčin: $$62 = (a^2+ b) - (b^2
+ a) = (a^2 - b^2)- (a - b) = (a - b)(a + b - 1).$$
Súčin celých čísel $u = a - b$ a $v = a + b - 1$ je teda rovný súčinu dvoch prvočísel $2 \cdot 31$.
Keďže $v \geq 1 + 1 - 1 = 1$, je nutne aj číslo $u$ kladné a zrejme $u < v$, takže $(u, v)$ je jedna
z~dvojíc $(1, 62)$ alebo $(2, 31)$. Ak vyjadríme naopak $a, b$ pomocou $u, v$, dostaneme $$a =\frac{u+v+1}{2} \ \ \ \  \textrm{a} \ \ \ \  b=\frac{v-u+1}{2}.$$ Pre $(u, v) = (1, 62)$ tak dostávame riešenie $(a, b) = (32, 31)$, dvojici $(u, v) = (2, 31)$ zodpovedá druhé riešenie $(a, b) = (17, 15)$. Iné riešenia úloha nemá.\\
\\
\kom Úloha je veľmi podobná tej, ktorou sme sa zaoberali na stretnutí, slúži tak na overenie toho, či si študenti princíp riešenia osvojili. Zároveň ale zadanie nie je zapísané priamo rovnosťou, takže úloha precvičí aj schopnosť transformovať slovný text na matematický zápis.\\
\\
}


% Do not delete this line (pandoc magic!)

\problem{60-I-1-D1}{seminar06,rovnice,domacekolo,doplnujuca}{
Nech $n$ je prirodzené číslo väčšie ako 2. Máme $n$ čísel so súčtom~$n$, pričom každé z~nich je aritmetickým priemerom ostatných čísel. Aké sú to čísla?
}{
\rieh Usporiadajme si naše čísla podľa veľkosti, nech $x_1 \leq x_2 \leq\,\ldots \leq x_n$. Aritmetický priemer skupiny čísel je aspoň taký, ako najmenšie z~nich. Aritmetický priemer čísel $x_2, x_3,\ldots , x_n$ je preto aspoň $x_2$, a je rovný $x_1$ len v~prípade, že žiadne z~čísel $x_3,\,\ldots , x_n$ nie je väčšie ako $x_2$. Z~toho hneď dostávame, že všetky naše čísla musia byť rovnaké a teda rovné 1.
}

}
\seminar{7}

\subsection*{Téma}
Teória čísel I~-- úlohy o~deliteľnosti
\teachernote{
\subsection*{Ciele}

\textbf{Úvodný komentár.} Keďže ide o~prvé stretnutie zo série seminárov zameraných na elementárnu teóriu čísel, je potrebné so študentmi zopakovať základné znalosti, ktoré by mali mať zo základnej školy:
\begin{itemize}
\item chápať rozdiel medzi číslom a cifrou,
\item používať rozvinutý a skrátený zápis čísla v~desiatkovej sústave,
\item rozumieť pojmom prvočíslo a zložené číslo,
\item vedieť určiť najmenší spoločný násobok a najväčší spoločný deliteľ daných dvoch celých čísel,
\item poznať pravidlá deliteľnosti číslami 2, 3, 4, 5, 6, 8, 9, 10.
\end{itemize}
Študenti by mali byť schopní zdôvodniť všetky pravidlá deliteľnosti. Ak si ich nepamätajú, môže byť táto úloha vhodnou rozcvičkou pred riešením pripravených problémov.

Takisto je vhodné zjednotiť značenie, ktoré budeme používať.  Fakt, že celé číslo $a$ delí celé číslo $b$ budeme zapisovať v tvare $a \mid b$ . V~tomto texte tiež označujeme $(a,b)$ najväčší spoločný deliteľ čísel $a$ a $b$ a $[a,b]$ ich najmenší spoločný násobok.

Pripomenieme ešte, že ak pre prirodzené čísla $a, b, c$ platí $a \mid (b\cdot c)$ a zároveň $(a,b)=1$, musí nutne $a\mid c$. Toto tvrdenie budeme v~priebehu seminárov využívať často, je preto dôležité, aby ho študenti vzali za svoje. Vyzbrojení všetkými spomenutými znalosťami sa môžeme pustiť do riešenia úloh.

}
\subsection*{Úlohy a riešenia}

% Do not delete this line (pandoc magic!)

\problem{~\cite{holton2010}, úloha 38, str. 115}{seminar07,delitelnost,cifry}{
Nech $N$ je päťciferné kladné číslo také, že $N=\overline{a679b}$. Ak je $N$ deliteľné 72, určte prvú cifru $a$ a poslednú cifru $b$.
}{
\rieh Keďže je číslo $N$ deliteľné $72=8\cdot 9$, musí byť súčasne deliteľné ôsmimi aj deviatimi. Z~pravidla pre deliteľnosť ôsmimi vyplýva, že číslo $\overline{79b}$ musí byť násobkom ôsmich a teda $b=2$. Pravidlo pre deliteľnosť deviatimi diktuje, že ciferný súčet hľadaného čísla $a+6+7+9+2=a+24$ je násobkom deviatich, dostávame tak $a=3$. Hľadaným číslom je $N=36792$.\\
\\
\kom Úloha nie je náročná a je zaradená ako zahrievacie cvičenie a ukážka práce s~deliteľnosťou zloženým číslom.\\
\\
}


\problem{66-I-2-N1}{
Dokážte, že v~nekonečnom rade čísel
$$ 1 \cdot 2 \cdot 3, 2 \cdot 3 \cdot 4, 3 \cdot 4 \cdot 5, 4 \cdot 5 \cdot 6, \ldots ,$$
je číslo prvé deliteľom všetkých čísel ďalších.
}{
\rie Prvé číslo v~nekonečnom rade je číslo 6. Dokazujeme tak, že všetky výrazy tvaru $n(n+1)(n+2)$, kde $n\geq 2$ je prirodzené číslo, sú deliteľné šiestimi. To ale zjavne platí, keďže z~troch po sebe idúcich čísel je vždy práve jedno deliteľné tromi a minimálne jedno z~nich je tiež párne. Deliteľnosť dvomi a tromi zároveň nám tak zaručí deliteľnosť šiestimi a požadované tvrdenie je dokázané.\\
\\
\kom Táto jednoduchá úloha zoznámi žiakov s~poznatkom často využívaným v~úlohách zameraných na dokazovanie deliteľnosti číslom, ktoré je násobkom troch: z~troch po sebe idúcich prirodzených čísel je vždy práve jedno deliteľné tromi.\\
\\
}


% Do not delete this line (pandoc magic!)

\problem{63-I-5-N1}{seminar07,delitelnost,domacekolo}{
Dokážte, že pre každé prirodzené $n$ je číslo $n^3+ 2n$ deliteľné tromi.
}{
\rie Každé prirodzené číslo $n$ je tvaru $n=3k$, $n=3k+1$ alebo $n=3k+2$, kde $k$ je prirodzené číslo alebo 0. Dokazované tvrdenie overíme pre každú z~týchto možností zvlášť.

a) $n=3k$: $n^3+2n=(3k)^3+2\cdot 3k=27k^3+6k=3k(9k^2+2)$, tvrdenie platí.

b) $n=3k+1$: $n^3+2n=(3k+1)^3+2(3k+1)=(27k^3+27k^2+9k+1)+(6k+2)=27k^3+27k^2+15k+3=3(9k^3+9k^2+5k+1)$, tvrdenie platí.

c) $n=3k+2$: $n^3+2n=(3k+2)^3+2(3k+2)=(27k^3+54k^2+36k+8)+(6k+4)=27k^3+54k^2+42k+12=3(9k^3+18k^2+14k+4)$, a preto $3\mid n^3+2n$ aj v~tomto prípade.\\
\\
\textbf{Iné riešenie.} Môžeme sa inšpirovať predchádzajúcou úlohou a opäť využiť poznatok, že súčin troch po sebe idúcich čísel je vždy deliteľný tromi. Číslo $n^3+2n$ môžeme napísať ako $n^3-n+3n$. Číslo $n^3-n$ je deliteľné tromi pre každé prirodzené $n$, keďže ide o súčin troch po sebe idúcich čísel: $n^3-n=n(n^2-1)=n(n-1)(n+1)$. $3\mid 3n$ a preto aj $3\mid n^3+2n$.\\
\\
\kom Úloha zoznamuje študentov s~ďalším možným postupom pri dokazovaní deliteľnosti výrazu daným prirodzeným číslom $m$: rozdelenie na $m$ možností podľa zvyšku po delení číslom $m$ a dokázanie tvrdenia pre každú z~týchto možností zvlášť. Je vhodné diskutovať so študentmi o~výhodnosti tejto metódy pre (ne)veľké $m$.

V druhom prístupe k riešeniu sme využili to, že sme číslo $2n$ zapísali v na prvý pohľad zložitejšom tvare $-n+2n$, čo nám však v konečnom dôsledku pomohlo úlohu elegantne vyriešiť. Táto myšlienka nájde uplatnenie aj v iných úlohách (o.\i. aj v úlohe \textbf{7.10}), preto ak s ňou študenti neprídu sami, je vhodné na ňu upozorniť.

Úlohu je možné dokázať použitím matematickej indukcie, avšak tá nie je štandardnou náplňou osnov nematematických gymnázií, preto sme toto riešenie nezvolili ako vzorové. Ak sa však študenti s~dôkazom použitím indukcie stretli, je vhodné s~nimi rozobrať aj tento spôsob riešenia.\\
\\
}


% Do not delete this line (pandoc magic!)

\problem{63-I-5-N2}{seminar07,delitelnost}{
Dokážte, že pre každé nepárne číslo $n$ je číslo $n^2 - 1$ deliteľné ôsmimi.
}{
\rie Výraz $n^2-1$ upravíme na súčin $(n-1)(n+1)$. To je súčin dvoch po sebe idúcich párnych čísel, keďže $n$ je nepárne. Preto práve jedno z~čísel $n-1$ a $n+1$ je deliteľné 4 a druhé z~nich je nepárnym násobkom čísla 2. Celkovo je teda súčin $(n-1)(n+1)$ deliteľný ôsmimi.\\
\\
\kom Posledná úloha zo série jednoduchých dôkazov deliteľnosti využíva podobnú myšlienku ako úloha \textbf{7.2}, navyše však vyžaduje upravenie výrazu $n^2-1$ do vhodného tvaru. Následná diskusia o~riešení je už jednoduchá.\\
\\
}


% Do not delete this line (pandoc magic!)

\problem{63-I-5-N3+63-I-5-N4, resp. 55-I-1}{seminar07,delitelnost,domacekolo}{
\begin{enumerate}[a)]
\item Dokážte, že pre všetky celé kladné čísla $m$ je rozdiel $m^6 - m^2$ deliteľný šesťdesiatimi.
\item Určte všetky kladné celé čísla $m$, pre ktoré je rozdiel $m^6 - m^2$ deliteľný číslom 120.
\end{enumerate}
}{
\rieh a) Číslo $n = m^6 -m^2 = m^2 (m^2-1)(m^2 +1)$ je vždy deliteľné štyrmi, pretože pri párnom $m$ je $m^2$ deliteľné štyrmi a pri nepárnom $m$ sú čísla $m^2-1$, $m^2 +1$ obe párne, jedno z~nich je dokonca deliteľné štyrmi a ich súčin je teda deliteľný ôsmimi. Z~troch po sebe idúcich prirodzených čísel $m^2-1$, $m^2$, $m^2 + 1$ je práve jedno deliteľné tromi, a preto je aj číslo $n$ deliteľné tromi. Ak je $m$ deliteľné piatimi, je $m^2$ deliteľné piatimi, dokonca dvadsiatimi piatimi. V~opačnom prípade je $m$ tvaru $5k + r$, kde $r$ je rovné niektorému z~čísel 1, 2, 3, 4 a $k$ je prirodzené alebo 0. Potom $m^2 = 25k^2 + 10kr + r^2$ a $r^2$ sa rovná niektorému z~čísel 1, 4, 9, 16. V~prvom a v~poslednom prípade je číslo $m^2-1$ deliteľné piatimi, v~ostatných dvoch prípadoch je číslo $m^2 + 1$ deliteľné piatimi. Teda číslo $n$ je vždy deliteľné nesúdeliteľnými číslami 4, 3 a 5, a teda aj ich súčinom 60.\\

b) Už sme ukázali, že v~prípade nepárneho $m$ je súčin $(m^2-1)(m^2 + 1)$ deliteľný ôsmimi a číslo $n = m^6- m^2$ je teda deliteľné číslom $120 = 8 \cdot 3 \cdot 5$. Ak je však číslo $m$ párne, sú čísla $m^2 -1$, $m^2 + 1$ nepárne, žiadne nie je deliteľné dvoma. Číslo $n$ je potom deliteľné ôsmimi iba v~prípade, že $m^2$ je deliteľné ôsmimi, teda $m$ je deliteľné štyrmi. Číslo $n$ je potom deliteľné šestnástimi, tromi a piatimi, a preto dokonca číslom 240.

\textit{Záver.} Naše výsledky môžeme zhrnúť. Číslo $n = m^6 - m^2$ je deliteľné číslom 120 práve vtedy, keď $m$ je nepárne alebo deliteľné štyrmi.\\
\\
\kom Sada dvoch na seba nadväzujúcich úloh využíva poznatky získané pri riešení jednoduchších prípravných úloh zo začiatku seminára a vyžaduje sústredené a starostlivé aplikovanie všetkých z~nich.\\
\\
}


\problem{59-II-1}{
Dokážte, že pre ľubovoľné celé čísla $n$ a $k$ väčšie ako 1 je číslo $n^{k+2} - n^k$ deliteľné dvanástimi.
}{
\rieh Vzhľadom na to, že $12 = 3 \cdot 4$, stačí ukázať, že číslo $a = n^{k+2} -  {n^k} = n^k (n^2 - 1) = (n - 1)n(n + 1)n^{k-1}$ je deliteľné tromi a štyrmi. Prvé tri činitele posledného výrazu sú tri po sebe idúce prirodzené čísla, takže práve jedno z~nich je deliteľné tromi, a preto aj číslo $a$ je deliteľné tromi. Je deliteľné aj štyrmi, lebo pri párnom $n$ je v~poslednom výraze druhý a štvrtý činiteľ párny, zatiaľ čo pri nepárnom $n$ je párny prvý a tretí činiteľ. Tým je dôkaz hotový.\\
\\
\textbf{Iné riešenie.} Položme $a = n^{k+2} - n^k = n^k (n^2 - 1) = (n - 1)n^k (n + 1)$. Opäť ukážeme, že $a$ je deliteľné štyrmi a tromi. Ak je $n$ párne, je $n^k$ deliteľné štyrmi pre každé celé $k \geq 2$. Ak je $n$ nepárne, sú činitele $n - 1$ a $n + 1$ párne čísla, takže $a$ je deliteľné štyrmi pre každé celé $n = 2$.

Deliteľnosť tromi je zrejmá pre $n = 3l$. Ak $n = 3l + 1$, pričom $l$ je celé kladné číslo, je tromi deliteľný činiteľ $n - 1$ (a teda aj číslo $a$). Ak $n = 3l + 2$ ($l$ je celé nezáporné), je tromi deliteľný činiteľ $n + 1$. Keďže iné možnosti pre zvyšok čísla $n$ po delení tromi nie sú, je číslo $a$ deliteľné tromi. Tým je požadovaný dôkaz ukončený.\\
\\
\kom  Deliteľnosť štyrmi je tiež možné dokázať aj rozborom možností $n=4l$, $n=4l+1$, $n=4l+2$ a $n=4l+3$, pre $l$ celé a nezáporné. Kľúčovým krokom v~riešení bolo vhodné rozloženie čísla $a$ na súčin. To však súdiac podľa priemerného počtu bodov udelených za túto úlohu v~krajských kolách\footnote{3,0\,b v~prípade úspešných riešiteľov, 1,8\,b v~prípade všetkých riešiteľov, najmenej zo všetkých úloh krajského kola daného ročníka} na Slovensku bola úloha pre riešiteľov neľahká.\\
\\
}


% Do not delete this line (pandoc magic!)

\problem{58-S-3}{
Keď isté dve prirodzené čísla v~rovnakom poradí sčítame, odčítame, vydelíme a vynásobíme a všetky štyri výsledky sčítame, dostaneme 2 009. Určte tieto dve čísla.
}{
\rie Pre hľadané prirodzené čísla $x$ a $y$ sa dá podmienka zo zadania vyjadriť rovnicou
\begin{equation} \label{eq:58S3}
    (x + y) + (x - y) +\frac{x}{y}+ (x \cdot y) = 2 009,
\end{equation}
v~ktorej sme čiastočné výsledky jednotlivých operácií dali do zátvoriek.

Vyriešme rovnicu \ref{eq:58S3} vzhľadom na neznámu $x$ (v~ktorej je, na rozdiel od neznámej $y$, rovnica lineárna):
\begin{align} \label{eqs:58S3}
2x +\frac{x}{y}+ xy &= 2 009, \nonumber \\
2xy + x + xy^2 &= 2 009y, \nonumber\\
x(y + 1)^2 &= 2 009y, \nonumber \\
x &= \frac{2009y}{(y + 1)^2}.
\end{align}
Hľadáme práve tie prirodzené čísla $y$, pre ktoré má nájdený zlomok celočíselnú hodnotu, čo možno vyjadriť vzťahom $(y + 1)^2 \mid 2009y$. Keďže čísla $y$ a $y + 1$ sú nesúdeliteľné, sú nesúdeliteľné aj čísla $y$ a $(y +1)^2$, takže musí platiť $(y +1)^2 \mid 2009 = 7^2 \cdot41$. Keďže $y +1$ je celé číslo väčšie ako 1 (a činitele 7, 41 sú prvočísla), poslednej podmienke vyhovuje iba hodnota $y = 6$, ktorej po dosadení do \ref{eqs:58S3} zodpovedá $x = 246$. (Skúška nie je nutná, lebo rovnice \ref{eq:58S3} a \ref{eqs:58S3} sú v~obore prirodzených čísel ekvivalentné.)

Hľadané čísla v~uvažovanom poradí sú 246 a 6.\\
\\
\kom Úloha je zaujímavá v~tom, že na prvý pohľad nemusí riešiteľ tušiť, že ide o~problém využívajúci poznatky z~deliteľnosti. Zároveň vyžaduje netriviálnu zručnosť a nápad pri upravovaní počiatočnej rovnice do vhodného tvaru, nadväzuje tým na predchádzajúce semináre o~algebraických výrazoch a rovniciach. Úloha je tak peknou ukážkou toho, že v~matematike (a~nielen tam) nie sú znalosti a koncepty nesúvisiace, ale často sú vzájomne prepojené.\\
\\
}


% Do not delete this line (pandoc magic!)

\problem{66-I-2-N2}{seminar07,delitelnost,domacekolo}{
Nájdite všetky celé $d > 1$, pri ktorých hodnoty výrazov $U(n) = n^3+ 17n^2-1$ a $V (n) = n^3+ 4n^2+ 12$ dávajú po delení číslom $d$ rovnaké zvyšky, nech je celé číslo $n$ zvolené akokoľvek.
}{
\rieh Hľadané $d$ sú práve tie, ktoré delia rozdiel $U(n) - V~(n) = 13n^2 - 13 = 13(n - 1)(n + 1)$ pre každé celé $n$. Tento rozdiel je tak určite deliteľný 13. Aby sme ukázali, že (zrejme vyhovujúce) $d = 13$ je jediné, dosaďme do rozdielu $U(n) - V~(n)$ hodnotu $n = d$: číslo $d$ je s~číslami $d - 1$ a $d + 1$ nesúdeliteľné, takže delí súčin $13(d - 1)(d + 1)$ jedine vtedy, keď delí činiteľ 13, teda keď $d = 13$. Vyhovuje jedine $d = 13$.\\
\\
\kom Úloha stavia na myšlienke deliteľnosti rozdielu $U(n)$ a $V(n)$ hľadaným $d$, čo je ďalší užitočný nástroj: namiesto upravovania výrazov $U(n)$ a $V(n)$ ich odčítať. Taktiež je vhodné upozorniť študentov, že druhá časť riešenia -- dokázanie, že nájdené riešenie je jediné -- je taktiež podstatnou súčasťou riešenia (nielen) tejto úlohy.\\
\\
}


% Do not delete this line (pandoc magic!)

\problem{66-I-2-D1}{seminar07,delitelnost,domacekolo,doplnujuca}{
Pre ktoré prirodzené čísla $n$ nie je výraz $V (n) = n^4+ 11n^2 - 12$ násobkom ôsmich?
}{
\rieh Upravme výraz $V(n)$ do tvaru súčinu: $V (n) = (n^2-1)(n^2+12)=(n-1)(n+1)(n^2+12)$. Vidíme, že $V(n)$ je určite násobkom ôsmich v~prípade nepárneho $n$ (viď. tretia úloha tohto seminára). Keďže pre párne $n$ je súčin $(n-1)(n+1)$ nepárny, hľadáme práve tie $n$ tvaru $n = 2k$, pre ktoré nie je deliteľný ôsmimi výraz $n^2+ 12 = 4(k^2+ 3)$, čo nastane práve vtedy, keď $k$ je párne. Hľadané $n$ sú teda práve tie, ktoré sú deliteľné štyrmi.\\
\\
\kom Úloha využíva vhodnú úpravu výrazu $V$ na súčin. Tu študenti zúročia zručnosti nadobudnuté v~algebraických seminároch. Zároveň využijú skôr dokázané tvrdenie o~deliteľnosti ôsmimi a napokon, úloha ich pripraví na nasledujúci komplexnejší problém.\\
\\
}


% Do not delete this line (pandoc magic!)

\problem{66-I-2}{seminar07,delitelnost,domacekolo}{
Nájdite najväčšie prirodzené číslo $d$, ktoré má tú vlastnosť, že pre ľubovoľné prirodzené číslo $n$ je hodnota výrazu $$V (n) = n^4+ 11n^2-12$$
deliteľná číslom $d$.
}{
\rieh Vypočítajme najskôr hodnoty $V (n)$ pre niekoľko najmenších prirodzených čísel $n$ a ich rozklady na súčin prvočísel zapíšme do tabuľky:
\begin{center}
\begin{tabular}{c c}
$n$ & $V (n) $\\
\hline
1 & 0\\
2 & $48 = 2^4 \cdot 3$\\
3 & $168 = 2^3 \cdot 3 \cdot 7$\\
4 & $420 = 2^2 \cdot 3 \cdot 5 \cdot 7$
\end{tabular}
\end{center}
Z~toho vidíme, že hľadaný deliteľ $d$ všetkých čísel $V (n)$ musí byť deliteľom čísla $2^2 \cdot 3 = 12$, spĺňa teda nerovnosť $d \leq 12$. Preto ak ukážeme, že číslo $d = 12$ zadaniu vyhovuje,  t.\,j. že $V (n)$ je násobkom čísla 12 pre každé prirodzené $n$, budeme s~riešením hotoví.

Úprava $$V (n) = n^4+ 11n^2 - 12 = (12n^2 - 12) + (n^4 - n^2),$$ pri ktorej sme z~výrazu $V (n)$ ”vyčlenili“ dvojčlen 1$2n^2 -12$, ktorý je zrejmým násobkom čísla 12, redukuje našu úlohu na overenie deliteľnosti číslom 12 (teda deliteľnosti číslami 3 a 4) dvojčlena $n^4 - n^2$ . Využijeme na to jeho rozklad $$n^4 - n^2 = n^2(n^2 - 1) = (n - 1)n^2(n + 1).$$
Pre každé celé $n$ je tak výraz $n^4 - n^2$ určite deliteľný tromi (také je totiž jedno z~troch
po sebe idúcich celých čísel $n - 1$, $n$, $n + 1$) a súčasne aj deliteľný štyrmi (zaručuje to
v~prípade párneho $n$ činiteľ $n^2$, v~prípade nepárneho $n$ dva párne činitele $n-1$ a $n+1$).

Dodajme, že deliteľnosť výrazu $V (n)$ číslom 12 možno dokázať aj inými spôsobmi, napríklad môžeme využiť rozklad $V (n) = n^4+ 11n^2 - 12 = (n^2+ 12)(n^2 - 1)$ z~predchádzajúcej úlohy alebo prejsť k~dvojčlenu $n^4 + 11n^2$ a podobne.

\textit{Záver.} Hľadané číslo $d$ je rovné 12.\\
\\
\kom Úloha je okrem využitia všetkých doterajších poznatkov zaradená aj z~dôvodu prvého kroku riešenia. Je vhodné študentom ukázať, že preskúmanie výrazu pre niekoľko malých hodnôt $n$ nám môže pomôcť utvoriť si predstavu o~tom, ako sa bude výraz správať ďalej, príp. vytvoriť hypotézu, ktorú sa neskôr pokúsime dokázať. Táto metóda nájde uplatnenie nielen v~tejto konkrétnej úlohe, ale aj v~ďalších partiách matematiky.\\
}




\subsection*{Domáca práca}

% Do not delete this line (pandoc magic!)

\problem{66-S-2}{
Označme $M$ množinu všetkých hodnôt výrazu $V (n) = n^4 + 11n^2 - 12$, pričom $n$ je nepárne prirodzené číslo. Nájdite všetky možné zvyšky po delení číslom 48, ktoré dávajú prvky množiny $M$.
}{
\rieh Najskôr vypočítame prislúchajúce hodnoty výrazu $V$ pre niekoľko prvých nepárnych čísel:
\begin{center}
\begin{tabular}{c c}
$n$ & $V (n)$\\
\hline
1 & 0\\
3 & $168 = 3 \cdot 48 + 24$ \\
5 & $888 = 18 \cdot 48 + 24$ \\
7 & $2928 = 61 \cdot 48$\\
9 & $7440 = 155 \cdot 48$
\end{tabular}
\end{center}

Medzi hľadané zvyšky teda patria čísla 0 a 24. Ukážeme, že iné zvyšky už možné nie sú. Na to stačí dokázať, že pre každé nepárne číslo $n$ platí $24 \mid V~(n)$. Z~školskej časti seminára vieme, že pre každé prirodzené číslo $n$ platí $12 \mid V~(n)$, teda aj $3 \mid V~(n)$. Keďže čísla 3 a 8 sú nesúdeliteľné, stačí ukázať, že pre každé nepárne číslo $n$ platí $8 \mid V~(n)$. Využijeme
pritom rozklad daného výrazu na súčin
\begin{equation} \label{eq:66S2}
    V (n) = n^4+ 11n^2 - 12 = (n^2 - 1)(n^2+ 12) = (n - 1)(n + 1)(n^2+ 12).
\end{equation}
Ľubovoľné nepárne prirodzené číslo $n$ možno zapísať v~tvare $n = 2k - 1$, pričom $k \in \NN$ . Pre také $n$ potom dostávame
$$V (2k - 1) = [(2k - 1) - 1][(2k - 1) + 1][(2k - 1)^2
+ 12] = 4(k - 1)k(4k^2 - 4k + 13),$$
a keďže súčin $(k - 1)k$ dvoch po sebe idúcich celých čísel je deliteľný dvoma, je celý výraz deliteľný ôsmimi.

\textit{Záver}. Daný výraz môže dávať po delení číslom 48 práve len zvyšky 0 a 24.

\textit{Poznámka}. Poznatok, že $8 \mid V~(n)$ pre každé nepárne $n$, možno dokázať aj inak, bez použitia rozkladu \ref{eq:66S2}. Ak je totiž $n = 2k - 1$, pričom $k \in \NN$ , tak číslo
$$n^2= (2k - 1)^2= 4k^2 - 4k + 1 = 4k(k - 1) + 1$$
dáva po delení ôsmimi (vďaka tomu, že jedno z~čísel $k$, $k - 1$ je párne) zvyšok 1, a teda rovnaký zvyšok dáva aj číslo $n^4$ (ako druhá mocnina nepárneho čísla $n^2$). Platí teda $n^2 = 8u + 1$ a $n^4 = 8v + 1$ pre vhodné celé $u$ a $v$, takže hodnota výrazu
$$V (2k - 1) = (8v + 1) + 11(8u + 1) - 12 = 8(v + 11u)$$
je naozaj násobkom ôsmich.

Pripojme aj podobný dôkaz poznatku $3 \mid V~(n)$ zo seminárneho stretnutia. Pre čísla $n$ deliteľné tromi je to zrejmé, ostatné $n$ sú tvaru $n = 3k \pm 1$, takže číslo
$$n^2= (3k \pm 1)^2= 9k^2 \pm 6k + 1 = 3k(3k \pm 2) + 1$$
dáva po delení tromi zvyšok 1, rovnako tak aj číslo $n^4 = (n^2)^2$. Dosadenie $n^2 = 3u + 1$ a $n^4 = 3v + 1$ do výrazu $V (n)$ už priamo vedie k~záveru, že $3 \mid V~(n)$.\\
\\
}


% Do not delete this line (pandoc magic!)

\problem{60-I-2}{seminar07,delitelnost}{
Dokážte, že výrazy $23x + y$, $19x + 3y$ sú deliteľné číslom 50 pre rovnaké dvojice prirodzených čísel $x$, $y$.
}{
\rieh Predpokladajme, že pre dvojicu prirodzených čísel $x, y$ platí $50 \mid 23x + y$. Potom pre nejaké prirodzené číslo $k$ platí $23x + y = 50k$. Z~tejto rovnosti dostaneme $y = 50k - 23x$, čiže $19x + 3y = 19x + 3(50k - 23x) = 150k - 50x = 50(3k - x)$, takže číslo $19x + 3y$ je násobkom čísla 50.

Podobne to funguje aj z~druhej strany. Ak pre nejakú dvojicu prirodzených čísel $x,y$ platí $50 \mid 19x + 3y$, tak $19x + 3y = 50l$ pre nejaké prirodzené číslo $l$. Z~tejto rovnosti vyjadríme číslo~$y$; dostaneme $y = (50l - 19x)/3$ (ďalší postup by bol podobný, aj keby sme vyjadrili $x$ namiesto~$y$). Po dosadení dostaneme $$23x + y = 23x + \frac{50l - 19x}{3}=\frac{69x + 50l - 19x}{3}=\frac{50 \cdot (x + l)}{3}.$$
O~výslednom zlomku vieme, že je to prirodzené číslo. Čitateľ tohto zlomku je deliteľný číslom 50. V~menovateli je len číslo 3, ktoré je nesúdeliteľné s~50, preto sa číslo 50 nemá s~čím z~menovateľa vykrátiť a teda číslo $23x + y$ je deliteľné 50.\\
\\
\textbf{Iné riešenie*.} Zrejme $3 \cdot (23x + y) - (19x + 3y) = 50x$, čiže ak 50 delí jedno z~čísel $23x + y$ a $19x + 3y$, tak delí aj druhé z~nich.\\
}



\teachernote{
\subsection*{Doplňujúce zdroje a materiály}
Výborným zdrojom úloh jednoduchých aj zložitejších je publikácia ~\cite{holton2010}, najmä jej časti 4.1, 4.2 a 4.3, ktoré obsahujú mnoho jednoduchších aj zložitejších príkladov na dokazovanie deliteľnosti, spoločné delitele a násobky aj úlohy o~ciferných zápisoch, preto môže byť vhodným doplnením banky úloh nielen pre tento, ale aj nasledujúce dva semináre.

}


\seminar{8}

\subsection*{Téma}
Teória čísel II -- úlohy o~najmenšom spoločnom násobku a najväčšom spoločnom deliteli

\teachernote{
\subsection*{Ciele}
Zoznámiť sa s~metódami riešenia príkladov o~spoločných deliteľoch a násobkoch, upevniť znalosti zo seminára predchádzajúceho.

}
\subsection*{Úlohy a riešenia}

% Do not delete this line (pandoc magic!)

\problem{61-I-3-N1}{seminar08,nsdnsn,domacekolo,navodna}{
Určte, pre ktoré prirodzené čísla $a, b$ platí $(a, b) = 10$ a zároveň  $[a, b] = 150$.
}{
\rieh Pretože $10 = 2 \cdot 5$ a $150 = 2 \cdot 3 \cdot 5^2$, požadované rovnosti sú splnené práve vtedy, keď $a = 2 \cdot 3^s \cdot 5^t$ a $b = 2 \cdot 3^u \cdot 5^v$, kde $\{s, u\} = \{0, 1\}$ a $\{t, v\} = \{1, 2\}$. Riešením je teda jedna zo štvoríc $\{a, b\} = \{10, 150\}$ alebo $\{a, b\} = \{30, 50\}$.\\
\\
\kom Úloha je relatívne jednoduchá a nevyžaduje žiadne špeciálne znalosti, zároveň však nie je triviálna. Tvorí tak príjemné preklenutie medzi školskými a olympiádnymi príkladmi.\\
\\
}


% Do not delete this line (pandoc magic!)

\problem{69-I-5-N1}{
Nech $d$ je najväčší spoločný deliteľ prirodzených čísel $a$ a $b$. Ukážte, že čísla $a/d$ a $b/d$ sú celé a nesúdeliteľné.
}{
\rie Ak je $d$ najväčším spoločným deliteľom čísel $a$ a $b$, potom existujú prirodzené čísla $u$ a $v$ také, že $a=ud$ a $b=vd$, čím sme dokázali prvú časť tvrdenia. Druhú dokážeme sporom. Predpokladajme, že $a/d$ a $b/d$ nie sú nesúdeliteľné. Potom existuje ich najväčší spoločný deliteľ $d_1$. Číslo $d_1$ však potom delí aj čísla $a$ a $b$, čo je spor s~predpokladom, že $d=(a,b)$.\\
\\
\kom Táto mini-úloha je prípravným krokom k~nasledujúcemu všeobecnejšiemu tvrdeniu a zároveň môže pripomenúť použitie dôkazu sporom.\\
\\
}


% Do not delete this line (pandoc magic!)

\problem{60-I-5-N2}{
Dokážte, že pre ľubovoľné prirodzené čísla $a, b$ platí vzťah $$[a, b] \cdot (a, b) = ab.$$
}{
\rie Nech $d = (a, b)$, potom $a = ud$, $b = vd$ pre nesúdeliteľné $u$ a $v$, a teda $[a, b] = uvd$. Porovnaním ľavej a pravej strany dokazovanej nerovnosti dostávame $uvd\cdot d = ud\cdot vd$, čo je pravdivé tvrdenie, teda vzťah je dokázaný.

Alternatívne môžeme vzťah dokázať úvahou o~exponentoch prvočísel, z~ktorých sú čísla $a$ a $b$ zložené. Nech $a=p_1^{\alpha_1}\cdot p_2^{\alpha_2} \cdots p_k^{\alpha_k}$ a $b=p_1^{\beta_1}\cdot p_2^{\beta_2} \cdots p_k^{\beta_k}$, kde $p_1$ až $p_k$ sú prvočísla a $\alpha_k, \beta_k$ prirodzené čísla. Potom
\vspace{-25pt}
\begin{center}
\begin{align*}
(a,b) &=p_1^{\min\{\alpha_1, \beta_1\}}\cdot p_2^{\min\{\alpha_2, \beta_2\}}\cdots p_k^{\min\{\alpha_k, \beta_k\}},\\
[a,b] &=p_1^{\max\{\alpha_1, \beta_1\}}\cdot p_2^{\max\{\alpha_2, \beta_2\}}\cdots p_k^{\max\{\alpha_k, \beta_k\}},\\
ab &=p_1^{\alpha_1+\beta_1}\cdot p_2^{\alpha_2+\beta_2}\cdots p_k^{\alpha_k, \beta_k}.
\end{align*}
\end{center}
Keďže pre akékoľvek čísla $\alpha, \beta$ platí $\max\{\alpha, \beta\}+\min\{\alpha, \beta\}=\alpha+\beta$, a to vo všetkých prípadoch $\alpha < \beta$, $\alpha = \beta$, $\alpha > \beta$, je naše tvrdenie dokázané.\\
\\
\kom Predchádzajúce tvrdenie je stavebným kameňom mnohých úloh o~spoločných násobkoch a deliteľoch, najmä myšlienka zápisu prirodzených čísel $a$ a $b$ v~tvare $a=ud$ a $b=vd$, kde $u$ a $v$ sú prirodzené čísla také, že $(u,v)=1$ a $d=(a,b)$ nájde uplatnenie veľmi často. \\
\\
}


% Do not delete this line (pandoc magic!)

\problem{64-I-5-N4}{seminar08,nsdnsn}{
Platí pre každé tri prirodzené čísla $a, b, c$ a ich najväčší spoločný deliteľ $d$ a ich najmenší spoločný násobok $n$ rovnosť $abc = nd$?
}{
\rie Neplatí, uvedieme protipríklad. Napríklad pre čísla $15, 18$ a $24$ je $d=(15,18,24)=3$, $n=[15, 18, 24]=360$. Ďalej $15\cdot 18 \cdot 24 =6480$ a $(15,18,24)\cdot[15, 18, 24]=3\cdot 360=1080$, to však nie sú rovnaké čísla a tvrdenie neplatí. \\
\\
\kom Všeobecnejší pohľad na predchádzajúci problém by sme dostali skrz pohľad na exponenty prvočísel, z~ktorých sú čísla $a, b, c$ zložené. Skúmaná rovnosť nastane len v~prípade, že sú všetky tri čísla navzájom po dvoch nesúdeliteľné.

Zároveň úloha demonštruje riešenie uvedením protipríkladu, čo je princíp, s~ktorým sme sa v~seminároch zatiaľ nestretli a jeho spomenutie je určite vhodné.\\
\\
}


% Do not delete this line (pandoc magic!)

\problem{64-I-5-N5}{seminar08,nsdnsn}{
Ak majú prirodzené čísla $a, b$ najväčšieho spoločného deliteľa $d$, majú rovnakého najväčšieho spoločného deliteľa aj čísla $a$, $b$, $a - b$, $a + b$. Dokážte. Platí rovnaké tvrdenie pre najmenší spoločný násobok?
}{
\rie Najväčší spoločný deliteľ týchto štyroch čísel nebude určite väčší ako $d$ (ak by bol, potom by $d$ nebol najväčší spoločný deliteľ čísel $a$ a $b$, čo by bolo v~spore s~predpokladom úlohy). Stačí teda ukázať, že $d$ delí $a+b$ a $a-b$. Ak zapíšeme $a$ a $b$ v~tvare $a=ud$ a $b=vd$, pričom pre prirodzené čísla $u$, $v$ platí $(u,v)=1$, bude potom $a+b=ud+vd=(u+v)d$, $a-b=ud-vd=(u-v)d$. Vidíme, že $d$ delí súčet aj rozdiel čísel $a$ a $b$, tvrdenie je teda dokázané.

Tvrdenie pre najmenší spoločný násobok neplatí, uvedieme protipríklad. Pre čísla $a=12$, $b=8$, $a+b=20$, $a-b=4$, $[12,8]=24$, avšak $[12,8,20,4]=120$.\\
\\
\kom Úloha precvičuje dôkaz všeobecného tvrdenia a opäť prináša protipríklad ako dostatočný argument.\\
\\
}


% Do not delete this line (pandoc magic!)

\problem{61-I-3-N4, resp. 50-II-1}{
Nájdite všetky dvojice prirodzených čísel $a, b$, pre ktoré platí $a+b+[a, b]+(a, b) = 50$.
}{
\rieh Položme $a = ud$, $b = vd$, kde $d$ je najväčší spoločný deliteľ čísel $a, b$, prirodzené čísla $u, v$ sú nesúdeliteľné a $[a,b]=uvd$. Podľa zadania má platiť $ud+vd+uvd+d= 50$. Inak napísané, $(1 + u)(1 + v)d = 50$. Nájdime preto všetky rozklady čísla 50 na súčin troch prirodzených čísel $d, u+1, v+1$, z~ktorých posledné dve sú väčšie ako 1. Bez ujmy na všeobecnosti môžeme predpokladať, že $a \leq b$, tj. $u \leq v$. Dostaneme nasledujúce možnosti.
\begin{center}
\begin{tabular}{|c|c|c|c|c|c|c|}
\hline
$d$ & $u+1$ & $v+1$ & $u$ & $v$ & $a$ & $b$\\
\hline
1 & 2 & 25 & 1 & 24 & 1 & 24 \\
1 & 5 & 10 & 4 & 9 & 4 & 9\\
2 & 5 & 5 & 4 & 4 & 8 & 8 \\
5 & 2 & 5 & 1 & 4 & 5 & 20\\
\hline
\end{tabular}
\end{center}
V~prípade $d=2$ dostaneme $u=v=4$, to je však spor s~tým, že $u$ a $v$ sú nesúdeliteľné. Preto má úloha práve tri riešenia. \\
\\
\kom Úloha okrem vhodného zapísania čísel $a$, $b$ a $[a,b]$ vyžaduje ešte vhodnú úpravu rovnosti zo zadania, opäť tak kombinuje algebraické poznatky s~poznatkami z~oblasti teórie čísel.\\
\\
}


\problem{61-S-1}{
Nájdite všetky dvojice prirodzených čísel $a, b$, pre ktoré platí rovnosť množín
$$\{a \cdot [a, b], b \cdot (a, b)\} = \{45, 180\}.$$
}{
\rieh Z~danej rovnosti vyplýva, že číslo $b$ je nepárne (inak by obe čísla naľavo boli párne), a teda číslo $a$ je párne (inak by obe čísla naľavo boli nepárne). Rovnosť množín preto musí byť splnená nasledovne:
$$a \cdot [a, b] = 180 \ \ \ \text{a} \ \ \  b \cdot (a, b) = 45. \ \ \ (1)$$
Keďže číslo $a$ delí číslo $[a, b]$, je číslo $180 = 2^2 \cdot3^2 \cdot 5$ deliteľné druhou mocninou (párneho) čísla $a$, takže musí platiť buď $a = 2$, alebo $a = 6$.

V~prípade $a = 2$ (vzhľadom na to, že $b$ je nepárne) platí
$$a \cdot [a, b] = 2 \cdot [2, b] = 2 \cdot 2b = 4b,$$
čo znamená, že prvá rovnosť v~(1) je splnená jedine pre $b = 45$. Vtedy $b \cdot (a, b) = 45 \cdot (2, 45) = 45$, takže je splnená aj druhá rovnosť v~(1), a preto dvojica $a = 2, b = 45$ je riešením úlohy.

V~prípade $a = 6$ podobne dostaneme
$$a \cdot [a, b] = 6 \cdot [6, b] = 6 \cdot 2 \cdot [3, b] = 12 \cdot [3, b],$$
čo znamená, že prvá rovnosť v~(1) je splnená práve vtedy, keď $[3, b] = 15$. Tomu vyhovujú jedine hodnoty $b = 5$ a $b = 15$. Z~nich však iba hodnota $b = 15$ spĺňa druhú rovnosť v~(1), ktorá je teraz v~tvare $b\cdot(6, b) = 45$. Druhým riešením úlohy je teda dvojica $a = 6$, $b = 15$, žiadne ďalšie riešenia neexistujú.\\
\textit{Záver}. Hľadané dvojice sú dve, a to $a = 2, b = 45$ a $a = 6, b = 15$.\\
\\
\textbf{Iné riešenie.} Označme $d = (a, b)$. Potom $a = ud$ a $b = vd$, pričom $u, v$ sú nesúdeliteľné prirodzené čísla, takže $[a, b] = uvd$. Z~rovností
$$a \cdot [a, b] = ud \cdot uvd = u^2vd^2 \ \ \ \text{a} \ \ \ \ b \cdot (a, b) = vd \cdot d = vd^2$$
vidíme, že číslo $a \cdot [a, b]$ je $u^2$-násobkom čísla $b \cdot (a, b)$, takže zadaná rovnosť množín môže byť splnená jedine tak, ako sme zapísali vzťahmi (1) v~prvom riešení. Tie teraz môžeme vyjadriť rovnosťami
$$u^2vd^2= 180 \ \ \  \text{a} \ \ \ \ vd^2= 45.$$
Preto platí $u^2 =180/45= 4$, čiže $u = 2$. Z~rovnosti $vd^2 = 45 = 3^2 \cdot 5$ vyplýva, že buď $d = 1$ (a $v = 45$), alebo $d = 3$ (a $v = 5$). V~prvom prípade $a = ud = 2 \cdot 1 = 2$ a $b = vd = 45 \cdot 1 = 45$, v~druhom $a = ud = 2 \cdot 3 = 6$ a $b = vd = 5 \cdot 3 = 15$.\\
\\
\textit{Poznámka}. Keďže zo zadanej rovnosti okamžite vyplýva, že obe čísla $a, b$ sú deliteľmi čísla 180 (takým deliteľom je dokonca aj ich najmenší spoločný násobok $[a, b]$), je možné úlohu vyriešiť rôznymi inými cestami, založenými na testovaní konečného počtu dvojíc konkrétnych čísel $a$ a $b$. Takýto postup urýchlime, keď vopred zistíme niektoré nutné podmienky, ktoré musia čísla $a, b$ spĺňať. Napríklad spresnenie rovnosti množín na dvojicu rovností (1) možno (aj bez použitia úvahy o~parite čísel $a, b$) vysvetliť všeobecným postrehom: súčin $a \cdot [a, b]$ je vždy deliteľný súčinom $b \cdot (a, b)$, pretože ich
podiel možno zapísať v~tvare
$$ \frac{a \cdot [a, b]}{b \cdot (a, b)}=\frac{a}{(a, b)}\cdot\frac{[a, b]}{b},$$
teda ako súčin dvoch \textit{celých} čísel.\\
\\
\kom Úloha je zložitejšia ako predchádzajúce, dá sa však riešiť mnohými spôsobmi a bude iste zaujímavé vidieť rôzne študentské riešenia. Je taktiež vhodným miestom na to, aby sme študentov nechali diskutovať o~prístupoch medzi sebou a prípadne skúšali hľadať slabiny jednotlivých zdôvodnení. Určite považujeme za vhodné zmieniť poslednú rovnosť z~poznámky, keďže ide o~zaujímavý postreh a metóda vhodného zapísania tvaru zlomku je užitočná nielen tu. Na túto úlohu nadväzuje komplexnejšia domáca práca, ktorá však vychádza z~veľmi podobného princípu.\\
\\
}


% Do not delete this line (pandoc magic!)

\problem{64-I-5}{
Rozdiel dvoch prirodzených čísel je $2010$ a ich najväčší spoločný deliteľ je $2014$-krát menší ako ich najmenší spoločný násobok. Určte všetky také dvojice čísel.
}{
\rieh Označme hľadané čísla $a$ a $b$ $(a > b)$ a $d$ ich najväčší spoločný deliteľ. Potom $a = ud$, $b = vd$, pričom $u > v$ sú nesúdeliteľné čísla. Keďže najmenší spoločný násobok čísel $a, b$ je číslo $uvd$, dosadením do zadaných vzťahov dostaneme rovnosti
$$ a - b = (u~- v)d = 2 010,$$
$$uvd = 2 014d, \ \ \ \text{čiže} \ \ \  uv = 2 014.$$
Podľa rozkladu na súčin prvočísel $2 014 = 2\cdot19\cdot53$ vypíšeme všetky možné dvojice $(u, v)$ a pre každú z~nich sa presvedčíme, či číslo $u- v$ je deliteľom čísla $2010$. V~pozitívnom prípade príslušný podiel udáva číslo $d$ a výpočet neznámych $a = ud$ a $b = vd$ je už jednoduchý:

a) $u = 2 014$ a $v = 1$: $u - v~= 2 013$ nedelí 2 010;

b) $u = 19 \cdot 53 = 1 007$ a $v = 2$: $u - v~= 1 005 \mid 2 010$, $d = 2$, $a = 1 007 \cdot 2 = 2 014$, $b = 2 \cdot 2 = 4$;

c) $u = 2 \cdot 53 = 106$ a $v = 19$: $u - v~= 87$ nedelí 2 010;

d) $u = 53$ a $v = 2 \cdot 19 = 38$: $u - v= 15 \mid 2010$, $d = 134$, $a = 53 \cdot 134 = 7 102$, $b = 38 \cdot 134 = 5 092$.

\textit{Záver}. Hľadané čísla tvoria jednu z~dvojíc $(2014, 4)$ alebo $(7 102, 5 092)$.\\
\\
\kom Úloha neprináša žiadne nové poznatky a princípy, je však vhodná na trénovanie riešenia sústavy dvoch rovníc s~dvomi neznámymi a opäť tak vytvorí prepojenie s~minulými seminármi.\\
\\
}


% Do not delete this line (pandoc magic!)

\problem{60-I-5-D3}{seminar08,nsdnsn,vyrazy,domacekolo,doplnujuca}{
Nájdite všetky dvojice kladných celých čísel $a, b$, pre ktoré má výraz
$$\frac{a}{b}+\frac{14b}{9a}$$
celočíselnú hodnotu.
}{
\rieh Nech $d = (a, b)$, potom $a = ud$, $b = vd$ pre nesúdeliteľné prirodzené $u$ a $v$. Skúmaný výraz bude po dosadení $(9u^2+ 14v^2)/(9uv)$, takže $9u \mid 14v^2$ a z~nesúdeliteľnosti~$u$ a $v$ máme $u \mid 14$, navyše $3 \mid v$. Podobne $v \mid 9$; vyskúšame konečne veľa možností.\\
\\
\kom Úloha je zaujímavá tým, že prácu s~najväčším spoločným deliteľom obsahuje nepriamo a využíva tiež poznatky o~deliteľnosti z~minulého seminára.\\
\\
}


% Do not delete this line (pandoc magic!)

\problem{60-I-5}{
Dokážte, že najmenší spoločný násobok $[a, b]$ a najväčší spoločný deliteľ $(a, b)$ ľubovoľných dvoch kladných celých čísel $a, b$ spĺňajú nerovnosť
$$a \cdot (a, b) + b \cdot [a, b] \geq 2ab.$$
Zistite, kedy v~tejto nerovnosti nastane rovnosť.
}{
\rieh Nerovnosť by bolo ľahké dokázať, ak by niektorý z~dvoch sčítancov na ľavej strane bol sám osebe aspoň taký, ako pravá strana. Číslo $[a, b]$ je zjavne násobkom čísla $a$. Ak $[a, b] \geq 2a$, tak $b[a, b] \geq 2ab$ a v~zadanej nerovnosti platí dokonca ostrá nerovnosť, lebo číslo $a(a, b)$ je kladné. Ak $[a, b] < 2a$, tak neostáva iná možnosť ako $[a, b] = a$. To však nastane iba v~prípade, keď $b \mid a$. V~tomto prípade $(a, b) = b$ a v~zadanej nerovnosti
nastane rovnosť.\\
\\
\textbf{Iné riešenie*.} Označme $d = (a, b)$, takže $a = ud$ a $b = vd$ pre nesúdeliteľné prirodzené čísla $u, v$. Z~toho hneď vieme, že $[a, b] = uvd$. Keďže
\begin{align*}
a \cdot (a, b) + b \cdot [a, b]& = ud^2+ uv^2d^2= u(1 + v^2)d^2,\\
2ab&  = 2uvd^2,
\end{align*}
je vzhľadom na $ud^2 > 0$ nerovnosť zo zadania ekvivalentná s~nerovnosťou $1 + v^2 \geq 2v$, čiže $(v - 1)^2 \geq0$, čo platí pre každé $v$. Rovnosť nastane práve vtedy, keď $v = 1$, čiže $b \mid a$.\\
\\
\textbf{Iné riešenie*.} Označme $d = (a, b)$. Je známe, že $[a, b] \cdot (a, b) = ab$. Po vyjadrení $[a, b]$ z~tohto vzťahu, dosadení do zadanej nerovnosti a ekvivalentnej úprave dostaneme ekvivalentnú nerovnosť $d^2 + b^2 \geq 2bd$, ktorá platí, lebo $(d - b)^2 \geq 0$. Rovnosť nastáva
pre $d = b$, čiže v~prípade $b \mid a$.\\
\\
\kom Na úspešné zvládnutie úlohy je opäť potrebná znalosť z~predchádzajúceho seminára o~nerovnostiach a taktiež ponúka široké spektrum prístupov, takže bude zaujímavé sledovať, ako k~nej študenti pristúpia.
}




\subsection*{Domáca práca}

% Do not delete this line (pandoc magic!)

\problem{61-I-3}{
Nájdite všetky trojice prirodzených čísel $a, b, c$, pre ktoré platí množinová rovnosť
$$\{(a, b), (a, c), (b, c), [a, b], [a, c], [b, c]\}= \{2, 3, 5, 60, 90, 180\},$$
pričom $(x, y)$ a $[x, y]$ označuje postupne najväčší spoločný deliteľ a najmenší spoločný násobok čísel $x$ a $y$.
}{
\rieh Prvky danej množiny $M$ rozložíme na prvočinitele:
$$M = \{2, 3, 5, 2^2 \cdot 3 \cdot 5, 2 \cdot 3^2 \cdot 5, 2^2 \cdot 3^2 \cdot 5\}.$$
Odtiaľ vyplýva, že v~rozklade hľadaných čísel $a, b, c$ vystupujú iba prvočísla 2, 3 a 5. Každé z~nich je pritom prvočiniteľom práve dvoch z~čísel $a, b, c$: keby bolo prvočiniteľom len jedného z~nich, chýbalo by v~rozklade troch najväčších spoločných deliteľov a jedného najmenšieho spoločného násobku, teda v~štyroch číslach z~$M$; keby naopak bolo prvočiniteľom všetkých troch čísel $a, b, c$, nechýbalo by v~rozklade žiadneho čísla z~$M$. Okrem toho vidíme, že v~rozklade každého z~čísel $a, b, c$ je prvočíslo 5 najviac v~jednom exemplári.

Podľa uvedených zistení môžeme čísla $a, b, c$ usporiadať tak, že rozklady čísel $a, b$ obsahujú po jednom exemplári prvočísla 5 (potom $(c, 5) = 1$) a že $(a, 2) = 2$ (ako vieme, aspoň jedno z~čísel $a, b$ musí byť párne). Číslo 5 z~množiny $M$ je potom nutne rovné $(a, b)$, takže platí $(b, 2) = 1$, a preto $(b, 3) = 3$ (inak by platilo $(b, c) = 1$), odtiaľ zase s~ohľadom na $(a, b) = 5$ vyplýva $(a, 3) = 1$. Máme teda $a = 5 \cdot 2^s$ a $b = 5 \cdot 3^t$ pre vhodné prirodzené čísla $s$ a $t$.

Z~rovnosti $[a, b] = 2^s \cdot3^t \cdot5$ vyplýva, že nastane jeden z~troch nasledujúcich prípadov.

(1) $2^s \cdot 3^t \cdot 5 = 60 = 2^2 \cdot 3^1 \cdot 5$. Vidíme, že platí $s = 2$ a $t = 1$, čiže $a = 20$ a $b = 15.$ Ľahko určíme, že tretím číslom je $c = 18$.

(2) $2^s \cdot 3^t \cdot 5 = 90 = 2^1 \cdot 3^2 \cdot 5$. V~tomto prípade $a = 10$, $b = 45$ a $c = 12$.

(3) $2^s \cdot 3^t \cdot 5 = 180 = 2^2 \cdot 3^2 \cdot 5$. Teraz $a = 20$, $b = 45$ a $c = 6$.\\
\textit{Záver}. Hľadané čísla $a, b, c$ tvoria jednu z~množín $\{20, 15, 18\}$, $\{10, 45, 12\}$ a $\{20, 45, 6\}$.\\
\\
\textbf{Iné riešenie.} V~danej rovnosti je množina napravo tvorená šiestimi rôznymi číslami väčšími ako 1, takže čísla $(a, b), (a, c), (b, c)$ musia byť netriviálnymi deliteľmi postupne čísel $[a, b], [a, c], [b, c]$. Čísla 2, 3, 5 ale žiadne netriviálne delitele nemajú, musí teda platiť
$$\{(a, b), (a, c), (b, c)\}= \{2, 3, 5\} \ \ \ \text{a} \ \ \  \{[a, b], [a, c], [b, c]\} = \{60, 90, 180\}.$$
Pretože poradie čísel $a, b, c$ nehrá žiadnu úlohu, môžeme predpokladať, že platí $(a, b) = 2, (a, c) = 3$ a $(b, c) = 5$. Odtiaľ vyplývajú vyjadrenia
$$a = 2 \cdot 3 \cdot x = 6x, \ \ \ b = 2 \cdot 5 \cdot y = 10y, \ \ \ c = 3 \cdot 5 \cdot z~= 15z$$
pre vhodné prirodzené čísla $x, y, z$. Zo známej rovnosti $[x, y]\cdot(x, y) = xy$ tak dostaneme vyjadrenia najmenších spoločných násobkov v~tvare
$$[a, b] =\frac{6x \cdot 10y}{2}= 30xy, \ \ \ [a, c] =\frac{6x \cdot 15z}{3}= 30xz,\ \ \  [b, c] =\frac{10y \cdot 15z}{5}= 30yz.$$
Z~rovnosti $\{30xy, 30xz, 30yz\} = \{60, 90, 180\}$ upravenej na $\{xy, xz, yz\} = \{2, 3, 6\}$ potom vďaka tomu, že 2 a 3 sú prvočísla, vyplýva $\{x, y, z\} = \{1, 2, 3\}$. Pretože z~podmienky $5 = (b, c) = (10y, 15z)$ vyplýva $y \neq 3$ a $z \neq 2$, prichádzajú do úvahy len trojice $(x, y, z)$ rovné (1, 2, 3), (2, 1, 3) a (3, 2, 1), ktorým postupne zodpovedajú trojice $(a, b, c)$ rovné (6, 20, 45), (12, 10, 45), (18, 20, 15). Skúškou sa presvedčíme, že všetky tri vyhovujú množinovej rovnosti zo zadania úlohy.\\
\\
}


% Do not delete this line (pandoc magic!)

\problem{63-S-2}{seminar08,nsdnsn,skolskekolo}{
Čísla 1, 2,\,\ldots , 10 rozdeľte na dve skupiny tak, aby najmenší spoločný násobok súčinu všetkých čísel prvej skupiny a súčinu všetkých čísel druhej skupiny bol čo najmenší.
}{
\rieh Pre uvažované súčiny $a$ a $b$ určite platí $a \cdot b = 1 \cdot 2 \cdot\,\ldots \cdot 10 = 2^8 \cdot 3^4 \cdot 5^2 \cdot 7$. Aspoň jedno z~čísel $a, b$ je preto deliteľné $2^4$, aspoň jedno deliteľné $3^2$, aspoň jedno deliteľné 5 a práve jedno deliteľné 7. Pre najmenší spoločný násobok $n$ čísel $a, b$ preto platí $n \geq 2^4 \cdot 3^2 \cdot 5 \cdot 7 = 5 040$, pritom rovnosť tu nastane práve vtedy, keď ani jedno z~čísel $a$, $b$ nebude deliteľné žiadnym z~čísel $2^5, 3^3$ a $5^2$.

Ak zvolíme napríklad $a = 2 \cdot 3 \cdot 4 \cdot 5 \cdot 6 = 720$ a $b = 1 \cdot 7 \cdot 8 \cdot 9 \cdot 10 = 5040$, bude najmenší spoločný násobok oboch čísel práve $5040$. Tým je ukázané, že $5040$ je naozaj najmenšia zo všetkých možných hodnôt $n$.

I~keď bolo úlohou nájsť iba jeden príklad, pre úplnosť uvedieme všetky rozdelenia s~minimálnou hodnotou $n = 5040$:
\begin{center}
\begin{tabular}{c c}
\hline
Prvá skupina čísel & Druhá skupina čísel \\
\hline
2, 3, 4, 5, 6 &1, 7, 8, 9, 10\\
3, 5, 6, 8 & 1, 2, 4, 7, 9, 10\\
2, 5, 8, 9 & 1, 3, 4, 6, 7, 10\\
1, 2, 3, 4, 5, 6 & 7, 8, 9, 10\\
1, 3, 5, 6, 8 & 2, 4, 7, 9, 10\\
1, 2, 5, 8, 9 & 3, 4, 6, 7, 10\\
2, 3, 4, 5, 6, 7 & 1, 8, 9, 10\\
3, 5, 6, 7, 8 & 1, 2, 4, 9, 10\\
2, 5, 7, 8, 9 & 1, 3, 4, 6, 10\\
1, 2, 3, 4, 5, 6, 7 & 8, 9, 10\\
1, 3, 5, 6, 7, 8 & 2, 4, 9, 10\\
1, 2, 5, 7, 8, 9 & 3, 4, 6, 10
\end{tabular}
\end{center}

Nájsť ich nie je ťažké, keď si uvedomíme, že čísla 1 a 7 môžeme dať do ľubovoľnej z~oboch skupín, zatiaľ čo v~tej istej skupine spolu nemôžu byť 4 s~8, 5 s~10, 3 s~9 ani 6 s~9; s~8 spolu môže byť práve jedno z~párnych čísel 2, 6 a 10. Získame tak iba tri základné rozdelenia (prvé tri riadky tabuľky), z~ktorých možno každé štyrmi spôsobmi doplniť číslami 1 a 7.\\
\\
\textit{Poznámka}. Úlohu možno vyriešiť aj bez výpočtu súčinu $a \cdot b$. Deliteľnosť $n$ číslami $3^2, 5$ a 7 vyplýva z~ich priameho zastúpenia medzi rozdeľovanými číslami, deliteľnosť číslom $2^4$ z~jednoduchej úvahy o~rozdelení všetkých piatich párnych čísel: ak nie je číslo 8 vo svojej skupine ako párne jediné, je všetko jasné, v~opačnom prípade sú v~rovnakej skupine čísla 2, 4 a 6 (aj 10, ale to už ani nepotrebujeme).\\
}



\teachernote{
\subsection*{Doplňujúce zdroje a materiály}
Materiály vhodné na ďalšie počítanie nájdeme v~minulom seminári. Keďže témy sú si veľmi blízke, publikácie zvyčajne obsahujú úlohy zamerané na obe témy.
%Prečo bol Heisenberg zlý milenec? Lebo keď našiel správnu polohu, nemal správnu rýchlosť. A keď mal energiu, nemal čas.

}

\seminar{9}

\subsection*{Téma}
Geometria I~-- základné poznatky
\teachernote{
\subsection*{Ciele}
Zopakovať a upevniť základné poznatky z~planimetrie, ktoré by študenti mali mať zo základnej školy. Venovať sa vlastnostiam uhlov, trojuholníkov, štvoruholníkov a kružníc. Niektoré z~poznatkov odvodiť.

\textbf{Úvodný komentár.} Keďže planimetria nie je súčasťou osnov 1. ročníka gymnázií, je potrebné poznatky žiakov z~tejto oblasti o~to starostlivejšie zopakovať. Geometrické úlohy majú veľmi často najhoršiu úspešnosť v~krajských kolách MO, čo môže mať viacero dôvodov. Nepopierateľne však študentom tréning pomôže, preto je geometrii v~priebehu roka venovaných $6+1$ seminárov.

Zo zmienených dôvodov má preto tento seminár odlišnú štruktúru ako predchádzajúce -- viac ako riešeniu úloh z~olympiád sa venujeme opakovaniu základných vlastností uhlov, trojuholníkov, štvoruholníkov a kružníc, ktorých znalosti budú nenahraditeľné v~ďalších piatich geometrických seminároch. Spolu so študentmi tak vytvoríme základnú výbavu, ktorá im pomôže v~boji s~geometrickými záludnosťami.

Študenti by mali mať nasledujúce znalosti (voľne spracované podľa [XX]TODO):
\begin{itemize}
\item uhly
\begin{itemize}
\item chápať pojmy vrcholové, vedľajšie, súhlasné a striedavé uhly, vedieť nájsť dvojice takých uhlov a používať ich pri riešení úloh,
\end{itemize}
\item trojuholníky
\begin{itemize}
\item poznať základné vlastnosti strán a vnútorných uhlov trojuholníka: trojuholníková nerovnosť, súčet vnútorných uhlov,
\item vedieť popísať rozdiely medzi ostrouhlým, pravouhlým, tupouhlým, všeobecným, rovnoramenným a rovnostranným trojuholníkom,
\item chápať pojmy os uhla, os strany, výška, ťažnica, stredná priečka, kružnica vpísaná a opísaná trojuholníku a poznať ich vlastnosti,
\item poznať a vedieť používať vzorec na výpočet obsahu trojuholníka,
\item poznať a vhodne používať vety o~zhodnosti ($sss$, $sus$, $usu$, $Ssu$) a podobnosti ($sss$, $sus$, $uu$, $Ssu$) trojuholníkov,
\item poznať a používať Pytagorovu vetu pre pravouhlý trojuholník,
\end{itemize}
\item štvoruholníky
\begin{itemize}
\item vedieť popísať všeobecný štvoruholník a jeho špecifické prípady: rovnobežník, štvorec, obdĺžnik, kosoštvorec, kosodĺžnik, lichobežník,
\item poznať základné vzorce pre výpočet obsahu rôznych rovnobežníkov a lichobežníkov,
\item vedieť, že uhlopriečky v~pravouholníku a rovnobežníku sa polia a vedieť tento fakt využiť pri riešení úloh,
\end{itemize}
\item kružnice a kruhy
\begin{itemize}
\item chápať pojmy kružnica, kruh, kružnicový oblúk, dotyčnica, sečnica, tetiva, stredový a obvodový uhol,
\item poznať a vedieť používať Talesovu kružnicu,
\end{itemize}
\item riešenie konštrukčných úloh
\begin{itemize}
\item náčrt, rozbor, popis konštrukcie, diskusia o~počte riešení.
\end{itemize}
\end{itemize}
\kom Skôr než frontálny výklad je vhodné nechať skladať mozaiku vedomostí študentov. Ak pracujeme s~malou skupinou, môžeme o~vyššie spomenutých bodoch diskutovať všetci spoločne. Ak seminár navštevuje väčšie množstvo záujemcov o~matematiku, rozdelíme študentov na menšie skupiny, pričom každá spracuje poznatky o~zadanej neprázdnej podmnožine vyššie spomenutých oblastí. Tie si potom študenti navzájom odprezentujú, vedúci seminára nepresnosti vhodnými otázkami koriguje.
\\
\kom V~druhej polovici seminára niektoré zo základných tvrdení, ktoré budeme v~priebehu ďalších stretnutí využívať, dokážeme.\\
}

\subsection*{Úlohy a riešenia}

% % Do not delete this line (pandoc magic!)

\problem{anonymous 1}{
Dokážte, že súčet veľkostí vnútorných uhlov ľubovoľného trojuholníka je $180^\circ$.
}{
\rie Veďme rovnobežku $XY$ so stranou $AB$ vrcholom $C$ trojuholníka $ABC$, tak že bod~$C$ leží medzi bodmi $X$ a $Y$ \todo{(obr.1)}. \\
\\
\todo{DOPLNIŤ Obr. 1}\\
\\
Potom $|\ma BAC|=|\ma ACX|$ a $|\ma ABC|=|\ma BCY|$, pretože ide o~dvojice striedavých uhlov. Keďže $|\ma ACX|+|\ma ACB|+ |\ma BCY|=180^\circ$, pretože uhol $XCY$ je priamy, platí aj $|\ma BAC|+|\ma ABC|+|\ma ACB|=180^\circ$.
}


% % Do not delete this line (pandoc magic!)

\problem{66-I-3-N1}{seminar10,geomlah,domacekolo}{
Z~trojuholníkových nerovností medzi dĺžkami strán ľubovoľného trojuholníka odvoďte známe
pravidlo $\alpha < \beta \Rightarrow a < b$ o~porovnaní veľkostí vnútorných uhlov a dĺžok protiľahlých strán v~ľubovoľnom trojuholníku $ABC$.
}{
\rieh Ak je $\alpha  < \beta$, môžeme nájsť vnútorný bod $X$ strany $AC$, pre ktorý platí $|\ma ABX| = \alpha$, a teda $|AX| = |BX|$, takže z~trojuholníkovej nerovnosti $|BC| < |BX| + |XC|$ už vyplýva $a < b$.
\begin{figure}[h]
    \centering
    \includegraphics{images/66I3N1\imagesuffix}
    \caption{}
    \label{fig:66I3N1}
\end{figure}
}


% % Do not delete this line (pandoc magic!)

\problem{63-I-4-N3}{
Dokážte vety:

a) Ak majú dva trojuholníky rovnakú výšku, potom pomer ich obsahov sa rovná pomeru dĺžok príslušných základní.

b) Ak majú dva trojuholníky zhodné základne, potom pomer ich obsahov sa rovná pomeru príslušných výšok.
}{
\rie a) Označme rovnakú výšku dvoch trojuholníkov $v$. V~trojuholníku $T_1$ je táto výškou na základňu $a_1$, v~trojuholníku $T_2$ na základňu $a_2$. Pomer obsahov týchto trojuholníkov je potom $$\frac{S_{T_1}}{S_{T_2}}=\frac{\frac{1}{2}a_1v}{\frac{1}{2}a_2v}=\frac{a_1}{a_2},$$ čo sme chceli dokázať.

b) Označme zhodnú základňu dvoch trojuholníkov $z$, v~trojuholníku $T_1$ je výška na túto základňu $v_1$, v~trojuholníku $T_2$ je výška na túto základňu $v_2$. Pomer obsahov trojuholníkov $T_1$ a $T_2$ je
$$\frac{S_{T_1}}{S_{T_2}}=\frac{\frac{1}{2}zv_1}{\frac{1}{2}zv_2}=\frac{v_1}{v_2},$$
čo je pomer príslušných výšok.\\
\\
}


% % Do not delete this line (pandoc magic!)

\problem{61-I-5-N1}{
Pre všeobecný trojuholník $ABC$ so stranami $a$, $b$, $c$ a obsahom $S$ platí pre polomer $r$ vpísanej kružnice vzorec $r = 2S/(a + b + c)$. Dokážte.
}{
\rieh Stred $M$ vpísanej kružnice rozdeľuje uvažovaný trojuholník $ABC$ na tri menšie trojuholníky $BCM$, $ACM$, $ABM$ s~obsahmi $\frac{1}{2}ar$, $\frac{1}{2}br$, $\frac{1}{2}cr$, ktorých súčet je $S$, odkiaľ vyplýva dokazovaný vzorec.\\
\\
\todo{DOPLNIŤ Obr.3}
}


% % Do not delete this line (pandoc magic!)

\problem{anonymous 2}{
Dokážte, že uhlopriečky v~rovnobežníku sa navzájom polia.
}{
\rie Označme $U$ priesečník uhlopriečok $AC$ a $BD$ rovnobežníka $ABCD$ (obr.~\ref{fig:anon2}).
\begin{figure}[h]
    \centering
    \includegraphics{images/anon2\imagesuffix}
    \caption{}
    \label{fig:anon2}
\end{figure}
Keďže uhly $ABD$ a $BDC$ sú striedavé, majú rovnakú veľkosť. Podobne uhly $BAC$ a $ACD$ sú rovnako veľké, pretože sú takisto dvojicou striedavých uhlov. Potom sú trojuholníky $ABU$ a $CDU$ zhodné, keďže sa zhodujú v~jednej strane $|AB|=|CD|$ a v~dvoch k~nej priľahlých uhloch. Preto aj $|AU|=|UC|$, $|BU|=|UD|$ a tvrdenie je dokázané.
}


% % Do not delete this line (pandoc magic!)

\problem{58-I-4-N1}{
Označme $U$ priesečník uhlopriečok daného konvexného štvoruholníka $ABCD$. Dokážte, že priamky $AB$ a $CD$ sú rovnobežné práve vtedy, keď trojuholníky $ADU$ a $BCU$ majú rovnaký obsah.
}{
\rie Rovnosť obsahov trojuholníkov $ADU$ a $BCU$ je ekvivalentná s~rovnosťou obsahov trojuholníkov $ABC$ a $ABD$ so spoločnou stranou $AB$, pretože $S_{ABC}=S_{ABU}+S_{BCU}$ a $S_{ABD}=S_{ABU}+S_{AUD}$. Trojuholníky $ABC$ a $ABD$ majú spoločnú základňu $AB$, takže ich obsahy budú rovnaké práve vtedy, ak výšky na túto stranu budú rovnaké, resp. ak body $C$ a $D$ budú od priamky $AB$ rovnako vzdialené. To nastane len v~prípade, ak body $C$ a $D$ ležia na priamke rovnobežnej s~priamkou $AB$, čo sme chceli dokázať.\\
\\
\todo{DOPLNIŤ Obr. 5}
}


% \problem{64-I-4-N1}{
Lichobežník $ABCD$ má základne s~dĺžkami $|AB|=a$ a $|CD|=C$ a jeho uhlopriečky sa pretínajú v~bode $U$. Aký je pomer obsahov trojuholníkov $ABU$ a $CDU$?
}{
\rie Trojuholníky $ABU$ a $CDU$ sú zrejme podobné ($|\ma BAU|=|\ma UCD|$, $|\ma ABU|=|\ma CDU|$, $|\ma AUB|=|\ma CUD|$, pretože prvé dve sú dvojice striedavých uhlov, posledné dva sú uhly vrcholové) s~koeficientom podobnosti $k=a/c$. Preto pre výšku $v_1$ na stranu $AB$ v~trojuholníku $ABU$ a výšku $v_2$ na stranu $CD$ v~trojuholníku $CDU$ platí $v_1/v_2=k$, resp. $v_1=kv_2=(av^2)/c$. Potom pre pomer obsahov trojuholníkov $ABU$ a $CDU$ máme
$$\frac{S_{ABU}}{S_{CDU}}=\frac{\frac{1}{2}av_1}{\frac{1}{2}cv_2}=\frac{a\frac{a}{cv_2}}{cv_2}=\frac{a^2}{c^2}.$$\\
\\
\textbf{Záverečný komentár} Na prvý pohľad by sa mohlo zdať, že študenti budú o(c)hromení množstvom nových poznatkov v~tomto seminári. Dúfame však, že sa tak nestane, keďže veľká väčšina obsahu by mala byť prinajmenšom povedomá, ak nie úplne zrozumiteľná. Seminár tiež patrí k~tým menej náročným, avšak je veľmi dôležitou prípravou pred tvrdšími orieškami.
}




\subsection*{Domáca práca}

% Do not delete this line (pandoc magic!)

\problem{58-I-2-D1}{}{
Nech $k$ je kružnica opísaná pravouhlému trojuholníku $ABC$ s~preponou $AB$ dĺžky $c$. Označme $S$ stred strany $AB$ a $D$ a $E$ priesečníky osí strán $BC$ a $AC$ s~jedným oblúkom $AB$ kružnice $k$. Vyjadrite obsah trojuholníka $DSE$ pomocou dĺžky prepony $c$.
}{
\rie Trojuholník $DSE$ je pravouhlý rovnoramenný s~pravým uhlom pri vrchole $S$, pretože odvesny $DS$ a $ES$ ležia na osiach navzájom kolmých strán. Odvesny majú dĺžku $\frac{c}{2}$, pretože sú to polomery kružnice opísanej trojuholníku $ABC$. Obsah trojuholníka $DSE$ je $\frac{1}{2}\cdot|DS|\cdot |DE|=\frac{1}{2}\cdot \frac{c}{2}\cdot\frac{c}{2}=\frac{c^2}{8}.$ \\
\\
}


% Do not delete this line (pandoc magic!)

\problem{58-I-2-D2}{seminar10,geomlah,domacekolo}{
Vyjadrite obsah rovnoramenného lichobežníka $ABCD$ so základňami $AB$ a $CD$ pomocou dĺžok $a$, $c$ jeho základní a dĺžky $b$ jeho ramien.
}{
\rie Bez ujmy na všeobecnosti môžeme predpokladať, že $a>b$. Najprv vyjadríme výšku $v$ pomocou dĺžok základní a odvesien. Nech je $P$ päta výšky z~bodu $D$ na stranu $AB$. Potom $|AP|=(a-c)/2$. Použitím Pytagorovej vety v~pravouhlom trojuholníku $APD$ máme
$$\bigg(\frac{a-c}{2}\bigg)^2+v^2=b^2,$$
odkiaľ $v=\sqrt{b^2-(\frac{a-c}{2})^2}=\frac{1}{2}\sqrt{4b^2-(a-c)^2}$ a preto pre obsah lichobežníka dostávame $$S_{ABCD}=\frac{1}{2}(a+c)\cdot v=\frac{1}{4}(a+c)\sqrt{4b^2-(a-c)^2}.$$
}


% % Do not delete this line (pandoc magic!)

\problem{anonymous 3}{
Použitím viet o~podobnosti trojuholníkov a Pytagorovej vety odvoďte Euklidove vety o~odvesne a o~výške pravouhlého trojuholníka.
}{
\rie Prehľadné odvodenie je možne nájsť v ~\cite{kadlecek1996}.
}



\teachernote{
\subsection*{Doplňujúce zdroje a materiály}
Ak študenti budú stále neistí v~používaní základných geometrických poznatkov, je možné ich odkázať na základoškolské učebnice geometrie, v~ktorých nájdu aj jednoduchšie príklady na precvičenie, príp. vhodným doplnkom geometrického vzdelania je aj publikácia [~\cite{kadlecek1996}].
}


\section{December}
\seminar{10}

\weblinks{Na stiahnutie: \href{pdf/seminar10-teacher.pdf}{učiteľská verzia}, \href{pdf/seminar10-student.pdf}{študentská verzia}}

\subsection*{Téma}
Geometria II -- podobné trojuholníky a Pytagorova veta

\subsection*{Ciele}
Precvičiť riešenie úloh vhodným (viacnásobným) využitím Pytagorovej vety a dvojíc podobných trojuholníkov

\subsection*{Úlohy a riešenia}

% Do not delete this line (pandoc magic!)

\problem{66-S-3}{seminar11,pytveta,skolskekolo}{
Päta $P$ výšky z~vrcholu $C$ v~trojuholníku $ABC$ delí stranu $AB$ v~pomere $|AP| : |PB|= 1 : 3$. V~rovnakom pomere sú aj obsahy štvorcov nad jeho stranami $AC$ a $BC$.
Dokážte, že trojuholník $ABC$ je pravouhlý.
}{
\rieh Označme $d$ dĺžku úsečky $AP$ a $v$ dĺžku výšky $CP$ trojuholníka $ABC$. Dĺžky jeho strán označíme zvyčajným spôsobom $a, b, c$. Zo zadania teda vyplýva $|PB| = 3d$.
\begin{figure}
    \centering
    \includegraphics{images/66S3\imagesuffix}
    \caption{}
    \label{fig:66S3}
\end{figure}

Použitím Pytagorovej vety v~trojuholníkoch $APC$ a $PBC$ dostávame rovnosti $b^2= d^2 +v^2$ a $a^2 = 9d^2 +v^2$. Z~druhého predpokladu úlohy potom vyplýva rovnosť $a^2 = 3b^2$, čiže $9d^2 + v^2 = 3d^2 + 3v^2$, odkiaľ $v^2 = 3d^2$. Dosadením do prvých dvoch rovností tak dostávame $a^2 = 12d^2$ a $b^2 = 4d^2$. A~keďže $c = 4d$, čiže $c^2 = 16d^2$, dokázali sme, že pre dĺžky strán trojuholníka $ABC$ platí $a^2 + b^2 = c^2$.

Trojuholník $ABC$ je preto podľa obrátenej Pytagorovej vety pravouhlý.\\
\\
\textit{Poznámka.} Ak zvážime pomocný pravouhlý trojuholník s~odvesnami $a$ a $b$, tak pre jeho preponu~$c'$ podľa Pytagorovej vety platí $c' = a^2 + b^2$. Porovnaním s~odvodenou rovnosťou $c^2 = a^2 + b^2$ tak dostávame $c'= c$, takže pôvodný trojuholník je podľa vety $sss$ zhodný s~trojuholníkom pomocným, a je teda skutočne pravouhlý. Môžeme tolerovať názor, že samotná Pytagorova veta udáva nielen nutnú, ale aj postačujúcu podmienku na to, aby bol daný trojuholník pravouhlý.\\
\\
\kom Úloha relatívne priamočiaro využíva viacnásobné využitie Pytagorovej vety, je tak vhodným zahrievacím problémom tohto seminára.\\
\\
}


% Do not delete this line (pandoc magic!)

\problem{66-I-3}{seminar11,pytveta,geompoc,domacekolo}{
Päta výšky z~vrcholu $C$ v~trojuholníku $ABC$ delí stranu $AB$ v~pomere $1 : 2$. Dokážte, že pri zvyčajnom označení dĺžok strán trojuholníka $ABC$ platí nerovnosť $$3|a - b| < c.$$
}{
\rieh Päta $D$ uvažovanej výšky je podľa zadania tým vnútorným bodom strany $AB$, pre ktorý platí $|AD| = 2|BD|$ alebo $|BD| = 2|AD|$. Obe možnosti sú znázornené na obr.~\ref{fig:66I3_1} s~popisom dĺžok strán $AC$, $BC$ a oboch úsekov rozdelenej strany $AB$.
\begin{figure}[h]
    \centering
    \includegraphics{images/66D31\imagesuffix}
    \caption{}
    \label{fig:66I3_1}
\end{figure}
Pytagorova veta pre pravouhlé trojuholníky $ACD$ a $BCD$ vedie k~dvojakému vyjadreniu druhej mocniny spoločnej odvesny $CD$, pričom v~situácii naľavo dostaneme
$$|CD|^2= b^2- \bigg(\frac{2}{3}c\bigg)^2= a^2 - \bigg(\frac{1}{3}c\bigg)^2,$$
odkiaľ po jednoduchej úprave poslednej rovnosti dostaneme vzťah
$$3(b^2 - a^2) = c^2.$$
Pre druhú situáciu vychádza analogicky
$$3(a^2 -b^2) = c^2.$$
Závery pre obe možnosti možno zapísať jednotne ako rovnosť s~absolútnou hodnotou
$$3|a^2 - b^2 | = c^2.$$
Ak použijeme rozklad $|a^2 - b^2 | = |a - b|(a + b)$ a nerovnosť $c < a + b$ (ktorú ako je známe spĺňajú dĺžky strán každého trojuholníka $ABC$), dostaneme z~odvodenej rovnosti
$$3|a - b|c < 3|a - b|(a + b) = c^2,$$
odkiaľ po vydelení kladnou hodnotou $c$ dostaneme $3|a - b| < c$, ako sme mali dokázať. Zdôraznime, že nerovnosť $3|a-b|c < 3|a-b|(a+b)$ sme správne zapísali ako ostrú -- v~prípade $a = b$ by síce prešla na rovnosť, avšak podľa nášho odvodenia by potom platilo $c^2 = 0$, čo odporuje tomu, že ide o~dĺžku strany trojuholníka.\\
\\
\textbf{Iné riešenie*.} Nerovnosť, ktorú máme dokázať, možno po vydelení tromi zapísať bez
absolútnej hodnoty ako dvojicu nerovností
$$-\frac{1}{3}c < a - b < \frac{1}{3}c.$$
Opäť ako v~pôvodnom riešení rozlíšime dve možnosti pre polohu päty $D$ uvažovanej výšky a ukážeme, že vypísanú dvojicu nerovností možno upresniť na tvar
$$-\frac{1}{3} < a - b < 0,\ \ \ \ \text{respektíve} \ \ \ \  0 < a - b <\frac{1}{3}c,$$
podľa toho, či nastáva situácia z~ľavej či pravej časti~obr.~\ref{fig:66I3_1}.

Pre situáciu z~obr.~\ref{fig:66I3_1} naľavo prepíšeme avizované nerovnosti $-\frac{1}{3}c < a - b < 0$ ako $a < b < a +\frac{1}{3}c$ a odvodíme ich z~pomocného trojuholníka $ACE$, pričom $E$ je stred úsečky $AD$, takže body $D$ a $E$ delia stranu $AB$ na tri zhodné úseky dĺžky $\frac{1}{3}c$.
\begin{figure}[h]
    \centering
    \includegraphics{images/66D32\imagesuffix}
    \caption{}
    \label{fig:66I3_2}
\end{figure}
V~obr.~\ref{fig:66I3_2} sme rovno vyznačili, že úsečka $EC$ má dĺžku $a$ ako úsečka $BC$, a to v~dôsledku zhodnosti trojuholníkov $BCD$ a $ECD$ podľa vety $sus$. Preto je pravá z~nerovností $a < b < a +\frac{1}{3}c$ porovnaním dĺžok strán trojuholníka $ACE$, ktoré má všeobecnú platnosť.

Ľavú nerovnosť $a < b$ odvodíme z~druhého všeobecného poznatku, že totiž v~každom trojuholníku oproti väčšiemu vnútornému uhlu leží dlhšia strana. Stačí nám teda zdôvodniť, prečo pre uhly vyznačené na~obr.~\ref{fig:66I3_2} platí $|\ma CAE| < |\ma AEC|$. To je však jednoduché: zatiaľ čo uhol $CAE$ je vďaka pravouhlému trojuholníku $ACD$ ostrý, uhol $AEC$ je naopak tupý, pretože k~nemu vedľajší uhol $CED$ je ostrý vďaka pravouhlému trojuholníku $CED$.

Pre prípad situácie z~obr.~\ref{fig:66I3_1} napravo možno predchádzajúci postup zopakovať s~novým bodom $E$, tentoraz stredom úsečky $BD$. Môžeme však vďaka súmernosti podľa osi $AB$ konštatovať, že z~dokázaných nerovností $-\frac{1}{3}c < a - b < 0$ pre situáciu naľavo vyplývajú nerovnosti $-\frac{1}{3}c < b - a < 0$ pre situáciu napravo, z~ktorých po vynásobení číslom $-1$ dostaneme práve nerovnosti $0 < a - b <\frac{1}{3}$, ktoré sme mali v~druhej situácii dokázať.\\
\\
\kom Nosným prvkom úlohy je opäť Pytagorova veta, väčšiu pozornosť však vyžaduje rozbor úlohy, keďže päta výšky sa môže nachádzať v~dvoch rôznych polohách.\\
\\
}


% Do not delete this line (pandoc magic!)

\problem{63-S-3}{
Daný je trojuholník $ABC$ s~pravým uhlom pri vrchole $C$. Stredom $I$ kružnice trojuholníku vpísanej vedieme rovnobežky so stranami $CA$ a $CB$, ktoré pretnú preponu postupne v~bodoch $X$ a $Y$. Dokážte, že platí $|AX|^2 + |BY |^2 = |XY |^2$.
}{
\rieh Trojuholník $AIX$ je rovnoramenný, pretože $|\ma IAX| = |\ma IAC| = | \ma AIX|$ (prvá rovnosť vyplýva z~podmienky, že bod $I$ leží na osi uhla $BAC$, druhá potom z~vlastností striedavých uhlov, obr. 4). Preto $|AX| = |IX|$. Analogicky zistíme, že $|BY | = |Y I|$. Keďže úsečky $IX$ a $IY$ zvierajú (rovnako ako s~nimi rovnobežné úsečky $CA$ a $CB$) pravý uhol, podľa Pytagorovej vety pre pravouhlý trojuholník $XIY$ platí $$|AX|^2+ |BY |^2= |IX|^2+ |Y I|^2= |XY |^2,$$
čo sme mali dokázať.
\begin{center}
\includegraphics{images/63S3\imagesuffix}\\

Obr. 4
\end{center}
\kom Úloha už vyžaduje trochu viac invencie a postrehu, keďže kľúčovým krokom v~riešení je všimnúť si, že trojuholníky $AIX$ a $BIY$ sú rovnoramenné. K~tomu však študentov môže naviesť poloha bodu $I$, ktorý leží na osi uhlov a to, že rovnobežky $AC$ a $XI$, resp. $BC$ a $YI$ sú preťaté priečkami $AI$, resp. $BI$, takže v~náčrtku vieme nájsť niekoľko dvojíc zhodných uhlov. Úloha tak kombinuje použitie Pytagorovej vety aj vlastnosti rovnoramenných trojuholníkov.\\
\\
}


% Do not delete this line (pandoc magic!)

\problem{58-S-2}{seminar11,trojuholniky,anglechas}{
V~pravouhlom trojuholníku $ABC$ označíme $P$ pätu výšky z~vrcholu $C$ na preponu $AB$. Priesečník úsečky $AB$ s~priamkou, ktorá prechádza vrcholom $C$ a stredom kružnice vpísanej trojuholníku $PBC$, označíme $D$. Dokážte, že úsečky $AD$ a $AC$ sú zhodné.
}{
\rieh V~pravouhlom trojuholníku $ABC$ s~preponou $AB$ pre veľkosti $\alpha, \beta$ uhlov pri vrcholoch $A$, $B$ platí $\alpha+\beta= 90^\circ$, preto $|\ma ACP| = 90^\circ -\alpha = \beta$ a $|\ma BCD| = | \ma DCP|= \frac{1}{2}(90^\circ -\beta) = \frac{1}{2}\alpha$ lebo priamka $CD$ je osou uhla $BCP$ (obr.~\ref{fig:58S2}). Pre vonkajší uhol $ADC$ trojuholníka $BCD$ tak zrejme platí $|\ma ADC| = |\ma DBC| + |\ma BCD| = \beta  +\frac{1}{2}\alpha = |\ma DCA|.$

Zistili sme, že trojuholník $ADC$ má pri vrcholoch $C, D$ zhodné vnútorné uhly, je
teda rovnoramenný, a preto $|AD| = |AC|$.
\begin{figure}[h]
    \centering
    \includegraphics{images/58S21\imagesuffix}
    \caption{}
    \label{fig:58S2}
\end{figure}
\\
\kom Úloha je zameraná na nájdenie veľkosti vhodných uhlov\footnote{V anglickej literatúre sa tejto metóde -- počítaniu veľkostí všemožných uhlov -- hovorí \textit{angle-chasing}.} a využitie poznatku, že uhly pri základni rovnoramenného trojuholníka majú rovnakú veľkosť.
}


% Do not delete this line (pandoc magic!)

\problem{64-I-4}{seminar11,podtroj,pomery,domacekolo}{
Označme $E$ stred základne $AB$ lichobežníka $ABCD$, v~ktorom platí $|AB| : |CD| = 3 : 1$. Uhlopriečka $AC$ pretína úsečky $ED$, $BD$ postupne v~bodoch $F$, $G$. Určte postupný pomer $|AF| : |FG| : |GC|$.
}{
\rieh  Keďže v~zadaní aj v~otázke úlohy sú iba pomery, môžeme si dĺžky strán lichobežníka zvoliť ako vhodné konkrétne čísla. Zvoľme teda napr. $|AB| = 6$, potom $|AE| = |BE| = 3$ a $|CD| = 2$. Hľadané dĺžky označme $|AF| = x$, $|FG| = y$, $|GC| = z$. Tieto dĺžky sme vyznačili na obr.~\ref{fig:64I4}, taktiež aj tri dvojice zhodných uhlov, ktoré teraz využijeme pri úvahách o~trojuholníkoch podobných podľa vety $uu$.

Trojuholníky $ABG$ a $CDG$ sú podobné, preto $(x + y) : z~= 6 : 2 = 3 : 1$. Aj trojuholníky $AEF$ a $CDF$ sú podobné, preto $x : (y + z) = 3 : 2$.
\begin{figure}[h]
    \centering
    \includegraphics{images/64D4\imagesuffix}
    \caption{}
    \label{fig:64I4}
\end{figure}
Odvodené úmery zapíšeme ako sústavu rovníc
\begin{align*}
x + y - 3z &= 0,\\
2x - 3y - 3z &= 0.
\end{align*}
Ich odčítaním získame rovnosť $x = 4y$, čiže $x : y = 4 : 1$. Dosadením tohto výsledku do prvej rovnice dostaneme $5y = 3z$, čiže $y : z~= 3 : 5$. Spojením oboch pomerov získame výsledok $x : y : z~= 12 : 3 : 5$.\\
\\
\kom Úloha je výborným tréningom na hľadanie vhodných dvojíc podobných trojuholníkov tak, aby sme pomocou údajov zo zadania boli schopní určiť hľadaný pomer, keďže jedna dvojica trojuholníkov na nájdenie odpovede zjavne stačiť nebude. Okrem toho tiež pozorovania z~náčrtu vedú k~sústave dvoch rovníc, takže študenti uplatnia aj svoje algebraické zručnosti.\\
\\
}


% Do not delete this line (pandoc magic!)

\problem{63-I-4}{seminar11,trojuholniky,podtroj,pytveta,obsahy}{
Vo štvorci $ABCD$ označme $K$ stred strany $AB$ a $L$ stred strany $AD$. Úsečky $KD$ a $LC$ sa pretínajú v~bode $M$ a rozdeľujú štvorec na dva trojuholníky a dva štvoruholníky. Vypočítajte ich obsahy, ak úsečka $LM$ má dĺžku 1\,cm.
}{
\rieh Platí $|AK| = |DL|$ a $|AD| = |DC| = 2|AK|$ (obr.~\ref{fig:63I4}), takže pravouhlé trojuholníky $AKD$ a $DLC$ sú zhodné podľa vety $sus$. Okrem toho sú trojuholníky $MLD$ a $AKD$ podobné podľa vety $uu$, lebo $|\ma LDM| = |\ma KDA|$ a $|\ma DLM| = |\ma DLC| = |\ma AKD|$. Analogicky sa dá overiť i podobnosť trojuholníkov $MDC$ a $AKD$. Z~podobnosti trojuholníkov $AKD$, $MLD$ a $MDC$ vyplýva, že $|MD| = 2|ML| = 2$\,cm a $|MC| = 2|MD| = 4$\,cm. Obsahy útvarov $MLD$, $MDC$ a $AKML$ sú
$$S_{MLD} =\frac{1\cdot 2}{2}= 1\,\text{cm}^2, \ \ \ \  S_{MDC} = \frac{2\cdot 4}{2}= 4\,\text{cm}^2$$
a
$$S_{AKML} = S_{AKD}- S_{MLD} = S_{DLC} - S_{MLD} = S_{MDC} = 4\,\text{cm}^2.$$
Nakoniec pomocou Pytagorovej vety dostávame $S_{ABCD} = |DC|^2 = |DM|^2 + |CM|^2= 20$\,cm$^2$, takže
$$S_{KBCM} = S_{ABCD} - (S_{MLD} + S_{MDC} + S_{AKML}) = 11\,\text{cm}^2.$$
\textit{Záver.} Obsahy trojuholníkov $MLD$, $MDC$ a štvoruholníkov $AKML$, $KBCM$ sú postupne 1\,cm$^2$, 4\,cm$^2$, 4\,cm$^2$ a 11\,cm$^2$.
\begin{figure}[h]
    \centering
    \includegraphics{images/63D41\imagesuffix}
    \caption{}
    \label{fig:63I4}
\end{figure}
\\
\kom Opäť je potrebné identifikovať podobné trojuholníky a potom pomocou známeho koeficientu určiť ich obsahy. Oproti predchádzajúcej úlohe ešte študenti navyše využijú Pytagorovu vetu.
}


% Do not delete this line (pandoc magic!)

\problem{65-II-3}{seminar11,podtroj,obsahy,pomery,krajskekolo}{
V~pravouhlom lichobežníku $ABCD$ s~pravým uhlom pri vrchole $A$ základne $AB$ je bod $K$ priesečníkom výšky $CP$ lichobežníka s~jeho uhlopriečkou $BD$. Obsah štvoruholníka $APCD$ je polovicou obsahu lichobežníka $ABCD$. Určte, akú časť obsahu trojuholníka $ABC$ zaberá trojuholník $BCK$.
}{
\rieh V~pravouholníku $APCD$ označme $c = |CD| = |AP|$ a $v = |AD| = |CP|$ (obr.~\ref{fig:65II3}, pričom sme už vyznačili ďalšie dĺžky, ktoré odvodíme v~priebehu riešenia)\footnote{Keďže podľa zadania uhlopriečka $BD$ pretína výšku $CP$, musí jej päta $P$ ležať medzi bodmi $A$ a $B$, takže ide o~\uv{zvyčajný} lichobežník $ABCD$ s~dlhšou základňou $AB$ a kratšou základňou $CD$.}.
\begin{figure}[h]
    \centering
    \includegraphics{images/65K3\imagesuffix}
    \caption{}
    \label{fig:65II3}
\end{figure}
Z~predpokladu $S_{APCD} =\frac{1}{2}S_{ABCD}$ vyplýva pre druhú polovicu obsahu $ABCD$ vyjadrenie $\frac{1}{2}S_{ABCD} = S_{PBC}$, takže $S_{APCD} = S_{PBC}$ čiže $cv =\frac{1}{2}|PB|v$, odkiaľ vzhľadom na to, že $v \neq 0$, vychádza $|PB| = 2c$, v~dôsledku čoho $|AB| = 3c$.

Trojuholníky $CDK$ a $PBK$ majú pravé uhly pri vrcholoch $C$, $P$ a zhodné (vrcholové) uhly pri spoločnom vrchole $K$, takže sú podľa vety $uu$ podobné, a to s~koeficientom $|PB| : |CD| = 2c : c = 2$. Preto tiež platí $|PK| : |CK| = 2 : 1$, odkiaľ $|KP| =\frac{2}{3}v$ a $|CK| =\frac{1}{3}v$.

Posudzované obsahy trojuholníkov $ABC$ a $BCK$ tak majú vyjadrenie
$$S_{ABC} = \frac{|AB| \cdot |CP|}{2}=\frac{3cv}{2} \ \ \ \ \text{a} \ \ \ \  S_{BCK} =\frac{|CK|\cdot |BP|}{2}=\frac{\frac{1}{3}v\cdot 2c}{2}=\frac{cv}{3},$$
preto ich pomer má hodnotu
$$\frac{S_{BCK}}{S_{ABC}}=\frac{\frac{1}{3}cv}{\frac{3}{2}cv}=\frac{2}{9}.$$
\textit{Záver.} Trojuholník $BCK$ zaberá $2/9$ obsahu trojuholníka $ABC$.\\
\\
\kom Najkomplexnejšia úloha tohto seminára precvičí študentov v~používaní vlastností podobných trojuholníkov a taktiež vo vyjadrovaní obsahov trojuholníkov pomocou určiteľných hodnôt. Tvorí tak dôstojnú bodku za týmto seminárom.
}




\subsection*{Domáca práca}

% Do not delete this line (pandoc magic!)

\problem{58-I-2}{}{
Pravouhlému trojuholníku $ABC$ s~preponou $AB$ je opísaná kružnica. Päty kolmíc z~bodov $A$, $B$ na dotyčnicu k~tejto kružnici v~bode $C$ označme $D$, $E$. Vyjadrite dĺžku úsečky $DE$ pomocou dĺžok odvesien trojuholníka $ABC$.
}{
\rieh Označme odvesny trojuholníka $ABC$ zvyčajným spôsobom $a$, $b$ a protiľahlé uhly $\alpha$, $\beta$. Stred prepony $AB$ (ktorý je súčasne stredom opísanej kružnice) označíme $O$ (obr.~\ref{fig:58I2_1}).

Výška $v = CP$ rozdeľuje trojuholník $ABC$ na trojuholníky $ACP$ a $CBP$ podobné trojuholníku $ABC$ podľa vety $uu$ ($\alpha + \beta = 90^\circ$), úsečka $OC$ je kolmá na $DE$ a navyše $|OC| = |OA| = r$ (polomer opísanej kružnice). Odtiaľ $|\ma OCA| = |\ma OAC| = \alpha$ a $|\ma DCA| = 90^\circ - |\ma OCA| = \beta$.

Pravouhlé trojuholníky $ACP$ a $ACD$ so spoločnou preponou $AC$ sa teda zhodujú aj v~uhloch pri vrchole $C$. Sú preto zhodné, dokonca súmerne združené podľa priamky $AC$. Analogicky sú trojuholníky $CBP$ a $CBE$ súmerne združené podľa $BC$. Takže $|CD|= |CE| = v$, čiže $|DE| = 2v = 2ab/\sqrt{a^2 + b^2}$, lebo z~dvojakého vyjadrenia dvojnásobku obsahu trojuholníka $ABC$ vyplýva $v = ab/|AB|$, pričom $|AB| =\sqrt{a^2 + b^2}$.

\textit{Poznámka.} Namiesto dvojakého vyjadrenia obsahu môžeme na výpočet výšky $CP$ využiť podobnosť trojuholníkov $CBP$ a $ABC$: $\sin \alpha = |CP|/|AC| = |BC|/|AB|$.
\begin{figure}[h]
    \centering
    \begin{minipage}{0.45\textwidth}
        \centering
        \includegraphics[width=0.9\textwidth]{images/58D21\imagesuffix}
        \caption{}
        \label{fig:58I2_1}
    \end{minipage}\hfill
    \begin{minipage}{0.45\textwidth}
        \centering
        \includegraphics[width=0.9\textwidth]{images/58D22\imagesuffix}
        \caption{}
        \label{fig:58I2_2}
    \end{minipage}
\end{figure}

\textbf{Iné riešenie*.} Úsečka $OC$ je strednou priečkou lichobežníka $DABE$, lebo je rovnobežná so základňami a prechádza stredom $O$ ramena $AB$. Preto $D$ je obrazom bodu $E$ v~súmernosti podľa stredu $C$. Obraz $F$ bodu $B$ v~tej istej súmernosti leží na polpriamke $AD$ za bodom $D$ (obr.~\ref{fig:58I2_2}). Máme $|CF| = |BC| = a$, uhol $ACF$ je pravý, a teda trojuholníky $AFC$ a $ABC$ sú zhodné. Vidíme, že $CD$ je výška v~trojuholníku $AFC$ zhodná s~výškou $v_c$ trojuholníka $ABC$, a $DE$ je jej dvojnásobkom. Veľkosť výšky $v_c$ dopočítame rovnako ako v~predchádzajúcom riešení.

\textit{Záver.} $|DE| = 2ab/\sqrt{a^2 + b^2}$.\\
\\
}


% Do not delete this line (pandoc magic!)

\problem{58-II-2}{
V~pravouhlom trojuholníku $ABC$ označíme $P$ pätu výšky z~vrcholu $C$ na preponu $AB$ a $D, E$ stredy kružníc vpísaných postupne trojuholníkom $APC$, $CPB$. Dokážte, že stred
kružnice vpísanej trojuholníku $ABC$ je priesečníkom výšok trojuholníka $CDE$.
}{
\rieh V~pravouhlom trojuholníku $ABC$ s~preponou $AB$ označme $\alpha$ veľkosť vnútorného uhla pri vrchole $A$, zrejme potom platí $|\ma ACP| = 90^\circ -\alpha, |\ma PCB| = \alpha.$ Stred $D$ kružnice vpísanej trojuholníku $APC$ leží na osi uhla $PAC$, takže $|\ma DAC| = \frac{1}{2}\alpha$, a podobne aj $|\ma PCE| = \frac{1}{2}\alpha$. Odtiaľ pre veľkosť uhla $AUC$ v~trojuholníku $AUC$, pričom $U$ je priesečník polpriamok $AD$ a $CE$ (obr. 11), vychádza
$$|\ma AUC| = 180^\circ -\bigg(90^\circ -\alpha + \frac{1}{2}\alpha\bigg) -\frac{1}{2}\alpha = 90^\circ.$$
To znamená, že polpriamka $AD$ je kolmá na $CE$, úsečka $DU$ je teda výška v~trojuholníku $DEC$. Úplne rovnako zistíme, že aj polpriamka $BE$ (ktorá je zároveň osou uhla $ABC$) je kolmá na $CD$. Dostávame tak, že priesečník polpriamok $AD$ a $BE$, čo je stred kružnice vpísanej trojuholníku $ABC$, je zároveň aj priesečníkom výšok trojuholníka $DEC$.
\begin{center}
\includegraphics{images/58K21\imagesuffix}\\

Obr. 11
\end{center}
\textbf{Iné riešenie*.} Označme $F$ a $G$ zodpovedajúce priesečníky priamok $CD$ a $CE$ so stranou $AB$ (obr. 12). Podľa úlohy vyriešenej na seminári v~škole je trojuholník $CAG$
\begin{center}
\includegraphics{images/58K22\imagesuffix}\\

Obr. 12
\end{center}
rovnoramenný so základňou $CG$. Os $AD$ uhla $CAG$ rovnoramenného trojuholníka $CAG$ je tak aj jeho osou súmernosti, a je preto kolmá na základňu $CG$, teda aj na $CE$. Podobne zistíme, že aj trojuholník $CBF$ je rovnoramenný so základňou $CF$, takže os $BE$ uhla $FBC$ je kolmá na $CF$, teda aj na $CD$. Priesečník oboch osí $AD$ a $BE$ je tak nielen stredom kružnice vpísanej trojuholníku $ABC$, ale aj priesečníkom výšok trojuholníka $CDE$, čo sme mali dokázať.
}



\subsection*{Doplňujúce zdroje a materiály}
Vhodným doplnkom nielen tohto, ale všetkých ďalších geometrických seminárov je publikácia [~\cite{andreescu2013}], ktorá obsahuje veľké množstvo riešených úloh z~euklidovskej geometrie, od jednoduchých až po úroveň medzinárodných súťaží.

\url{https://old.kms.sk/~mazo/matematika/pocitanieUhlov.pdf}


\seminar{11}

\weblinks{Na stiahnutie: \href{pdf/seminar11-teacher.pdf}{učiteľská verzia}, \href{pdf/seminar11-student.pdf}{študentská verzia}}

\subsection*{Téma}
Geometria III -- obsahy trojuholníkov a štvoruholníkov

\subsection*{Ciele}
Precvičenie úloh zaoberajúcich sa obsahmi trojuholníkov a štvoruholníkov,  rôznorodé určovanie obsahu, príp. pomeru obsahov trojuholníkov v~úlohách.

\subsection*{Úlohy a riešenia}
\begin{tcolorbox}[breakable,notitle,boxrule=0pt,colback=light-gray,colframe=light-gray]\ul{11.1} [57-S-2] V~danom rovnobežníku $ABCD$ je bod $E$ stred strany $BC$ a bod $F$ leží vnútri strany $AB$. Obsah trojuholníka $AFD$ je $15$\,cm$^2$ a obsah trojuholníka $FBE$ je $14$\,cm$^2$. Určte obsah štvoruholníka $FECD$.

\end{tcolorbox}

\rieh Označme $v$ vzdialenosť bodu $C$ od priamky $AB$, $a = |AB|$ a $x = |AF|$. Pre obsahy trojuholníkov $AFD$ a $FBE$ (obr. 1) platí $\frac{1}{2}x\cdot v~= 15$, $\frac{1}{2}(a - x) \cdot \frac{1}{2}v = 14$. Odtiaľ $xv = 30$, $av - xv = 56$. Sčítaním oboch rovností nájdeme obsah rovnobežníka $ABCD$: $S_{ABCD} = av = 86$\,cm$^2$. Obsah štvoruholníka $FECD$ je teda $S_{FECD} = S_{ABCD}- (S_{AFD} + S_ {FBE}) = 57$\,cm$^2.$
\begin{center}
\includegraphics{images/57S21\imagesuffix} \includegraphics{images/57S22\imagesuffix}\\

Obr. 1  \ \ \ \ \ \hspace{130pt} Obr. 2
\end{center}
\textbf{Iné riešenie.} Trojuholníky $BEF$ a $ECF$ majú spoločnú výšku z~vrcholu $F$ a zhodné základne $BE$ a $EC$. Preto sú obsahy oboch trojuholníkov rovnaké. Z~obr. 2 vidíme, že obsah trojuholníka $CDF$ je polovicou obsahu rovnobežníka $ABCD$ (oba útvary majú spoločnú základňu $CD$ a rovnakú výšku). Druhú polovicu tvorí súčet obsahov trojuholníkov $AFD$ a $BCF$. Odtiaľ $S_{FECD} = S_{ECF} + S_{CDF} = S_{ECF} + (S_{AFD} + S_{BCF}) = S_{AFD} + 3 S_{FBE} = 57$\,cm$^2$.\\
\\
\textbf{Iné riešenie.} Do rovnobežníka dokreslíme úsečky $FG$ a $EH$ rovnobežné so stranami $BC$ a $AB$ tak, ako znázorňuje obr. 3.
\begin{center}
\includegraphics{images/57S23\imagesuffix}\\

Obr. 3
\end{center}
Rovnobežníky $AFGD$ a $FBEH$ sú svojimi uhlopriečkami $DF$ a $EF$ rozdelené na dvojice zhodných trojuholníkov. Takže $S_{GDF} = S_{AFD} = 15$\,cm$^2$ a $S_{HFE} = S_{BEF} = 14$\,cm$^2$. Zo zhodnosti rovnobežníkov $HECG$ a $FBEH$ navyše ľahko usúdime, že všetky štyri trojuholníky $FBE$, $EHF$, $HEC$ a $CGH$ sú zhodné, takže obsah štvoruholníka $FECD$ je $S_{AFD} + 3S_{FBE} = 57$\,cm$^2$.\\
\\
\kom Úloha je zaradená ako rozcvička pred komplexnejšími problémami, nie je totiž veľmi náročná na vyriešenie. Pekne tiež demonštruje, že niekedy nám vhodný prístup, náčrtok alebo správne nakreslená priamka v~obrázku riešenie úlohy významne zjednoduší.\\
\\
\begin{tcolorbox}[breakable,notitle,boxrule=0pt,colback=light-gray,colframe=light-gray]\ul{11.2} [62-II-2] Vnútri rovnobežníka $ABCD$ je daný bod $K$ a v~páse medzi rovnobežkami $BC$ a $AD$ v~polrovine opačnej k~$CDA$ je daný bod $L$. Obsahy trojuholníkov $ABK, BCK, DAK$ a $DCL$ sú $S_{ABK} = 18$\,cm$^2$, $S_{BCK} = 8$\,cm$^2$, $S_{DAK} = 16$\,cm$^2$, $S_{DCL} = 36$\,cm$^2$. Vypočítajte obsahy trojuholníkov $CDK$ a $ABL$.

\end{tcolorbox}

\rieh Trojuholníky $ABK$ a $CDK$ majú zhodné strany $AB$ a $CD$ a súčet ich výšok $v_1$ a $v_2$ (vzdialeností bodu $K$ od priamky $AB$, resp. $CD$) je rovný výške v~rovnobežníka $ABCD$ (vzdialenosti rovnobežných priamok $AB$ a $CD$, obr. 4). Preto súčet ich obsahov dáva polovicu súčtu obsahu daného rovnobežníka:
$$S_{ABK} + S_{CDK} = \frac{1}{2} |AB|v_1 +\frac{1}{2} |CD|v_2 = \frac{1}{2}|AB| \cdot (v_1 + v_2 ) =\frac{1}{2}|AB| \cdot v~=\frac{1}{2} S_{ABCD}.$$
Podobne aj $S_{BCK} + S_{DAK} =\frac{1}{2} S_{ABCD}$, teda
$$S_{CDK} = S_{BCK} + S_{DAK} - S_{ABK}= 6\,\text{cm}^2.$$
\begin{center}
\includegraphics{images/62K2\imagesuffix}

Obr. 4
\end{center}
Trojuholníky $ABL$ a $DCL$ majú zhodné strany $AB$ a $CD$. Ak $v_3$ označuje príslušnú výšku druhého z~nich, je výška prvého z~nich rovná $v + v_3$, takže pre rozdiel obsahov týchto trojuholníkov platí
\begin{align*}
S_{ABL} - S_{DCL} &= \frac{1}{2} |AB| \cdot (v~+ v_3 ) - \frac{1}{2}|CD|\cdot v_3 =\frac{1}{2} |AB| \cdot (v~+ v_3 - v_3 ) =\\
&= \frac{1}{2} |AB| \cdot v~= \frac{1}{2} S_{ABCD} = S_{BCK} + S_{DAK}.
\end{align*}
Odtiaľ vyplýva
$$S_{ABL} = S_{BCK} + S_{DAK} + S_{DCL} = 60\,\text{cm}^2.$$
\\
\kom Úloha precvičuje použitie tvrdenia, ktoré sme dokázali v~prvom geometrickom seminári, a to, že ak majú dva trojuholníky základňu rovnakej dĺžky, potom ich obsahy sú v~rovnakom pomere ako ich výšky na túto základňu.\\
\\
\begin{tcolorbox}[breakable,notitle,boxrule=0pt,colback=light-gray,colframe=light-gray]\ul{11.3} [64-S-2] Označme $K$ a $L$ postupne body strán $BC$ a $AC$ trojuholníka $ABC$, pre ktoré platí $|BK|= \frac{1}{3}|BC|$, $|AL| =\frac{1}{3}|AC|$. Nech $M$ je priesečník úsečiek $AK$ a $BL$. Vypočítajte pomer obsahov trojuholníkov $ABM$ a $ABC$.

\end{tcolorbox}

\rieh Označme $v$ výšku trojuholníka $ABC$ na stranu $AB$, $v_1$ výšku trojuholníka $ABM$ na stranu $AB$ a $v_2$ výšku trojuholníka $KLM$ na stranu $KL$ (obr. 5). Z~podobnosti trojuholníkov $LKC$ a $ABC$ (zaručenej vetou $sus$) vyplýva, že $|KL| =\frac{2}{3} |AB|$. Z~porovnania ich výšok zo spoločného vrcholu $C$ vidíme, že výška $v$ trojuholníka $ABC$ je rovná trojnásobku vzdialenosti priečky $KL$ od strany $AB$, teda $v = 3(v_1 +v_2)$. Keďže $AK$ a $BL$ sú priečky rovnobežiek $KL$ a $AB$, vyplýva zo zhodnosti prislúchajúcich striedavých uhlov podobnosť trojuholníkov $ABM$ a $KLM$.
\begin{center}
\includegraphics{images/64S2\imagesuffix}

Obr. 5
\end{center}
Keďže $|KL| =\frac{2}{3}|AB|$, je tiež $v_2 =\frac{2}{3}v_1$, a preto $v_1 + v_2 =\frac{5}{3}v_1$, čiže
$$v = 3(v_1 + v_2) = 5v_1.$$
Trojuholníky $ABM$ a $ABC$ majú spoločnú stranu $AB$, preto ich obsahy sú v~pomere výšok na túto stranu, takže obsah trojuholníka $ABC$ je päťkrát väčší ako obsah trojuholníka $ABM$.\\
\\
\kom Ďalšia úloha, ktorá precvičuje rovnaké tvrdenie ako predchádzajúca. Pomery výšok je tentoraz potrebné určiť z~podobnosti trojuholníkov. Tu sa teda uplatnia znalosti precvičované na minulom seminárnom stretnutí. \\
\\
\begin{tcolorbox}[breakable,notitle,boxrule=0pt,colback=light-gray,colframe=light-gray]\ul{11.4} [64-II-3]  Daný je lichobežník $ABCD$ so základňami $AB$, $CD$, pričom $2|AB| = 3|CD|$.

a) Nájdite bod $P$ vnútri lichobežníka tak, aby obsahy trojuholníkov $ABP$ a $CDP$ boli v~pomere $3 : 1$ a aj obsahy trojuholníkov $BCP$ a $DAP$ boli v~pomere $3 : 1$.

b) Pre nájdený bod $P$ určte postupný pomer obsahov trojuholníkov $ABP$, $BCP$, $CDP$ a $DAP$.

\end{tcolorbox}

\rieh Predpokladajme, že bod $P$ má požadované vlastnosti. Priamka rovnobežná so základňami lichobežníka a prechádzajúca bodom $P$ pretína ramená $AD$ a $BC$ postupne v~bodoch $M$ a $N$ (obr. 6). Označme $v$ výšku daného lichobežníka, $v_1$ výšku trojuholníka $CDP$ a $v_2$ výšku trojuholníka $ABP$.
\begin{center}
\includegraphics{images/64K3\imagesuffix}

Obr. 6
\end{center}
a) Keďže obsahy trojuholníkov $ABP$ a $CDP$ sú v~pomere $3 : 1$, platí
$$\frac{|AB|v_2}{2}:\frac{|CD|v_1}{2}= 3 : 1, \ \ \ \ \text{čiže} \ \ \ \ \frac{v_1}{v_2}=\frac{1}{3}\cdot \frac{|AB|}{|CD|}=\frac{1}{3}\cdot \frac{3}{2}=\frac{1}{2}.$$
Z~vyznačených dvojíc podobných pravouhlých trojuholníkov vyplýva, že v~práve určenom pomere $2 : 1$ výšok $v_2$ a $v_1$ delí aj bod $M$ rameno $AD$ a bod $N$ rameno $BC$ (v~prípade pravého uhla pri jednom z~vrcholov $A$ či $B$ je to zrejmé rovno). Tým je konštrukcia bodov $M$ a $N$, a teda aj úsečky $MN$ určená. Teraz zistíme, v~akom pomere ju delí uvažovaný bod $P$.

Keďže obsahy trojuholníkov $BCP$ a $DAP$ sú v~pomere $3 : 1$, platí
$$ \bigg(\frac{|NP|v_1}{2}+\frac{|NP|v_2}{2}\bigg): \bigg(\frac{|MP|v_1}{2}+\frac{|MP|v_2}{2}\bigg)= 3 : 1,$$
$$ \frac{|NP|(v_1 + v_2 )}{2}: \frac{|MP|(v_1 + v_2 )}{2}= 3 : 1, \ \ \ \  |NP| : |MP| = 3 : 1.$$
Tým je konštrukcia (jediného) vyhovujúceho bodu $P$ úplne opísaná.

b) Doplňme trojuholník $DAC$ na rovnobežník $DAXC$. Jeho strana $CX$ delí priečku $MN$ na dve časti, a keďže $v_1 =\frac{1}{3}v$, môžeme dĺžku priečky $MN$ vyjadriť ako $|MN| = |MY | + |Y N| = |AX| +\frac{1}{3} |XB| = |CD| +\frac{1}{3} (|AB| - |CD|) = \frac{1}{3}|AB| +\frac{2}{3}|CD| = \frac{7}{6}|CD|$, lebo podľa zadania platí $|AB| =\frac{3}{2}|CD|$. Preto
$$|MP| =\frac{1}{4}|MN| =\frac{1}{4} \cdot \frac{7}{6}|CD| = \frac{7}{24}|CD|,$$
takže pre pomer obsahov trojuholníkov $CDP$ a $DAP$ platí
$$ \frac{|CD|v_1}{2}:\frac{|MP|(v_1 + v_2 )}{2}= (|CD|v_1 ) : \bigg( \frac{7}{24}\cdot |CD| \cdot 3v_1\bigg)= 1 :\frac{7}{8} = 8 : 7.$$
Pomer obsahov trojuholníkov $BCP$ a $CDP$ je teda $21 : 8$ a pomer obsahov trojuholníkov $ABP$ a $BCP$ je tak $24 : 21$. Postupný pomer obsahov trojuholníkov $ABP$, $BCP$, $CDP$ a $DAP$ je preto $24 : 21 : 8 : 7$.\\
\\
\kom Táto komplexná úloha je vrcholom tohto seminárneho stretnutia. Vyžaduje umnú prácu s~pomermi obsahov, podobnými trojuholníkmi aj netriviálny nápad doplnenia trojuholníka $DAC$ na rovnobežník. Je tak vhodné skôr než samostatne úlohu riešiť spoločne na tabuľu. Študentom tiež pripomenieme, že podobne ako v~úvodnej úlohe, aj tu našlo vhodné rozdelenie zadaného útvaru svoje opodstatnenie a prispelo k~úspešnému rozklúsknutiu problému. \\
\\
\begin{tcolorbox}[breakable,notitle,boxrule=0pt,colback=light-gray,colframe=light-gray]\ul{11.5} [62-I-6] Vnútri pravidelného šesťuholníka $ABCDEF$ s~obsahom 30\,cm$^2$ je zvolený bod $M$. Obsahy trojuholníkov $ABM$ a $BCM$ sú postupne 3\,cm$^2$ a 2\,cm$^2$. Určte obsahy trojuholníkov $CDM$, $DEM$, $EFM$ a $FAM$.

\end{tcolorbox}

\rieh Úloha je o~obsahu šiestich trojuholníkov, na ktoré je daný pravidelný šesťuholník rozdelený spojnicami jeho vrcholov s~bodom $M$ (obr. 7). Celý šesťuholník s~daným obsahom, ktorý označíme $S$, možno rozdeliť na šesť rovnostranných trojuholníkov s~obsahom $S/6$ (obr. 8). Ak označíme $r$ ich stranu, $v$ vzdialenosť rovnobežiek $AB$, $CD$ a $v_1$ vzdialenosť bodu $M$ od priamky $AB$, dostaneme
$$S_{ABM} + S_{EDM} =\frac{1}{2}rv_1 +\frac{1}{2}r(v - v_1 ) = \frac{1}{2} rv =\frac{S}{3},$$
lebo $S/3$ je súčet obsahov dvoch vyfarbených rovnostranných trojuholníkov. Vďaka symetrii majú tú istú hodnotu $S/3$ aj súčty $S_{BCM} +S_{EFM}$ a $S_{CDM} +S_{FAM}$. Odtiaľ už dostávame prvé dva neznáme obsahy $S_DEM = S/3 - S_{ABM} = 7$\,cm$^2$ a $S_{EFM}= S/3 - S_{BCM} = 8$\,cm$^2$.
\begin{center}
\includegraphics{images/62D61\imagesuffix} \includegraphics{images/62D62\imagesuffix}\\

Obr. 7  \hspace{160pt} Obr. 8
\end{center}
Ako určiť zvyšné dva obsahy $S_{CDM}$ a $S_{FAM}$, keď zatiaľ poznáme len ich súčet $S/3$? Všimnime si, že súčet zadaných obsahov trojuholníkov $ABM$ a $BCM$ má významnú hodnotu $S/6$, ktorá je aj obsahom trojuholníka $ABC$ (to vyplýva opäť z~obr. 8). Taká zhoda obsahov znamená práve to, že bod $M$ leží na uhlopriečke $AC$. Trojuholníky $ABM$ a $BCM$ tak majú zhodné výšky zo spoločného vrcholu $B$ a to isté platí aj pre výšky trojuholníkov $CDM$ a $FAM$ z~vrcholov $F$ a $D$ ( t.\,j. bodov, ktoré majú od priamky $AC$ rovnakú vzdialenosť). Pre pomery obsahov týchto dvojíc trojuholníkov tak dostávame
$$\frac{S_{CDM}}{S_{FAM}}=\frac{|CM|}{|AM|}=\frac{S_{BCM}}{S_{ABM}}=\frac{2}{3}.$$
V~súčte $S_{CDM} + S_{FAM}$ majúcom hodnotu $S/3$ sú teda sčítance v~pomere $2 : 3$. Preto $S_{CDM} =4$\,cm$^2$ a $S_{FAM} = 6$\,cm$^2$.\\
\\
\kom Úloha je odľahčeným a netradičným príkladom využitia princípu, na ktorom sme stavali celé toto seminárne stretnutie: súčty obsahov \uv{protiľahlých} trojuholníkov sú stále rovnaké. Posledná časť úlohy vyžaduje netriviálny nápad a študenti tak možno budú potrebovať malú radu.
\subsection*{Domáca práca}
\begin{tcolorbox}[breakable,notitle,boxrule=0pt,colback=light-gray,colframe=light-gray]\ul{11.6} [65-I-4] Vnútri strán $AB$, $AC$ daného trojuholníka $ABC$ sú zvolené postupne body $E$, $F$, pričom $EF \parallel BC$. Úsečka $EF$ je potom rozdelená bodom $D$ tak, že platí $$p = |ED| : |DF | = |BE| : |EA|.$$

a) Ukážte, že pomer obsahov trojuholníkov $ABC$ a $ABD$ je pre $p = 2 : 3$ rovnaký ako pre $p = 3 : 2$.

b) Zdôvodnite, prečo pomer obsahov trojuholníkov $ABC$ a $ABD$ má hodnotu aspoň 4.

\end{tcolorbox}

\rieh Pre spoločnú hodnotu $p$ oboch pomerov zo zadania platí
$$|ED| = p|DF| \ \ \ \ \text{a zároveň} \ \ \ \ |BE| = p|EA|.  \ \ \ \ (1)$$
Pred vlastným riešením oboch úloh a) a b) vyjadríme pomocou daného čísla $p$ skúmaný pomer obsahov trojuholníkov $ABC$ a $ABD$. Ten je rovný -- keďže trojuholníky majú spoločnú stranu AB -- pomeru dĺžok ich výšok $CC_0$ a $DD_0$ (obr. XXX), ktorý je rovnaký ako
\\
OBRAZOK
\\
pomer dĺžok úsečiek $BC$ a $ED$, a to na základe podobnosti pravouhlých trojuholníkov $BCC_0$ a $EDD_0$ podľa vety $uu$ (uplatnenej vďaka $BC \parallel ED$).\footnote{V prípade pravých uhlov $ABC$ a $AED$ to platí triviálne, lebo vtedy $B = C_0$ a $E = D0$.} Platí teda rovnosť
$$\frac{S_{ABC}}{S_{ABD}} =\frac{|BC|}{|ED|}.\ \ \ \  (2)$$
Vráťme sa teraz k~rovnostiam (1), podľa ktorých
$$|EF| = (1 + p)|DF| \ \ \ \ \text{a} \ \ \ \ |AB| = (1 + p)|EA|,$$
a všimnime si, že trojuholníky $ABC$ a $AEF$ majú spoločný uhol pri vrchole $A$ a zhodné uhly pri vrcholoch $C$ a $F$ (pretože $BC \parallel EF$), takže sú podľa vety $uu$ podobné. Preto
pre dĺžky ich strán platí
$$\frac{|AB|}{|AE|}=\frac{|BC|}{|EF|},\ \ \ \ \text{čiže} \ \ \ \  1 + p =\frac{|BC|}{(1 + p)|DF|}, \ \ \ \ \text{odkiaľ} \ \ \ \ |BC| = (1 + p)^2 |DF|.$$
Keď vydelíme posledný vzťah hodnotou $|ED|$, ktorá je rovná $p|DF|$ podľa (1), získame podiel z~pravej strany (2) a tým aj hľadané vyjadrenie
$$\frac{S_{ABC}}{S_{ABD}}=\frac{(1 + p)^2}{p}. \ \ \ \  (3)$$

a) Algebraickou úpravou zlomku zo vzťahu (3)
$$ \frac{(1 + p)^2}{p}=\frac{1 + 2p + p^2}{p}= 2 + p + \frac{1}{p}$$
zisťujeme, že hodnota pomeru $S_{ABC} : S_{ABD}$ je pre akékoľvek dve navzájom prevrátené hodnoty $p$ a $1/p$ rovnaká, teda nielen pre hodnoty $2/3$ a $3/2$, ako sme mali ukázať.

b) Podľa vzťahu (3) je našou úlohou overiť pre každé $p > 0$ nerovnosť
$$\frac{(1 + p)^2}{p}\geq4,\ \ \ \ \text{čiže} \ \ \ \  (1 + p)^2\geq 4p.$$
To je však zrejme ekvivalentné s~nerovnosťou $(1 -p)^2\geq 0$, ktorá skutočne platí, nech je základ druhej mocniny akýkoľvek (rovnosť nastane jedine pre $p = 1$).

Dodajme, že pre iný dôkaz bolo možné využiť aj vyššie uvedené \uv{symetrické} vyjadrenie
$$\frac{(1 + p)^2}{p}= 2 + p +\frac{1}{p}$$
a uplatniť naň dobre známu nerovnosť $p + 1/p \geq 2$, ktorej platnosť pre každé $p > 0$ vyplýva napr. z~porovnania aritmetického a geometrického priemeru dvojice čísel $p$ a $1/p$, nazývaného AG-nerovnosť:
$$\frac{1}{2}\bigg(p +\frac{1}{p}\bigg)\geq \sqrt{p\cdot \frac{1}{p}}= 1, \ \ \ \ \text{pretože všeobecne} \ \ \ \ \frac{a+b}{2} \geq \sqrt{a\cdot b} \ \ \ \ (\forall a, b \geq 0).$$

\subsection*{Doplňujúce zdroje a materiály}
Rovnako ako v~predchádzajúcich geometrických seminároch ostávame v~odporúčaniach verní publikáciám [~\cite{andreescu2013}] a [~\cite{kadlecek1996}].


\section*{Seminár 12}

\weblinks{Na stiahnutie: \href{pdf/seminar12-teacher.pdf}{učiteľská verzia}, \href{pdf/seminar12-student.pdf}{študentská verzia}}

\subsection*{Téma}
Geometria IV -- kružnice vpísaná a opísaná trojuholníku

\subsection*{Ciele}
Precvičiť úlohy zamerané najmä na vlastnosti kružnice vpísanej a opísanej trojuholníku.

\subsection*{Úlohy a riešenia}
\begin{tcolorbox}[breakable,notitle,boxrule=0pt,colback=light-gray,colframe=light-gray]\ul{12.1} [57-II-1] Trojuholník $ABC$ spĺňa pri zvyčajnom označení dĺžok strán podmienku $a \leq b \leq c$. Vpísaná kružnica sa dotýka strán $AB$, $BC$ a $AC$ postupne v~bodoch $K$, $L$ a $M$. Dokážte, že z~úsečiek $AK$, $BL$ a $CM$ možno zostrojiť trojuholník práve vtedy, keď platí $b + c < 3a$.

\end{tcolorbox}

\rieh Označme $x = |AK| = |AM|$, $y = |BL| = |BK|$, $z = |CM| = |CL|$ (obr. 1) zhodné úseky dotyčníc z~jednotlivých vrcholov trojuholníka ku vpísanej kružnici.
\begin{center}
\includegraphics{images/57K1\imagesuffix}\\

Obr. 1
\end{center}
Zrejme
$$a= y + z, \ \ \ \ b = z~+ x, \ \ \ \ c = x + y. \ \ \ \ (1)$$
Z~uvedených rovností vidíme, že daná podmienka
$$b + c < 3a \ \ \ \ (2)$$
je ekvivalentná nerovnosti
$$x < y + z, \ \ \ \ (3)$$
čo je nutná podmienka existencie trojuholníka so stranami dĺžok $x$, $y$ a $z$.

Dosadením z~(1) do podmienok $b \leq c$ a $a \leq b$ zistíme, že $z \leq y$ a $y \leq x$. To znamená, že ďalšie dve trojuholníkové nerovnosti $y < z~+ x$ a $z < x + y$ sú automaticky splnené, takže nerovnosť (3), a tým aj (2) je podmienkou postačujúcou. Tým je tvrdenie úlohy dokázané.\\
\\
\kom Úloha využíva poznatok, že spojnice vrcholov a bodov dotyku so stredom vpísanej kružnice rozdelia trojuholník na tri dvojice zhodných trojuholníkov. Ten využijeme v~nasledujúcej úlohe aj domácej práci. Okrem toho, aj keď úloha nie je na výpočet nijako extrémne náročná, je študentov potrebné upozorniť, že dokazujú ekvivalenciu, takže nerovnosť zo zadania musí byť nielen podmienkou nutnou, ale aj postačujúcou.\\
\\
\begin{tcolorbox}[breakable,notitle,boxrule=0pt,colback=light-gray,colframe=light-gray]\ul{12.2} [61-S-2] Označme $S$ stred základne $AB$ daného rovnoramenného trojuholníka $ABC$. Predpokladajme, že kružnice vpísané trojuholníkom $ACS$, $BCS$ sa dotýkajú priamky $AB$ v~bodoch, ktoré delia základňu $AB$ na tri zhodné diely. Vypočítajte pomer $|AB| : |CS|$.

\end{tcolorbox}

\rieh Vďaka súmernosti podľa priamky $CS$ sa obe vpísané kružnice dotýkajú výšky $CS$ v~rovnakom bode, ktorý označíme $D$. Body dotyku týchto kružníc s~úsečkami $AS$, $BS$, $AC$, $BC$ označíme postupne $E$, $F$, $G$, $H$ (obr. 2). Pre vyjadrenie všetkých potrebných dĺžok ešte zavedieme označenie $x = |SD|$ a $y = |CD|$.
\begin{center}
\includegraphics{images/61S2\imagesuffix}\\

Obr. 2
\end{center}
Vzhľadom na symetriu dotyčníc z~daného bodu k~danej kružnici platia rovnosti
$$|SD| = |SE| = |SF| = x \ \ \ \ \text{a} \ \ \ \ |CD| = |CG| = |CH| = y.$$
Úsečka $EF$ má preto dĺžku $2x$, ktorá je podľa zadania zároveň dĺžkou úsečiek $AE$ a $BF$, a teda aj dĺžkou úsečiek $AG$ a $BH$ (opäť vďaka symetrii dotyčníc). Odtiaľ už bezprostredne vyplývajú rovnosti
$$|AB| = 6x, \ \ \ \ |AC| = |BC| = 2x + y \ \ \ \ \text{a} \ \ \ \  |CS| = x + y.$$

Závislosť medzi dĺžkami $x$ a $y$ zistíme použitím Pytagorovej vety pre pravouhlý trojuholník $ACS$ (s~odvesnou $A$ dĺžky $3x$):
$$(2x + y)^2= (3x)^2+ (x + y)^2.$$
Roznásobením a ďalšími úpravami odtiaľ dostaneme ($x$ a $y$ sú kladné hodnoty)
\begin{align*}
4x^2+ 4xy + y^2 &= 9x^2+ x^2+ 2xy + y^2,\\
2xy & = 6x^2,\\
y &= 3x.
\end{align*}
Hľadaný pomer tak má hodnotu
$$|AB| : |CS| = 6x : (x + y) = 6x : 4x = 3 : 2.$$
Poznamenajme, že prakticky rovnaký postup celého riešenia možno zapísať aj pri štandardnom označení $c = |AB|$ a $v = |CS|$. Keďže podľa zadania platí $|AE| =\frac{1}{3}c$, a teda $|SE| =\frac{1}{6}c$, z~rovnosti $|SD| = |SE|$ vyplýva $|CD| = |CS|-|SD| = v-\frac{1}{6}c$, odkiaľ
$$|AC| = |AG| + |CG| = |AE| + |CD| =\tfrac{1}{3}c + (v-\tfrac{1}{6}c) = v~+\tfrac{1}{6}c,$$
takže z~Pytagorovej vety pre trojuholník $ACS$,
$$(v +\tfrac{1}{6}c)^2= (\tfrac{1}{2}c)^2+ v^2,$$
vychádza $3v = 2c$, čiže $c : v~= 3 : 2$.\\
\\
\kom Úloha vychádza z~poznatku, ktorý si študenti osvojili v~úlohe predchádzajúcej a pridáva k~nemu ešte prácu s~Pytagorovou vetou a manipuláciu s~algebraickými výrazmi, takže tvorí prirodzené pokračovanie úlohy predchádzajúcej.\\
\\
\begin{tcolorbox}[breakable,notitle,boxrule=0pt,colback=light-gray,colframe=light-gray]\ul{12.3} [62-S-1] Danému rovnostrannému trojuholníku vpíšme a opíšme kružnicu. Označme $S$ obsah vzniknutého medzikružia a $T$ obsah kruhu, ktorého priemer je zhodný s~dĺžkou strany daného trojuholníka. Ktorý z~obsahov $S$, $T$ je väčší? Svoju odpoveď zdôvodnite.

\end{tcolorbox}

\rieh Ukážeme, že sa oba obsahy rovnajú. Označme $A$, $B$, $C$ vrcholy daného trojuholníka a $r$ a $R$ zodpovedajúce polomery jeho vpísanej a opísanej kružnice; dĺžku jeho strany označme $a$. Obe uvedené kružnice majú spoločný stred $S$. Označme ešte $P$ bod dotyku vpísanej kružnice so stranou $AB$. Keďže trojuholník $ABC$ je rovnostranný, je $P$ zároveň stredom strany $AB$. Použitím Pytagorovej vety v~pravouhlom trojuholníku $PSB$ dostávame
$$R^2 - r^2=  (\tfrac{1}{2}a)^2,$$
čo je ekvivalentné s~dokazovaným tvrdením $S = \pi (R^2 - r^2) = \pi \big( \frac{1}{2}a\big)^2= T$.\\
\\
\textit{Poznámka.} Rovnostranný trojuholník so stranou $a$ má výšku veľkosti $v = \frac{1}{2}a \sqrt{3}$, takže skúmané polomery sú $R =\frac{2}{3}v \big(=\frac{1}{3}a\sqrt{3}\big)$ a $r =\frac{1}{3}v \big(=\frac{1}{6}a\sqrt{3}\big)$, a preto
$$S = \pi ( R^2 - r^2) = \pi \big( \tfrac{4}{9} -\tfrac{1}{9})v^2= \pi \cdot \tfrac{1}{3}\cdot\tfrac{3}{4}a^2= \pi \big( \tfrac{1}{2}a\big)^2= T.$$
\\
\kom Úloha je relatívne jednoduchá, využíva znalosť o~bode dotyku vpísanej kružnice a taktiež pripravuje študentov na nasledujúcu zložitejšiu analýzu. \\
\\
\begin{tcolorbox}[breakable,notitle,boxrule=0pt,colback=light-gray,colframe=light-gray]\ul{12.4} [61-I-5] Daný je rovnoramenný trojuholník so základňou dĺžky $a$ a ramenami dĺžky $b$. Pomocou nich vyjadrite polomer $R$ kružnice opísanej a polomer $r$ kružnice vpísanej tomuto trojuholníku. Potom ukážte, že platí $R \geq 2r$ a zistite, kedy nastane rovnosť.

\end{tcolorbox}

\rieh Označme $S$ stred základne $BC$ daného rovnoramenného trojuholníka $ABC$, $O$ stred jeho opísanej kružnice, $M$ stred vpísanej kružnice a $P$ pätu kolmice z~bodu $M$ na rameno $AC$ (obr. 3).
\begin{center}
\includegraphics{images/61D5\imagesuffix}

Obr. 3
\end{center}
Z~pravouhlého trojuholníka $BSA$ pomocou Pytagorovej vety vyjadríme veľkosť $v$ výšky $AS$, pričom v~pravouhlom trojuholníku $BSO$ s~preponou dĺžky $R$ pre odvesnu $OS$ platí $|OS| =||AS|-|AO|| = |v-R|$ (musíme si uvedomiť, že v~tupouhlom trojuholníku $ABC$ bude bod $S$ ležať medzi bodmi $A$ a $O$!). Dostávame tak dve rovnosti
\begin{align*}
v^2 &= b^2 -\frac{a^2}{4},\\
R^2 &= \frac{a^2}{4}+ (v~-R)^2;
\end{align*}
ich sčítaním vyjde
$$v^2+ R^2= b^2 + (v~- R)^2,\ \ \ \ \text{čiže} \ \ \ \  b^2= 2vR.$$
Dosadením z~prvej rovnice $v =\frac{1}{2}\sqrt{4b^2- a^2}$ do poslednej rovnosti dostaneme hľadaný vzorec pre $R$.

Dodajme, že rovnosť $b^2 = 2vR$, ktorú sme práve odvodili a z~ktorej už ľahko vyplýva vzorec pre polomer $R$, je Euklidovou vetou o~odvesne $AB$ pravouhlého trojuholníka $ABA'$ s~preponou $AA'$, ktorá je priemerom kružnice opísanej trojuholníku $ABC$ (obr. 3).

Nájdený vzorec pre polomer $R$ zapíšeme prehľadne spolu s~druhým hľadaným vzorcom pre polomer $r$, ktorého odvodeniu sa ešte len budeme venovať:
$$R =\frac{\sqrt{b^2}}{\sqrt{4b^2 - a^2}}\ \ \ \ \text{a}\ \ \ \  r = \frac{a\sqrt{4b^2-a^2}}{2(a+2b)}.\ \ \ \  (\ast)$$
Druhý zo vzorcov ($\ast$) sa dá získať okamžite zo známeho vzťahu $r = 2S/(a + b + c)$ pre polomer $r$ kružnice vpísanej do trojuholníka so stranami $a$, $b$, $c$ a obsahom $S$;
v~našom prípade stačí len dosadiť $b = c$ a $2S = av$, kde $v = \frac{1}{2}\sqrt{4b^2 - a^2}$ podľa úvodnej časti riešenia.

Ďalšie dva spôsoby odvodenia druhého zo vzorcov ($\ast$) založíme na úvahe o~pravouhlom trojuholníku $AMP$, ktorého strany majú dĺžky
$$|AM| = v~-r, \ \ \ \ |MP| = r, \ \ \ \ |AP| = |AC| - |PC| = b - |SC| = b - \frac{a}{2}.$$
Pre tento trojuholník môžeme napísať Pytagorovu vetu alebo využiť jeho podobnosť s~trojuholníkom $ACS$, konkrétne zapísať rovnosť sínusov ich spoločného uhla pri vrchole $A$. Podľa toho dostaneme rovnice
$$(v - r)^2= r^2+\big(b -\frac{a}{2}\big)^2, \ \ \ \ \text{resp.} \ \ \ \ \frac{r}{v-r}=  \frac{\frac{1}{2}a}{b},$$
ktoré sú obidve lineárne vzhľadom na neznámu $r$ a majú riešenie
$$r = \frac{v}{2}-\frac{1}{2v}\cdot \big( b - \frac{a}{2} \big)^2, \ \ \ \ \text{resp.}\ \ \ \  r=
\frac{av}{a+2b}.$$
Po dosadení za $v$ v~oboch prípadoch dostaneme hľadaný vzorec pre $r$. V~druhom prípade
je to zrejmé, v~prvom to ukážeme:
$$r =\frac{v}{2}  - \frac{1}{2v} \cdot \big(b \frac{a}{2}\big)^2= \frac{v^2 - b^2 + ab \frac{1}{4}a^2}{2v}=\frac{2ab - a^2}{4v}=\frac{a(2b - a)}{2\sqrt{(2b -a)(2b + a)}}=\frac{2\sqrt{2b-a}}{2\sqrt{2b-a}}= \\ =\frac{a \sqrt{4b^2 -a^2}}{2(a + 2b)}.$$

Ešte ostáva dokázať nerovnosť $R \geq 2r$. Využijeme na to odvodené vzorce ($\ast$), z~ktorých dostávame (pripomíname, že $2b > a > 0$)
$$ \frac{R}{2r}= R \cdot \frac{1}{2r}=\frac{b^2}{\sqrt{4b^2-a^2}}\cdot \frac{a+2b}{a \sqrt{4b^2-a^2}}=\frac{b^2}{a(2b-a)}.$$
Nerovnosť $R \geq 2r$ teda platí práve vtedy, keď $b^2\geq a(2b -a)$. Posledná nerovnosť je však ekvivalentná s~nerovnosťou $(a - b)^2\geq 0$, ktorej platnosť je už zrejmá. Tým je dôkaz nerovnosti $R \geq 2r$ hotový. Navyše vidíme, že rovnosť v~nej nastane jedine v~prípade, keď $(a - b)^2 = 0$, čiže $a = b$, teda práve vtedy, keď je pôvodný trojuholník nielen rovnoramenný, ale dokonca rovnostranný.\\
\\
\kom Úloha poskytuje mnoho prístupov k~riešeniu a bude zaujímavé nechať študentov porovnať ich výsledky. Spája tiež zistenia z~predchádzajúcich úloh, v~niektorých prípadoch študenti využijú Euklidovu vetu a nezaobídu sa ani bez zručnej manipulácie s~algebraickými výrazmi. \\
\\
\begin{tcolorbox}[breakable,notitle,boxrule=0pt,colback=light-gray,colframe=light-gray]\ul{12.5} [63-I-2]  V~rovine sú dané body $A$, $P$, $T$ neležiace na jednej priamke. Zostrojte trojuholník $ABC$ tak, aby $P$ bola päta jeho výšky z~vrcholu $A$ a $T$ bod dotyku strany $AB$ s~kružnicou jemu vpísanou. Uveďte diskusiu o~počte riešení vzhľadom na polohu daných bodov.

\end{tcolorbox}

\rieh Vrchol $B$ je určený polpriamkou $AT$ a kolmicou $p$ na výšku $AP$ v~bode $P$ (obr. 4), na ktorej leží strana $BC$. Pritom bod $T$ musí byť vnútorným bodom úsečky $AB$. Stred $S$ kružnice vpísanej trojuholníku $ABC$ potom dostaneme ako priesečník kolmice $q$
\begin{center}
\includegraphics{images/63D2\imagesuffix}\\

Obr. 4
\end{center}
na priamku $AT$ v~bode $T$ s~osou uhla ohraničeného priamkou $p$ a polpriamkou $BA$. Jej polomer bude mať veľkosť $|ST|$.

Ostáva zostrojiť vrchol $C$ hľadaného trojuholníka $ABC$. Ten bude ležať jednak na priamke $p$, jednak na druhej dotyčnici vpísanej kružnice z~vrcholu $A$, ktorá je súmerne združená so stranou $AB$ podľa priamky $AS$. Stačí teda zostrojiť bod $U$ dotyku strany $AC$ s~kružnicou vpísanou ako obraz bodu $T$ v~uvedenej osovej súmernosti.

Odtiaľ vyplýva \textit{konštrukcia}:
\begin{enumerate}
\item $p$: $P \in p$ a $p \perp AP$;
\item $B$: $B \in AT \cap p$, bod $B$ musí ležať na polpriamke $AT$ za bodom $T$;
\item $q$: $T \in q$ a $q \perp AT$;
\item $u_1$, $u_2$: dve (navzájom kolmé) osi rôznobežiek $AB$, $p$;
\item $S_1$, $S_2$: $S_1 \in q \cap u_1$, $S_2 \in q \cap u_2$;
\item $U_1$, $U_2$: obrazy bodu $T$ v~súmernostiach podľa priamok $AS_1$ a $AS_2$;
\item $C_1$, $C_2$: priesečníky priamky $p$ s~polpriamkami $AU_1$ a $AU_2$;
\item trojuholníky $ABC_1$ a $ABC_2$.
\end{enumerate}
\textit{Diskusia.} Bod $B$ konštruovaný v~2. kroku existuje, len ak uhol $PAT$ je ostrý (inak ani polpriamka $AT$ nepretne priamku $p$) a zároveň bod $T$ leží vnútri polroviny $pA$, čo je ekvivalentné s~tým, že aj uhol $APT$ je ostrý. Body $S_1$, $S_2$ existujú vždy a sú rôzne, lebo ležia v~opačných polrovinách určených priamkou $AB$. Kružnica vpísaná leží celá v~trojuholníku $ABC$, a teda i v~páse určenom priamkou $p$ a priamkou s~ňou rovnobežnou, ktorá prechádza vrcholom $A$, takže stred $S$ vpísanej kružnice musí padnúť do pásu tvoreného priamkou $p$ a priamkou $p'$ s~ňou rovnobežnou, ktorá rozpoľuje výšku $AP$. V~takom prípade dotyčnica ku kružnici $(S; |ST|)$ (súmerne združená s~dotyčnicou $AB$ podľa priamky $AS$) určite pretne priamku $p$ v~hľadanom vrchole $C$.

Diskusiu zhrnieme takto: Ak pre vnútorné uhly trojuholníka $APT$ platí $|\ma PAT| \geq 90^\circ$ alebo $|\ma APT| \geq 90^\circ$, nemá úloha riešenie. Ak platí $|\ma PAT| < 90^\circ$ a zároveň $|\ma APT| < 90^\circ$, je počet riešení 0 až 2 podľa toho, koľko zo zostrojených bodov $S_1$ a $S_2$ leží medzi rovnobežkami $p$ a $p'$.\\
\\
\kom V~posledných rokoch sa v~MO nevyskytlo veľké množstvo konštrukčných úloh. Napriek tomu však považujeme za dôležité vyriešiť so študentmi aspoň jeden takýto problém a poukázať na to, že zostrojením vyhovujúceho útvaru riešenie úlohy nekončí a je potrebné uviesť aj diskusiu, ktorá je častokrát aspoň tak náročná ako vhodná konštrukcia. Zaradenie úlohy v~tomto seminári považujeme za vhodné tiež preto, lebo úloha využíva vlastnosti kružnice vpísanej, a tak so cťou uzavrie toto seminárne stretnutie.

\subsection*{Domáca práca}
\begin{tcolorbox}[breakable,notitle,boxrule=0pt,colback=light-gray,colframe=light-gray]\ul{12.6} [59-I-4] Kružnica $k(S; r)$ sa dotýka priamky $AB$ v~bode $A$. Kružnica $l(T; s)$ sa dotýka priamky $AB$ v~bode $B$ a pretína kružnicu k~v~krajných bodoch $C$, $D$ jej priemeru. Vyjadrite dĺžku a úsečky $AB$ pomocou polomerov $r$, $s$. Dokážte ďalej, že priesečník $M$ priamok $CD$, $AB$ je stredom úsečky $AB$.

\end{tcolorbox}

\rieh Keďže kružnica $l$ má ako tetivu priemer $CD$ kružnice $k$ a dané kružnice nie sú totožné, platí pre ich polomery nerovnosť $s > r$. Ak označíme $P$ pätu kolmice z~bodu $S$ na úsečku $BT$ (obr. 5), tak z~Pytagorovej vety pre pravouhlé trojuholníky
\begin{center}
\includegraphics{images/59D4\imagesuffix}\\

Obr. 5
\end{center}
$CST$ a $SPT$ vyplýva
$$|ST|^2 = s^2 - r^2\ \ \ \ \text{a} \ \ \ \  |ST|^2 = |SP|^2 + (s~-r)^2. \ \ \ \  (1)$$
Odtiaľ pre veľkosť úsečky $SP$ vychádza
$$|SP|^2 = (s^2 - r^2 ) - (s~- r)^2 = 2r(s - r).$$
A~keďže $ABPS$ je pravouholník, dostávame
$$|AB| = |SP| =\sqrt{2r(s - r)}.$$

Z~pravouhlých trojuholníkov $AMS$ a $MTS$ ďalej podľa prvej rovnosti v~(1) vyplýva
$$|AM|^2 = |SM|^2 - r^2 = |MT|^2- |ST|^2 - r^2 = |MT|^2 -s^2,$$
pritom z~pravouhlého trojuholníka $MBT$ máme
$$|BM|^2 = |MT|^2 - s^2.$$
Preto $|AM| = |BM|$ a bod $M$ je teda stredom úsečky $AB$.

\textit{Poznámka.} Záver, že $M$ je stredom úsečky $AB$, vyplýva okamžite aj z~mocnosti bodu $M$ k~obom kružniciam (bod $M$ leží na tzv. chordále oboch kružníc). Tieto pojmy sú však pre súťažiacich kategórie C zväčša neznáme a nebudú nutné ani pre riešenia ďalších súťažných kôl.\\
\\
\begin{tcolorbox}[breakable,notitle,boxrule=0pt,colback=light-gray,colframe=light-gray]\ul{12.7} [61-I-2] Dĺžky strán trojuholníka sú v~metroch vyjadrené celými číslami. Určte ich, ak má trojuholník obvod 72\,m a ak je najdlhšia strana trojuholníka rozdelená bodom dotyku vpísanej kružnice v~pomere $3 : 4.$

\end{tcolorbox}

\rieh Využijeme všeobecný poznatok, že body dotyku vpísanej kružnice delia hranicu trojuholníka na šesť úsečiek, a to tak, že každé dve z~nich, ktoré vychádzajú z~toho istého vrcholu trojuholníka, sú zhodné. (Dotyčnice z~daného bodu k~danej kružnici sú totiž súmerne združené podľa spojnice daného bodu so stredom danej kružnice.)

V~našej úlohe je najdlhšia strana trojuholníka rozdelená na úseky, ktorých dĺžky označíme $3x$ a $4x$; dĺžku úsekov z~vrcholu oproti najdlhšej strane označíme $y$ (obr. 6). Strany trojuholníka majú teda dĺžky $7x$, $4x + y$ a $3x + y$, kde $x$, $y$ sú neznáme kladné čísla (dĺžky berieme bez jednotiek). Ak má byť $7x$ dĺžka najdlhšej strany, musí platiť $7x > 4x + y$, čiže $3x > y$. Zdôraznime, že hľadané čísla $x, y$ nemusia byť nutne celé, podľa zadania to však platí o~číslach $7x$, $4x + y$ a $3x + y$.
\begin{center}
\includegraphics{images/61D1\imagesuffix}\\

Obr. 6
\end{center}
Údaj o~obvode trojuholníka zapíšeme rovnosťou
$$72 = 7x + (3x + y) + (4x + y), \ \ \ \ \text{čiže} \ \ \ \ 36 = 7x + y.$$
Pretože $7x$ je celé číslo, je celé i číslo $y = 36 - 7x$; a pretože podľa zadania i čísla $4x + y$ a $3x + y$ sú celé, je celé i číslo $x = (4x + y) - (3x + y)$. Preto od tohto okamihu už hľadáme dvojice celých kladných čísel $x$, $y$, pre ktoré platí
$$3x > y \ \ \ \ \text{a}  \ \ \ \ 7x + y = 36.$$
Odtiaľ vyplýva $7x < 36 < 7x + 3x = 10x$, teda $x \leq 5$ a súčasne $x \geq 4$.

Pre $x = 4$ je $y = 8$ a $(7x, 4x+y, 3x+y) = (28, 24, 20)$, pre $x = 5$ je $y = 1$ a $(7x, 4x+ + y, 3x + y) = (35, 21, 16)$. Strany trojuholníka sú teda $(28, 24, 20)$ alebo $(35, 21, 16)$. (Trojuholníkové nerovnosti sú zrejme splnené.)


\section*{Seminár 13}
\subsection*{Téma}
Matematická súťaž v~riešení úloh \textit{Náboj}

\subsection*{Príprava a prevedenie súťaže}
Predvianočné seminárne stretnutie sa ponesie v~odľahčenom duchu. Pre študentov bude pripravená tímová súťaž v~štýle súťaže \textit{Náboj}.

Pred seminárom je potrebné vytlačiť a pripraviť zadania úloh pre študentov. Úlohy sú dostupné na XXX v~rôznych jazykových mutáciách a voľne prístupné na použitie. Každý tím bude mať vlastnú sadu zadaní, rozstrihaných na jednotlivé úlohy. Okrem toho je potrebné mať pripravenú aj verziu pre
opravovateľov, kde nájdu úlohy spolu s~výsledkami a stručným riešením.

Posledným prípravným krokom je osloviť kolegov, ktorí by nám boli ochotní s~organizáciou pomôcť, keďže pre jedného človeka je kontrolovanie výsledkov, vydávanie nových úloh a zapisovanie priebežného poradia dosť náročné, ak nie nemožné. S~dobrovoľnými pomocníkmi sa potom pred seminárom dohodneme na rozdelení rolí, aby si prípadný opravovateľ mal čas prejsť zadania úloh.

Na seminárnom stretnutí potom rozdelíme študentov do štvorčlenných až päťčlenných tímov, vysvetlíme pravidlá, zodpovieme prípadné otázky, rozdáme zadania a súťaž môže začať. Ak máme študentov na seminári menej, je možné súťažiť aj v~trojčlenných tímoch.


\section{Január}
\seminar{14}

\weblinks{Na stiahnutie: \href{pdf/seminar14-teacher.pdf}{učiteľská verzia}, \href{pdf/seminar14-student.pdf}{študentská verzia}}

\subsection*{Téma}
Domáce kolo MO -- analýza úloh

\subsection*{Priebeh}
Seminár je venovaný rozboru úloh domáceho kola aktuálneho ročníka MO. Od vedúceho seminára tak bude vyžadovať domácu prípravu navyše, keďže bude musieť samostatne vyriešiť zadané úlohy.

Na seminári potom so študentmi prediskutujeme ich spôsoby riešenia a prístupu k~jednotlivým úlohám. Ak už budú mať študenti úlohy domáceho kola opravené, môžeme sa venovať aj najčastejším konkrétnym chybám a nejasnostiam, ktoré sa pri riešení úloh našimi zverencami objavili.

\subsection*{Domáca práca}
Na záver seminára študentov požiadame, aby si zopakovali obsah seminára od začiatku školského roka a poznačili si prípadné nejasnosti alebo náročné partie. Nasledujúci seminár totiž bude príležitosťou všetky nejasnosti odstrániť.


\input{seminars/seminar15.tex}
\seminar{16}{Konzultačné stretnutie v réžii študentov}
\teachernote{
\subsection*{Priebeh}

Na seminár nie je zámerne naplánovaný žiadny konkrétny obsah, pretože je zamýšľaný ako priestor na konzultáciu, prinesenie vlastných otázok a nejasností, ktoré študenti majú, aby sme sa spoločne zbavili všetkých pochybností pred školským kolom.
}

\section{Február}
% \seminar{17}

\subsection*{Téma}
Algebraické výrazy a (ne)rovnice IV -- zložitejšie nerovnosti


\teachernote{
\subsection*{Ciele}
Zoznámiť a precvičiť so študentami riešenie úloh zameraných na dokazovanie zložitejších nerovností, AG-nerovnosť

}
\subsection*{Úlohy a riešenia}

% Do not delete this line (pandoc magic!)

\problem{66-II-4}{
Dokážte, že pre všetky kladné reálne čísla $a \leq b \leq c$ platí $$(-a + b + c)\bigg( \frac{1}{a}+\frac{1}{b}+\frac{1}{c}\bigg) \geq 3.$$
}{
\rieh  Nerovnosť vynásobíme kladným výrazom $abc$ a po roznásobení ju postupne (ekvivalentne) upravíme:
\begin{align*}
-a(bc + ac + ab) + b(bc + ac + ab) + c(bc + ac + ab) &\geq 3abc,\\
-abc - a^2c - a^2b + b^2c + abc + ab^2+ bc^2+ ac^2+ abc &\geq 3abc,\\
(b^2c - abc) + (bc^2 - abc) + (ac^2 - a^2c) + (ab^2 - a^2b) &\geq 0,\\
bc(b - a) + bc(c - a) + ac(c - a) + ab(b - a) &\geq 0.
\end{align*}
Vzhľadom na predpoklad $0 < a \leq b \leq c$ je výsledná, a teda aj pôvodná nerovnosť splnená.\\
\\
\textbf{Iné riešenie*.} Dokazovanú nerovnosť postupne upravíme, pričom využijeme známu nerovnosť $b/c + c/b \geq 2$, ktorá je pre kladné čísla $b, c$ ekvivalentná s nerovnosťou $(b - c)^2\geq 0$: $$(-a + b + c)\bigg( \frac{1}{a}+\frac{1}{b}+\frac{1}{c}\bigg) = 1+\bigg(\frac{b}{a}-\frac{a}{b}\bigg)+\bigg(\frac{c}{a}-\frac{a}{c}\bigg)+\bigg(\frac{b}{c}+\frac{c}{b}\bigg)\geq$$ $$
\geq 1+\frac{b^2-a^2}{ab}+\frac{c^2-a^2}{ac}+2\geq3,$$
pretože zrejme platí aj $a^2\leq b^2\leq c^2$.\\
\\
\textbf{Iné riešenie*.} Podľa predpokladov úlohy platia nerovnosti $-a + b + c \geq c$ a $\frac{1}{a}+\frac{1}{b}+\frac{1}{c}\geq \frac{2}{b}+\frac{1}{c}$.
Obe nerovnosti (s kladnými stranami) medzi sebou vynásobíme a získame tak $$(-a + b + c)\bigg(\frac{1}{a}+\frac{1}{b}+\frac{1}{c}\bigg) \geq c \bigg( \frac{2}{b}+\frac{1}{c}\bigg)=1 +\frac{2c}{b}\geq 3$$
pretože $c/b \geq 1$ podľa zadania.
\\
\\
\kom Ďalšia úloha, ktorú je možné rozlúsknuť spektrom rozličných prístupov. Ak študenti zvolia len cestu ekvivalentných úprav, ukážeme im aj riešenie, ktoré využíva nerovnosť $b/c+c/b \geq 2$ z predchádzajúceho seminára o nerovnostiach, rovnako ako riešenie pomocou vynásobenia nerovností medzi sebou. Takto dáme študentom príležitosť poznať aj iné prístupy, ktoré môžu byť užitočné pri ďalšom riešení úloh.\\
\\
%\\
%\\
%\kom Úvodná úloha slúži na opakovanie, pripomenutie naučených postupov a overenie toho, čo sa študenti doteraz naučili o nerovnostiach. Považujeme za vhodné ukázať všetky tri zmienené postupy riešenia, keďže ekvivalentné úpravy rovníc, využívanie známych rovností aj sčítanie dvoch a viac rovností sú všetko užitočné metódy, ktoré sa oplatí mať v našej riešiteľskej zásobe.
}




 % Seminar chyba!
\seminar{18}

\subsection*{Téma}
Algebraické výrazy a (ne)rovnice IV -- zložitejšie nerovnosti


\teachernote{
\subsection*{Ciele}
Zoznámiť a precvičiť so študentami riešenie úloh zameraných na dokazovanie zložitejších nerovností, AG-nerovnosť

}
\subsection*{Úlohy a riešenia}

\todo{DOKONČiŤ.}

% Do not delete this line (pandoc magic!)

\problem{66-II-4}{
Dokážte, že pre všetky kladné reálne čísla $a \leq b \leq c$ platí $$(-a + b + c)\bigg( \frac{1}{a}+\frac{1}{b}+\frac{1}{c}\bigg) \geq 3.$$
}{
\rieh  Nerovnosť vynásobíme kladným výrazom $abc$ a po roznásobení ju postupne (ekvivalentne) upravíme:
\begin{align*}
-a(bc + ac + ab) + b(bc + ac + ab) + c(bc + ac + ab) &\geq 3abc,\\
-abc - a^2c - a^2b + b^2c + abc + ab^2+ bc^2+ ac^2+ abc &\geq 3abc,\\
(b^2c - abc) + (bc^2 - abc) + (ac^2 - a^2c) + (ab^2 - a^2b) &\geq 0,\\
bc(b - a) + bc(c - a) + ac(c - a) + ab(b - a) &\geq 0.
\end{align*}
Vzhľadom na predpoklad $0 < a \leq b \leq c$ je výsledná, a teda aj pôvodná nerovnosť splnená.\\
\\
\textbf{Iné riešenie*.} Dokazovanú nerovnosť postupne upravíme, pričom využijeme známu nerovnosť $b/c + c/b \geq 2$, ktorá je pre kladné čísla $b, c$ ekvivalentná s nerovnosťou $(b - c)^2\geq 0$: $$(-a + b + c)\bigg( \frac{1}{a}+\frac{1}{b}+\frac{1}{c}\bigg) = 1+\bigg(\frac{b}{a}-\frac{a}{b}\bigg)+\bigg(\frac{c}{a}-\frac{a}{c}\bigg)+\bigg(\frac{b}{c}+\frac{c}{b}\bigg)\geq$$ $$
\geq 1+\frac{b^2-a^2}{ab}+\frac{c^2-a^2}{ac}+2\geq3,$$
pretože zrejme platí aj $a^2\leq b^2\leq c^2$.\\
\\
\textbf{Iné riešenie*.} Podľa predpokladov úlohy platia nerovnosti $-a + b + c \geq c$ a $\frac{1}{a}+\frac{1}{b}+\frac{1}{c}\geq \frac{2}{b}+\frac{1}{c}$.
Obe nerovnosti (s kladnými stranami) medzi sebou vynásobíme a získame tak $$(-a + b + c)\bigg(\frac{1}{a}+\frac{1}{b}+\frac{1}{c}\bigg) \geq c \bigg( \frac{2}{b}+\frac{1}{c}\bigg)=1 +\frac{2c}{b}\geq 3$$
pretože $c/b \geq 1$ podľa zadania.
\\
\\
\kom Ďalšia úloha, ktorú je možné rozlúsknuť spektrom rozličných prístupov. Ak študenti zvolia len cestu ekvivalentných úprav, ukážeme im aj riešenie, ktoré využíva nerovnosť $b/c+c/b \geq 2$ z predchádzajúceho seminára o nerovnostiach, rovnako ako riešenie pomocou vynásobenia nerovností medzi sebou. Takto dáme študentom príležitosť poznať aj iné prístupy, ktoré môžu byť užitočné pri ďalšom riešení úloh.\\
\\
%\\
%\\
%\kom Úvodná úloha slúži na opakovanie, pripomenutie naučených postupov a overenie toho, čo sa študenti doteraz naučili o nerovnostiach. Považujeme za vhodné ukázať všetky tri zmienené postupy riešenia, keďže ekvivalentné úpravy rovníc, využívanie známych rovností aj sčítanie dvoch a viac rovností sú všetko užitočné metódy, ktoré sa oplatí mať v našej riešiteľskej zásobe.
}

\section*{Seminár 19}
\subsection*{Téma}
Teória čísel IV -- prvočísla

\subsection*{Ciele}
Precvičiť so študentmi rôzne úlohy o~prvočíslach, pri riešení ktorých sa uplatnia poznatky o~deliteľnosti nadobudnuté v~seminároch 7 a 8.

\subsection*{Úlohy a riešenia}
\begin{tcolorbox}[breakable,notitle,boxrule=0pt,colback=light-gray,colframe=light-gray]\ul{19.1} [63-I-3-N2] Číslo $n$ je súčinom dvoch rôznych prvočísel. Ak zväčšíme menšie z~nich o~1 a druhé ponecháme, ich súčin sa zväčší o~7. Určte číslo $n$.

\end{tcolorbox}

\rie Označme $p<q$ prvočísla zo zadania. Potom platí $(p+1)q=pq+7$. Po roznásobení ľavej strany a odčítaní výrazu $pq$ od oboch strán rovnosti dostávame $q=7$. Prvočíslo $p$ má byť menšie ako $q$, preto $p\in \{2,3,5\}$ a hľadaným číslom $n$ je tak jedno z~čísel 14, 21 alebo 35.\\
\\
\begin{tcolorbox}[breakable,notitle,boxrule=0pt,colback=light-gray,colframe=light-gray]\ul{19.2} [63-I-3-N4] Číslo $n$ je súčinom dvoch prvočísel. Ak zväčšíme každé z~nich o~1, ich súčin sa zväčší o~35. Určte číslo $n$.

\end{tcolorbox}

\rie Podobne ako v~predchádzajúcom prípade označme $p\leq q$ (nie nutne rôzne) prvočísla zo zadania a to prepíšme do tvaru rovnosti $(p+1)(q+1)=pq+35$. Po úprave dostávame $p+q=34$. Hľadáme teda dvojice prvočísel, ktorých súčet bude 34. Takými sú jedine 3 a 31, 5~a 29, 11 a 23, 17 a 17. Riešením úlohy je potom $n \in \{93, 145, 253, 289\}$.\\
\\
\kom Úvodné dve jednoduché úlohy majú prípravný charakter na úlohu nasledujúcu a sú skôr rozcvičkou, než náročnou aplikáciou vedomostí o~prvočíslach.\\
\\
\begin{tcolorbox}[breakable,notitle,boxrule=0pt,colback=light-gray,colframe=light-gray]\ul{19.3} [63-I-3]
Číslo $n$ je súčinom troch rôznych prvočísel. Ak zväčšíme dve menšie z~nich o~1 a najväčšie ponecháme nezmenené, zväčší sa ich súčin o~915. Určte číslo $n$.

\end{tcolorbox}

\rieh Nech $n = pqr, p < q < r$. Rovnosť $(p + 1)(q + 1)r = pqr + 915$ ekvivalentne upravíme na tvar $(p + q + 1) \cdot r = 915 = 3 \cdot 5 \cdot 61$, z~ktorého vyplýva, že prvočíslo $r$ môže nadobudnúť len niektorú z~hodnôt 3, 5 a 61. Pre $r = 3$ ale z~poslednej rovnice dostávame $(p + q + 1) \cdot 3 = 3 \cdot 5 \cdot 61$, čiže $p + q = 304$. To je spor s~tým, že $r$ je najväčšie. Analogicky zistíme, že nemôže byť ani $r = 5$. Je teda $r = 61$ a $p + q = 14$. Vyskúšaním všetkých možností pre $p$ a $q$ vyjde $p = 3$, $q = 11$, $r = 61$ a $n = 3 \cdot 11 \cdot 61 = 2 013$.\\
\\
\kom Úloha vyžaduje vhodnú manipuláciu rovnosti zo zadania a potom už len dostatočne pozornú analýzu vzniknutých možností.\\
\\
\begin{tcolorbox}[breakable,notitle,boxrule=0pt,colback=light-gray,colframe=light-gray]\ul{19.4} [64-S-3]
Nájdite najmenšie prirodzené číslo $n$ s~ciferným súčtom 8, ktoré sa rovná súčinu troch rôznych prvočísel, pričom rozdiel dvoch najmenších z~nich je 8.

\end{tcolorbox}

\rieh Hľadané číslo $n$ je súčinom troch rôznych prvočísel, ktoré označíme $p, q, r$, $p < q < r$. Číslo $n = pqr$ má ciferný súčet 8, ktorý nie je deliteľný tromi, preto ani $n$ nie je deliteľné tromi a teda $p, q, r \neq 3$. Napokon hľadané číslo $n$ nie je deliteľné ani dvoma, pretože by muselo byť $p = 2$ a $q = p + 8 = 10$, čo nie je prvočíslo. Musí teda byť $p = 5$.

Ak je $p = 5$, je $q = p + 8 = 13$, takže $r \in \{17, 19, 23, 29, 31, \ldots \}$ a $n \in \{1 105,1 235, 1 495,$ $1 885, 2 015, \ldots\}$. V~tejto množine je zrejme najmenšie číslo s~ciferným súčtom 8 číslo $2 015$.

Ak je $p > 5$, je $p = 11$ najmenšie prvočíslo také, že aj $q = p + 8$ je prvočíslo. Preto $p = 11$, $q = 19$, a teda $r = 23$, takže pre zodpovedajúce čísla $n$ platí $n = 11 \cdot 19 \cdot 23= 4 807 > 2 015$.\\
\\
\kom Úloha príjemne spája poznatky o~deliteľnosti a prvočíslach a nemala by pre študentov byť neprekonateľnou výzvou.\\
\\
\begin{tcolorbox}[breakable,notitle,boxrule=0pt,colback=light-gray,colframe=light-gray]\ul{19.5} [57-S-1]
Nájdite všetky dvojice prirodzených čísel $a, b$ väčších ako 1 tak, aby ich súčet aj súčin boli mocniny prvočísel.

\end{tcolorbox}

\rieh Z~podmienky pre súčin vyplýva, že $a$ aj $b$ sú mocninami toho istého prvočísla $p$: $a = p^r$, $b = p^s$, pričom $r, s$ sú celé kladné čísla. Keby bolo $p$ nepárne, bol by súčet $a + b$ deliteľný okrem čísla $p$ aj číslom 2, takže by nebol mocninou prvočísla. Teda $p = 2$. Ak $r < s$, je súčet $a + b = 2^r (1 + 2^{s-r})$ opäť číslo párne deliteľné nepárnym číslom väčším ako 1, nie je teda mocninou prvočísla. K~rovnakému záveru dôjdeme aj v~prípade, keď $r > s$. Ostáva preto jediná možnosť: $a = b = 2^r$ , pričom $r$ je celé kladné číslo. Skúška $a+b = 2^r +2^r = 2^{r+1}$ a $ab = 2^{2r}$ potvrdzuje, že riešením sú všetky dvojice $(a, b) = (2^r, 2^r)$, kde $r$ je celé kladné číslo.\\
\\
\begin{tcolorbox}[breakable,notitle,boxrule=0pt,colback=light-gray,colframe=light-gray]\ul{19.6} [65-I-1-D2 resp. 55-C-II-4] Nájdite všetky dvojice prvočísel $p$ a $q$, pre ktoré platí $p + q^2= q + 145p^2$.

\end{tcolorbox}

\rieh Pre prvočísla $p, q$ má platiť $q(q - 1) = p(145p -1)$, takže prvočíslo $p$ delí $q(q -1)$. Prvočíslo $p$ nemôže deliť prvočíslo $q$, pretože to by znamenalo, že $p = q$, a teda $145p = p$, čo nie je možné. Preto $p$ delí $q-1$,  t.\,j. $q - 1 = kp$ pre nejaké prirodzené $k$. Po dosadení do daného vzťahu dostaneme podmienku $$p=\frac{k+1}{145-k^2}.$$ Vidíme, že menovateľ zlomku na pravej strane je kladný jedine pre $k \leq 12$, zároveň však pre $k \leq 11$ je jeho čitateľ menší ako menovateľ: $k + 1 \leq 12 < 24 \leq 145 k^2$. Iba pre $k = 12$ tak vyjde $p$ prirodzené a prvočíslo, $p = 13$. Potom $q = 157$, čo je tiež prvočíslo. Úloha má jediné riešenie.\\
\\
\kom Úloha opäť ukazuje, že upravenie podmienok zo zadania do vhodného tvaru, o~ktorom môžeme ďalej diskutovať, je často kľúčovým krokom v~riešení. V~tomto prípade ide o~podmienku $q=kp+1$ a následný rozbor hodnôt v~čitateli a menovateli zlomku. To by v~študentoch malo umocniť dojem, že zručné narábanie s~algebraickými výrazmi nájde svoje široké uplatnenie.\\
\\
\begin{tcolorbox}[breakable,notitle,boxrule=0pt,colback=light-gray,colframe=light-gray]\ul{19.7} [66-I-2-D4, resp. 62-I-5]
Určte všetky celé čísla $n$, pre ktoré $2n^3 -3n^2 +n+3$ je prvočíslo.

\end{tcolorbox}

\rieh Ukážeme, že jedinými celými číslami, ktoré vyhovujú úlohe, sú $n = 0$ a $n = 1$.

Upravme najskôr výraz $V = 2n^3 - 3n^2 + n + 3$ nasledujúcim spôsobom:
$$V = (n^3 - 3n^2+ 2n) + (n^3 - n) + 3 = (n - 2)(n - 1)n + (n - 1)n(n + 1) + 3.$$
Oba súčiny $(n-2)(n-1)n$ a $(n-1)n(n+1)$ v~upravenom výraze $V$ sú deliteľné tromi pre každé celé číslo $n$ (v~oboch prípadoch ide o~súčin troch po sebe idúcich celých čísel), takže výraz $V$ je pre všetky celé čísla $n$ deliteľný tromi. Hodnota výrazu $V$ je preto prvočíslom práve vtedy, keď $V = 3$, teda práve vtedy, keď súčet oboch spomenutých súčinov je rovný nule:
$$0 = (n - 2)(n - 1)n + (n - 1)n(n + 1) = n(n - 1)[(n - 2) + (n + 1)] = n(n - 1)(2n - 1).$$
Poslednú podmienku však spĺňajú iba dve celé čísla $n$, a to $n = 0$ a $n = 1$. Tým je úloha vyriešená.\\
\textit{Poznámka}. Fakt, že výraz $V$ je deliteľný tromi pre ľubovoľné celé $n$, môžeme odvodiť aj tak, že doňho postupne dosadíme $n = 3k$, $n = 3k + 1$ a $n = 3k + 2$, pričom $k$ je celé číslo, rozdelíme teda všetky celé čísla $n$ na tri skupiny podľa toho, aký dávajú zvyšok po delení tromi.\\
\\
\kom Aj keď vzorové riešenie môže spočiatku vyzerať trikovo, po vyskúšaní niekoľkých málo hodnôt $n$ je vždy hodnota čísla zo zadania deliteľná 3, čo by študentov malo priviesť k~myšlienke skúsiť dokázať deliteľnosť čísla zo zadania tromi.\\
\\
\begin{tcolorbox}[breakable,notitle,boxrule=0pt,colback=light-gray,colframe=light-gray]\ul{19.8} [MŘMUI TODO, 2.3, str 174] Nájdite všetky prvočísla, ktoré sú súčasne súčtom a rozdielom dvoch vhodných prvočísel.

\end{tcolorbox}

\rieh Predpokladajme, že prvočíslo $p$ je súčasne súčtom aj rozdielom dvoch prvočísel. Potom je však $p>2$ a teda je $p$ nepárne. Pretože je $p$ zároveň súčet aj rozdiel dvoch prvočísel, jedno z~nich musí byť vždy párne, teda 2. Takže hľadáme prvočísla $p, p_1, p_2$ tak, že $p=p_1+2=p_2-2$, teda $p_1, p, p-2$ sú tri po sebe idúce nepárne čísla a teda práve jedno z~nich je deliteľné troma (študenti by si mali rozmyslieť prečo). Avšak troma je deliteľné jediné prvočíslo 3, odkiaľ vzhľadom na to, že $p_1\geq 1$ vyplýva $p_1=3$, $p=5$ a $p_2=7$. Jediné prvočíslo vyhovujúce zadaniu je teda $p=5$.\\
\\
\kom Úloha, ktorá vyžaduje viac uvažovania, než tvrdého počítania, je zaujímavá práve jediným výsledkom.\\
\\
\begin{tcolorbox}[breakable,notitle,boxrule=0pt,colback=light-gray,colframe=light-gray]\ul{19.9} \cite[str. 95]{thiele1986} Nájdite celočíselné riešenia rovnice $$\frac{1}{x}+\frac{1}{y}=\frac{1}{p},$$ kde $p$ je pevne dané prvočíslo.

\end{tcolorbox}

\rieh Ak existujú vôbec nejaké riešenia vyšetrovanej rovnice, potom sú nenulové. Preto môžeme rovnicu upraviť na ekvivalentný tvar $yx-px-py=0$, resp. $(x-p)(y-p)-p^2=0$, a teda $$(x-p)(y-p)=p^2.$$ Odtiaľ je vidieť, že celočíselné riešenia môžeme dostať len vtedy, ak $x-p$ prebehne všetkých deliteľov čísla $p^2$, pričom $y-p$ prebehne doplnkové delitele. Pretože je $p$ prvočíslo, musí byť nutne $$x-p \in \{1, p, p^2, -1, -p, -p^2\}.$$ Pretože $x\neq 0$, odpadá $x-p=-p$. Ostáva teda $$x \in \{1+p, 2p, p+p^2, p-1, p-p^2\} \ \ \ \ \text{a teda} \ \ \ \ y \in \{p+p^2, 2p, 1+p, p-p^2, p-1\}.$$ Tieto hodnoty sú skutočne riešením, o~čom sa môžeme presvedčiť skúškou.\\
\\
\kom Úloha, v~ktorej opäť predtým, než uplatníme znalosti o~deliteľnosti, príp. prvočíslach, musíme umne upraviť východiskový tvar rovnice.


\subsection*{Domáca práca}

\begin{tcolorbox}[breakable,notitle,boxrule=0pt,colback=light-gray,colframe=light-gray]\ul{19.10} [65-I-1]
Nájdite všetky možné hodnoty súčinu prvočísel $p$, $q$, $r$, pre ktoré platí
$$p^2 - (q + r)^2= 637.$$

\end{tcolorbox}

\rieh Ľavú stranu danej rovnice rozložíme na súčin podľa vzorca pre $A^2 - B^2$. V~takto upravenej rovnici
$$(p + q + r)(p - q - r) = 637$$
už ľahko rozoberieme všetky možnosti pre dva celočíselné činitele naľavo. Prvý z~nich je väčší a kladný, preto aj druhý musí byť kladný (lebo taký je ich súčin), takže podľa rozkladu na súčin prvočísel čísla $637 = 7^2 \cdot 13$ ide o~jednu z~dvojíc $(637, 1)$, $(91, 7)$
alebo $(49, 13)$. Prvočíslo $p$ je zrejme aritmetickým priemerom oboch činiteľov, takže sa musí rovnať jednému z~čísel $\frac{1}{2}(637 + 1) = 319$, $\frac{1}{2}(91 + 7) = 49$, $\frac{1}{2}(49 + 13) = 31$. Prvé dve z~nich však prvočísla nie sú ($319 = 11 \cdot  29$ a $49 = 7^2$), tretie áno. Takže nutne $p = 31$ a prislúchajúce rovnosti $31 + q + r = 49$ a $31 - q - r = 13$ platia práve vtedy, keď $q + r = 18$. Také dvojice prvočísel $\{q, r\}$ sú iba $\{5, 13\}$ a $\{7, 11\}$ (stačí prebrať
všetky možnosti, alebo si uvedomiť, že jedno z~prvočísel $q$, $r$ musí byť aspoň $18 : 2 = 9$, nanajvýš však $18 - 2 = 16$). Súčin $pqr$ tak má práve dve možné hodnoty, a to $31 \cdot  5\cdot  13 = 2 015$ a $31 \cdot  7 \cdot  11 = 2 387$.\\

\subsection*{Doplňujúce zdroje a materiály}
Ďalšie zaujímavé príklady je možné nájsť v~[~\cite{herman2011}], paragraf 2, taktiež v~[~\cite{holton2010}] alebo na [PP] TODO.
\seminar{20}{Teória čísel IV -- prvočísla}

\teachernote{
\subsection*{Ciele}
Precvičiť so študentmi rôzne úlohy o~prvočíslach, pri riešení ktorých sa uplatnia poznatky o~deliteľnosti nadobudnuté v~seminároch 7 a 8.

}
\subsection*{Úlohy a riešenia}

% Do not delete this line (pandoc magic!)

\problem{63-I-3-N2}{
Číslo $n$ je súčinom dvoch rôznych prvočísel. Ak zväčšíme menšie z~nich o~1 a druhé ponecháme, ich súčin sa zväčší o~7. Určte číslo $n$.
}{
\rie Označme $p<q$ prvočísla zo zadania. Potom platí $(p+1)q=pq+7$. Po roznásobení ľavej strany a odčítaní výrazu $pq$ od oboch strán rovnosti dostávame $q=7$. Prvočíslo $p$ má byť menšie ako $q$, preto $p\in \{2,3,5\}$ a hľadaným číslom $n$ je tak jedno z~čísel 14, 21 alebo 35.\\
\\
}


% Do not delete this line (pandoc magic!)

\problem{63-I-3-N4}{seminar20,prvocisla,domacekolo}{
Číslo $n$ je súčinom dvoch prvočísel. Ak zväčšíme každé z~nich o~1, ich súčin sa zväčší o~35. Určte číslo $n$.
}{
\rie Podobne ako v~predchádzajúcom prípade označme $p\leq q$ (nie nutne rôzne) prvočísla zo zadania a to prepíšme do tvaru rovnosti $(p+1)(q+1)=pq+35$. Po úprave dostávame $p+q=34$. Hľadáme teda dvojice prvočísel, ktorých súčet bude 34. Takými sú jedine 3 a 31, 5~a 29, 11 a 23, 17 a 17. Riešením úlohy je potom $n \in \{93, 145, 253, 289\}$.\\
\\
\kom Úvodné dve jednoduché úlohy majú prípravný charakter na úlohu nasledujúcu a sú skôr rozcvičkou, než náročnou aplikáciou vedomostí o~prvočíslach.\\
\\
}


% Do not delete this line (pandoc magic!)

\problem{63-I-3}{seminar20,prvocisla}{
Číslo $n$ je súčinom troch rôznych prvočísel. Ak zväčšíme dve menšie z~nich o~1 a najväčšie ponecháme nezmenené, zväčší sa ich súčin o~915. Určte číslo $n$.
}{
\rieh Nech $n = pqr, p < q < r$. Rovnosť $(p + 1)(q + 1)r = pqr + 915$ ekvivalentne upravíme na tvar $(p + q + 1) \cdot r = 915 = 3 \cdot 5 \cdot 61$, z~ktorého vyplýva, že prvočíslo $r$ môže nadobudnúť len niektorú z~hodnôt 3, 5 a 61. Pre $r = 3$ ale z~poslednej rovnice dostávame $(p + q + 1) \cdot 3 = 3 \cdot 5 \cdot 61$, čiže $p + q = 304$. To je spor s~tým, že $r$ je najväčšie. Analogicky zistíme, že nemôže byť ani $r = 5$. Je teda $r = 61$ a $p + q = 14$. Vyskúšaním všetkých možností pre $p$ a $q$ vyjde $p = 3$, $q = 11$, $r = 61$ a $n = 3 \cdot 11 \cdot 61 = 2 013$.\\
\\
\kom Úloha vyžaduje vhodnú manipuláciu rovnosti zo zadania a potom už len dostatočne pozornú analýzu vzniknutých možností.\\
\\
}


\problem{64-S-3}{
Nájdite najmenšie prirodzené číslo $n$ s~ciferným súčtom 8, ktoré sa rovná súčinu troch rôznych prvočísel, pričom rozdiel dvoch najmenších z~nich je 8.
}{
\rieh Hľadané číslo $n$ je súčinom troch rôznych prvočísel, ktoré označíme $p, q, r$, $p < q < r$. Číslo $n = pqr$ má ciferný súčet 8, ktorý nie je deliteľný tromi, preto ani $n$ nie je deliteľné tromi a teda $p, q, r \neq 3$. Napokon hľadané číslo $n$ nie je deliteľné ani dvoma, pretože by muselo byť $p = 2$ a $q = p + 8 = 10$, čo nie je prvočíslo. Musí teda byť $p = 5$.

Ak je $p = 5$, je $q = p + 8 = 13$, takže $r \in \{17, 19, 23, 29, 31, \ldots \}$ a $n \in \{1 105,1 235, 1 495,$ $1 885, 2 015, \ldots\}$. V~tejto množine je zrejme najmenšie číslo s~ciferným súčtom 8 číslo $2 015$.

Ak je $p > 5$, je $p = 11$ najmenšie prvočíslo také, že aj $q = p + 8$ je prvočíslo. Preto $p = 11$, $q = 19$, a teda $r = 23$, takže pre zodpovedajúce čísla $n$ platí $n = 11 \cdot 19 \cdot 23= 4 807 > 2 015$.\\
\\
\kom Úloha príjemne spája poznatky o~deliteľnosti a prvočíslach a nemala by pre študentov byť neprekonateľnou výzvou.\\
\\
}


% Do not delete this line (pandoc magic!)

\problem{57-S-1}{seminar20,prvocisla}{
Nájdite všetky dvojice prirodzených čísel $a, b$ väčších ako 1 tak, aby ich súčet aj súčin boli mocniny prvočísel.
}{
\rieh Z~podmienky pre súčin vyplýva, že $a$ aj $b$ sú mocninami toho istého prvočísla $p$: $a = p^r$, $b = p^s$, pričom $r, s$ sú celé kladné čísla. Keby bolo $p$ nepárne, bol by súčet $a + b$ deliteľný okrem čísla $p$ aj číslom 2, takže by nebol mocninou prvočísla. Teda $p = 2$. Ak $r < s$, je súčet $a + b = 2^r (1 + 2^{s-r})$ opäť číslo párne deliteľné nepárnym číslom väčším ako 1, nie je teda mocninou prvočísla. K~rovnakému záveru dôjdeme aj v~prípade, keď $r > s$. Ostáva preto jediná možnosť: $a = b = 2^r$ , pričom $r$ je celé kladné číslo. Skúška $a+b = 2^r +2^r = 2^{r+1}$ a $ab = 2^{2r}$ potvrdzuje, že riešením sú všetky dvojice $(a, b) = (2^r, 2^r)$, kde $r$ je celé kladné číslo.\\
\\
}


\input{problems/65-I-1-D2.tex}

\input{problems/62-I-5.tex}

\kom Aj keď vzorové riešenie môže vyzerať trikovo, po vyskúšaní niekoľko málo hodnôt $n$ je vždy hodnota zo zadania deliteľná 3, čo by študentov mohlo priviesť na myšlienku skúsiť dokázať deliteľnosť čísla zo zadania tromi.

% Do not delete this line (pandoc magic!)

\problem{MŘMUI TODO, 2.3, str 174}{
Nájdite všetky prvočísla, ktoré sú súčasne súčtom a rozdielom dvoch vhodných prvočísel.
}{
\rieh Predpokladajme, že prvočíslo $p$ je súčasne súčtom aj rozdielom dvoch prvočísel. Potom je však $p>2$ a teda je $p$ nepárne. Pretože je $p$ zároveň súčet aj rozdiel dvoch prvočísel, jedno z~nich musí byť vždy párne, teda 2. Takže hľadáme prvočísla $p, p_1, p_2$ tak, že $p=p_1+2=p_2-2$, teda $p_1, p, p-2$ sú tri po sebe idúce nepárne čísla a teda práve jedno z~nich je deliteľné troma (študenti by si mali rozmyslieť prečo). Avšak troma je deliteľné jediné prvočíslo 3, odkiaľ vzhľadom na to, že $p_1\geq 1$ vyplýva $p_1=3$, $p=5$ a $p_2=7$. Jediné prvočíslo vyhovujúce zadaniu je teda $p=5$.\\
\\
\kom Úloha, ktorá vyžaduje viac uvažovania, než tvrdého počítania, je zaujímavá práve jediným výsledkom.\\
\\
}


\problem{str. 95}{
{thiele1986} Nájdite celočíselné riešenia rovnice $$\frac{1}{x}+\frac{1}{y}=\frac{1}{p},$$ kde $p$ je pevne dané prvočíslo.
}{
\rieh Ak existujú vôbec nejaké riešenia vyšetrovanej rovnice, potom sú nenulové. Preto môžeme rovnicu upraviť na ekvivalentný tvar $yx-px-py=0$, resp. $(x-p)(y-p)-p^2=0$, a teda $$(x-p)(y-p)=p^2.$$ Odtiaľ je vidieť, že celočíselné riešenia môžeme dostať len vtedy, ak $x-p$ prebehne všetkých deliteľov čísla $p^2$, pričom $y-p$ prebehne doplnkové delitele. Pretože je $p$ prvočíslo, musí byť nutne $$x-p \in \{1, p, p^2, -1, -p, -p^2\}.$$ Pretože $x\neq 0$, odpadá $x-p=-p$. Ostáva teda $$x \in \{1+p, 2p, p+p^2, p-1, p-p^2\} \ \ \ \ \text{a teda} \ \ \ \ y \in \{p+p^2, 2p, 1+p, p-p^2, p-1\}.$$ Tieto hodnoty sú skutočne riešením, o~čom sa môžeme presvedčiť skúškou.\\
\\
\kom Úloha, v~ktorej opäť predtým, než uplatníme znalosti o~deliteľnosti, príp. prvočíslach, musíme umne upraviť východiskový tvar rovnice.
}



\home{
\subsection*{Domáca práca}


% Do not delete this line (pandoc magic!)

\problem{65-I-1}{seminar20,prvocisla}{
Nájdite všetky možné hodnoty súčinu prvočísel $p$, $q$, $r$, pre ktoré platí
$$p^2 - (q + r)^2= 637.$$
}{
\rieh Ľavú stranu danej rovnice rozložíme na súčin podľa vzorca pre $A^2 - B^2$. V~takto upravenej rovnici
$$(p + q + r)(p - q - r) = 637$$
už ľahko rozoberieme všetky možnosti pre dva celočíselné činitele naľavo. Prvý z~nich je väčší a kladný, preto aj druhý musí byť kladný (lebo taký je ich súčin), takže podľa rozkladu na súčin prvočísel čísla $637 = 7^2 \cdot 13$ ide o~jednu z~dvojíc $(637, 1)$, $(91, 7)$
alebo $(49, 13)$. Prvočíslo $p$ je zrejme aritmetickým priemerom oboch činiteľov, takže sa musí rovnať jednému z~čísel $\frac{1}{2}(637 + 1) = 319$, $\frac{1}{2}(91 + 7) = 49$, $\frac{1}{2}(49 + 13) = 31$. Prvé dve z~nich však prvočísla nie sú ($319 = 11 \cdot  29$ a $49 = 7^2$), tretie áno. Takže nutne $p = 31$ a prislúchajúce rovnosti $31 + q + r = 49$ a $31 - q - r = 13$ platia práve vtedy, keď $q + r = 18$. Také dvojice prvočísel $\{q, r\}$ sú iba $\{5, 13\}$ a $\{7, 11\}$ (stačí prebrať
všetky možnosti, alebo si uvedomiť, že jedno z~prvočísel $q$, $r$ musí byť aspoň $18 : 2 = 9$, nanajvýš však $18 - 2 = 16$). Súčin $pqr$ tak má práve dve možné hodnoty, a to $31 \cdot  5\cdot  13 = 2 015$ a $31 \cdot  7 \cdot  11 = 2 387$.\\
}

}

\teachernote{
\subsection*{Doplňujúce zdroje a materiály}
Ďalšie zaujímavé príklady je možné nájsť v  \cite{herman2011}, paragraf 2 a taktiež v \cite{holton2010}. %alebo na \todo{[PP]}.
}



\section{Marec}
\seminar{21}

\subsection*{Téma}
Teória čísel V~-- miš-maš
\teachernote{
\subsection*{Ciele}
Trénovať riešenie rôznorodých úloh z~oblasti elementárnej teórie čísel bez špecifického zamerania

\textbf{Úvodný komentár.}
Toto seminárne stretnutie sa nevyznačuje špecifickou témou, ale ide skôr o~panoptikum rôznych úloh, ktoré sa dajú zahrnúť do oblasti teórie čísel lepšie ako do akejkoľvek inej.

}
\subsection*{Úlohy a riešenia}


% Do not delete this line (pandoc magic!)

\problem{65-II-4}{seminar21,mismas}{
Adam s~Barborou hrajú so zlomkom
$$ \frac{10a + b}{10c + d}$$
takúto hru na štyri ťahy: Hráči striedavo nahrádzajú ľubovoľné z~doposiaľ neurčených písmen $a$, $b$, $c$, $d$ nejakou cifrou od 1 do 9. Barbora vyhrá, keď výsledný zlomok bude rovný buď celému číslu, alebo číslu s~konečným počtom desatinných miest; inak vyhrá Adam (napríklad keď vznikne zlomok $\frac{11}{29}$). Ak začína Adam, ako má hrať Barbora, aby zaručene vyhrala? Ak začína Barbora, je možné poradiť Adamovi tak, aby vždy vyhral?
}{
\rieh Ak má prvý ťah Adam, môže Barbora hrať tak, aby bol výsledný zlomok rovný jednej, čo podľa zadania prinesie Barbore výhru. Taký zlomok vyjde, keď budú súčasne platiť obe rovnosti $a = c$ a $b = d$, ktoré Barbora dosiahne ťahmi \uv{súmerne združenými} podľa zlomkovej čiary s~Adamovými ťahmi.

Ak začína Barbora, môže Adam hrať tak, aby vyšiel zlomok s~menovateľom $10c+d$ deliteľným tromi, ktorého čitateľ $10a + b$ však deliteľný tromi nebude. Na to Adamovi stačí po každom z~oboch Barboriných ťahov vhodne \uv{doplniť} čitateľ či menovateľ, napríklad podľa kritéria deliteľnosti tromi mu stačí zabezpečiť, aby sa ciferný súčet $a+b$ čitateľa rovnal 10 a aby sa ciferný súčet $c+d$ menovateľa rovnal 9 alebo 12. Adam tak vyhrá, pretože výsledný zlomok nebude možné krátiť tromi, takže sa nebude rovnať žiadnemu zlomku s~mocninou čísla 10 v~menovateli, akým sa dá zapísať každé číslo s~konečným počtom desatinných miest.\\
\\
\kom Úlohu je možné najprv zadať ako hru medzi dvoma hráčmi a až po tom, čo študenti odohrajú niekoľko kôl a vypozorujú zákonitosti, je vhodné pustiť sa do tvrdého riešenia. Zaujímavé tiež môže byť porovnať stratégie jednotlivých študentov medzi sebou, príp. ich po samostatnej práci nechať niekoľko súbojov odohrať znova, aby svoju stratégiu overili v~praxi.\\
\\
}


% Do not delete this line (pandoc magic!)

\problem{57-I-1-N1}{
Ak $m, k$ a $\sqrt[k]{m}$ sú celé čísla väčšie ako 1, tak v~rozklade čísla $m$ na súčin prvočísel sa každé prvočíslo vyskytuje v~mocnine, ktorej exponent je násobkom čísla~$k$. Dokážte.
}{
\rieh Rozklad čísla $m$ dostaneme, keď rozklad čísla $\sqrt[k]{m}$ umocníme na $k$-tu, každý exponent v~rozklade čísla $m$ tak bude súčinom exponentu v~rozklade čísla $\sqrt[k]{m}$ a čísla $k$. \\ %Nech je rozklad čísla $\sqrt[k]{m}=p_1^{\alpha_1}\cdot p_2^{\alpha_2}\cdots p_n^{\alpha_n}$ a rozklad čísla $m=p_1^{\beta_1}\cdot p_2^{\beta_2}\cdots p_n^{\beta_n}$, kde $p_1<\,\ldots < p_n$ sú prvočísla\\
\\
\kom Úloha je prípravou k~riešeniu komplexnejšieho problému, ktorý nasleduje.\\
\\
}


% Do not delete this line (pandoc magic!)

\problem{57-I-1}{
Určte najmenšie prirodzené číslo $n$, pre ktoré aj čísla
$\sqrt{2n}, \sqrt[3]{3n}, \sqrt[5]{5n}$ sú prirodzené.
}{
\rieh Vysvetlíme, prečo prvočíselný rozklad hľadaného čísla musí obsahovať len vhodné mocniny prvočísel 2, 3 a 5. Každé prípadné ďalšie prvočíslo by sa v~rozklade čísla $n$ muselo vyskytovať v~mocnine, ktorej exponent je deliteľný dvoma, tromi aj piatimi zároveň (viď predchádzajúca úloha). Po vyškrtnutí takého prvočísla by sa číslo $n$ zmenšilo a skúmané odmocniny by pritom ostali celočíselné.

Položme preto $n = 2^a \cdot 3^b \cdot 5^c$, pričom $a, b, c$ sú prirodzené čísla. Čísla $\sqrt[3]{3n}$ a $\sqrt[5]{5n}$ sú celé, preto je exponent $a$ násobkom troch a piatich. Aj $\sqrt{2n}$ je celé číslo, preto musí byť číslo $a$ nepárne. Je teda nepárnym násobkom pätnástich: $a \in \{15, 45, 75,\,\ldots\}$. Analogicky je exponent~$b$ taký násobok desiatich, ktorý po delení tromi dáva zvyšok 2: $b \in \{20, 50, 80,\,\ldots\}$. Napokon $c$ je násobok šiestich, ktorý po delení piatimi dáva zvyšok 4: $c \in \{24, 54, 84,\,\ldots\}$. Z~podmienky, že $n$ je najmenšie, dostávame $n = 2^{15} \cdot3^{20} \cdot 5^{24}$.

Presvedčíme sa ešte, že dané odmocniny sú prirodzené čísla:
$$\sqrt{2n} = 2^8 \cdot 3^{10} \cdot 5^{12},\ \ \ \sqrt[3]{3n} = 2^5 \cdot 3^7 \cdot 5^8, \ \ \ \sqrt[5]{5n} = 2^3 \cdot 3^2 \cdot5^5.$$

\textit{Záver}. Hľadaným číslom je $n = 2^{15} \cdot 3^{20} \cdot 5^{24}$.\\
\\
\kom Úloha je netradičným príkladom uplatnenia poznatkov o~rozklade čísla na súčin prvočísel a vďaka návodnej úlohe by študenti mali byť dostatočne pripravení na jej samostatné riešenie.\\
\\
}


% Do not delete this line (pandoc magic!)

\problem{60-II-2}{seminar21,mismas,vyrazy,krajskekolo}{
Nájdite všetky kladné celé čísla $n$, pre ktoré je číslo $n^2 + 6n$ druhou mocninou celého čísla.
}{
\rieh Zrejme $n^2 +6n > n^2$ a zároveň $n^2 +6n < n^2 +6n+9 = (n+3)^2$. V~uvedenom intervale ležia iba dve druhé mocniny celých čísel: $(n + 1)^2$ a $(n + 2)^2$.

V~prvom prípade máme $n^2 + 6n = n^2 + 2n + 1$, teda $4n = 1$, tomu však žiadne celé číslo $n$ nevyhovuje.

V~druhom prípade máme $n^2 + 6n = n^2 + 4n + 4$, teda $2n = 4$. Dostávame tak jediné riešenie $n = 2$.\\
\\
\textbf{Iné riešenie*.} Budeme skúmať rozklad $n^2 + 6n = n(n+ 6)$. Spoločný deliteľ oboch čísel $n$ a $n + 6$ musí deliť aj ich rozdiel, preto ich najväčším spoločným deliteľom môžu byť len čísla 1, 2, 3 alebo 6. Tieto štyri možnosti rozoberieme.

Keby boli čísla $n$ a $n+6$ nesúdeliteľné, muselo by byť každé z~nich druhou mocninou. Rozdiel dvoch druhých mocnín prirodzených čísel však nikdy nie je 6. Pre malé čísla sa o~tom ľahko presvedčíme, a pre $k = 4$ už je rozdiel susedných štvorcov $k^2$ a $(k - 1)^2$ aspoň 7. Vlastnosť, že 1, 3, 4, 5 a 7 je päť najmenších rozdielov dvoch druhých mocnín, využijeme aj ďalej.

Ak je najväčším spoločným deliteľom čísel $n$ a $n+6$ číslo 2, je $n = 2m$ pre vhodné $m$, ktoré navyše nie je deliteľné tromi. Ak $n(n + 6) = 4m(m + 3)$ je štvorec, musí byť aj $m(m + 3)$ štvorec. Čísla $m$ a $m + 3$ sú však nesúdeliteľné, preto musí byť každé z~nich druhou mocninou prirodzeného čísla. To nastane len pre $m = 1$, čiže $n = 2$. Ľahko overíme, že $n(n + 6)$ je potom naozaj druhou mocninou celého čísla.

Ak je najväčším spoločným deliteľom čísel $n$ a $n + 6$ číslo 3, je $n = 3m$ pre vhodné nepárne $m$. Ak $n(n+6) = 9m(m+2)$ je štvorec, musia byť nesúdeliteľné čísla $m$ a $m+2$ tiež štvorce. Také dva štvorce však neexistujú.

Ak je najväčším spoločným deliteľom čísel $n$ a $n + 6$ číslo 6, je $n = 6m$ pre vhodné $m$. Ak $n(n + 6) = 36m(m + 1)$ je štvorec, musia byť štvorce aj obe nesúdeliteľné čísla $m$ a $m + 1$, čo nastane len pre $m = 0$, my však hľadáme len kladné čísla $n$.

Úlohe vyhovuje jedine $n = 2$.\\
\\
\kom K~správnemu riešeniu úlohy vedú mnohé cesty. Prvé uvedené riešenie je trochu trikové, avšak nápadité, a preto ak ho študenti neobjavia, je vhodné im ho na záver ukázať. Nie je tiež nepravdepodobné, že študenti budú skúšať, ako sa číslo $n^2+6n$ správa pre rôzne hodnoty $n$, čo by ich mohlo naviesť na správu cestu nájdenia jediného riešenia.\\
\\
}


% Do not delete this line (pandoc magic!)

\problem{66-II-1}{seminar21,mismas,mnohocleny,sustavy,krajskekolo}{
Nájdite všetky mnohočleny $P(x) = ax^2 +bx+c$ s~celočíselnými koeficientami spĺňajúce
$$1 < P(1) < P(2) < P(3) \ \ \ \text{a súčasne} \ \  \
\frac{P(1) \cdot P(2) \cdot P(3)}{4}= 17^2.$$
}{
\rieh Rovnosť zo zadania je ekvivalentná rovnosti $P(1)\cdot P(2)\cdot P(3) = 4\cdot17^2$, takže čísla $P(1)$, $P(2)$, $P(3)$ môžu byť iba z~množiny deliteľov čísla $4 \cdot 17^2$ väčších ako 1:
$$2 < 4 < 17 < 2 \cdot 17 < 4 \cdot 17 < 17^2< 2 \cdot 17^2< 4 \cdot 17^2.$$

Ak by platilo $P(1) = 4$, bol by súčin $P(1)\cdot P(2)\cdot P(3)$ aspoň $4 \cdot 17 \cdot (2 \cdot 17) = 8 \cdot 17^2$, čo nevyhovuje zadaniu. Preto $P(1) = 2$ a tak je nutne $P(2) = 17$, pretože keby bolo $P(2) = 4$, musel by byť daný súčin $4 \cdot 17^2$ deliteľný číslom $P(1)\cdot P(2) = 8$, čo neplatí, a pre $P(2) = 2 \cdot 17$ by bol súčin $P(1)\cdot P(2)\cdot P(3)$ opäť príliš veľký. Pre tretiu neznámu
hodnotu $P(3)$ potom vychádza $P(3) = 4 \cdot 17^2 /(2 \cdot 17) = 2 \cdot 17$.

Hľadané koeficienty $a$, $b$, $c$ tak sú práve také celé čísla, ktoré vyhovujú sústave
\begin{align*}
P(1) &= a + b + c = 2,\\
P(2) &= 4a + 2b + c = 17,\\
P(3) &= 9a + 3b + c = 34.
\end{align*}
Jej vyriešením dostaneme $a = 1$, $b = 12$, $c = -11$.

\textit{Záver}. Úlohe vyhovuje jediný mnohočlen $P(x) = x^2 + 12x - 11$.\\
\\
\kom Úloha spája poznatky o~deliteľnosti, mnohočlenoch a na jej úspešné doriešenie je nutná aj schopnosť popasovať sa so sústavou troch rovníc s~tromi neznámymi. Zaujímavé bude tiež pozorovať, koľko študentov si spomenie, že podobnou úlohou sa už zaoberali v~seminári 6.\\
\\
}


% Do not delete this line (pandoc magic!)

\problem{64-II-1}{seminar21,mismas}{
Celé čísla od 1 do 9 rozdelíme ľubovoľne na tri skupiny po troch a potom čísla v~každej skupine medzi sebou vynásobíme.
\begin{enumerate}[a)]
    \item Určte tieto tri súčiny, ak viete, že dva z~nich sa rovnajú a sú menšie ako tretí súčin.
    \item Predpokladajme, že jeden z~troch súčinov, ktorý označíme $S$, je menší ako dva ostatné súčiny (ktoré môžu byť rovnaké). Nájdite najväčšiu možnú hodnotu $S$.
\end{enumerate}
}{
\rieh Najskôr vyjadríme súčin všetkých deviatich čísel pomocou jeho rozkladu na súčin prvočísel:
$$ 1 \cdot 2 \cdot 3 \cdot 4 \cdot 5 \cdot 6 \cdot 7 \cdot 8 \cdot 9 = 2^7 \cdot 3^4 \cdot 5 \cdot 7.$$

a) Označme dva z~uvažovaných (rôznych) súčinov $S$ a $Q$, pričom $S < Q$. Z~rovnosti
$$S \cdot S~\cdot Q = 2^7 \cdot 3^4 \cdot 5 \cdot 7$$
vidíme, že prvočísla 5 a 7 musia byť zastúpené v~súčine $Q$, takže $Q = 5 \cdot 7 \cdot x = 35x$, pričom $x$ je jedno zo zvyšných čísel 1, 2, 3, 4, 6, 8 a 9. Ďalej vidíme, že v~rozklade dotyčného $x$ musí mať prvočíslo 2 nepárny exponent a prvočíslo 3 párny exponent -- tomu vyhovujú iba čísla 2 a 8. Pre $x = 2$ ale vychádza $Q = 35 \cdot 2 = 70 < S= 2^3 \cdot 3^2 = 72$, čo odporuje predpokladu $S < Q$. Preto je nutne $x = 8$, pre ktoré vychádza $Q = 35 \cdot 2 = 280$ a $S^2 = 2^4 \cdot 3^4$ čiže $S = 2^2 \cdot 3^2 = 36$. Trojica súčinov je teda
$(36, 36, 280)$.

Ostáva ukázať, že získanej trojici naozaj zodpovedá rozdelenie daných deviatich čísel na trojice:
$$S = 1 \cdot 4 \cdot 9 = 36, \ \ \ S~= 2 \cdot 3 \cdot 6 = 36, \ \ \ Q = 5 \cdot 7 \cdot 8 = 280.$$

b) Označme uvažované súčiny $S, Q$ a $R$, pričom $S < Q$ a $S < R$ (nie je ale vylúčené, že $Q = R$). V~riešení časti a) sme zistili, že platí rovnosť
$$S \cdot Q \cdot R = 70 \cdot 72 \cdot 72.$$
Ak teda ukážeme, že existuje rozdelenie čísel, pri ktorom $S = 70$ a $R = Q = 72$, bude $S = 70$ hľadaná najväčšia hodnota, lebo keby pri niektorom rozdelení platilo $S \geq 71$, muselo by byť $R \geq 72$ aj $Q \geq 72$ a tiež $S \cdot Q \cdot R \geq 71 \cdot 72 \cdot 72$, čo zrejme odporuje predchádzajúcej rovnosti. Nájsť potrebné rozdelenie je jednoduché:
$$ S~= 2 \cdot 5 \cdot 7 = 70, \ \ \ Q = 1 \cdot 8 \cdot 9 = 72, \ \ \  R = 3 \cdot 4 \cdot 6 = 72.$$ \\
\\
\kom Zaujímavá úloha, ktorá dôvtipne využíva rozklady čísel na súčin prvočísel.
}




\subsection*{Domáca práca}

% Do not delete this line (pandoc magic!)

\problem{58-I-5}{
Z~množiny $\{1, 2, 3, \ldots, 99\}$ vyberte čo najväčší počet čísel tak, aby súčet žiadnych dvoch vybraných čísel nebol násobkom jedenástich. (Vysvetlite, prečo zvolený výber má požadovanú vlastnosť a prečo žiadny výber väčšieho počtu čísel nevyhovuje.)
}{
\rieh Čísla od 1 do 99 rozdelíme podľa ich zvyšku po delení číslom 11 do jedenástich deväťprvkových skupín $T_0, T_1 ,\ldots, T_{10}$:
\begin{center}
\begin{align*}
T_0 &= \{11, 22, 33, . . . , 99\},\\
T_1 &= \{1, 12, 23, . . . , 89\},\\
T_2 &= \{2, 13, 24, . . . , 90\},\\
\vdots\\
T_{10} &= \{10, 21, 32, . . . , 98\}.\\
\end{align*}
\end{center}
Ak vyberieme jedno číslo z~$T_0$ (viac ich ani vybrať nesmieme) a všetky čísla z~$T_1, T_2, T_3, T_4$ a $T_5$, dostaneme vyhovujúci výber $1 + 5 \cdot 9 = 46$ čísel, lebo súčet dvoch čísel z~$0, 1, 2, 3, 4, 5$ je deliteľný jedenástimi jedine v~prípade 0 + 0, z~množiny $T_0$ sme však vybrali iba jedno číslo.

Na druhej strane, v~ľubovoľnom vyhovujúcom výbere je najviac jedno číslo zo skupiny $T_0$ a najviac 9 čísel z~každej zo skupín
$$ T_1 \cup T_{10}, \ \ T_2 \cup T_9, \ \ T_3 \cup T_8, \ \  T_4 \cup T_7, \ \ T_5 \cup T_6,$$
lebo pri výbere 10 čísel z~niektorej skupiny $T_i \cup T_{11-i}$ by medzi vybranými bolo niektoré číslo zo skupiny $T_i$ a aj niektoré číslo zo skupiny $T_{11-i}$; ich súčet by potom bol deliteľný jedenástimi. Celkom je teda vo výbere najviac $1 + 5 \cdot 9 = 46$ čísel.

\textit{Poznámka}. Možno uvedené \uv{učesané} riešenie vyzerá príliš trikovo. Avšak počiatočné úvahy každého riešiteľa k~nemu rýchlo vedú: iste záleží len na zvyškoch vybraných čísel, takže rozdelenie na triedy $T_i$ a vyberanie z~nich je prirodzené. Je jasné, že z~$T_0$ môže byť vybrané len jedno číslo a všetko ďalšie, o~čo sa musíme starať, je požiadavka, aby sme nevybrali zároveň po čísle zo skupiny $T_i$ aj zo skupiny $T_{11-i}$. Ak je už vybrané niektoré číslo z~triedy $T_i$, kde $i\neq 0$, môžeme pokojne vybrať všetky čísla z~$T_i$, to už skúmanú vlastnosť nepokazí. Je preto dokonca jasné, ako všetky možné výbery najväčšieho počtu čísel vyzerajú.\\
\\
\kom Je veľmi vhodné sa k tejto úlohe v nasledujúcom seminári vrátiť s poznámkou, že množiny $T_1, \ldots, T_{10}$ nazývame zvyškové triedy. \\
\\
}


\input{problems/66-I-6.tex}



\seminar{22}

\subsection*{Téma}
Geometria VI -- miš-maš
\subsection*{Ciele}
Precvičenie geometrických poznatkov, rôznorodné netradičné úlohy

\subsection*{Úlohy a riešenia}
\begin{tcolorbox}[breakable,notitle,boxrule=0pt,colback=light-gray,colframe=light-gray]\ul{22.1} [66-II-3] Dokážte, že obdĺžnik s~rozmermi $32 \times 120$ sa dá zakryť siedmimi zhodnými štvorcami so stranou 30.

\end{tcolorbox}

\rieh Štyrmi štvorcami so stranou 30 zrejme zakryjeme obdĺžnik $30\times 120$. Zvyšnú časť $2 \times 120$ rozdelíme na tri zhodné časti, konkrétne obdĺžniky $2 \times 40$, a ukážeme, ako každý z~nich (rovnako) pokryť jedným z~troch zvyšných štvorcov so stranou 30. Dosiahneme to, keď štvorec položíme na obdĺžnik tak, že obe uhlopriečky štvorca budú ležať na osiach súmernosti dotyčného obdĺžnika. Stačí potom ukázať, že obdĺžnik so stranou 2 vpísaný do štvorca podľa obr. 1 má druhú stranu dlhšiu ako 40. Jej dĺžka je zrejme $30\sqrt{2}-2$ (od uhlopriečky štvorca odčítame na každej strane 1 ako veľkosť výšky
\begin{center}
\includegraphics{images/66K3\imagesuffix} \\

Obr. 1
\end{center}
pravouhlého trojuholníka so stranami $2, \sqrt{2}, \sqrt{2}$, pozri zväčšenú časť obr. 1), takže stačí ukázať, že $30\sqrt{2}-2\geq 40$. To je ekvivalentné s~nerovnosťou $5\sqrt{2}\geq 7$, čiže $50 \geq 49$, čo je splnené. Daný obdĺžnik $32 \times 120$ teda naozaj možno zakryť siedmimi štvorcami so stranou 30.\\
\\
\begin{tcolorbox}[breakable,notitle,boxrule=0pt,colback=light-gray,colframe=light-gray]\ul{22.2} [60-S-2]  Daný je štvorec so stranou dĺžky 6\,cm. Nájdite množinu stredov všetkých priečok štvorca, ktoré ho delia na dva štvoruholníky, z~ktorých jeden má obsah 12\,cm$^2$. (Priečka štvorca je úsečka, ktorej krajné body ležia na stranách štvorca.)

\end{tcolorbox}

\rieh Ak priečka delí štvorec na dva štvoruholníky, musia ich koncové body ležať na protiľahlých stranách štvorca. V~takom prípade sú oba štvoruholníky lichobežníkmi alebo pravouholníkmi (pre potreby tohto riešenia budeme pravouholník považovať za špeciálny lichobežník). Označme daný štvorec $ABCD$, koncové body priečky označme $K$ a $L$. Predpokladajme, že bod $K$ leží na strane $AD$, potom bod $L$ leží na strane $BC$. Jeden zo štvoruholníkov $KABL$ a $KDCL$ má podľa zadania obsah 12\,cm$^2$; nech je to napr. lichobežník $KABL$.

Obsah lichobežníka vypočítame ako súčin jeho výšky s~dĺžkou strednej priečky. Výška je v~našom prípade rovná dĺžke strany štvorca, čiže 6\,cm. Jeho stredná priečka má teda dĺžku 2\,cm. Z~toho vyplýva, že stred úsečky $KL$ musí ležať na osi strany $AB$ vo
\begin{center}
\includegraphics{images/60S21\imagesuffix} \ \ \ \ \ \includegraphics{images/60S22\imagesuffix} \\

Obr. 2 \hspace{130pt} Obr. 3
\end{center}
vzdialenosti 2\,cm od stredu strany $AB$ (obr. 2). Platí to aj naopak: Ak stred úsečky $KL$ leží v~opísanej polohe, bude štvoruholník $KABL$ lichobežník s~obsahom 12\,cm$^2$.

Ak budeme namiesto lichobežníka $KABL$ uvažovať lichobežník $KDCL$, vyjde stred priečky $KL$ na osi úsečky $CD$ vo vzdialenosti 2\,cm od stredu strany $CD$.

Ak priečka $KL$ spája body na stranách $AB$ a $CD$, dostaneme ďalšie dva možné body ležiace na spojnici stredov úsečiek $AD$ a $BC$. Hľadanú množinu teda tvoria štyri body, ktoré ležia na priečkach spájajúcich stredy protiľahlých strán štvorca vo vzdialenosti 1\,cm od jeho stredu (obr. 3).

\begin{tcolorbox}[breakable,notitle,boxrule=0pt,colback=light-gray,colframe=light-gray]\ul{22.4} [65-S-3] V~kružnici so stredom $S$ zostrojíme priemer $AB$ a ľubovoľnú naň kolmú tetivu $CD$. Zdôvodnite, prečo je obvod trojuholníka $ACD$ menší ako dvojnásobok obvodu trojuholníka $SBC$.

\end{tcolorbox}

\rieh Želaný vzťah medzi obvodmi trojuholníkov $ACD$ a $SBC$ vyplynie, keď pre dĺžky ich strán objavíme nerovnosti
$$|AC| < 2|SB|,\ \ \ \  |AD| < 2|SC|\ \ \ \  \text{a} \ \ \ \  |CD| < 2|BC|.$$
Prvé dve z~nich sú dôsledkom toho, že tetivy $AC$ a $AD$ danej kružnice sú kratšie ako jej priemer $AB$ (obr. 1), tretia nerovnosť zapísaná v~tvare $\frac{1}{2}|CD| < |BC|$ je nerovnosťou medzi dĺžkami odvesny a prepony dvoch zhodných pravouhlých trojuholníkov, na ktoré je trojuholník $BCD$ rozdelený priamkou $AB$, ktorá je totiž (vďaka predpokladu $AB \perp CD$) osou tetivy $CD$. Dodajme, že rovnako dobre možno využiť aj trojuholníkovú nerovnosť $|CD| < |BC| + |BD| = 2|BC|$.
\begin{center}
\includegraphics{images/63S3\imagesuffix}\\

Obr. 5
\end{center}
\textbf{Iné riešenie.} Označme $\alpha$ veľkosti vnútorných uhlov pri základni $AC$ rovnoramenného trojuholníka $SAC$. Potom jeho vonkajší uhol pri vrchole $S$, čiže uhol $CSB$, má veľkosť $2\alpha$, ktorú má aj uhol $CAD$, pretože polpriamka $AB$ je jeho osou (obr. 5). 2Rovnoramenné trojuholníky $ACD$ a $SCB$ sa tak zhodujú vo vnútorných uhloch pri svojich hlavných vrcholoch $A$ a $S$, a sú teda podobné. Preto je pomer ich obvodov rovný pomeru dĺžok ich ramien, a ten má naozaj hodnotu menšiu ako 2, lebo ramená trojuholníka $ACD$ sú kratšie ako priemer danej kružnice, zatiaľ čo ramená trojuholníka $SCB$ majú dĺžku jej polomeru.\\
\\
\begin{tcolorbox}[breakable,notitle,boxrule=0pt,colback=light-gray,colframe=light-gray]\ul{22.5} [59-S-2] Kružnice $k(S; 6\,\text{cm})$ a $l(O; 4\,\text{cm})$ majú vnútorný dotyk v~bode $B$. Určte dĺžky strán trojuholníka $ABC$, pričom bod $A$ je priesečník priamky $OB$ s~kružnicou $k$ a bod $C$ je priesečník kružnice $k$ s~dotyčnicou z~bodu $A$ ku kružnici $l$.

\end{tcolorbox}

\rieh Bod dotyku kružnice $l$ s~dotyčnicou z~bodu $A$ označme $D$ (obr. 6). Z~vlastností dotyčnice ku kružnici vyplýva, že uhol $ADO$ je pravý. Zároveň je pravý aj uhol
\begin{center}
\includegraphics{images/59S2\imagesuffix}\\

Obr. 6
\end{center}
$ACB$ (Tálesova veta). Trojuholníky $ABC$ a $AOD$ sú tak podobné podľa vety $uu$, lebo sa zhodujú v~uhloch $ACB$, $ADO$ a v~spoločnom uhle pri vrchole $A$. Z~uvedenej podobnosti vyplýva
$$\frac{|BC|}{|OD|}=\frac{|AB|}{|AO|}. \ \ \ \  (1)$$
Zo zadaných číselných hodnôt vychádza $|OD| = |OB| = 4$\,cm, $|OS| = |SB| - |OB| = 2$\,cm, $|OA| = |OS| + |SA| = 8$\,cm a $|AB| = 12$\,cm. Podľa (1) je teda $|BC| : 4\,\text{cm} = 12 : 8$ a odtiaľ $|BC| = 6$\,cm. Z~Pytagorovej vety pre trojuholník $ABC$ nakoniec zistíme, že $|AC| = \sqrt{12^2 - 6^2}\,\text{cm}= 6$\,cm.\\
\\
\begin{tcolorbox}[breakable,notitle,boxrule=0pt,colback=light-gray,colframe=light-gray] \ul{22.6} [63-II-4]  Daný je konvexný štvoruholník $ABCD$ s bodom$E$ vnútri strany $AB$ tak, že platí $|\ma ADE| = |\ma DEC| = |\ma ECB|$. Obsahy trojuholníkov $AED$ a $CEB$ sú postupne 18\,cm$^2$ a 8\,cm$^2$ . Určte obsah trojuholníka $ECD$.
\end{tcolorbox}

\rieh Hľadaný obsah trojuholníka $ECD$ označme $S$. Uhol $DEC$ je striedavý s uhlami $ADE$ a $ECB$, odtiaľ $AD  \parallel EC$ a $ED \parallel BC$ (obr. 7). Trojuholníky $EDA$
\begin{center}
\includegraphics{images/63K41\imagesuffix}\\

Obr. 7
\end{center}
a $EDC$ majú spoločnú stranu $ED$, pomer ich obsahov je teda rovný pomeru prislúchajúcich výšok. Ak navyše postupne označíme $P, Q a R$ kolmé priemety vrcholov $A, B a C$ na priamku $DE$ a označíme $v = |AP|$, $w = |BQ| = |CR|$, dostaneme z podobných
pravouhlých trojuholníkov $AEP$ a $BEQ$ úmeru
$$\frac{18}{S}=\frac{v}{w}=\frac{|AE|}{|EB|}.$$
Analogicky pre trojuholníky $ECD$ a $ECB$ zistíme, že
$$\frac{8}{S}=\frac{|EB|}{|AE|}.$$
(V obr. 7 sú prislúchajúce priemety iba naznačené, ale jedná sa o ten istý výpočet
ako v predošlom odseku, len v ňom zameníme zodpovedajúce body $A\leftrightarrow B, C \leftrightarrow D$ a prislúchajúce obsahy trojuholníkov $AED$ a $BEC$.) Dokopy teda je $S : 8 = 18 : S$ čiže $S^2 = 144$, takže trojuholník $ECD$ má obsah $S = 12$\,cm$^2$ .






\seminar{23}

\weblinks{Na stiahnutie: \href{pdf/seminar23-teacher.pdf}{učiteľská verzia}, \href{pdf/seminar23-student.pdf}{študentská verzia}}

Seminár sa nebude konať, namiesto neho sa študenti zúčastnia medzinárodnej matematickej súťaže 5-členných družstiev \textit{Náboj}.


\seminar{24}

Seminár sa nebude konať, namiesto neho sa študenti zúčastnia medzinárodnej matematickej súťaže 5-členných družstiev \textit{Náboj}.



\section{Apríl}
\section*{Seminár 25}

\weblinks{Na stiahnutie: \href{pdf/seminar25-teacher.pdf}{učiteľská verzia}, \href{pdf/seminar25-student.pdf}{študentská verzia}}

\subsection*{Téma}
Kombinatorika II -- hry s hľadaním víťaznej stratégie a logické úlohy.
\subsection*{Ciele}
Pokračovať v precvičovaní úloh zameraných na hľadanie víťaznej stratégie a úloh, ktoré nevyžadujú špeciálne matematické znalosti.

\subsection*{Úlohy a riešenia}
\begin{tcolorbox}[breakable,notitle,boxrule=0pt,colback=light-gray,colframe=light-gray]\ul{25.1} [61-I-6-N2] Na tabuli sú napísané všetky prvočísla menšie ako 100. Gitka a Terka sa striedajú v~ťahoch pri nasledujúcej hre. Najprv Gitka zmaže jedno z~prvočísel. Ďalej vždy hráčka, ktorá je na ťahu, zmaže jedno z~prvočísel, ktoré má s~predchádzajúcim zmazaným prvočíslom jednu zhodnú číslicu (tak po prvočísle 3 je možné zmazať trebárs 13 alebo 37). Hráčka, ktorá je na ťahu a nemôže už žiadne prvočíslo zmazať, prehráva. Ktorá z~oboch hráčok môže hrať tak, že vyhrá nezávisle od ťahov súperky?

\end{tcolorbox}

\rieh Pretože prvočísel menších ako 100 je nepárny počet (25), ponúka sa hypotéza, že víťaznú stratégiu bude mať prvá hráčka. Ukážme, že to tak naozaj je. Táto hráčka si vopred v~duchu spáruje (podľa spoločnej číslice) napísané prvočísla (dá sa to urobiť viacerými spôsobmi, uvedieme ten, pri ktorom v~každom kroku párujeme najmenšie doposiaľ nespárované prvočíslo s~najmenším ďalším doposiaľ nespárovaným prvočíslom so spoločnou číslicou): (2, 23), (3, 13), (5, 53), (7, 17), (11, 19), (29, 59), (31, 37), (41, 43), (47, 67), (61, 71), (73, 79), (83, 89); jediné zostávajúce nespárované prvočíslo 97 preto Gitka zmaže ako prvé a ďalej pri hre bude mazať vždy prvočíslo, ktoré je v~páre s~predchádzajúcim zmazaným prvočíslom. Týmto postupom musí vyhrať.\\
\\
\kom Úloha je malým opakovaním toho, ktoré čísla do 100 sú prvočísla a môžeme ju využiť pripomenutie definície prvočísla. Ako aj v nasledujúcich úlohách, môžeme študentov nechať najprv odohrať niekoľko hier a potom skúsiť ich čiastkové zistenia spoločne pretaviť do univerzálnej stratégie.\\
\\
\begin{tcolorbox}[breakable,notitle,boxrule=0pt,colback=light-gray,colframe=light-gray]\ul{25.2} [61-II-4] Na tabuli je napísaných prvých $n$ celých kladných čísel. Marína a Tamara sa striedajú v~ťahoch pri nasledujúcej hre. Najskôr Marína zotrie jedno z~čísel na tabuli. Ďalej vždy hráčka, ktorá je na ťahu, zotrie jedno z~čísel, ktoré sa od predchádzajúceho zotretého čísla ani nelíši o~1, ani s~ním nie je súdeliteľné. Hráčka, ktorá je na ťahu a nemôže už žiadne číslo zotrieť, prehrá. Pre $n = 6$ a pre $n = 12$ rozhodnite, ktorá z~hráčok môže hrať tak, že vyhrá nezávisle na ťahoch druhej hráčky.

\end{tcolorbox}

\rieh Úloha dvoch po sebe zotieraných čísel je v~zadanej hre symetrická: ak je po čísle $x$ možné zotrieť číslo $y$, je (pri inom priebehu hry) po čísle $y$ možné zotrieť číslo $x$. Preto si môžeme celú hru (so zadaným číslom $n$) \uv{sprehľadniť} tak, že najskôr vypíšeme všetky takéto (nazývajme ich prípustné) dvojice $(x, y)$. Keďže na poradí čísel v~prípustnej dvojici nezáleží, stačí vypisovať len tie dvojice $(x, y)$, v~ktorých $x < y$.

V~prípade $n = 6$ všetky prípustné dvojice sú
$$(1, 3), (1, 4), (1, 5), (1, 6), (2, 5), (3, 5).$$
Z~tohto zoznamu ľahko odhalíme, že víťaznú stratégiu má (prvá) hráčka Marína. Ak totiž zotrie na začiatku hry číslo 4, musí Tamara zotrieť číslo 1, a keď potom Marína zotrie číslo 6, nemôže už Tamara žiadne ďalšie číslo zotrieť. Okrem tohto priebehu $4 \rightarrow 1\rightarrow 6$ si môže Marína zaistiť víťazstvo aj inými, pre Tamaru ”vynútenými“ priebehmi, napríklad $6\rightarrow 1 \rightarrow 4$ alebo $4 \rightarrow 1 \rightarrow 3 \rightarrow 5 \rightarrow 2$.

V~prípade $n = 12$ je všetkých prípustných dvojíc výrazne väčšie množstvo. Preto si položíme otázku, či všetky čísla od 1 do 12 možno rozdeliť na šesť prípustných dvojíc. Ak totiž nájdeme takú šesticu, môžeme opísať víťaznú stratégiu druhej hráčky (Tamary): ak zotrie Marína pri ktoromkoľvek svojom ťahu číslo $x$, Tamara potom vždy zotrie to číslo $y$, ktoré s~číslom $x$ tvorí jednu zo šiestich nájdených dvojíc. Tak nakoniec Tamara zotrie aj posledné (dvanáste) číslo a vyhrá (prípadne hra skončí skôr tak, že Marína nebude môcť zotrieť žiadne číslo).

Hľadané rozdelenie všetkých 12 čísel do šiestich dvojíc naozaj existuje, napríklad
$$(1, 4), (2, 9), (3, 8), (5, 12), (6, 11), (7, 10).$$
Iné vyhovujúce rozdelenie dostaneme, keď v~predošlom dvojice (1, 4) a (6, 11) zameníme dvojicami (1, 6) a (4, 11). Ďalšie, menej podobné vyhovujúce rozdelenie je napríklad
$$(1, 6), (2, 5), (3, 10), (4, 9), (7, 12), (8, 11).$$
\textit{Záver.} Pre $n = 6$ má víťaznú stratégiu Marína, pre $n = 12$ Tamara.\\
\\
\kom Úloha je náročnejšia ako predchádzajúca, no študenti by mali prvú časť zvládnuť samostatne, v časti druhej môžu svoje sily spojiť so s ďalšími spolužiakmi, príp. stratégie, ktoré vymysleli, otestovať pri vzájomnej hre.\\
\\
\begin{tcolorbox}[breakable,notitle,boxrule=0pt,colback=light-gray,colframe=light-gray]\ul{25.3} [61-I-6-N3] Dve hráčky majú k~dispozícii pre hru, ktorú opíšeme, neobmedzený počet dvadsaťcentových mincí a stôl s~kruhovou doskou s~priemerom 1\,m. Hra prebieha tak, že sa hráčky pravidelne striedajú v~ťahoch. Najprv prvá hráčka položí jednu mincu kamkoľvek na prázdny stôl. Ďalej vždy hráčka, ktorá je na ťahu, položí na voľnú časť stola ďalšiu mincu (tak, aby nepresahovala okraj stola a aby sa skôr položených mincí nanajvýš dotýkala). Ktorá z~oboch hráčok môže hrať tak, že vyhrá nezávisle od ťahov súperky?

\end{tcolorbox}

\rieh Víťaznú stratégiu má prvá hráčka: prvú mincu položí doprostred stola a v~každom ďalšom kroku položí mincu na miesto súmerne združené podľa stredu stola s~miestom práve položenej mince.\\
\\
\kom Úloha je zaujímavým príkladom, kde zohráva špeciálnu úlohu jeden konkrétny bod hracej plochy. Nájdenie víťaznej stratégie je po uvedomení si tejto vlastnosti už úlohou jednoduchou. Na príklade tejto hry môžeme študentov upozorniť na ďalší všeobecný princíp, ktorý pri riešení matematických problémov môže prísť vhod -- hľadanie symetrií, príp. špeciálnych bodov týchto symetrií a skúmanie ich vlastností, pričom symetrie nemusíme chápať nutne iba v geometrickom kontexte. \\
\\
\begin{tcolorbox}[breakable,notitle,boxrule=0pt,colback=light-gray,colframe=light-gray]\ul{25.4} [59-D-1] Erika a Klárka hrali hru ”slovný logik“ s týmito pravidlami: Hráč $A$ si myslí slovo zložené z piatich rôznych písmen. Hráč $B$ vysloví ľubovoľné slovo zložené z piatich rôznych písmen a hráč $A$ mu prezradí, koľko písmen uhádol na správnej pozícii a koľko na nesprávnej. Písmená považujeme za rôzne, aj keď sa líšia iba mäkčeňom alebo dĺžňom (napríklad písmena $A$, \textit{Á} sú rôzne). Keby si hráč $A$ myslel napríklad slovo \textit{LOĎKA} a $B$ by vyslovil slovo \textit{KOLÁČ}, odpovie hráč $A$, že jedno písmeno uhádol hráč $B$ na správnej pozícii a dve na nesprávnej. Skrátene oznámi \uv{1 + 2}, lebo sa naozaj obe slová zhodujú iba v písmene $O$ vrátane pozície (druhej zľava) a v písmenách $K$ a $L$, ktorých pozície sú odlišné. Erika si myslela slovo z piatich rôznych písmen a Klárka vyslovila slová \textit{KABÁT, STRUK, SKOBA, CESTA} a \textit{ZÁPAL}. Erika na tieto slová v danom poradí odpovedala 0 + 3, 0 + 2, 1 + 2, 2 + 0 a 1 + 2. Zistite, aké slovo si Erika mohla myslieť.

\end{tcolorbox}

\rieh Slová \textit{ZÁPAL} a $STRUK$ nemajú spoločné písmená. Preto sa, ako vyplýva z odpovedí 1 + 2 a 0 + 2, medzi ich písmenami, ktoré dokopy tvoria množinu $M= \{Z$, \textit{Á}, $P, A, L, S, T, R, U, K\}$, nachádza všetkých päť písmen hľadaného slova. V slove $SKOBA$ majú byť práve tri z hľadaných písmen. Sú to teda písmená $S, K, A$. (Zvyšné písmená $B$ a $O$ totiž do množiny $M$ nepatria.) V slove $CESTA$ majú byť len dve z hľadaných písmen, a obe na správnej pozícii. Sú to už nájdené $S$ a $A$, ktoré teda patria na tretie, resp. piate miesto hľadaného slova (a písmeno $T$ môžeme z množiny $M$ ”vylúčiť“). Písmeno $K$ nemôže byť ani na prvom, ani na druhom mieste: vyplýva to z odpovedí pre slová \textit{KABÁT} (0 + 3) a $SKOBA$ (1 + 2). Takže je na štvrtom mieste a ostáva určiť prvé dve písmená. V slove $STRUK$ sú len dve z hľadaných písmen (musia to teda byť $S$ a $K$), obe na nesprávnych pozíciách. Preto z množiny $M$ \uv{ vylúčime} aj písmená $R, U$ (a $T$, ak sme to doteraz neurobili). Zvyšné dve hľadané písmená potom patria do množiny $\{Z$, \textit{Á}, $P, L\}$. Z podmienok pre slovo \textit{KABÁT} vyplýva, že jedno z nich je \textit{Á}. V slove \textit{ZÁPAL} je práve jedno písmeno na správnej pozícii. Keby to bolo $Z$, nemali by sme kam uložiť písmeno \textit{Á}. Takže \textit{Á} je na druhom mieste a navyše môžeme vylúčiť písmeno $Z$. Na prvom mieste hľadaného slova môže byť $L$ alebo $P$. Ľahko sa presvedčíme, že nájdené slová \textit{LÁSKA} aj \textit{PÁSKA} vyhovujú všetkým podmienkam úlohy.\\
\\
\kom Úloha opäť nevyžaduje žiadne matematické znalosti, je však výbornou previerkou toho, ako sú študenti schopní narábať s veľkým množstvom informácií, nestratiť v nich prehľad a využiť ich na zdarné vyriešenie zadaného problému. \\
\\

\begin{tcolorbox}[breakable,notitle,boxrule=0pt,colback=light-gray,colframe=light-gray]\ul{25.5} [63-D-6]
Šachového turnaja sa zúčastnilo 8 hráčov a každý s každým odohral jednu partiu. Za víťazstvo získal hráč 1 bod, za remízu pol bodu, za prehru žiadny bod. Na konci turnaja mali všetci účastníci rôzne počty bodov. Hráč, ktorý skončil na 2. mieste, získal rovnaký počet bodov ako poslední štyria dokopy. Určte výsledok partie medzi 4. a 6. hráčom v celkovom poradí.

\end{tcolorbox}

\rieh Poslední štyria hráči odohrali medzi sebou 6 partií, takže počet bodov, ktoré dokopy získali, je aspoň 6. Hráč, ktorý skončil na 2. mieste, teda získal aspoň 6 bodov. Keby získal viac ako 6, teda aspoň 6,5 bodov, musel by najlepší hráč (vďaka podmienke rôznych počtov) získať všetkých 7 možných bodov; porazil by tak i hráča na 2. mieste, ktorý by v dôsledku toho získal menej ako 6,5 bodov, a to je spor. Hráč v poradí druhý preto získal práve 6 bodov. Presne toľko ale získali dokopy i poslední štyria, a tak mohli tieto body získať len zo vzájomných partií, čo znamená, že prehrali všetky partie s hráčmi z prvej polovice výsledného poradia. Hráč, ktorý skončil na 6. mieste, preto prehral partiu s hráčom, ktorý skončil na 4. mieste.\\
\\
\kom Posledná logická úloha seminára poskytuje niekoľko rôznych ciest k riešeniu, nielen jedinú popísanú v zadaní a tak môže byť pre študentov inšpiratívne svoje riešenia medzi sebou pozdieľať. Na úlohu tiež nadväzuje domáca práca.
\subsection*{Domáca práca}

\begin{tcolorbox}[breakable,notitle,boxrule=0pt,colback=light-gray,colframe=light-gray]\ul [63-K-2]{25.6}
Šachového turnaja sa zúčastnilo 5 hráčov a každý s každým odohral jednu partiu. Za prvenstvo získal hráč 1 bod, za remízu pol bodu, za prehru žiadny bod. Poradie hráčov na turnaji sa určuje podľa počtu získaných bodov. Jediným ďalším kritériom rozhodujúcim o konečnom umiestnení hráčov v prípade rovnosti bodov je počet výhier (kto má viac výhier, je na tom v umiestnení lepšie). Na turnaji získali všetci hráči rovnaký počet bodov. Vojto porazil Petra a o prvé miesto sa delil s Tomášom. Ako dopadla partia
medzi Petrom a Martinom?

\end{tcolorbox}

\rieh Každý hráč odohral po jednej partii so zvyšnými štyrmi. Bolo teda odo hraných celkom $\frac{1}{2}\cdot5 \cdot 4 = 10$ partií, takže každý hráč získal práve 2 body. Sú len tri možnosti, ako získať odohraním štyroch partií 2 body, a podľa toho obsahovala celková tabuľka nanajvýš tri rovnocenné skupiny hráčov. Tieto skupiny, $A, B$ a $C$, uvádzame v poradí, v ktorom by sa v konečnej tabuľke umiestnili:

Skupina $A$ obsahuje všetkých hráčov, ktorí majú po dvoch výhrach a dvoch prehrách. Skupina $B$ pozostáva z hráčov s jednou výhrou, jednou prehrou a dvoma remízami. Skupina $C$ obsahuje hráčov so štyrmi remízami.

Vojto a Tomáš sú jediní víťazi, preto nepatria do skupiny $C$. Nepatria ani do skupiny $B$, pretože v opačnom prípade by s nimi museli všetci traja hráči zo skupiny $C$ s horším výsledkom remizovať (a každý hráč skupiny B má len dve remízy).

Z toho vyplýva, že Vojto a Tomáš majú po dvoch výhrach a dvoch prehrách a skupina $C$ je prázdna. Zvyšní traja hráči tak majú po jednej výhre, jednej prehre a dvoch remízach, ktoré museli uhrať navzájom medzi sebou.\\
\textit{Záver}. Peter a Martin spolu remizovali.\\

\textbf{Iné riešenie.} Využijeme (nadbytočný) údaj, že Vojto porazil Petra: Keby mali Vojto a Tomáš po jednej výhre, jednej prehre a dvoch remízach, musel by aj Peter patriť medzi víťazov turnaja. Jediný v poradí nižší celkový výsledok sú totiž štyri remízy, Peter však jednu partiu prehral, a tak musel aj jednu vyhrať. Vojto a Tomáš majú 1preto po dvoch výhrach a dvoch prehrách. Ak Peter prehral s Vojtom, musel poraziť Tomáša. (Nemohol mať dve prehry, keďže bol v poradí nižšie ako Tomáš a Vojto. Ani nemohol s Tomášom, ktorý žiadnu remízu nemá, remizovať.) Potrebný druhý bod získal dvoma remízami -- s Martinom a nepomenovaným piatym hráčom.\\
\textit{Záver}. Peter a Martin spolu remizovali.\\
\\
\begin{tcolorbox}[breakable,notitle,boxrule=0pt,colback=light-gray,colframe=light-gray]\ul {25.7}[64-D-3] Simona a Lenka hrajú hru. Pre dané celé číslo $k$ také, že $0 \leq k \leq 64$, vyberie Simona $k$ políčok šachovnice $8 \times 8$ a každé z nich označí krížikom. Lenka potom šachovnicu nejakým spôsobom vyplní tridsiatimi dvoma dominovými kockami. Ak je počet kociek pokrývajúcich dva krížiky nepárny, vyhráva Lenka, inak vyhráva Simona. V závislosti
od $k$ určte, ktoré z dievčat má vyhrávajúcu stratégiu.

\end{tcolorbox}

\rieh Riešenie rozdeľme podľa hodnoty čísla $k$.

Ak $k = 0$, je počet kociek pokrývajúcich dva krížiky rovný nule, preto vyhrá Simona.

Ak $0 < k \leq 32$, umiestni Simona krížiky napr. iba na biele políčka šachovnice. Potom pod žiadnou kockou nie sú dva krížiky, preto vyhrá Simona.

Ak $k > 32$, pričom $k$ je párne, umiestni Simona 32 krížikov na biele políčka a zvyšné krížiky kamkoľvek. Potom pod párnym počtom kociek sú dva krížiky (takých kociek je totiž práve $k - 32$, pretože každá dominová kocka pokrýva jedno biele a jedno čierne
políčko šachovnice), takže vyhrá Simona.

Ak $32 < k \leq 61$, pričom k je nepárne, nenapíše Simona krížiky do troch políčok v jednom z \uv{bielych rohov}, t. j. do rohového bieleho a do dvoch susedných čiernych políčok, ale napíše ich do všetkých ostatných 31 bielych políčok a zvyšok do akýchkoľvek čiernych políčok (okrem spomenutých dvoch). Na bielych políčkach je teda nepárny počet krížikov a na čiernych párny počet krížikov. Okolo každého čierneho políčka s krížikom sú všetky biele políčka tiež s krížikom, preto každá kocka, ktorá zakrýva čierne políčko s krížikom, zakrýva dva krížiky. Iné kocky dva krížiky nezakrývajú. Preto opäť vyhrá Simona.

Ak $k = 63$, dva krížiky nie sú iba pod jedinou kockou, preto v takom prípade vyhrá Lenka, a to bez potreby akejkoľvek stratégie.\\
\textit{Záver.} Pre každé $0 \leq k \leq 64$, $k \neq 63$, má vyhrávajúcu stratégiu Simona, pri $k = 63$ vyhráva automaticky Lenka.\\
\\

\subsection*{Doplňujúce zdroje a materiály}
Výborným zdrojom všemožných matematických hier, spolu s ich kategorizáciou a možnosťou využitia v triede je  [~\cite{burjan1999}].


\section*{Seminár 26}

\weblinks{Na stiahnutie: \href{pdf/seminar26-teacher.pdf}{učiteľská verzia}, \href{pdf/seminar26-student.pdf}{študentská verzia}}

\subsection*{Téma}
Krajské kolo MO

\subsection{Ciele}
Analyzovať so študentmi úlohy krajského kola MO, ukázať na prípadné spojitosti s kolom domácim a školským, objasniť najčastejšie chyby.


\seminar{27}{Krajské kolo MO}
\teachernote{
\subsection*{Ciele}
Analyzovať so študentmi úlohy krajského kola MO, ukázať na prípadné spojitosti s kolom domácim a školským, objasniť najčastejšie chyby.
}
\seminar{28}

\subsection*{Téma}
Algebra -- sústavy rovníc, rovnice s parametrom

\subsection*{Ciele}

\subsection*{Úlohy a riešenia} 

\todo{DOPLNIŤ komentáre.}

% Do not delete this line (pandoc magic!)

\problem{B-66-II-1}{seminar29,rovnice}{
Nájdite všetky dvojice prirodzených čísel $a$, $b$, pre ktoré platí
$$a +\frac{66}{a}= b +\frac{66}{b}.$$
}{
\rieh Anulovaním pravej strany upravíme danú rovnicu na tvar
$$a - b + 66\bigg(\frac{1}{a}-\frac{1}{b}\bigg)= (a - b)\bigg(1-\frac{66}{ab}\bigg)=\frac{1}{ab}(a - b)(ab - 66) = 0.$$
Z toho vyplýva, že hľadané dvojice $(a, b)$ prirodzených čísel sú práve tie, pre ktoré platí $a = b$ alebo $ab = 66$.

Úlohe teda vyhovuje nekonečne veľa dvojíc prirodzených čísel tvaru $(a, b) = (k, k)$, pričom $k$ je ľubovoľné prirodzené číslo, a keďže číslo $66 = 2\cdot3\cdot11$ má osem deliteľov, tak aj osem dvojíc $(a, b) \in \{$$(1, 66)$, $(2, 33)$, $(3, 22)$, $(6, 11)$, $(11, 6)$, $(22, 3)$, $(33,2)$, $(66,1)$$\}$.\\
\\
\kom Úloha je relatívne jednoduchá a vhodná ako rozcvička na začiatok seminára. Pripomenie študentom metódu riešenia rovníc rozkladom na súčin výrazov, ktorý je rovný nule. Zároveň v záverečnej diskusii zľahka využijú vedomosti o deliteľnosti prirodzených čísel.\\
\\
}


% Do not delete this line (pandoc magic!)

\problem{B-58-II-1}{
V obore reálnych čísel riešte sústavu rovníc
\begin{align*}
    x + y & = 1,\\
    x - y & = a,\\
    -4ax + 4y & = z^2 + 4
\end{align*}
s neznámymi $x$, $y$, $z$ a reálnym parametrom $a$. 
}{
\rieh Sčítaním prvej a druhej rovnice danej sústavy dostaneme $2x = 1 + a$, odčítaním druhej rovnice od prvej $2y = 1 - a$. Odtiaľ 
\begin{equation} \label{eq:B58II1_1}
    x =\frac{1}{2}(1 + a), \ \ \ \  y =\frac{1}{2}(1 - a).
\end{equation}
Keď dosadíme za $x$ a $y$ do tretej rovnice pôvodnej sústavy, dostaneme rovnicu
$$-2a(1 + a) + 2(1 - a) = z^2 + 4,\ \ \ \ \text{čiže} \ \ \ \ z^2 + 2a^2 + 4a + 2 = 0,$$
ktorú upravíme na tvar
$$z^2 + 2(a + 1)^2 = 0.$$
Oba sčítance na ľavej strane poslednej rovnice sú nezáporné čísla. Ich súčet je 0 práve vtedy, keď $z = 0$, $a = -1$. Dosadením týchto hodnôt do \ref{eq:B58II1_1} dostaneme $x = 0, y = 1$.

\textit{Záver.} Daná sústava rovníc má riešenie iba pre $a = -1$, a to $x = 0$, $y = 1$, $z = 0$. Skúška pri tomto postupe nie je nutná.
\\
\\
\kom Úloha vyžaduje umné narábanie so sústavou troch rovníc tak, aby bolo možné uskutočniť záverečnú diskusiu o existencii riešenia pre rôzne hodnoty parametra $a$. Je tiež vhodné so študentami prediskutovať, prečo v tomto prípade nie je nutné robiť skúšku správnosti.\\
\\
}


% Do not delete this line (pandoc magic!)

\problem{B-60-S-1}{
V obore reálnych čísel vyriešte rovnicu
$$\sqrt{x + 3} +\sqrt{x} = p$$
s neznámou $x$ a reálnym parametrom $p$.
}{
\rieh Aby bola ľavá strana rovnice definovaná, musia byť oba výrazy pod odmocninami nezáporné, čo je splnené práve pre všetky $x \geq 0$. Pre nezáporné $x$ potom $p =\sqrt{x + 3}+\sqrt{x} \geq \sqrt{3}$, rovnica môže teda mať riešenie iba pre $p \geq \sqrt{3}$.

Upravujme danú rovnicu:
\begin{align*}
\sqrt{3} +\sqrt{x + 3} & = p,\\
2x + 3 + 2\sqrt{x(x + 3)} & = p^2,\\
2\sqrt{x(x + 3)} & = p^2- 2x - 3,\\
4x(x + 3) & = (p^2 -2x - 3)^2,\\
4x^2+ 12x & = p^4+ 4x^2+ 9 - 4p^2x - 6p^2+ 12x,\\
x & = \frac{(p^2 - 3)^2}{4p^2}.\ \ \ \ (1)  
\end{align*}
Keďže sme danú rovnicu umocňovali na druhú, je nutné sa presvedčiť skúškou, že vypočítané $x$ je pre hodnotu parametra $p \geq \sqrt{3}$ riešením pôvodnej rovnice:
\begin{align*}
    \sqrt{\frac{(p^2 - 3)^2}{4p^2}+ 3} +\sqrt{\frac{(p^2 - 3)^2}{4p^2}} & = \sqrt{\frac{p^4 - 6p^2 + 9 + 12p^2}{4p^2}}+\sqrt{\frac{(p^2 - 3)^2}{4p^2}} =\\
 & = \sqrt{\frac{(p^2 + 3)^2}{4p^2}}+\sqrt{\frac{(p^2 -3)^2}{4p^2}}=\frac{p^2 + 3}{2p}+\frac{p^2 - 3}{2p}= p.
\end{align*}
Pri predposlednej úprave sme využili podmienku $p \geq \sqrt{3}$ (a teda aj $p^2 -3 \geq 0$ a $p > 0$), takže $\sqrt{(p^2 - 3)^2} = p^2 - 3$ a $\sqrt{4p^2} = 2p$.

\textit{Poznámka.} Namiesto skúšky stačí overiť, že pre nájdené $x$ sú všetky umocňované výrazy nezáporné, teda vlastne stačí overiť, že
$$p^2 - 2x - 3 =\frac{(p^2 - 3)(p^2 + 3)}{2p^2}\geq 0.$$
Pre $p \geq \sqrt{3}$ to tak naozaj je.

Vynechať skúšku možno aj takouto úvahou: Funkcia $\sqrt{x + 3}+\sqrt{x}$ je zrejme rastúca, v bode 0 (ktorý je krajným bodom jej definičného oboru) nadobúda hodnotu $\sqrt{3}$ a zhora je neohraničená. Preto každú hodnotu $p \geq \sqrt{3}$ nadobúda pre práve jedno $x \geq 0$. Z toho vyplýva, že pre $p \geq \sqrt{3}$ má zadaná rovnica práve jedno riešenie, a teda (jediné) nájdené riešenie \todo{fixni (1)} musí vyhovovať.
}


% Do not delete this line (pandoc magic!)

\problem{B-58-I-2}{seminar29,rovnice,sustavy,domacekolo}{
Určte všetky trojice $(x, y, z)$ reálnych čísel, pre ktoré platí
\begin{align*}
    x^2 + xy & = y^2 + z^2,\\
    z^2 + zy & = y^2 + x^2 .
\end{align*}
}{
\rieh Odčítaním prvej rovnice od druhej dostaneme po úprave
$$(z - x)(2z + 2x + y) = 0.$$
Sú preto možné dva prípady, ktoré rozoberieme samostatne.
\begin{enumerate}[a)]
\item Prípad $z - x = 0$. Dosadením $z = x$ do prvej rovnice sústavy dostaneme $x^2+ xy = y^2 + x^2$, čiže $y(x - y) = 0$. To znamená, že platí $y = 0$ alebo $x = y$. V prvom prípade dostávame trojice $(x, y, z) = (x, 0, x)$, v druhom $(x, y, z) = (x, x, x)$; také trojice sú riešeniami danej sústavy pre ľubovoľné reálne číslo $x$, ako ľahko overíme dosadením
(aj keď taká skúška pri našom postupe vlastne nie je nutná). \label{part:a}
\item Prípad $2z + 2x + y = 0$. Dosadením $y = -2x - 2z$ do prvej rovnice sústavy dostaneme
$$x^2 + x(-2x - 2z) = (-2x - 2z)^2 + z^2,\ \ \ \ \text{čiže} \ \ \ \  5(x + z)^2 = 0.$$
Posledná rovnica je splnená práve vtedy, keď $z = -x$, vtedy však $y = -2x - 2z = 0$.Dostávame trojice $(x, y, z) = (x, 0, -x)$, ktoré sú riešeniami danej sústavy pre každé reálne $x$, ako overíme dosadením. (O takej skúške platí to isté čo v prípade \ref{part:a}.
\end{enumerate}
\textit{Odpoveď.} Všetky riešenia $(x, y, z)$ danej sústavy sú trojice troch typov:
$$(x, x, x), \ \  (x, 0, x), \ \  (x, 0, -x),$$
kde $x$ je ľubovoľné reálne číslo.

\textbf{Iné riešenie*.} Obe rovnice sústavy sčítame. Po úprave dostaneme rovnicu
$$y(x + z - 2y) = 0$$
a opäť rozlíšime dve možnosti.
\begin{enumerate}[a)]
\item Prípad $y = 0$. Z prvej rovnice sústavy ihneď vidíme, že $x^2 = z^2$, čiže $z =\pm x$. Skúškou overíme, že každá z trojíc $(x, 0, x)$ a $(x, 0, -x)$ je pre ľubovoľné reálne $x$
riešením. \label{part:a2}
\item Prípad $x + z - 2y = 0$. Dosadením $y = \frac{1}{2}(x + z)$ do prvej rovnice sústavy dostaneme
$$x^2 + x(x + z)^2=\frac{(x + z)^2}{4}+ z^2, \ \ \ \ \text{po úprave} \ \ \ \  x^2 = z^2.$$
Platí teda $z = -x$ alebo $z = x$. Dosadením do rovnosti $x + z - 2y = 0$ v prvom prípade dostaneme $y = 0$, v druhom prípade $y = x$. Zodpovedajúce trojice $(x, 0, -x)$ a $(x, x, x)$ sú riešeniami pre každé reálne $x$ (prvé z nich sme však našli už v časti \ref{part:a2}.
\end{enumerate}
}


% Do not delete this line (pandoc magic!)

\problem{B-60-I-1}{
V obore reálnych čísel vyriešte sústavu
\begin{align*}
    \sqrt{x^2 + y^2} & = z + 1,\\
    \sqrt{y^2 + z^2} & = x + 1,\\
    \sqrt{z^2 + x^2} & = y + 1.
\end{align*}
}{
\rieh Umocnením a odčítaním prvých dvoch rovností dostaneme $x^2 - z^2= (z + 1)^2 - (x + 1)^2$, čo upravíme na $2(x^2 - z^2 ) + 2(x - z) = 0$, čiže
$$(x - z)(x + z + 1) = 0. \ \ \ \ (1)$$
Analogicky by sme dostali ďalšie dve rovnice, ktoré vzniknú z \todo{(1)} cyklickou zámenou neznámych $x \rightarrow y \rightarrow z$. Vzhľadom na túto symetriu (daná sústava sa nezmení dokonca pri ľubovoľnej permutácii neznámych) stačí rozobrať len nasledovné dve možnosti:

Ak $x = y = z$, prejde pôvodná sústava na jedinú rovnicu $\sqrt{2x^2} = x + 1$, ktorá má dve riešenia $x_{1,2} = 1 \pm \sqrt{2}$. Každá z trojíc $(1 \pm \sqrt{2}, 1 \pm \sqrt{2}, 1 \pm{2})$ je zrejme riešením pôvodnej sústavy.

Ak sú niektoré dve z čísel $x, y, z$ rôzne, napríklad $x \neq z$, vyplýva z \todo{(1)} rovnosť $x+z = - 1$. Dosadením $x+1 = - z$ do druhej rovnice sústavy dostávame $y = 0$ a potom z tretej rovnice máme $x^2 + (x + 1)^2 = 1$, čiže $x(x + 1) = 0$. Posledná rovnica má dve riešenia $x = 0$ a $x = - 1$, ktorým zodpovedajú $z = - 1$ a $z = 0$. Ľahko overíme, že obe nájdené trojice $(0, 0, - 1)$ a $(- 1, 0, 0)$ sú riešeniami danej sústavy, rovnako aj trojica $(0, - 1, 0)$, ktorú dostaneme ich permutáciou.

Daná sústava má päť riešení:
$(0, 0, - 1), (0, - 1, 0), ( - 1, 0, 0), (1 +\sqrt{2}, 1 +\sqrt{2}, 1 +\sqrt{2})$ a $(1 -\sqrt{2}, 1 -\sqrt{2}, 1 -\sqrt{2})$.
}


%\input{problems/}

%\input{problems/}

\section{Máj}
%\seminar{29}

\subsection*{Téma}
Algebraické výrazy a rovnice VI -- Kvadratické rovnice

\teachernote{
\subsection*{Ciele}
Precvičiť metódy používané pri práci s~kvadratickými rovnicami

\subsubsection*{Úvodný komentár}

Na začiatku seminára si spolu so študentami osviežime znalosti o kvadratických rovniciach, počte ich riešení a vzťahoch medzi reálnymi koreňmi a koeficientmi (Viètove vzorce). V čase konania seminára už študenti pravdepodobne budú mať za sebou preberanie tohto učiva na hodinách matematiky, takže by opakovanie nemalo zabrať priveľa času.}

\subsection*{Úlohy a riešenia}


% Do not delete this line (pandoc magic!)

\problem{B-57-I-5-N3}{
Nájdite všetky dvojice $(a,b)$ reálnych čísel, pre ktoré má každá z rovníc $x^2+(a-2)x+b-3=0$, $x^2+(a+2)x+3b-5=0$ dvojnásobný koreň.
}{
\rie Kvadratická rovnica má dvojnásobný koreň práve vtedy, ak jej diskriminant je rovný nule. Z tejto podmienky pre rovnice zo zadania dostávame
\begin{equation}
    \begin{aligned}
        a^2-4a-4b+16 & = 0, \\
        a^2+4a-12b+24 & =0.
    \end{aligned}
 \label{eq:B57I5N3}
\end{equation}
Odčítaním druhej rovnice od prvej máme po úprave $a=b-1$. Dosadením tohto vzťahu do jednej z rovníc v \ref{eq:B57I5N3} potom určíme možné hodnoty $b$, ktoré sú 3 a 7. K nim odpovedajúce hodnoty $a$ sú tak 2 a 6 a teda hľadané dvojice reálnych čísel $(a,b)$ sú $(2, 3)$ a $(6, 7)$.\\
\\
\kom Jednoduchá úloha na úvod, v ktorej študenti aplikujú znalosti o závislosti medzi hodnotou diskriminantu a počtom riešení kvadratickej rovnice. Ten potom vedie na riešenie sústavy dvoch rovníc s dvomi neznámymi.\\
\\
}
\\
\\
\kom Jednoduchá úloha na úvod, v ktorej študenti aplikujú znalosti o závislosti medzi hodnotou diskriminantu a počtom riešení kvadratickej rovnice. Ten potom vedie na riešenie sústavy dvoch rovníc s dvomi neznámymi.

% Do not delete this line (pandoc magic!)

\problem{B-57-I-5}{
Určte všetky dvojice $a, b$ reálnych čísel, pre ktoré má každá z~kvadratických rovníc
$$ax^2 + 2bx + 1 = 0, \ \ \ \ bx^2 + 2ax + 1 = 0$$
dva rôzne reálne korene, pričom práve jeden z~nich je spoločný obom rovniciam.
}{
\rieh Zo zadania vyplýva, že a $6\neq 0$, $b \neq 0$ (inak by rovnice neboli kvadratické)
a $a \neq b$ (inak by rovnice boli totožné, a ak by mali dva reálne korene, boli by oba
spoločné).

Označme $x_0$ spoločný koreň oboch rovníc, takže
$$ax_0^2+ 2bx_0 + 1 = 0,\ \ \ bx_0^2+ 2ax_0 + 1 = 0.$$
Odčítaním oboch rovníc dostaneme $(a - b)(x_0^2- 2x_0 ) = x_0 (a - b)(x_0 - 2) = 0$. Keďže $a \neq b$ a 0 zrejme koreňom daných rovníc nie je, musí byť spoločným koreňom číslo $x_0 = 2$. Dosadením do daných rovníc tak dostaneme jedinú podmienku $4a + 4b + 1 = 0$, čiže
$$b = -a -\frac{1}{4}.$$


Diskriminant druhej z~daných rovníc je potom $4a^2 - 4b = 4a^2 + 4a + 1 = (2a + 1)^2$, takže rovnica má dva rôzne reálne korene pre ľubovoľné $a \neq -\frac{1}{2}$. Podobne diskriminant prvej z~daných rovníc je $4b^2- 4a = 4b^2 + 4b +1 = (2b +1)^2$. Rovnica má teda dva rôzne reálne korene pre ľubovoľné $b \neq -\frac{1}{2}$, čiže $a\neq  \frac{1}{4}$

Z~uvedených predpokladov však zároveň vyplýva $a \neq -\frac{1}{4}$ $(b \neq 0)$ a $a \neq - \frac{1}{8}$ $(a \neq b)$.

\textit{Záver.} Vyhovujú všetky dvojice $(a, -a - \frac{1}{4})$, kde $a \in \RR \ \{-\frac{1}{2}, -\frac{1}{4}, -\frac{1}{8}, 0, \frac{1}{4}\}$.\\
}


\kom V úlohe sa k správnemu riešeniu dostaneme pomocou vhodného odčítania dvoch rovníc (a potom vhodnou úpravou takto vzniknutej rovnice). Považujeme za vhodné študentov na tento \uv{trik} upozorniť, keďže nájde uplatnenie nielen v nasledujúcej úlohe, ale aj v rôznych iných príkladoch.

% Do not delete this line (pandoc magic!)

\problem{B-57-II-1}{}{
Uvažujme dve kvadratické rovnice
$$x^2-ax-b = 0,\ \ \ \  x^2-bx-a = 0$$
s~reálnymi parametrami $a$, $b$. Zistite, akú najmenšiu a akú najväčšiu hodnotu môže nadobudnúť súčet $a + b$, ak existuje práve jedno reálne číslo $x$, ktoré súčasne vyhovuje obom rovniciam. Určte ďalej všetky dvojice $(a, b)$ reálnych parametrov, pre ktoré tento súčet tieto hodnoty nadobúda.
}{
\rieh Odčítaním oboch daných rovníc dostaneme rovnosť $(b-a)x+a-b = 0$, čiže $(b-a)(x-1) = 0$. Odtiaľ vyplýva, že $b = a$ alebo $x = 1$.

Ak $b = a$, majú obidve rovnice tvar $x^2-ax-a = 0$. Práve jedno riešenie existuje práve vtedy, keď diskriminant $a^2 + 4a$ je nulový. To platí pre $a = 0$ a pre $a = -4$. Pretože $b = a$, má súčet $a + b$ v~prvom prípade hodnotu $0$ a v~druhom prípade hodnotu $-8$.

Ak $x = 1$, dostaneme z~daných rovníc $a + b = 1$, teda $b = 1-a$. Rovnice potom majú tvar
$$x^2-ax + a-1 = 0 \ \ \ \ \text{a} \ \ \ \ x^2 + (a-1)x-a = 0.$$
Prvá má korene $1$ a $a-1$, druhá má korene $1$ a $-a$. Práve jedno spoločné riešenie tak dostaneme vždy s~výnimkou prípadu, keď $a-1 = -a$, čiže $a = \frac{1}{2}$ -- vtedy sú spoločné riešenia dve.

\textit{Záver.} Najmenšia hodnota súčtu $a + b$ je $-8$ a je dosiahnutá pre $a = b = -4$. Najväčšia hodnota súčtu $a + b$ je $1$; túto hodnotu má súčet $a + b$ pre všetky dvojice $(a, 1-a)$, kde $a\neq \frac{1}{2}$ je ľubovoľné reálne číslo.\\
\\\kom Úloha nadväzuje na predchádzajúcu, opäť rovnice v zadaní sčítame. Viac ako náročnosťou výpočtu je úloha zaujímavá svojim rozborom, kde je potrebné dať pozor na to, aby študenti správne zvážili oba prípady ($a=b$, $x=1$).\\
\\
}

\kom Úloha nadväzuje na predchádzajúcu, opäť rovnice v zadaní sčítame. Viac ako náročnosťou výpočtu je úloha zaujímavá svojim rozborom, kde je potrebné dať pozor na to, aby študenti správne zvážili oba prípady ($a=b$, $x=1$).

\input{problems/B-62-II-1.tex}


%\kom Je dôležité, aby si študenti uvedomili, že v tomto prípade dokazujú ekvivalenciu, teda je potrebné dokázať obe implikácie a dávať si pozor na, čo sú predpoklady, a čo je tvrdenie, ktoré sa snažíme dokázať. Prvá časť riešenia je relatívne priamočiara, v druhej \todo{XXX}.


% Do not delete this line (pandoc magic!)

\problem{B-59-I-6}{
Reálne čísla $a$, $b$ majú túto vlastnosť: rovnica $x^2 -ax+b-1 = 0$ má v~množine reálnych čísel dva rôzne korene, ktorých rozdiel je kladným koreňom rovnice $x^2 - ax + b + 1 = 0$.
\begin{enumerate}[a)]
    \item Dokážte nerovnosť $b > 3$.
    \item Pomocou $b$ vyjadrite korene oboch rovníc.
\end{enumerate}
}{
\rieh  Označme $x_1$ menší a $x_2$ väčší koreň prvej rovnice. Potom platí $x_1 + x_2 = a$, $x_1 x_2 = b - 1$. Druhá rovnica má koreň $x_2 - x_1$, a keďže súčet oboch koreňov je $a$, musí byť druhý koreň $a - (x_2 - x_1 ) = x_1 + x_2 - x_2 + x_1 = 2x_1$. Súčin koreňov druhej rovnice je $(x_2 -x_1 )\cdot2x_1 = b+1$. Odtiaľ dostávame $b = -1+2x_1 x_2 -2x_1^2= -1+2(b-1)-2x_1^2$, a teda
\begin{equation} \label{eq:B59I6_1}
    b = 3 + 2x_1^3> 3,
\end{equation}
lebo z~rovnosti $x_1 = 0$ by vyplývalo $b + 1 = b - 1 = 0$.

Keďže $x_2 - x_1 > 0$ a $b + 1 > 0$, musí byť aj $x_1 > 0$; z~\ref{eq:B59I6_1} máme $x_1 =\sqrt{(b - 3)/2}$ a ďalej
$$x_2 =\frac{b-1}{x_1}=\frac{(b - 1)\sqrt{2}}{\sqrt{b-3}}.$$
Korene druhej rovnice sú potom
$$x_2 - x_1 = \frac{b+1}{} \ \ \ \ \text{a} \ \ \ \  2x_1=\sqrt{2(b - 3)}.$$
\\
\textbf{Iné riešenie*.} Korene prvej rovnice sú
$$x_1 = \frac{a -\sqrt{a^2 - 4b + 4}}{2}, \ \ \ \  x_2 =\frac{a +\sqrt{a^2 - 4b + 4}}{2},$$
pričom pre diskriminant máme
\begin{equation} \label{eq:B59I6_2}
    D = a^2 - 4(b - 1) > 0.
\end{equation}
Rozdiel koreňov $x_2 - x_1 =\sqrt{a^2 - 4b + 4}$ je koreňom druhej rovnice, a preto
\begin{equation} \label{eq:B59I6_3}
    \begin{aligned}
        a^2 - 4b + 4 - a \sqrt{a^2 - 4b + 4} + b + 1 &= 0,\\
a^2 - 3b + 5 &= a\sqrt{a 2 - 4b + 4},\\
a^4 + 2a^2 (5 - 3b) + (3b - 5)^2 &= a^4 - 4a^2 b + 4a^2,\\
(3b - 5)^2 &= a^2 (2b - 6).
    \end{aligned}
\end{equation}
Rovnosť  $a = 0$ nastáva práve vtedy, keď $3b - 5 = 0$; potom by ale neplatilo \ref{eq:B59I6_2}. Preto $a^2 > 0$, $(3b - 5)^2 > 0$, a teda aj $2b - 6 > 0$, čiže $b > 3$. Z \ref{eq:B59I6_2} a \ref{eq:B59I6_3} potom vyplýva $a > 0$, a teda $a = (3b - 5)/\sqrt{2(b - 3)}$; ďalej potom
\begin{align*}
x_1 &=\frac{1}{2}\bigg( \frac{3a-5}{\sqrt{2(b-3)}}-\sqrt{\frac{(3b-5)^2}{2(b-3)}}-4b+4\bigg)=\sqrt{\frac{b-3}{2}},\\
x_2 &=\frac{1}{2} \bigg( \frac{3a-5}{\sqrt{2(b-3)}}+\sqrt{\frac{(3b-5)^2}{2(b-3)}}-4b+4\bigg)=\frac{(b-1)\sqrt{2}}{\sqrt{b-3}}.
\end{align*}
Druhá rovnica má korene
\begin{align*}
x_3 &=\frac{a-\sqrt{a^2-4b-4}}{2}=\frac{b+1}{\sqrt{2(b-3)}}=x_2-x_1\\
x_4 &=\frac{a+\sqrt{a^2-4b-4}}{2}=\sqrt{2(b-3)}.
\end{align*}
}


\kom Úloha sa dá vyriešiť relatívne \uv{netrikovo} vyjadrením koreňov prvej rovnice, dosadením ich rozdielu do druhej rovnice a odpovedajúcou diskusiou. Takýto prístup je síce zrozumiteľný, avšak dosť pracný. Ak študenti neprídu na prvý spôsob riešenia, považujeme za vhodné im ho ukázať ako dobrý príklad toho, ako nám použitie Viètovych vzorcov môže výraznej zjednodušiť výpočet.

% Do not delete this line (pandoc magic!)

\problem{B-64-II-4}{
Na tabuli je zoznam čísel $1, 2, 3, 4, 5, 6$ a \uv{rovnica}
$$\frac{\fbox{$\phantom{7}$}}{\fbox{$\phantom{7}$}}x^2+\frac{\fbox{$\phantom{7}$}}{\fbox{$\phantom{7}$}}x + \frac{\fbox{$\phantom{7}$}}{\fbox{$\phantom{7}$}}= 0.$$
Marek s~Tomášom hrajú nasledujúcu hru. Najskôr Marek vyberie ľubovoľné číslo zo zoznamu, napíše ho do jedného z~prázdnych políčok v~\uv{rovnici} a číslo zo zoznamu zotrie. Potom Tomáš vyberie niektoré zo zvyšných čísel, napíše ho do iného prázdneho políčka a v~zozname ho zotrie. Nato Marek urobí to isté a nakoniec Tomáš doplní tri zvyšné čísla na tri zvyšné voľné políčka v~\uv{rovnici}. Marek vyhrá, ak vzniknutá kvadratická rovnica s~racionálnymi koeficientmi bude mať dva rôzne reálne korene, inak vyhrá Tomáš. Rozhodnite, ktorý z~hráčov môže vyhrať nezávisle na postupe druhého
hráča.
}{
\rieh Označme $a$, $b$, $c$ koeficienty výslednej rovnice $ax^2 + bx + c = 0$. Tá má dva rôzne reálne korene práve vtedy, keď je jej diskriminant (v~symbolickej podobe)
$$b^2 - 4ac =\bigg( \frac{{\fbox{$\phantom{7}$}}}{{\fbox{$\phantom{7}$}}} \bigg)^2-4\bigg( \frac{{\fbox{$\phantom{7}$}}}{{\fbox{$\phantom{7}$}}}\bigg) \bigg(\frac{{\fbox{$\phantom{7}$}}}{{\fbox{$\phantom{7}$}}}\bigg)$$
kladný.

Ukážeme, že vyhrávajúcu stratégiu má Marek. Najskôr do menovateľa zlomku pre koeficient $b$ napíše $1$.
\begin{enumerate}[a)]
\item Ak Tomáš obsadí vo svojom prvom ťahu iné miesto ako v~čitateli $b$, napíše do neho Marek v~nasledujúcom ťahu najväčšie zostávajúce číslo zo zoznamu (teda 5 alebo 6). Hodnota $b^2$ potom bude aspoň 25 a zo zvyšných čísel možno zostaviť výraz $4ac$ s~hodnotou nanajvýš $4\cdot  \frac{6\cdot4}{3\cdot2}= 16$. Diskriminant vzniknutej kvadratickej rovnice tak bude určite kladný.
\item Predpokladajme, že Tomáš vo svojom ťahu doplní čitateľa $b$. Marek potom v~druhom ťahu napíše najmenšie zostávajúce číslo zo zoznamu (2 alebo 3) do čitateľa $a$ (alebo $c$).
\begin{enumerate}[(i)]
\item V~prípade, že Tomáš v~prvom ťahu napísal do čitateľa $b$ číslo 2, je hodnota $b^2$ rovná 4 a najväčšia možná hodnota $4ac$ (s~prihliadnutím na druhý Marekov ťah) je $4 \cdot \frac{3\cdot 6}{4\cdot 5}=\frac{18}{5}\leq  4$, teda diskriminant vzniknutej kvadratickej rovnice bude opäť kladný.
\item  V~prípade, že Tomáš v~prvom ťahu napísal do čitateľa $b$ iné číslo ako 2, je hodnota $b^2$ aspoň 9 a hodnota $4ac$ je nanajvýš $4 \cdot \frac{2\cdot 6}{3\cdot4} = 4$, takže diskriminant
vzniknutej kvadratickej rovnice bude aj v~tomto prípade kladný.

\end{enumerate}
\end{enumerate}
\textit{Záver.} V~danej hre môže vyhrať Marek nezávisle na ťahoch Tomáša. Jeho víťazná stratégia je opísaná vyššie.\\
\\
\kom Posledná úloha je zaujímavým spojením hľadania víťaznej stratégie a analýzy vlastností diskriminantu kvadratickej rovnice. Študentov necháme riešenie úlohy hľadať samostatne a potom ich vyzveme, aby stratégiu, ktorú našli, použili pri hre so spolužiakmi. Bude zaujímavé pozorovať, či nastane situácia, v ktorej aj neoptimálna stratégia zvíťazí.\\
\\
}


\kom Posledná úloha je zaujímavým spojením hľadania víťaznej stratégie a analýzy vlastností diskriminantu kvadratickej rovnice. Študentov necháme riešenie úlohy hľadať samostatne a potom ich vyzveme, aby stratégiu, ktorú našli, použili pri hre so spolužiakmi. Bude zaujímavé pozorovať, či nastane situácia, v ktorej aj neoptimálna stratégia zvíťazí.


\subsection*{Domáca práca}

% Do not delete this line (pandoc magic!)

\problem{B-57-S-2}{
Určte všetky dvojice $(a, b)$ reálnych čísel, pre ktoré majú rovnice
$$x^2 + (3a + b)x + 4a = 0, \ \ \ \  x^2 + (3b + a)x + 4b = 0$$
spoločný reálny koreň.
}{
\rieh Nech $x_0$ je spoločný koreň oboch rovníc. Potom platí
$$x_0^2+ (3a + b)x_0 + 4a = 0, \ \ \ \  x_0^2+ (3b + a)x_0 + 4b = 0.$$
Odčítaním týchto rovníc dostaneme $(2a-2b)x_0 +4(a-b) = 0$, odkiaľ po úprave získame $(a - b)(x_0 + 2) = 0$.

Rozoberieme dve možnosti:

Ak $a = b$, majú obidve dané rovnice rovnaký tvar $x^2 + 4ax + 4a = 0$. Aspoň jeden koreň (samozrejme spoločný) existuje práve vtedy, keď je diskriminant $16a^2-16a$ nezáporný, teda $a \in (-\infty, 0\rangle \cup \langle 1, \infty)$.

Ak $x_0 = -2$, dostaneme z~prvej aj z~druhej rovnice $4-2a-2b = 0$, teda $b = 2-a$. Dosadením do zadania dostaneme rovnice
$$x^2 + (2a + 2)x + 4a = 0, \ \ \ \ x^2 + (6-2a)x + 8-4a = 0,$$
ktoré majú pri ľubovoľnej hodnote parametra a spoločný koreň $-2$.

\textit{Záver.} Dané rovnice majú aspoň jeden spoločný koreň pre všetky dvojice $(a, a)$, kde $a \in (-\infty, 0\rangle \cup \langle 1, \infty)$, a pre všetky dvojice tvaru $(a, 2-a)$, kde $a$ je ľubovoľné.\\

}


% Do not delete this line (pandoc magic!)

\problem{B-59-S-1}{
Určte všetky hodnoty reálnych parametrov $p, q$, pre ktoré má každá z~rovníc
$$x(x - p) = 3 + q, \ \ \ \ x(x + p) = 3 - q$$
v~obore reálnych čísel dva rôzne korene, ktorých aritmetický priemer je jedným z~koreňov
zvyšnej rovnice.
}{
\rieh Z~Viètových vzťahov pre korene kvadratickej rovnice (ktoré vyplývajú z~rozkladu daného kvadratického trojčlena na súčin koreňových činiteľov) ľahko zistíme, že súčet koreňov prvej rovnice je $p$, takže ich aritmetický priemer je $\frac{1}{2}p$. Toto číslo má byť
koreňom druhej rovnice, preto
\begin{equation} \label{eq:B59S1_1}
    \frac{p}{2}\cdot \frac{3p}{2}= 3 - q. 
\end{equation}
Podobne súčet koreňov druhej rovnice je $-p$, ich aritmetický priemer je $-\frac{1}{2}p$, a preto
\begin{equation} \label{eq:B59S1_2}
    -\frac{p}{2}\cdot \bigg(- \frac{3p}{2}\bigg)= 3 + q. 
\end{equation}
Porovnaním oboch vzťahov~\ref{eq:B59S1_1} a~\ref{eq:B59S1_2} máme $3 - q = 3 + q$, čiže $q = 0$ a z~\ref{eq:B59S1_1} potom vyjde $p = 2$ alebo $p = -2$.

Z~oboch nájdených riešení dostaneme tú istú dvojicu rovníc $x(x - 2) = 3$, $x(x + 2) = 3$. Korene prvej z~nich sú čísla $-1$ a $3$, ich aritmetický priemer je $1$. Korene druhej rovnice sú čísla $1$ a $-3$, ich aritmetický priemer je $-1$.
}



%\teachernote{
%\subsection*{Doplňujúce zdroje a materiály}



%} % Seminar chyba!
\seminar{30}

\subsection*{Téma}
Kvadratické rovnice

\subsection*{Ciele}
Precvičiť metódy používané pri práci s~kvadratickými rovnicami

\subsection*{Úlohy a riešenia}

\input{problems/57-I-5.tex}

\input{problems/57-S-5.tex}

% Do not delete this line (pandoc magic!)

\problem{59-I-6}{cifry,domacekolo}{
Nájdite všetky prirodzené čísla, ktoré nie sú deliteľné desiatimi a ktoré vo svojom dekadickom zápise majú niekde vedľa seba dve nuly, po ktorých vyškrtnutí sa pôvodné číslo 89-krát zmenší.
}{
\rieh Rozoberieme niekoľko prípadov.

a) Predpokladajme najskôr, že nuly sú na treťom a druhom mieste sprava. Hľadané číslo $x$ má potom tvar $x = 1 000a + b$, pričom $a$ je prirodzené číslo (rovnako to bude aj v ďalších prípadoch, keď už to nebudeme pripomínať) a $b$ nenulová cifra. Podmienku
zo zadania $1 000a + b = 89(10a + b)$ upravíme na tvar $5a = 4b$, z ktorého vyplýva, že $b$ je násobok piatich. Vyhovuje tak iba $b = 5$ a $a = 4$, teda $x = 4 005$.

b) Ak hľadané číslo $x$ má nuly na štvrtom a treťom mieste sprava, je $x = 10 000a+b$, pričom $b$ je dvojciferné číslo. Podmienku zo zadania $10 000a+b = 89(100a+b)$ upravíme na tvar $25a = 2b$, z ktorého vyplýva, že $b$ je nepárny násobok čísla 25 (pripomíname, že $x$, a teda ani $b$, nie je deliteľné desiatimi). Odtiaľ $b = 25$, $a = 2$ alebo $b = 75$, $a = 6$, teda $x \in \{20 025, 60 075\}$.

c) Ak hľadané číslo $x$ má nuly na piatom a štvrtom mieste sprava, je $x = 100 000a+ b$, pričom $b$ je trojciferné číslo. Podmienku zo zadania $100 000a + b = 89(1 000a + b)$ upravíme na tvar $125a = b$, z ktorého vyplýva, že $b$ je nepárny násobok čísla 125.
Vyhovuje iba $b = 125$ a $a = 1$, $b = 375$ a $a = 3$, $b = 625$ a $a = 5$, $b = 875$ a $a = 7$, teda $x \in \{100 125, 300 375, 500 625, 700 875\}$.

d) Z predošlých prípadov vidíme, že pre hľadané číslo $x$ tvaru $x = 10^{n+2} a + b$, pričom $b$ je $n$-ciferné číslo, dostávame podmienku $10^{n+2} a + b = 89(10^n a + b)$, čiže $11 \cdot 10^n a = 88b$, odkiaľ pre $n \geq 4$ dostávame podmienku $125 \cdot 10^{n-3} a = b$, podľa ktorej je $b$ násobkom desiatich. Žiadne ďalšie $x$, ktoré by vyhovovalo zadaniu, teda neexistuje.

\textit{Záver.} Hľadané čísla sú $4 005, 20 025, 60 075, 100 125, 300 375, 500 625$, a $700 875$.
}

% Do not delete this line (pandoc magic!)

\problem{59-S-1}{
    Ak zväčšíme čitateľ aj menovateľ istého zlomku o 1, dostaneme zlomok o hodnotu 1/20 väčší. Ak urobíme s väčším zlomkom rovnakú operáciu, dostaneme zlomok o hodnotu 1/12 väčší, ako bola hodnota zlomku na začiatku. Určte všetky tri zlomky.
}{
\rieh Označme $a/b$ pôvodný zlomok. Podľa zadania platia rovnosti
$$\frac{a + 1}{b + 1}-\frac{a}{b}=\frac{1}{20} \ \ \ \ \text{a} \ \ \ \ \frac{a+2}{b+2}-\frac{a}{b}=\frac{1}{12} \ \ \ \  (a, b \in \NN),$$
ktoré sú ekvivalentné so vzťahmi
$$20b(a + 1)- 20a(b + 1) = b(b + 1) \ \ \ \ \text{a} \ \ \ \  12b(a + 2)- 12a(b + 2) = b(b + 2).$$
Tie upravíme na tvar $19b- 20a = b^2$ a $22b - 24a = b^2$. Po odčítaní oboch vzťahov zistíme, že $4a = 3b$, čo po dosadení do druhej rovnosti dá $22b- 18b = b^2$, čiže $b^2 = 4b$. Vzhľadom na podmienku $b \neq 0$ odtiaľ vyplýva $b = 4$ a $a = 3$.
Hľadané zlomky sú teda $\frac{3}{4}$, $\frac{4}{5}$ a $\frac{5}{6}$.

\textbf{Iné riešenie*.} Označme $a/b$ pôvodný zlomok. Zo vzťahov
$$\frac{1}{20}=\frac{1}{4\cdot 5} \ \ \ \ \text{a} \ \ \ \ \frac{1}{12}=\frac{1}{4\cdot 3}=\frac{2}{4\cdot 6}$$
možno odhadnúť, že riešením by mohlo byť $b = 4$. Potom
$$\frac{4(a + 1) - 5a}{4 \cdot 5}=\frac{1}{20} \ \ \ \ \text{a} \ \ \ \ \frac{4(a + 2) - 6a}{4 \cdot 6}=\frac{1}{12},$$
čiže $a = 3$. Musíme sa však ešte presvedčiť, že úloha iné riešenie nemá. Podmienky úlohy vedú ku vzťahom
$$\frac{b - a}{b(b + 1)}=\frac{1}{4\cdot 5} \ \ \ \ \text{a} \ \ \ \  \frac{2(b - a)}{b(b + 2)}=\frac{2}{4\cdot 6}.$$
Z podielu ich ľavých a pravých strán potom vyplýva
$$\frac{b + 2}{b + 1}=\frac{6}{5},$$
čomu vyhovuje jedine $b = 4$.

\textit{Poznámka.} V úplnom riešení nesmie chýbať vylúčenie možnosti $b \neq 4$. Napríklad z podobných rovností $1/20 = 30/(24 \cdot 25)$ a $1/12 = 52/(24 \cdot 26)$ by sme mohli hádať, že $b = 24$, čo riešením nie je.
}

% Do not delete this line (pandoc magic!)

\problem{62-II-1}{rozne,znamosti,kombi,krajskekolo}{
V tanečnej sa zišla skupina chlapcov a dievčat. Každý z prítomných 15 chlapcov pozná práve 4 dievčatá a každé dievča pozná práve 10 chlapcov. (Známosti sú vzájomné.) Dokážte, že ľubovoľní dvaja chlapci majú aspoň dve spoločné známe.
}{
\rieh Do každej známosti vstupuje práve jeden chlapec a každý z chlapcov má práve štyri známosti, spolu teda v tanečnej existuje $15 \cdot 4 = 60$ známostí. V každej známosti je však zastúpené práve jedno dievča a každé dievča má práve desať známostí. Ak označíme $d$ počet dievčat, tak $10 \cdot d = 60$. V tanečnej je teda 6 dievčat. Uvažujme ľubovoľného z chlapcov, povedzme Tomáša. Tomáš pozná 4 dievčatá, v tanečnej sú teda iba dve dievčatá, ktoré Tomáš nepozná. Ľubovoľný ďalší chlapec však pozná tiež štyri dievčatá, musí tak poznať aspoň dve z dievčat, ktoré pozná Tomáš.
}

% Do not delete this line (pandoc magic!)

\problem{64-II-4}{}{
Hovoríme, že kladné reálne číslo je copaté, ak nie je prirodzené a vo svojom dekadickom zápise obsahuje za desatinnou čiarkou iba konečne veľa nenulových cifier.

a) Nájdite dve copaté čísla $a, b$ také, že $a \cdot b =2015$.

b) Rozhodnite, či existujú tri copaté čísla $a, b, c$ také, že čísla $a \cdot b$, $b \cdot c$ a $c \cdot a$ sú
všetky prirodzené.
}{
\rieh a) Takých dvojíc copatých čísel je nekonečne veľa. Je to napr. dvojica $$a = 2 015 \cdot \frac{5}{2}=5037,5, \ \ \ b =\frac{2}{}5= 0,4.$$

Podobne vyhovuje každá z nekonečne veľa dvojíc
$$a = 2 015 \cdot \frac{5^m}{2^n},\ \ \  b =\frac{2^n}{5^m},$$
pričom $m$ a $n$ sú ľubovoľné prirodzené čísla. Uvedené číslo $a$ má $n$ desatinných miest, číslo $b$ ich má $m$.

b) Taká trojica copatých čísel neexistuje. Každé copaté číslo, ktoré má za desatinnou čiarkou poslednú nenulovú cifru na $k$-tom mieste, t. j. na mieste rádu $10^{-k}$, môžeme pre vhodné prirodzené číslo s zapísať ako $s \cdot 10^{-k} (k \geq 1)$. Pritom $s$ nie je deliteľné desiatimi, môže teda byť deliteľné iba  jedným z prvočísel 2 alebo 5, a to ľubovoľnou jeho mocninou.

Súčinom dvoch copatých čísel $a = s/10^k$ a $b = t/10^l$ dostaneme prirodzené číslo iba vtedy, keď je súčin $st$ deliteľný $10^{k+l}$ , čiže keď jedno z čísel $s, t$ je deliteľné $2^k+l$ a druhé $5^k+l$, pričom $k + l = 2$. Ak sú teda $a = s/10^k$, $b = t/10^l$, $c = u/10^m$ ľubovoľné copaté čísla také, že súčiny $a \cdot b$ a $a \cdot c$ sú prirodzené čísla, je z predchádzajúcej úvahy zrejmé, že obe čísla $t$ aj $u$ musia byť buď obe nepárne a deliteľné piatimi, alebo naopak obe párne a nedeliteľné piatimi, takže ich súčin tu nemôže byť deliteľný desiatimi, teda súčin $bc = tu/10^{l+m}$ nemôže byť celý.
}



\subsection*{Domáca práca}

Úlohou študentov bude vyhľadať, skonzultovať a zaslať vedúcemu seminára jeden príklad, problém alebo úlohu, ktorá sa viaže k témam algebry a teórie čísel, ktorými sme sa v seminári zaberali. Tieto úlohy budú použité ako zadania, ktoré využijeme v nasledujúcom seminári. Študenti môžu hľadať inšpiráciu v starších kolách MO, rôznych knižných publikáciách, zbierkach korešpondenčných seminárov alebo môžu úlohu dokonca sami vymyslieť.


\subsection*{Doplňujúce zdroje a materiály}




\seminar{31}

\subsection*{Téma}
Geometria -- stredové, obvodové, úsekové uhly, tetivové štvoruholníky
\teachernote{
\subsection*{Ciele}


\subsection*{Priebeh seminára}


\subsection*{Domáca práca}

}


\section{Jún}

\seminar{32}{Geometria VIII -- výpočtové úlohy}

\teachernote{
\subsection*{Ciele}
Precvičiť komplexnejšie úlohy zahŕňajúce geometrické výpočty
}


\subsection*{Úlohy a riešenia}

% Do not delete this line (pandoc magic!)

\problem{B-59-II-1}{
Kružnica $l(T; s)$ prechádza stredom kružnice $k(S; 2 cm)$. Kružnica $m(U; t)$ sa zvonka dotýka kružníc $k$ a $l$, pričom $US \perp ST$. Polomery $s$ a $t$ vyjadrené v centimetroch sú
celé čísla. Určte ich.
}{
\rieh Trojuholník $UST$ je pravouhlý. Jeho prepona $UT$ má dĺžku $s + t$, dĺžky odvesien sú $|US| = t + 2$, $|ST| = s$ (obr.~\ref{fig:B59II1}). Podľa Pytagorovej vety platí
$$(s + t)^2 = (t + 2)^2 + s^2.$$
Úpravami postupne dostávame
\begin{align*}
  s^2 + 2st + t^2 & = t^2 + 4t + 4 + s^2,\\
  st & = 2t + 2,\\
  t(s - 2) & = 2.  
\end{align*}
Čísla $t$ a $s - 2$ sú celé, preto $t$ musí byť deliteľom čísla $2$. Keďže $t$ je kladné, sú len dve možnosti; ak $t = 1$\,cm, tak $s = 4$\,cm, a ak $t = 2$\,cm, tak $s = 3$\,cm.
\begin{figure}[h]
  \centering
  \includegraphics{images/B59II1\imagesuffix}
  \caption{}
  \label{fig:B59II1}
\end{figure}
}


% Do not delete this line (pandoc magic!)

\problem{B-66-S-2}{}{
Na odvesnách $AC$ a $BC$ daného pravouhlého trojuholníka $ABC$ určte postupne body $K$ a $L$ tak, aby súčet
$$|AK|^2+ |KL|^2+ |LB|^2$$
nadobúdal najmenšiu možnú hodnotu a vyjadrite ju pomocou $c = |AB|$.
}{
\rieh V súlade s obr.~\ref{fig:B66S2} označme $x = |CL|$, $y = |CK|$, potom $|BL| = a - x$, a $|AK| = b - y$, pričom $a$, $b$ sú postupne dĺžky odvesien $BC$, $AC$.
\begin{figure}[h]
    \centering
    \includegraphics{images/B66S2\imagesuffix}
    \caption{}
    \label{fig:B66S2}
\end{figure}
Použitím Pytagorovej vety v pravouhlom trojuholníku $KLC$ dostaneme $|KL|^2= x^2 + y^2$, takže skúmaný súčet môžeme upraviť nasledujúcim spôsobom:
\begin{align*}
    |AK|^2+ |KL|^2+ |LB|^2 & = (b - y)^2+ x^2+ y^2+ (a - x)^2=\\
    & = 2x^2+ 2y^2 - 2ax - 2by + a^2+ b^2=\\
    & = 2\bigg(x-\frac{a}{2}\bigg)^2+ 2\bigg( y -\frac{b}{2}\bigg)^2+\frac{a^2 + b^2}{2}=\\
    & = 2\bigg(x -\frac{a}{2}\bigg)^2+ 2\bigg( y -\frac{b}{2}\bigg)^2+\frac{c^2}{2}.
\end{align*}
Vďaka nezápornosti druhých mocnín z toho vidíme, že skúmaný výraz nadobúda svoju najmenšiu hodnotu, konkrétne $\frac{1}{2}c$, práve vtedy, keď $x =\frac{1}{2}a$ a súčasne $y=\frac{1}{2}b$, teda práve vtedy, keď body $K$, $L$ sú postupne stredmi odvesien $AC$, $BC$ daného pravouhlého trojuholníka $ABC$.

\textit{Záver.} Najmenšia možná hodnota skúmaného súčtu je rovná $\frac{1}{1}c^2$. Túto hodnotu dostaneme práve vtedy, keď body $K$, $L$ budú postupne stredmi odvesien $AC$, $BC$ daného pravouhlého trojuholníka.
}


% Do not delete this line (pandoc magic!)

\problem{B-63-S-3}{}{
Na priamke $a$, na ktorej leží strana $BC$ trojuholníka $ABC$, sú dané body dotyku všetkých troch jemu pripísaných kružníc (body $B$ a $C$ nie sú známe). Nájdite na tejto priamke bod dotyku kružnice vpísanej.
}{
\rie
V danom trojuholníku $ABC$ označme $X$, $Y$, $Z$ body dotyku vpísanej kružnice s jeho stranami a $x = |AY | = |AZ|$, $y = |BX| = |BZ|$, $z = |CX| = |CY|$ zhodné úseky dotyčníc k vpísanej kružnici z jednotlivých vrcholov (obr.~\ref{fig:B63S3_1}). Ak označíme\\
\begin{figure}[h]
    \centering
    \includegraphics{images/B63S3_1\imagesuffix}
    \caption{}
    \label{fig:B63S3_1}
\end{figure}
zvyčajným spôsobom $a$, $b$, $c$ dĺžky jednotlivých strán, platí
$$a = y + z, \ \ \ \  b = z + x, \ \ \ \ c = x + y.$$
Sčítaním týchto troch rovníc dostaneme (pomocou $s$ ako zvyčajne označujeme polovičný obvod trojuholníka)
$$2s = a + b + c = 2x + 2y + 2z,$$
takže nám vyjde
\begin{equation} \label{eq:B63S3}
    x + y + z = s, \ \ \ \ x = s - a,\ \ \ \ y = s - b, \ \ \ \ z = s - c.
\end{equation}
Pozrime sa teraz na pripísanú kružnicu trojuholníku $ABC$, ktorá sa dotýka jeho strany $BC$ v bode $P$ a polpriamok $AB$ a $AC$ v bodoch $R$ a $Q$ (obr.~\ref{fig:B63S3_2}). Zo zhodnosti úsekov príslušných dotyčníc k~tejto kružnici máme
$$|AR| = |AQ|, \ \ \ \ |BR| = |BP|, \ \ \ \ |CP| = |CQ|,$$
odkiaľ vychádza
\begin{align*}
    2|AR| = |AR| + |AQ| & = |AB| + |BR| + |AC| + |CQ|  = \\
& = |AB| + |BP| + |AC| + |CP|  = a + b + c = 2s,
\end{align*}
čiže $|AR| = |AQ| = s$. Z tejto rovnosti ale vyplýva, že $|BP| = |BR| = s - c$, čo je podľa \ref{eq:B63S3} zároveň dĺžka z úsečky $CX$, teda $|BP| = |CX|$. To znamená, že body $P$ a $X$ sú súmerne združené podľa stredu úsečky $BC$.
\begin{figure}[h]
    \centering
    \includegraphics{images/B63S3_2\imagesuffix}
    \caption{}
    \label{fig:B63S3_2}
\end{figure}
Analogicky by sme odvodili rovnosti $|BK| = s$ a $|CL| = s$ pre body dotyku $K$ a $L$ kružníc pripísaných stranám $CA$ a $AB$ (obr.~\ref{fig:B63S3_2}) trojuholníka $ABC$ s priamkou $a$. Z týchto posledných rovností však vidíme, že $|BL| = s - a = |CK|$, teda aj body $K$ a $L$ sú súmerne združené podľa stredu úsečky $BC$. Body $K$ a $L$ sú známe (z troch daných bodov na priamke sú to tie dva krajné), poznáme teda aj stred $S$ strany $BC$ (je to stred úsečky $KL$) a bod $X$ nájdeme ako obraz tretieho daného bodu $P$ v stredovej súmernosti podľa stredu úsečky $BC$.
}


% Do not delete this line (pandoc magic!)

\problem{B-65-I-3}{seminar32,geompoc,pomery,netrgeo,domacekolo}{
V pravouhlom trojuholníku $ABC$ s preponou $AB$ a odvesnami dĺžok $|AC| = 4$\,cm a $|BC| = 3$\,cm ležia navzájom sa dotýkajúce kružnice $k_1(S_1; r_1 )$ a $k_2(S_2; r_2)$ tak, že $k_1$ sa dotýka strán $AB$ a $AC$, zatiaľ čo $k_2$ sa dotýka strán $AB$ a $BC$. Určte najmenšiu a najväčšiu možnú hodnotu polomeru $r_2$.
}{
\rieh Majme také dve kružnice, ktoré spĺňajú predpoklady úlohy (obr.~\ref{fig:B65I3_1}). Zrejme stred $S_1$ leží na osi uhla $BAC$ a stred $S_2$ na osi uhla $ABC$. Ďalej si uvedomme, že veľkosť
\begin{figure}[h]
    \centering
    \includegraphics{images/B65I3_1\imagesuffix}
    \caption{}
    \label{fig:B65I3_1}
\end{figure}
polomeru $r_1$ kružnice $k_1$ je priamo úmerná dĺžke úsečky $AS_1$ a podobne veľkosť $r_2$ priamo úmerná dĺžke úsečky $BS_2$. Keď zväčšíme polomer jednej z kružníc, musí sa nutne polomer druhej kružnice zmenšiť.

Kružnica $k_2$ nemôže mať polomer väčší ako najväčšia kružnica, ktorú možno do trojuholníka $ABC$ vpísať. Takou kružnicou je zrejme kružnica $k$ do trojuholníka $ABC$ vpísaná. A naopak najmenší polomer bude mať kružnica $k_2$, ak zvolíme $k_1 = k$. (Že v oboch opísaných prípadoch pre $k_2 = k$ aj pre $k_1 = k$ existuje príslušná \uv{vpísaná} kružnica $k_1$, resp. $k_2$, je vcelku zrejmé.)

Stačí teda vypočítať polomer $r$ kružnice $k$ do trojuholníka $ABC$ vpísanej a polomer kružnice $k_2$, ktorá sa dotýka kružnice $k$ a strán $AB$ a $BC$ daného trojuholníka.

Polomer $r$ vpísanej kružnice vypočítame napríklad zo vzorca $2S_{ABC} = ro$, pričom $S_{ABC}$ označuje obsah trojuholníka $ABC$ a $o$ jeho obvod.%\footnote{Iný postup využívajúci pravouhlosť trojuholníka $ABC$ je predmetom dopĺňajúcej úlohy.}
Obsah daného pravouhlého trojuholníka $ABC$ s preponou $AB$ je pri zvyčajnom označení dĺžok strán rovný $\frac{1}{2}ab.$ Prepona v trojuholníku $ABC$ má (v centimetroch) podľa Pytagorovej vety veľkosť $c= \sqrt{a^2 + b^2}=\sqrt{3^2 + 4^2} = 5$. Maximálny polomer kružnice $k_2$ je teda
$$r =\frac{2S_{ABC}}{o}=\frac{ab}{a+b+c}=\frac{3\cdot4}{3+4+5}= 1.$$

Pre výpočet polomeru $r_2$ kružnice $k_2$, ktorá sa dotýka kružnice $k$ a strán $AB$ a $BC$, označme $D$ a $E$ body, v ktorých sa kružnice $k$ a $k_2$ dotýkajú strany $AB$, a $F$, $G$ dotykové body kružnice k postupne so stranami $BC$ a $AC$ (obr.~\ref{fig:B65I3_2}). Keďže daný trojuholník je
\begin{figure}[h]
    \centering
    \includegraphics{images/B65I3_2\imagesuffix}
    \caption{}
    \label{fig:B65I3_2}
\end{figure}
pravouhlý, je $S_1FCG$ štvorec so stranou dĺžky $r = 1$, takže $|BF| = |BD| = 2$ a podľa Pytagorovej vety $|BS_1| =\sqrt{5}$. Z podobnosti pravouhlých trojuholníkov $BES_2$ a $BDS_1$ potom vyplýva
$$\frac{r_2}{|BS_2|}=\frac{r}{|BS_1|}, \ \ \ \ \text{čiže} \ \ \ \ \frac{r_2}{\sqrt{5}- r_2 - 1}=\frac{1}{\sqrt{5}}.$$
Po úprave tak pre hľadanú hodnotu neznámej $r_2$ dostaneme lineárnu rovnicu
$$r_2(\sqrt{5} + 1) =\sqrt{5}-1,$$
ktorú ešte zjednodušíme vynásobením $\sqrt{5}-1$. Zistíme tak, že najmenšia možná hodnota polomeru kružnice $k_2$ je rovná $$r_2 = \frac{3-\sqrt{5}}{2}.$$
}


%

\problem{B-61-II-3}{
Pravouhlému trojuholníku $ABC$ je vpísaná kružnica, ktorá sa dotýka prepony $AB$ v bode $K$. Úsečku $AK$ otočíme o $90^\circ$ do polohy $AP$ a úsečku $BK$ otočíme o $90^\circ$ do polohy $BQ$ tak, aby body $P$, $Q$ ležali v polrovine opačnej k polrovine $ABC$.
\begin{enumerate}[a)]
    \item Dokážte, že obsahy trojuholníkov $ABC$ a $PQK$ sú rovnaké.
    \item Dokážte, že obvod trojuholníka $ABC$ nie je väčší ako obvod trojuholníka $PQK$.Kedy nastane rovnosť obvodov?
\end{enumerate}
}{
\rieh a) Označme $S$ stred a $r$ polomer kružnice vpísanej trojuholníku $ABC$ a $L$, $M$ body dotyku tejto kružnice postupne so stranami $BC$, $CA$ \todo{(obr. 1)}. Ak označíme $|AK| = x$, $|BK| = y$, tak $|AP| = |AM| = x$, $|KP| = x\sqrt{2}$, $|BQ| = |BL| = y$, $|KQ| = y\sqrt{2}$. Keďže oba uhly $AKP$, $BKQ$ majú veľkosť $45^\circ$, je trojuholník $PQK$ pravouhlý, takže jeho obsah je 
$$S_{PQK} =\frac{x\sqrt{2}y\sqrt{2}}{2}=xy.$$

\todo{DOPLNIŤ Obr. 1}\\
\\
Štvoruholník $SLCM$ je štvorec so stranou dĺžky $r$ a $|AM| = x$, $|BL| = y$. Obsah trojuholníka $ABC$ je rovný súčtu obsahov trojuholníkov $ABS$, $BCS$ a $CAS$, teda
$$S_{ABC}=\frac{(x + y)r + (y + r)r + (x + r)r}{2}= (x + y + r)r.$$
Obsah trojuholníka $ABC$ je zároveň rovný
$$S_{ABC}=\frac{|AC| \cdot |BC|}{2}=\frac{(x + r)(y + r)}{2}=\frac{xy}{2}+\frac{(x + y + r)r}{2}=\frac{xy}{2}+\frac{S_{ABC}}{2}.$$
Odtiaľ dostávame $S_{ABC} = xy$, čiže $S_{ABC} = S_{PQK}$, čo sme mali dokázať.


b) V trojuholníku $ABC$ sú dĺžky strán $a = y + r$, $b = x + r$, $c = x + y$. Obvod trojuholníka $ABC$ je $a + b + |AB|$, obvod trojuholníka $PQK$ je $x\sqrt{2} + y\sqrt{2} + |PQ|$.

Zrejme platí $|AB| \leq |PQ|$ ($|AB|$ je vzdialenosťou rovnobežiek $AP$, $BQ$, \todo{obr. 1}). Rovnosť nastane jedine v prípade $|AP| = |BQ|$, čiže $x = y$.
Ešte dokážeme, že $a + b \leq  x\sqrt{2} + y\sqrt{2}$, teda že $a + b \leq c\sqrt{2}$. Posledná nerovnosť je ekvivalentná s nerovnosťou, ktorú dostaneme jej umocnením na druhú, pretože obe jej strany sú kladné. Dostaneme tak $a^2 +b^2 +2ab \leq 2c^2$. Keďže v pravouhlom trojuholníku $ABC$ platí $a^2 + b^2 = c^2$, máme dokázať nerovnosť $2ab \leq a^2 + b^2$, ktorá je však ekvivalentná s nerovnosťou $0 \leq (a - b)^2$. Tá platí pre všetky reálne čísla $a$, $b$ a rovnosť v nej nastane jedine pre $a = b$, t. j. $x = y$.

Celkovo vidíme, že obvod trojuholníka $ABC$ je menší alebo rovný obsahu trojuholníka $PQK$ a rovnosť nastane práve vtedy, keď je pravouhlý trojuholník $ABC$ rovnoramenný.
}

\subsection*{Domáca práca}

Keďže v nasledujúcom seminári je naplánované opakovanie, úlohou študentov bude si zbežne zopakovať, čomu sme sa posledných 9 mesiacov venovali. Zmyslom domácej práce nie je opätovné prepočítavanie všetkých príkladov, ale skôr získanie prehľadu a nadhľadu nad študovanými témami.
%Keďže v nasledujúcom seminári je naplánované opakovanie v réžii študentov, ich úlohou bude pripraviť si zhrnutie jednej zo štyroch oblastí, ktorými sme sa v seminári zaoberali (algebra, teória čísel, geometria, kombinatorika). Cieľom nie je znova prepočítavať všetky úlohy, ale zrozumiteľne zhrnúť kľúčové poznatky a stratégie, ktoré sme spolu so študentmi v seminároch používali. Študentov rozdelíme do štyroch skupín a každej z nich pridelíme jednu oblasť.


\seminar{33}

\subsection*{Téma}
Opakovanie I -- pohľad späť na všetko, čo sme sa naučili

\subsection*{Ciele}

Zopakovať kľúčové myšlienky, postupy a poznatky, ktoré v priebehu roka študenti získali.

\subsection*{Priebeh seminára}

%\todo{mind maps}

Študentov rozdelíme do 4 skupín a každej skupine pridelíme jednu zo študovaných oblastí (algebra, teória čísel, geometria, kombinatorika). Úlohou skupín bude vytvoriť myšlienkovú mapu, ktorá zhŕňa poznatky z priradenej oblasti -- mala by obsahovať nie len konkrétne fakty (napr. \uv{Súčet veľkostí protiľahlých vnútorných uhlov tetivového štvoruholníka je $180^\circ$.}), ale aj všeobecnejšie prístupy alebo metódy (napr. \uv{Pozor na kreslenie náčrtu.}).  Ak sa študenti ešte s myšlienkovými mapami nestretli, stručne im ich vysvetlíme, príp. ukážeme niekoľko príkladov. Na vypracovanie študentom necháme 20-30 minút. Po tomto čase jednotlivé skupiny prezentujú svoje výtvory a zvyšok osadenstva prispieva svojimi komentármi a otázkami. Na jednu skupinu si odporúčame vyhradiť aspoň 15 minút. Zaujímavé bude sledovať, či sa niektoré poznatky budú vyskytovať vo viacerých skupinách, prípadne či sa študenti budú odkazovať aj na inú oblasť, než akú spracovali.

Tento zvolený spôsob upevnenia a prepojenia poznatkov pokladáme za prínosný, pretože študentom dáva priestor premýšľať trochu iným spôsobom 


\subsection*{Doplňujúce materiály}

O myšlienkových mapách je možné nájsť viac na \url{https://www.mindmapping.com/mind-map.php} alebo \url{https://www.mindtools.com/pages/article/newISS_01.htm}, kde je tiež k dispozícii množstvo príkladov. 
O myšlienkových mapách je možné nájsť viac na \url{\detokenize{https://www.mindmapping.com/mind-map.php}} alebo \url{https://www.mindtools.com/pages/article/newISS_01.htm}, kde je tiež k dispozícii množstvo príkladov.


%Keďže seminárne stretnutie bude tentokrát najmä v réžii študentov, bude prebiehať trochu inak. Každá zo štyroch skupín predstaví svoju domácu prácu a spoločne so ostatnými študentmi 

%Úlohou učiteľa je výklad a myšlienky študentov vhodne korigovať tak, aby spoločne so študentami pokryli väčšinu oblastí. Cieľom tohto seminára nie je znova prechádzať všetky úlohy, ktoré sme doteraz so študentami vyriešili, ale skôr zopakovať všeobecné myšlienky a postupy, na ktoré je v jednotlivých oblastiach dobré myslieť.

%V prípade, že chce mať učiteľ istotu toho, že na seminári zaznejú všetky kľúčové myšlienky, môže študentov požiadať, aby mu domácu prácu zaslali v predstihu a skontrolovať, príp. doplniť ju.

%V prvej časti seminára prejdeme spolu so štyrmi skupinami základné oblasti problémov, ktorým sme sa na seminári venovali a v závere môžeme študentom nechať priestor na zdieľanie poznatkov alebo poučení, ktoré do žiadnej konkrétnej oblasti úloh nespadajú.





\chapter*{Záver}
\label{chap:zaver}
\addcontentsline{toc}{chapter}{\nameref{chap:zaver}}

V diplomovej práci sme sa venovali tvorbe prezenčného seminára pre študentov prvého ročníka strednej školy so záujmom o matematiku, konkrétnejšie príprave podrobných osnov vrátane riešených príkladov. Seminár má za cieľ podporovať študentov pri príprave na matematické súťaže, najmä matematickú olympiádu. Po tom, čo sme seminár zasadili do kontextu matematického vzdelávania stredoškolákov a preskúmali sme úlohy matematickej olympiády kategórie C v posledných desiatich rokoch, sme jednotlivé stretnutia tematicky rozdelili a doplnili o semináre konzultačné a zamerané aj na inú súťaž než olympiáda, konkrétne \textit{Náboj}.

Pri tvorbe obsahu sme sa snažili zakomponovať rôzne metódy výučby. Využili sme samostatnú i skupinovú prácu, matematické hry, tímové súťaže i prípravu obsahu seminára samotnými študentmi. To snáď prispelo k pestrosti seminára a rozvoju schopností a vedomostí študentov. Zároveň však pokladáme za dôležité poznamenať, že obsah a forma seminára nie sú jediným \uv{správnym} alebo najvhodnejším spôsobom práce so študentami a že prístupov k tejto úlohe je nespočetne mnoho. Taktiež nemá byť osnova seminára vnímaná ako nemenná dogma, ale skôr ako živý organizmus, ktorý sa väčšou alebo menšou mierou prispôsobuje potrebám konkrétnych študentov účastniacich sa konkrétneho seminára.

Výstup práce môže byť užitočný nielen vedúcim matematických krúžkov a seminárov, ale aj ostatným učiteľom matematiky. Tí v práci môžu nájsť náročnejšie úlohy než v bežných stredoškolských učebniciach matematiky a využiť ich v pravidelnej výuke pri práci s talentovanejšími študentmi v triede. Osnovy seminárov tiež poslúžia ako dobrý podporný materiál pre študentov, ktorí majú záujem pestovať svoje matematické zručnosti, jazyk práce a riešenia jednotlivých úloh sú totiž písané štýlom dobre zrozumiteľným aj pre stredoškolákov.

Na záver už len dodám, že čas strávený vypracovaním diplomovej práce bol pre mňa veľmi prínosný, vďaka hľadaniu materiálov som objavila mnoho zaujímavých zdrojov úloh a metód, literatúry, aktivít a príkladov a som zvedavá, aký ohlas budú mať, až ich zakomponujem do svojej učiteľskej praxe.

\renewcommand{\bibname}{Zoznam použitej literatúry}
\HlavickaLiteratura
\label{chap:bib}
\addcontentsline{toc}{chapter}{\nameref{chap:bib}}
%\nocite{*}
\printbibliography

\chapter*{Prílohy}
\label{chap:pril}
\addcontentsline{toc}{chapter}{\nameref{chap:pril}}

\section*{Príloha A}
TODO: Tu budu priklady pre studentov.
