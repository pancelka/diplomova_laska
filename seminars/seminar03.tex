\seminar{3}

\subsection*{Téma}
Prístup k riešeniu matematických úloh, typy dôkazov II

\teachernote{
\subsection*{Ciele}
Prediskutovať so študentmi rôzne prístupy k riešeniu neznámych problémov, zopakovať a/alebo zoznámiť s typmi dôkazov používaných v matematike.

\subsection*{Priebeh}

Seminár prebehne formou štrukturovanej diskusie, v ktorej so študentami rozoberieme korektný spôsob riešenia matematických úloh a problémov, zamyslíme sa nad tým, čo musí správne riešenie obsahovať a čoho sa, naopak, vyvarovať a vyzbrojíme študentov základnými stratégiami, ktoré môžu pri riešení úloh využiť. 

\subsubsection*{Typy úloh}

Pri riešení matematických problémov sa stretávame s dvoma základnými kategóriami: buď je úlohou dokázať (príp. vyvrátiť) dané tvrdenie, alebo nájsť objekty (čísla, tvary, výrazy, množiny bodov), ktoré vyhovujú zadanými podmienkam. 

\subsubsection*{Fázy riešenia}

Podľa \todo{Holton} má riešenie problémov nasledujúce fázy.
\begin{enumerate}[a)]
\item 
\item Prečítanie a porozumenie.
\item Kľúčové slová.
\item Panika.
\item Systém.
\item Vzorce.
\item Odhad. 
\item Matematická technika.
\item Vysvetlenie.
\item Zovšeobecnenie.
\end{enumerate}

\subsubsection*{Strátegie na začiatok}

\subsubsection*{Typy dôkazov}

\begin{enumerate}
\item Priamy dôkaz.
\item Dôkaz sporom.
\item Dôkaz použitím matematickej indukcie.
\end{enumerate}

\todo{DOPLNIŤ.} %Spoločne so študentami prečítame preriešime prvú kapitolu z~\cite{holton2010}
}

