% Do not delete this line (pandoc magic!)

\problem{57-S-2}{seminar12,obsahy,geompoc}{
V~danom rovnobežníku $ABCD$ je bod $E$ stred strany $BC$ a bod $F$ leží vnútri strany $AB$. Obsah trojuholníka $AFD$ je $15$\,cm$^2$ a obsah trojuholníka $FBE$ je $14$\,cm$^2$. Určte obsah štvoruholníka $FECD$.
}{
\rieh Označme $v$ vzdialenosť bodu $C$ od priamky $AB$, $a = |AB|$ a $x = |AF|$. Pre obsahy trojuholníkov $AFD$ a $FBE$ (obr.~\ref{fig:57S2_1}) platí $\frac{1}{2}x\cdot v~= 15$, $\frac{1}{2}(a - x) \cdot \frac{1}{2}v = 14$. Odtiaľ $xv = 30$, $av - xv = 56$. Sčítaním oboch rovností nájdeme obsah rovnobežníka $ABCD$: $S_{ABCD} = av = 86$\,cm$^2$. Obsah štvoruholníka $FECD$ je teda $S_{FECD} = S_{ABCD}- (S_{AFD} + S_ {FBE}) = 57$\,cm$^2.$
\begin{figure}[h]
    \centering
    \begin{minipage}{0.45\textwidth}
        \centering
        \includegraphics[width=0.9\textwidth]{images/57S21\imagesuffix}
        \caption{}
        \label{fig:57S2_1}
    \end{minipage}\hfill
    \begin{minipage}{0.45\textwidth}
        \centering
        \includegraphics[width=0.9\textwidth]{images/57S22\imagesuffix}
        \caption{}
        \label{fig:57S2_2}
    \end{minipage}
\end{figure}
\\
\textbf{Iné riešenie*.} Trojuholníky $BEF$ a $ECF$ majú spoločnú výšku z~vrcholu $F$ a zhodné základne $BE$ a $EC$. Preto sú obsahy oboch trojuholníkov rovnaké. Z~obr.~\ref{fig:57S2_2} vidíme, že obsah trojuholníka $CDF$ je polovicou obsahu rovnobežníka $ABCD$ (oba útvary majú spoločnú základňu $CD$ a rovnakú výšku). Druhú polovicu tvorí súčet obsahov trojuholníkov $AFD$ a $BCF$. Odtiaľ $S_{FECD} = S_{ECF} + S_{CDF} = S_{ECF} + (S_{AFD} + S_{BCF}) = S_{AFD} + 3 S_{FBE} = 57$\,cm$^2$.\\
\\
\textbf{Iné riešenie*.} Do rovnobežníka dokreslíme úsečky $FG$ a $EH$ rovnobežné so stranami $BC$ a $AB$ tak, ako znázorňuje~obr.~\ref{fig:58S2_3}.
\begin{figure}[h]
    \centering
    \includegraphics{images/57S23\imagesuffix}
    \caption{}
    \label{fig:58S2_3}
\end{figure}
Rovnobežníky $AFGD$ a $FBEH$ sú svojimi uhlopriečkami $DF$ a $EF$ rozdelené na dvojice zhodných trojuholníkov. Takže $S_{GDF} = S_{AFD} = 15$\,cm$^2$ a $S_{HFE} = S_{BEF} = 14$\,cm$^2$. Zo zhodnosti rovnobežníkov $HECG$ a $FBEH$ navyše ľahko usúdime, že všetky štyri trojuholníky $FBE$, $EHF$, $HEC$ a $CGH$ sú zhodné, takže obsah štvoruholníka $FECD$ je $S_{AFD} + 3S_{FBE} = 57$\,cm$^2$.\\
\\
\kom Úloha je zaradená ako rozcvička pred komplexnejšími problémami, nie je totiž veľmi náročná na vyriešenie. Pekne tiež demonštruje, že niekedy nám vhodný prístup, náčrtok alebo správne nakreslená priamka v~obrázku riešenie úlohy významne zjednoduší.\\
\\
}
