% Do not delete this line (pandoc magic!)

\problem{58-I-2-D1}{}{
Nech $k$ je kružnica opísaná pravouhlému trojuholníku $ABC$ s~preponou $AB$ dĺžky $c$. Označme $S$ stred strany $AB$ a $D$ a $E$ priesečníky osí strán $BC$ a $AC$ s~jedným oblúkom $AB$ kružnice $k$. Vyjadrite obsah trojuholníka $DSE$ pomocou dĺžky prepony $c$.
}{
\rie Trojuholník $DSE$ je pravouhlý rovnoramenný s~pravým uhlom pri vrchole $S$, pretože odvesny $DS$ a $ES$ ležia na osiach navzájom kolmých strán. Odvesny majú dĺžku $\frac{c}{2}$, pretože sú to polomery kružnice opísanej trojuholníku $ABC$. Obsah trojuholníka $DSE$ je $\frac{1}{2}\cdot|DS|\cdot |DE|=\frac{1}{2}\cdot \frac{c}{2}\cdot\frac{c}{2}=\frac{c^2}{8}.$ \\
\\
}
