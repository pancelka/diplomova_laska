% Do not delete this line (pandoc magic!)

\problem{~\cite{holton2010}, úloha 38, str. 115}{}{
Nech $N$ je päťciferné kladné číslo také, že $N=\overline{a679b}$. Ak je $N$ deliteľné 72, určte prvú cifru $a$ a poslednú cifru $b$.
}{
\rieh Keďže je číslo $N$ deliteľné $72=8\cdot 9$, musí byť súčasne deliteľné ôsmimi aj deviatimi. Z~pravidla pre deliteľnosť ôsmimi vyplýva, že číslo $\overline{79b}$ musí byť násobkom ôsmich a teda $b=2$. Pravidlo pre deliteľnosť deviatimi diktuje, že ciferný súčet hľadaného čísla $a+6+7+9+2=a+24$ je násobkom deviatich, dostávame tak $a=3$. Hľadaným číslom je $N=36792$.\\
\\
\kom Úloha nie je náročná a je zaradená ako zahrievacie cvičenie a ukážka práce s~deliteľnosťou zloženým číslom.\\
\\
}
