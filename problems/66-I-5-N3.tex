% Do not delete this line (pandoc magic!)

\problem{66-I-5-N3}{trojuholniky,podtroj,domacekolo}{
Dokážte, že ľubovoľná spojnica ramien daného lichobežníka $ABCD$, ktorá je rovnobežná s jeho základňami $AB \parallel CD$, je úsečka, ktorej stred leží na spojnici stredov oboch základní. Potom dokážte, že priesečník uhlopriečok P je stredom tej zo spomenutých spojníc ramien, ktorá týmto priesečníkom prechádza.
}{
\rieh Použite najskôr výsledok úlohy \todo{N2} pre podobné trojuholníky so spoločným vrcholom, ktorým je priesečník predĺžených ramien, a protiľahlými stranami, ktorými sú jednak základňa lichobežníka, jednak uvažovaná spojnica ramien. Na dôkaz vlastnosti priesečníka $P$ označte $E \in BC$, $F \in AD$ krajné body prislúchajúcej spojnice ramien a využite to, že podobnosť trojuholníkov $APF$, $ACD$ má taký istý koeficient ako podobnosť trojuholníkov $BEP$, $BCD$.
}
