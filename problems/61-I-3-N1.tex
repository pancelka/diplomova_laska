% Do not delete this line (pandoc magic!)

\problem{61-I-3-N1}{seminar08,nsdnsn,domacekolo}{
Určte, pre ktoré prirodzené čísla $a, b$ platí $(a, b) = 10$ a zároveň  $[a, b] = 150$.
}{
\rieh Pretože $10 = 2 \cdot 5$ a $150 = 2 \cdot 3 \cdot 5^2$, požadované rovnosti sú splnené práve vtedy, keď $a = 2 \cdot 3^s \cdot 5^t$ a $b = 2 \cdot 3^u \cdot 5^v$, kde $\{s, u\} = \{0, 1\}$ a $\{t, v\} = \{1, 2\}$. Riešením je teda jedna zo štvoríc $\{a, b\} = \{10, 150\}$ alebo $\{a, b\} = \{30, 50\}$.\\
\\
\kom Úloha je relatívne jednoduchá a nevyžaduje žiadne špeciálne znalosti, zároveň však nie je triviálna. Tvorí tak príjemné preklenutie medzi školskými a olympiádnymi príkladmi.\\
\\
}
